Chapter2
GalacticysicsAstroph
2.1orkHomewOne
−t
2.1.1.AssumethattheGalaxyis10Gyrold,therateofstarformationinthepastaswproportionaltoeT
wheretisthetimesincethegalaxyformedandT=3Gyr,andthestellarlifetimesareengivyb
()
M−3
t(M)=10GyrM
⊙
Calculatetheframctionsofall(a)2Mand(b)5Mstarserevformedthatarestillaround.toyda
⊙⊙
Solution:
Lettandtbethelifetimesof2Mstarsand5Mstars.Then
25⊙⊙
t2=10times(2M⊙)−3=11Gyr=1.25Gyr
M4
⊙
()
5M−32
t5=10times⊙=Gyr=0.8Gyr
M⊙25
IfNisthetotal2Mstarserevformed,then
2f⊙
10
∫−tk[−10]
N=keTdt=−e3−1=−0.32k
2fT
0
yAn2Mstarformedearlierthantfromtoydaareallgonesotheremaining2Mstarsareformed
⊙2⊙
eenbwet10−t2=10−1.25=8.75Gyrandtoyda(10Gyr)fromthebeginning.
10
∫−tk[−10−8.75]−3
N=keTdt=−e3−e3=−6.14×10k
2rT
8.75
Sotheratiooftotal2Mstarstillformedtothatarestillaroundis
⊙
N−6.14×10−3k
2r−2
N=−0.32k=1.91×10
2f
Sincethestarformationrateisindeptendenofmass,thetotal5Mstarserevformedisequaltothe
total2Mstars.So,N=−0.321k.yAn5Mstarformedearlierthan⊙tfromtoydaareallgoneso
⊙5f⊙5
42
CHAPTER2.CTICGALAOPHYSICSASTR43
theremaining5Mstarsareformedeenbwet10−t=10−0.08=9.92Gyrandtoyda(10Gyr)from
⊙2
thebeginning.
10
∫−tk[10−9.92]−4
N5r=keTdt=−e3−e3=−3.21×10k
T
9.92
Sotheratiooftotal5Mstarstillformedtothatarestillaroundis
⊙
N−3.21×10−4k
5r−4
N=−0.32k=9.99×10
5f
□
2.1.2.(a)Aclose(i.e.ed)unresolvbinaryconsistsofowtstarsheacoftapparenmagnitudem.Whatisthe
tapparenmagnitudeofthebinary?
(b)AstarhastapparenmagnitudemV=10andisdeterminedspectroscopicallytobeanA0main
sequencestar.Whatisitsdistance?(SeeeSpark&GallagherableT1.4.)
Solution:
Thelfux(f)magnitude(m)relationism=−2.5log(f).Sothelfuxofheacstarsisengiv.yb
−m
f=102.5
−m
Thelfuxiseadditivsothetotallfuxofbinaryisjustwicetofthisftot=2×f=2×102.5.wNothe
tapparanmagnitude(m)ofthebinaryis:
(−m)(m)
m=−2.5log(ftot)=−2.5log2×102.5=−2.5log(2)−=m−0.75
2.5
Sothetapparanmagnitudeofbinaryism−0.75.
enGivthatthetapparanmagnitudeofthestarismV=10,AsitisaA0fromthetablethealuev
forabsolutemagnitudeisfoundtobeM=0.80.eWwknothattherelationeenbwettheabsolute
V
magnitude(M)andtapparanmagnitude(m)andthedistaneofthestar(r),
VV
M−m=5(1−log(r))whererisinparsec
Vv
−9.2=5(1−log(r))
log(r)=2.84
r=102.84=691.83pc
Sothedistaneofthestaris691.83pc□
2.1.3.IfthemassfunctionforstarswsfollotheSalpeterdistribution,with
ξ(M)dN=AM2.35
dM
(wheredNisthebumernofstarswithmasseseenbwetMandM+dM;seeeSpark&Gallagher,p.
66),forMl<M<Mu,withMlMu,andthestellarymass–luminositrelationis
L(M)∝M4,
wshothatthetotalbumernandtotalmassofstarsdependmainlyonMl,whilethetotalyluminosit
dependsmainlyonMu.,SpeciifcallyforMl=0.2MandMu=100M,calculatethemassesM1
⊙⊙
andM2hsucthat50%ofthetotalmassistainedconinstarswithM<M1,while50%ofthetotal
yluminositistainedconinstarswithM>M2.
Solution:
ThetotalmassofdNstarsisMdN.Sothetotalmassoftherangeis
M=∫MuMdN=∫MuM·AM−2.35dN=A∫MuM−1.35dN=A(M−0.35−M−0.35)
tot−0.35ul
MlMlMl
CHAPTER2.CTICGALAOPHYSICSASTR44
SinceewevhatoifndM=Mhsucthathalfthetotalmasseenbwet0.2Mand100Mistobeequal
u1⊙⊙
tothetotalmassintherange0.2MandM.Let’ssupposeM=αM.So,
⊙11⊙
1[A{−0.35−0.35}][A{−0.35−0.35}]
(100M)−(0.2M)=(αM)−(0.2M)
2−0.35⊙⊙−0.35⊙⊙
1[−0.35−0.35][−0.35−0.35]
2100−0.2=α−0.2
[−0.35−0.35]1
α=100+0.2−0.35
2
α=1.06
Soforthestarintherange0.2Mto1.06Mevhahalfthetotalbumernofthestars.Theyluminosit
⊙⊙
ofheacstarofmassMisproportionaltoM4andtherearedNhsucstars.Sothetotalyluminositof
startseenbwetmassMandM+dMisproportionaltoM4dN,Sothetotalyluminositoftherange
MandMisatconstantimes
lu
∫M∫M∫M
uuuA()
L=M4dN=M4·AM−2.35dN=AM1.65dN=M2.65−M2.65
tot2.65ul
MlMlMl
SinceewevhatoifndM=Mhsucthathalfthetotalyluminositeenbwet0.2Mand100Mistobe
l2⊙⊙
equaltothetotalyluminositintherangeMand100M.Let’ssupposeM=βM.So,
l⊙1⊙
1[A{2.652.65}][A{2.652.65}]
(100M)−(0.2M)=(100M)−(βM)
22.65⊙⊙2.65⊙⊙
1[2.652.65][2.652.65]
2100−0.2=100−β
[2.652.65]1
100+0.22.65
β=2
β=76.98
M=76.98MSothestarsintherange77Mto100Mevhahalftheyluminositasthatoftotal
2⊙⊙⊙
starsintherange.□
2.1.4.Astronomersoftenximateapprothestellarmassfunction￿(M)ybaSalpeterwer-lapwowithaw-masslo
cutoff,buttheKroupadistribution
−0.3
CMforM≤0.1M
⊙
ξ(M)=BM−1.3for0.1M<M≤0.5M
⊙⊙
AM−2.35forM>0.5M
⊙
isactuallyahucmbetterdescription[Aisthesameasinpart(a)andtheothertsconstanBandC
arehosenctoensurethat￿isuous.]tinconIftheuppermasslimitinallcasesisMu=100Mandew
⊙
assumethesamesimpliifedymass–luminositrelationasinpart(a),whatw-masslocutoffMlustmbe
hosencinorderthatthetruncatedwer-lapwohasthesame(i)totalbumernofstars,(ii)totalmass,
and(iii)totalyluminositastheKroupadistribution?
Solution:
Sincetheengivfunctionξ(M)shouldbeuous,tunconheacpieceshouldevhaequalaluevatthe.boundary
B(0.5M)−1.3=A(0.5M)−2.35⇒B=2.070M−1.05A
⊙⊙⊙
C(0.1M)−0.3=B(0.1M)−1.35⇒C=10M−1B=20.70M−2.05A
⊙⊙⊙⊙
CHAPTER2.CTICGALAOPHYSICSASTR45
ThetotalbumernofstarsengivybKroupadistributionis
100M⊙0.1M⊙0.5M⊙100M⊙
N=∫ξ(M)dM=∫20.70AM−2.05M−0.3dM+∫2.07AM−1.05M−1.3dM+∫AM−2.35dM
⊙⊙
000.1M⊙0.5M⊙
N=5.90AM−1.35+5.27AM−1.35+1.88AM−1.35=13.05M−1.35A
⊙⊙⊙⊙
Alsothetotalbumernofstarengivybsalpeterdistributionwitherwlomasslimitas(αM⊙)
100M⊙
N=∫AM−2.35dM
αM⊙
=0.74(αM⊙)−1.35A−0.0014M−1.35A
⊙
Equatingthesealuesv
13.05M−1.05A=0.74(αM⊙)−1.35A−0.0014M−1.35A
⊙⊙
⇒α−1.35=17.63
⇒α=0.11
Thereforetheerwlolimitis0.11M⊙ifSalpeterdistributionandKroupadistributionevhathesame
bumernofstars.
orkingWintheunitsofM=1andA=1:
⊙
ThetotalMassofstarsengivybKroupadistributionis
1000.10.5100
M=∫Mξ(M)dM=∫20.70M0.7dM+∫2.07M−0.3dM+∫M−1.35dM
000.10.5
M=0.24+1.23+3.07=4.54
AlsothetotalMassofstarengivybsalpeterdistributionwitherwlomasslimitas(αM⊙)
100
M=∫M×M−2.35dM
α
=2.85α−0.35−0.57
Equatingthesealuesv
4.54=2.85α−0.35−0.57
⇒α−0.35=1.79
⇒α=0.19
Thereforetheerwlolimitis0.19MfortheSalpeterdistributionandKroupadistributiontoevhathe
⊙
sametotalmass.
ThetotalyLuminositofstarsengivybKroupadistributionis
1000.10.5100
L=∫M4ξ(M)dM=∫20.70AM3.7dM+∫2.07M2.7dM+∫M1.65dM
000.10.5
−5
L=8.78×10+0.042+75292.85=75292.89
CHAPTER2.CTICGALAOPHYSICSASTR46
AlsothetotalyLuminositofallstarsengivybsalpeterdistributionwitherwlomasslimitas(αM⊙)
100
L=∫M4×M−2.35dM
α
=75292.92−0.37α2.65
Equatingthesealuesv
75292.89=75292.92−0.37α2.65
⇒α2.65=0.07
⇒α=0.36
Thereforetheerwlolimitis0.36MforSalpeterdistributionandKroupadistributionevhathesame
⊙
.yLuminosit
□
2.1.5.(a)UseGauss’swlatoederivanexpressionforthevitationalgraforceinthezdirectionduetoan
inifnitesheetofsurfaceydensitΣlyinginthex–yplane.(b)Astarhaseloyvcit30km/sperpendicular
totheGalacticplaneasitcrossestheplane,andisedobservtoevhaaummaximdepartureeabvothe
planeof500pc.ximatingApprothediskasaninifnitevitatinggrasheetofmatter,estimateitssurface
ydensitΣ(i)inkgm2and(ii)inM⊙pc−2
Solution:
Thevitationalgralfux(Φ)thourghaclosedsurfaceenclosingmassMis
encl
Φ=4πGMencl(2.1)
Ifewassumethegalacticplaneasaninifnitesheetofmassuniformlydistributedervoasurfacewith
surfaceydensitΣandewetaktheGaussiansurfaceasacylynderofradiusaperpendiculartotheplane,
thenthetotalmassincludedwithinthecylinderouldwbeMencl=Area×Σ=πa2Σ.Butthetotal
surfaceareaofcylinderthatisperpendicular(zdirection)tothePlaneis2πa2.IfEisthevitationalGra
ifeldatthecylindersurface,thentotallfux(Φ)throughtheareaisE×2πa2Substutingthealuesvof
ΦandMin(2.1)ewget.
encl
2πa2E=4πG(πa2Σ)
⇒E=2πGΣ
Sothevitatationalgraforceperunitmassinthezdirectionis2πGΣ.
19
enGivthatastarwitheloyvcit=30andelsvtraamaxdistanceof=500=154310.
vkm/sspc.×m
Sinethevitationalgraifeldistconstanandisindeptendenofdistanceeabvothegalacticplane.eWcan
usethetconstanacclerationkinematicsrelationv2−v2=2as.Sincethespeedatuhmaximdistance
fi
iszero.
v2
a=i
2s
Buttheacclerationa=2πGΣ
v2(3×104)2
Σ=i==0.069kgm−2
19−11
4πGs2×1.543×10×4π×6.672×10
−31−232−2
Since1kg=5.02×10M⊙and1m=9.52×10pc
−3132−2−2
Σ=0.069×5.02×10×9.52×10Mpc=33.30Mpc
⊙⊙
SothesurfacemassydensitΣfortheengivplanargalaxyis0.069kgm2≡33.30M⊙pc−2.□
CHAPTER2.CTICGALAOPHYSICSASTR47
2.2orkHomewowT
2.2.1.Acertaintelescopehaslimitingvisualtapparenmagnitudem=22.Whatistheummaximdistance
V
athwhicitcoulddetect(a)theSun(absolutemagnitudeM=4.8),(b)anRRLyraeariablev(M
VV
=0.75),aCepheidariablev(M=−3.5),and(d)aypteIaasupverno(M=−20).
VV
Solution:
Ifmisthelimitingtaparranmagnitudeofthetelescope,ythinganwithtapparanmagnitudegreater
V
thanmouldwnotbeedresolvybthetelescope.Sotheummaximdistancethatthetelescopecanstill
V
eresolvisthedistanceinhwhicthetapparanmagnitudeofheacofthestarsisequaltothelimiting
tapparanmagnitude.
Ifewsupposedmaxistheummaximdistance.Then
(d)
M−m=−5logmax
vv10
m−M
vv
Rightarrowd=10×105
max
Sincethelimitingmagnitude(m)=22
v
22−Mv
d=10×105inarsecP
max
22−4.8
•orFSunMV=4.8,limitingdistancedmax=10×105=27.54kpc
22−0.75
•orFRRLyreMV=0.75,limitingdistancedmax=10×105=177.82kpc
22+3.5
•orFCepheidariablevM=−3.5,limitingdistanced=10×105=1.25Mpc
Vmax
22+20
•orFIaaSupvernoM=−20,limitingdistanced=10×105=2.511Gpc
Vmax
□
2.2.2.Asimpleaxisymmetricmodelofthestellarbumernydensitn(R,z)intheGalacticdiskis
R|z|
−/−/
hh
n(R,z)=n0eRez,
whereRisdistancefromtheGalacticter,cenzisdistancefromthediskplane,andhRandhzare
t)(constanscalets.heigh(a)IfallstarsevhathesameyluminositL∗,tegrateintheeabvoexpression
withrespecttoztodeterminethedisksurfacetnessbrighΣ(R)(thatis,thetotalyluminositperunit
areaatyanengivlocation).(b)wNotegrateinΣwithrespecttoRtodeterminethetotalyluminosit
10
Lofthe.Galaxy(c)IfL=2×10L,andh=4kpc,whatisthelocalsurfacetnessbrighinthe
GG⊙R
yvicinitoftheSun,atR=8kpc?(d)Ifh=250pcandL=L,calculatethelocalydensitofstars
z∗⊙
inthesolarbneighorhood(atz=0).
Solution:
enGivallstarsevhasameyluminositL∗theyluminositperunitareais:
∞∞
∫−R−|z|∫−R−z
hhhh
Σ(R)=LneRezdz=2LneRezdz
∗0∗0
−∞[]0
−R−z∞−R
hhh
=−LhneRez=2nheRL
∗z000z∗
wNoforthetotalyLuminositthefunctionΣ(R)istegratedinfromR=0to∞.
∞∞
∫∫−R
h
L=Σ(R)dR=2nheRLdR
G0z∗
00
([])
−R∞
h
=2nhL−heR=2nhhL
0z∗r00Rz∗
CHAPTER2.CTICGALAOPHYSICSASTR48
TheeabvoexpressionforLesgivthetotalyluminositofgalaxyintermsoftheyluminositofheac
G
starsL∗
orFyvicinitofsunatR=8kpcandL=2×1010Landh=4kpc
G⊙R
L2.5×109
L=2nhhL⇒h=G⇒h=L
G0Rz∗z2LhnznL⊙
∗R00∗
SothelocalsurfacetnessbrighΣ(R)attheyvicinitofsunthenis
R98
−h2.5×10−48
Σ(R)=2nheRL=2n··eL⇒6.76×10L
0z∗0nL∗⊙
0∗
Thelocalydensitofstarsaroundz=0is
80
n(8kpc,0)=n0e4e=0.13n0
□
2.2.3.(a)enGivthedeifnitionsoftheOorttsconstanAandBtedpreseninclass(Eqs.2.13and2.16inthe
text),
()
1V′1(RV)′
A=−RB=−

2R2RR=R
R=R0
0
erifyvthatA+B=V′(R)andAB=V/R,whereV(R)istheGalacticrotationw,laRisthe
0000
distancefromtheSuntotheGalacticter,cenandV=V(R0).
0
(b)HencewritewndoanestimateofV,ifR=8kpc.
00
(c)Considerthesphericallysymmetricydensitdistribution￿engivyb
()
R2−1
ρ(R)=ρ1+
0a2
eDerivanexpressionforthemassinsideradiusR.WhatisthecircularorbitalspeedV(R)atradius
R?HencedeterminetheformofA(R)andB(R)forR≫a.
Solution:
()
1V′1(RV)′
A=−2RRB=−2R
=−1R(V′−V)=−11(V+RV′)
2RR22R
11V1V1′
=−2V′+2R=−2R−2V
aluatingEvatR=RaluatingEvatR=R0
0
1V(R)1
11V(R)0′
A=−V′(R)+0B=−−V(R0)
02R2
22R00
wNothatewevhathealuesvforheactsconstanAandB.
V(R)
A+B=−V′(R)A−B=0
0R
0
CHAPTER2.CTICGALAOPHYSICSASTR49
LetusconiderawholloshellofradiusRwithknessthicdR.Thentheolumevofthetialdifferen
shellis
dV=4πR2dR
Thetialdifferenrelationformasscanbewrittenas.
dM=ρ(R)dV
()
R2−12
=ρ1+4πRdR
02
a
ThetotalmassenclosedinthesphereofradiusRisengivybthetegralinofdMfrom0toR
RR()
M(R)=∫dM=∫ρ01+R2−14πR2dR
2
a
00
R
=4πρ0∫R22dR
1+R
2
0a
2(-1(R))
=4πρ0aR−atan/
a
2(-1(R))
M(R)=4πρ0aR−atan/(2.2)
a
oTcalculatetheV(R)ewcanusetherelation.
V2(R)GM(R)
R=R2
22(-1(R))
V(R)G4πρ0aR−atan/
=aSubstutingM(R)from(2.2)
R√R2
(a-1(R))
V(R)=2aGπρ01−Rtana
-1(R)πa
ifR≫athentana≈2alsoR→0Then.√
V(R)=2aGπρ0
SinceVhasnodependenceonR,V′=0
√√
A(R)=−1V′+1V=0+12aGπρ0=aGπρ0
22R2RR
√√
B(R)=−1V−1V′=−12aGπρ0+0=−aGπρ0
2R22RR
□
2.2.4.IfourGalaxyhasalfatrotationecurvwithV0=210km/sandthetotalyluminositofthediskisas
inProblem2,whatistheGalacticmasstotlighratioM/Linside(a)thesolarcircle(R=8kpc),(b)
0
10R0?ComparethesewiththemasstotlighratioofaSalpeterstellarmassdistribution(seeorkHomew
1,Problem3)withM=0.2M,M=100M.
l⊙u⊙
Solution:
otalTyluminositinsideofradiusRcanbecalculatedas
RR
∫∫−R
h
L(R)=Σ(R)dR=2nheRL
0z∗
0(0)
−R
h
=2hRhzn01−eRL∗
CHAPTER2.CTICGALAOPHYSICSASTR50
Substutingh=4kpc,h=250pc,L=Lineabvoexpression
rz∗⊙
6(R)
L(R)=2×101−e4kpcn0L⊙
Iftherotationecurvislfat,themasscanbecalculatedas
RV2
M(R)=G=48.83RM⊙/pc
orFR=8kpc()
L(R)=2.0×1061−e−2noL⊙=1.72×106n0L⊙
RV2
5
M(R)==48.83×8000M=3.90×10M
G⊙⊙
Theratiothenis:
6
1.72×10nL
0⊙M⊙
M/L==0.22n0/
5L
3.90×10M⊙
⊙
orFR=10R=80kpc
0
6(−20)
L(R)=2.0×101−enoL⊙=1.99n0L⊙
RV2
M(R)==48.83×80000M=3.90×106M
G⊙⊙
Theratiothenis:
1.99L⊙−7M⊙
M/L==5.1×10n0/
6L
3.90×10M⊙
⊙
orFsalpeterdistributionξ(M)=AM−2.35Thetotalmassis
100M100M
∫⊙∫⊙
−1.356
M=Mξ(M)dM=AMdM=1.49×10A
.2M⊙.2M⊙
100M⊙100M⊙
L=∫M4ξ(M)dM=∫AM1.65dM=2.13×104A
.2M⊙.2M⊙
Theratiois
1.49×106A
M/L=4=69.95
2.13×10A□
2.3orkHomewThree
−3
n≈cm
2.3.1.NeutralydrogenhatomsinthecoolterstellarinmediumevhabumernydensitH1andtem-
peratureT100K.
123
(a)wShothattheeragevaspeedv¯oftheseatoms,deifnedybmv¯=kT(wheremisthemass
2H2H
ofydrogenhatomandkisBoltzmann’st),constanis
()1
/
T2
v¯≈2kms−1100K.
Solution:
CHAPTER2.CTICGALAOPHYSICSASTR51
123
enGivthattheeragevaspeedv¯oftheseatoms,deifnedybmv¯=kTItcanberearranged
2H2
toin
√√√√1
()1/2()/
3kT3×k×100T3×1.38×10−23×100TT2
v¯===−27=1.57kms−1
mm100K1.67×10100K100K
HH
□
(b)Hencewshothattheypicaltatomicter-of-masscenkineticenergyishucmgreaterthantheenergy
differenceeenbwettheypherifnestatesassociatedwiththe21-cmradioline.
Themeantimeeenbwetcollisionsforatomsinthistvironmenenisafewthousandears,ywhilethe
7
meantimeforanexcitedatomtoemita21cmphotonis≈1.1×10.Asaresult,thepopulations
oftheerwloandupperypherifnestatesaredeterminedtirelyenybcollisionalprocessesandthe
statesarepopulatedproportionaltotheirstatisticalts,eighwsothree-quartersofallydrogenh
atomsareintheupperstate.
Solution:
Theenergyassociatedwith21cmlineis
hc−25−6
E=λ=9.485×10J=5.92×10eV
TheypicaltenergyisV13.6ehwhicishucmgreaterthantheenergyassociatedwith21cmline□
9
(c)Calculatethetotal21-cmyluminositofagalaxytainingconatotalof5×10M⊙ofneutral
ydrogen.h
Solution:
5×109M
⊙57
ThetotalbumernofneutralydrogenhisN=mH=1.13×10.
7−1−15−1
Therateofemissionofphotonisf=(1.1×10)/yr=2.28×10s.
Sothetotalyluminositdueto21cmphotonisengivyb
L=N·f·E=3.11×1018W
Sothetotalyluminositfotheengivgalaxyis3.11×1018□
CHAPTER2.CTICGALAOPHYSICSASTR52
2.3.2.WhatarethesoundspeedandJeansmass.(Inallcases,assumeanadiabaticindexγ=5)
3
63
(a)inamolecularcloudcore(pureH)oftemperature10Kandbumernydensit1×10molecules/cm?
2
Solution:
MassofydrogenhmoleculeHism=3.34×10−27,T=10K
2H2
C=√γkT=262.41m/s;ρ=mn=3.34×10−15kgm−3
smH2
√πc24π
s9215
λ==8.048×10m;M=ρλ=7.29×10kg
jρj3j
□
3
(b)inatomicydrogenhgaswithtemperature100Kandbumernydensit1atom/cm
Solution:
MassofydrogenhatomismH=1.67×10−27,T=100K
√γkT−21−3
Cs=√m=117.56m/s;ρ=mHn=1.67×10kgm
πc24π
s13220
λ==5.09×10m;M=ρλ=9.22×10kg
jρj3j
□
6−33
(c)inhotionizedydrogenhwithtemperature1×10Kandbumernydensit1×10protons/cm?
Solution:
−27
Massofionizedydrogenhmoleculeism=1.67×10,T=100K
p
√γkT5−24−3
C==1.17×10m/s;ρ=mn=1.67×10kgm
smp
√πc24π
s17228
λ==1.67×10m;M=ρλ=2.91×10kg
jρj3j
□
2.3.3.AiratseaellevonEarthhasydensit￿=1.2kg/m3andsoundspeedvs=330m/s.
(a)WhatisitsJeanslength?WhatistheJeansmass?
Solution:
√πc24π
s228
λ==5.339×10m;M=ρλ=7.65×10kg
jρj3j
8
TheJeanslengthis533.9mandtheJeansmassis7.65×10kg□
(b)Bywhohucmdoestheyvitself-graofairhangecthefrequencyofasoundevawofelengthvaw1
m?
Solution:
Thefrequencyof1melengthvawevawonearthisf=vs/λ=330HzThehangecinfrequency
duetovitationgraisrelatedyb
f2−f2=Gρ(2.3)
nπ
CHAPTER2.CTICGALAOPHYSICSASTR53
Ifewsuppposehangedcfrequencyfn=f+∆fand∆fiseryvsmallthen
()()
∆f2∆f
f2−f2=f2−(f+∆f)2=f2−f21+≈f2−f21+2=2∆f·f
nff
Substutingthisdifferencetoin(2.3)ewget
∆f=Gρ=6.672×10−11·1.2=3.86×10−14Hz
πfπ330
22·
□
2.3.4.(a)Aparticleisdropped(fromradiusawithzeroeloy)vcittointhevitationalgraptialotencorre-
spondingtoastatichomogeneoussphereofradiusaandydensit￿.Calculatewholongtheparticle
estaktohreacthetercenofthesphere.
Solution:
Lettheydensitofthemassydensitofthehomogenousspherebeρ.Alsolethemassofthesphere
withintheshellofradiusrbeM(r).
M(r)=ρV(r)=ρ4πr3
3
ritingWtheequationofmotionfromNewton’sws.la
GM(r)Gρ4πr342(24)
r¨=−=−3=−Gπρr=−ωrwherew=Gπρ(2.4)
r2r233
Thesecondordertialdifferenequation(2.4)istheellwwnknoSHMequationhwhichasperiodic
solutionoftheform.
r(t)=Acos(ωt)+Bsin(ωt)(2.5)
tiatingDifferen(2.5)ewget
r˙(t)=−Aωsin(ωt)+Bωcos(ωt)
whereAandBaretheparametersdeterminedybtheboundaryalue.vSincetheparticlestarts
fromthesurfaceofthespherer(0)=aandtheinitialspeedr˙(0)=0.Usingtheseboundaryaluesv
ewifndthealuesvofAandB.TheusthdeterminedareA=aandB=0.So(2.5)becomes
√
4
r(t)=acos(ωt)Whereω=/Gπρ(2.6)
3
IfTisthetimetheparticleestaktohreacthetercenofthesphericaldistributionthenr(T)=0
sothesolutionof(2.6)esgiv
ωT=π⇒T=π=√3π
22ω16Gρ
ThetimeTisthetimetheparticleestaktohreacthetercenofsphericaldistribution.□
(b)Calculatethetimerequiredforahomogeneoussphereofradiusaandydensit￿withnoternalin
pressuresupporttocollapseunderitswno.yvitgra
Solution:
Ifthesphericaldistributioncollapsesybitswno,yvitgrathenastheparticleonthesurfaceis
pulledardwinardswtotheter,centhemasscompressesandsothemassinsidethesphericalshell
atyantimeistconstan
M(r)=ρ4πa3
3
CHAPTER2.CTICGALAOPHYSICSASTR54
ritingWtheequationofmotionfromNewton’sws.la
GM(r)Gρ4πa3ω2(243)
r¨=−=−3=−wherew=Gπρa(2.7)
r2r2r23
eWcantransformr¨
r¨≡dr˙≡drdr˙≡r˙dr˙(2.8)
dtdtdrdr
Onusing(2.8)(2.7)becomes
r˙dr˙=(−ω2)dr⇒1r˙2=ω2+K
r22r
Theboundaryconditionisthatatr=athestartingspeedofparticleisr˙=0Substutingthis
−ω2
kbacewifndK=/.eWget
a
√()1
√1111−2√
r˙=2ωr−a⇒r−adr=2ωdt(2.9)
1
Thesolutionof(2.9)is
3-1(√r)r−a√
a2sin−√=2ωt+C(2.10)
a1−1
ar
r−a-1(√r)ππ3/2
lim√=0;limsin=⇒C=a(2.11)
r→a1−1r→aa22
ar
Using(2.11)in(2.10)ewget
3(-1(√r)π)r−a√
a2sin−−√=2ωt(2.12)
a21−1
ar
IfTisthetimetheparticleestaktohreacthetercenr(T)=0sothesolutionof(2.10)esgiv
√3√32√
ππ3/2aπ3π
2ωT=a2⇒T=√a==
43
222ω8·/Gρπa32Gρ
3
ThetimeTisthetimetheparticleestaktohreacthetercenofsphericaldistributionhwhicisthe
timeofthecollapseofthemassdistributionunderitswnovitationalgrapull.□
2.4orkHomewourF
2.4.1.Estimatethemassesofstarclustersvingha
(a)rootmeansquareeloyvcit10km/sandhalf-massradius10pc,
Solution:⟨⟩
224
enGivv=10km/s,themeansquarespeedisv=(10km/s)=1×10Thetotalmassis
rms
engivyb
⟨2⟩416
M=6Rhv=6·1×10·10×3.08×10=2.78×1036kg=1.39×106M
G6.67×10−11⊙
Sothemassoftheclusteris1.39×106M□
⊙
1
edsolvybySymp1.1.1underythonp3.5
CHAPTER2.CTICGALAOPHYSICSASTR55
(b)meanydensit100pc−3,rmseloyvcit2km/s,andmeanstellarmass0.8M⊙,
Solution:
Ifthebumernydensitisnanderagevastellarmassism¯thenthemeanmassydensit
−33⟨2⟩4
ρ=n·m¯=100pc·0.8M=80M/pc;v=2km/s⇒v=4×10
⊙⊙rms
()
3M3M1/3
.Theydensitolumevrelationρ=4πR3⇒R=4πρ.
⟨⟩⟨⟩()(⟨⟩()1)3
6Rv26v23M1/36v2332
344
M==⇒M==4.53×10kg=2.27×10M
GG4πρG4πρ⊙
□
6
(c)dynamicaltime1×10yrandradius1pc.
Solution:
()1
Thedynamicaltimeτ=3π2.Usingρ=3Mewget
Gρ4πR3
4π2R3
354
M==1.75×10kg=8.79×10M
Gτ2⊙
□
2.4.2.terstellarIngasinymangalaxiesisinvirialequilibriumwiththestars,inthatthermsspeedofthe
gasparticlesisthesameasthermsstellarspeed.Consideralargeellipticalgalaxywithavirialradius
12
of100kpcandamassof1×10Msolarmasses.Calculatethermsstellareloyvcitusingthevirial
⊙
theorem.Henceestimatethetemperatureoftheterstellaringas,assumingthatitiscomposedtirelyen
ofydrogen.h
Solution:
()1(−111230)1
√GM26.67×10·1×10·1.9×102
v=⟨v2⟩===2.68×105m/s=268km/s
rms416
6R6·1×10·3.08×10
−27
Themassofydrogenhism=1.67×10kg.Ifalltheterstellarinmassaswcomposedofydrogenh
H
thenthetemperatureouldwbeengivybreation
⟨⟩m⟨v2⟩−2332
1mv2=3kT⇒T=H=1.67×10·(268×10)=2.86×106K
2H23k3·1.68×1023
□
2.4.3.Assuminganeragevastellarmassof0.5MandΛ=r/1AU,lookuptablealuesvandifndtherelaxation
⊙c
timetratthetercenofglobularcluster47ucanae.TwShothatthecrossingtimetcross≈2rc/σr∼
1×10−3trelax
Solution:
Thetotalbumernofstarsintheclusterisengivyb
otalTMass800M
N==⊙=1600
MeanMass0.5M
⊙
4.93
Theydensitofstarsfromtableisρ=10M/pc.Thedynamicaltimeofthestarscanbewno
⊙
calculatedas
()−1
GM2
5
τ=r3=3.09×10yr
c
CHAPTER2.CTICGALAOPHYSICSASTR56
.
wNotherelaxationtime
N160056
trelax=8.5ln(Λ)τ=8.5ln(rc/1AU)3.09×10=4.89×10yr
Thecrosstimeist2r2·0.7pc
cross=c==2.54×10−2
46
trelaxσrtrelax1.1×10·4.89×10
□
2.4.4.Theelovcitiesofstarsinastellarsystemaredescribedybathree-dimensionalellianMaxwdistribution—
thatis,
2−mv2/2kT
f(v)=Ave
Here,Aisanormalizationt,constanmisthestellarmass,assumedt,constankisBoltzmann’st,constan
andTisthetemperatureofthesystem.erifyVthemeanstellarkineticenergyis1m⟨v2⟩=3kT
22
Solution:
Thenormalizationconditionesgiv
∞∞
∫f(v)dv=∫Av2e−mv2/2kTdv=1
00
oTcarryoutthetegrationinletsemaksomehangecofariablesv
mv2=x;⇒v=√2kTx;dv=kTdx,Asv→{0,∞}x→{0,∞}
2kTmmv
Usingtheseariablevtransformation,ournormalizationtegralinbecomes.
∞∞√()3∞
∫2−xkT∫kT2kT−x√kT2∫√−x
Avemvdx=Ammxedx=A2mxedx=1
000
∞()()
∫n−1−x3111√
ButybdeifnitionofgammafunctionΓ(n)=xeewget.AndΓ2=2Γ2=2π
0
()3∞()3()
√kT2∫3−1−x√kT231
A2x2edx=A2Γ=1⇒A=
mm21√(kT)3
2π2
02m
Theexpectationaluevforthesquareofspeedcanbecalculatedas:
∞∞
2∫2∫4−mv2/2kT
⟨v⟩=vf(v)dv=Avedv
00
CHAPTER2.CTICGALAOPHYSICSASTR57
Carryingoutsametransformationsaseabvoewget.
∞()5
2∫3kT23−x
⟨v⟩=A22x2edx
m
0
()5∞()5()()5
3kT2∫5−1−x3kT253kT23√
=A22x2e=A22Γ=A22π
mm2m4
0
()5
13kT23√
=×22π
3√(kT)3m4
2π2
2m
=3kT
m
⇒1m⟨v2⟩=3kT
22
Sothekineticenergyofheacmassis3kTiftheeloyvcitdestributionofthebleensemofmasswfollo
ellianMaxwdistributionfunction.2□
CHAPTER2.CTICGALAOPHYSICSASTR58
2.4.5.orkWoutthedetailsofthesimpleapeorativevmodeldiscussedinclass.Starsaporateevfromacluster
ofmassMonatimescaletev=αtR,whereα≫1,so
dM=−M(2.13)
dtαtR
orFpureaporation,evheacescapingstarcarriesoffexactlyzeroenergy(i.e.starsbarelyescapethe
clusterptial),otensothetotalenergyoftheclusterremainst.constan
(a)IftheclusterptialotenenergycanysaalwbewrittenasU=−kGM2forifxedk,whereRisa
2R
haractersticcclusterradius,andassumingthattheclusterisysaalwinvirialequilibrium,wsho
thatR∝M2astheclusteres.olvev
Solution:
Theptialotenenergyrelationcanbereorganizedas
R=−kGM2;⇒R=βM2;⇒R∝M2;Whereβ=−kG
2U002U
SoR∝M2.□
(b)AssumingthattherelaxationtimetRscalesasM1/2R3/2so
()()
M1/2R3/2
tR=tR0MR(2.14)
00
eSolv(2.13)todeterminethelifetimeofthecluster(intermsofitsinitialrelaxationtimetR0).
Alsowritewndoanexpressionforthemeanclusterydensitasafunctionoftime.
Solution:
eWcanwriteEq.(2.14)ast=βM1/2R3/2.SinceR=βM2.eWwnoe,vhat=
R10R
βM1/2(βM2)3/2;
10
⇒t=βM7/2
R3
SupposeTisthelifetimeoftheclusterthathadinitialmassofMithenastimegoesfrom0toT
massgoesfromMto0.UsingtinEq.(2.13)ewget
iR
0T
dM=−1M;⇒∫M5/2dM=−β∫dt;⇒−2M7/2=−β(T)⇒T∝M7/2
dtαβM7/247i4i
3
M0
i
SothelifetimeoftheclusterisT∝M7/2■.
i
wNotheydensitρ∝M.ButforasystemindynamicalequilibriumewevhaR∝M2.Thisesgiv
R3
ρ∝M=M5⇒M∝ρ−5Eq.(2.13)canbeedsolvasafunctionoftimeaseabvoandwritten
(M2)3
as
M=βt2/7⇒M−5=βt−10/7⇒ρ=βt−10/7
556
□
(c)Estimatethisforaglobularclusterofmass5×105Mradius10pcandmeanstellarmass0.5M
⊙⊙
Solution:
Theydensitofthisclusteris
5
M5×10M
⊙−14323
ρ≈==9.86×10kg/m=5×10M/pc
R3103pc3⊙
CHAPTER2.CTICGALAOPHYSICSASTR59
.Thebumernofstaris
5
M5×10M
N=tot=⊙=1×106
mav0.5M⊙
Thetimescalethenis
()−1/2−115
GM6.67×10·5×10M
t==⊙=6.67×105yr
333
R10pc
□
2.5orkHomeweFiv
2.5.1.(a)Calculatethetotalvitationalgraptialotenenergiesof(i)ahomogeneoussphereofmassMand
radiusa,and(ii)aPlummersphereofmassMandscalelengtha
Solution:
TheptialotenenergyisU=GM(r)m.WhereM(r)isthemassinsideofsphericalshellofradiusr.
r43
orFahomogenoussphericaldistributionofρtheM(r)=3πrρandtheadditionalmassincrease
duetoincreaseintheradiusofmassisdm=ρ4πr2dr.Ifewbringdmfromyinifnittorthenthe
increaseinptialotenenergyis
GM(r)GM(r)G4πr3ρ
dU=dm=·ρ4πr2dr=3·ρ4πr2dr(2.15)
rrr
ThetotalptialotenenergyisobtainedybtegratinginEq.(2.15)from0totheradiusoftheifnal
spherea.
∫a525
162241622a16πGa2
U=3πρGrdr=3πGρ5=15ρ(2.16)
0
Butforahomogenoussphereofradiusatheydensitisρ=3M.UsingthisisEq.(2.16)ewget
4πa3
()
522
U=16π2Ga3M=3GM
354πa35a
SothevitationalgraptialotenenergyofhomogenousshpereofmassMandradiusais3GM2.■
5a
enGivyanptialotenfunctionewcanysaalwcalculatetheydensitfunctionusingthepoissonequa-
tion.
Φ=√GMPlummertialotenP(2.17)2
22∇Φ=4πGρ(r)oisson’sPequation(2.18)
r+a()
orFsphericalsystemtheLaplacianoperatoris∇2:=1∂r2∂.Calculating∂ϕewe.vha
r2∂r∂r∂r
∂ϕ=∂(√GM)=−GMr;⇒r2∂ϕ=−GMr3
∂r∂r22223/2∂r223/2
r+a(r+a)(r+a)
()[()3]
32−22
21∂GMr1∂a3GMa
∇Φ=r2∂r−223/2=r2∂rGM1+r2=225/2
(r+a)(r+a)
oisson’sPequationcanbeusedtocalculatetheydensitfunctionasρ(r)=∇2Φ.
4πG
22
ρ(r)=1·3GMa=3Ma1(2.19)
4πG225/24π225/2
(r+a)(r+a)
CHAPTER2.CTICGALAOPHYSICSASTR60
Eq.(2.19)esgivtheydensitfunctionoftheplummermodel.Thisydensitfunctioncanbeusedto
calculatethemassofsphericalolumevofradiusras:
rr
∫∫223
M(r)=ρ(r)4πr2dr=4π3Mardr=Mr(2.20)
4π225/2223/2
(r+a)(r+a)
00
eWcanuseEq.(2.15)tocalculatetheptialotenenergyequippedwiththemassfunctionandydensit
function.
∞∞
∫Mr323Ma2122∫r43πGM2
U=4πG223/2·r·4π225/2dr=3GMa224=32a
(r+a)(r+a)(r+a)
00
Sothetotalvitationalgraenergyofplummerptialotenfunctionis3πGM2.□
32a
(b)wShothatthetotalmassofthePlummermodelisindeedM.
Solution:
Eq.(2.20)esgivthemasstainedcanwithintheradiusrforplummersphere.Thetotalmassof
plummersphereisthetotalmasstainedconinsidetheradiusofr=∞.akingTlimitofEq.(2.20)
ewget.
Mr3M
M=lim=lim=M
tot3/2()3
r→∞22r→∞22
(r+a)1+a
r2
ThiswsshothatthetotalmassofplummermodelisMhwhicappearsintheptialotenfunction
engivybEq.(2.17).□
2.5.2.(a)erifyVthattheKuzminptialoten
Φ(r,z)=−√GM(2.21)
Kr2+(a+|z|)2
has∇2Φ=0forz̸=0,andsotsrepresenasurfaceydensitdistributionΣ(r)intheplanez=0.
Solution:
ritingWr2=x2+y2wherexandyarethecartesiancoordinatescorrespondingtothercoordinate
incylindricalsystem.eWgetΦ=−GM(x2+y2+(a+|z|)2)−1/2.Incartesiancoordinatesystem
2∂2∂2∂2
∇≡∂x2+∂y2+∂z2.Soheaccomptsonenofthsoperatorare.
(222)(222)
∂2Φ=GM2x−y−(a+z);∂2Φ=GM−x+2y−(a+z)
∂x2()5∂y2()5
x2+y2+(a+z)22x2+y2+(a+z)22
Sincetheptialotenisfunctionof|z|andtheeativderivof|z|t’tdosenexistatz=0.eWetakleft
handandtrighhandeativderivforthezcompt.onenUsing|z|=+zfortrighand|z|=−zfor
lefte,ativderiveWget.
(222)(222)
∂2Φ=GM−x−y+2(a+z)∂2Φ=GM−x−y+2(a−z)
∂z2()5∂z2()5
+x2+y2+(a+z)22−x2+y2+(a−z)22
Inheacofthecasesthetotalsum
∇2Φ=(∂2+∂2+∂2)Φ=(∂2+∂2+∂2)Φ=0
∂x2∂y2∂z2∂x2∂y2∂z2
+−
2
Byuseofoisson’sPequationρ(r)=1/4πG∇Φewconcludethemassydensitiszeroerywhereev
except(possibly?)atz=0.□
CHAPTER2.CTICGALAOPHYSICSASTR61
(b)UseGauss’swlatodetermineΣ(r).
Solution:H
ThegausswlaforvitationalgraifeldyssaSE·dA=4πGMenclwhereSisyanarbitraryclosed
surfaceandMenclisthemassinsidethatsurface.wNothatewwknothatthereisnomassexcept
atinifniteplanez=0,ewarecertainthatthevitationalGraforceifeldiscompletelyalongzˆ.The
forceifeldalongzˆisengivybE=∂Φ.Sincetheptialotenfunctionisnotsmooth,ewevhaowt
∂z
tdifferenaluesvforthiseativderivoneithersideofthedisc.
E+=∂Φ(−zˆ)=∂(√GM)(−zˆ)=−GM(a+z)(−zˆ)
()
3
22/
∂z+∂z+r+(a+z)r2+(a+z)22
E−=∂Φzˆ=∂(√GM)zˆ=GM(a−z)
()
3
22/
∂z−∂z−r+(a−z)r2+(a−z)22
IfewetakacylindricalgaussiansurfaceforSwithsurfaceAreaHAzˆ,Thetotalmassinsidethe
cylinderisMencl=Σ×AandthelfuxthoughthesurfaceE·dA=E+A+E−A.ButEzis
uniformsoewcancalculateE=∂Φ=GMa.AndsimilarlyforE=GMaSo,
+∂z+z=0223/2−223/2
(r+a)(r+a)
4πGΣ×A=GMaA+GMaA;⇒Σ=Ma
223/2223/2223/2
(r+a)(r+a)2π(r+a)
SothesurfacemassydensitoftheKuzmindiskisΣ(r)=Ma.□
223/2
2π(r+a)
(c)Whatisthecircularorbitspeedforaparticlevingmointheplaneofthedisk?
Solution:
orFthisptialotenthetotalmassinsidethesphericalshellofradiusrissimplythesurfaceydensit
timestheareaofgreatcircle,soM(r)=Σ(r)πr2.Theersetransvspeedforacircularorbit
v=√GM(r)=√Gπr2·Ma=√GMar
crr223/2223/2
2π(r+a)2(r+a)
ThisesgivthespeedofparticleincircularorbitforKuzminptial.oten□
12
2.5.3.orFstarsvingmoerticallyvinGalacticdisk,withenergyE=Φ(R,z)+/v,supposethedistribution
z02z
functionis
n0−E/σ2
f(z,vz)=√ez.
2πσ2
Findtheydensitn(z)andegivit’saluevn(0).oTconstructselftconsistenmodelletΦ(z)=σ2ϕ,wsho
that
d2ϕzσ2
2=e−ϕ,Wherey=andz2=
dy2z08πGmn
00
eSolvthisforϕ(y)andhenceifndΦ(z)andn(z).Whatisthealuevatlarge|z|?
Solution:
Thebumernydensitisthezerothtmomenofthisdistributionfunctionso
σ√π
2
∞∞∞
∫∫2∫✒

122−Φ/σv22
n0(−Φ−/v)/σ2n0e−z−Φ(R,z)/σ
2z2
n(z)=f(z,vz)dvz=√2e=√e2σ=n0e
2πσ22πσ2
−∞00

−Φ(z=0)/σ20
Thisesgivtheexpressionforn(z).SinceΦ(z=0)=0isen.givn(0)=ne=ne=n.
000
CHAPTER2.CTICGALAOPHYSICSASTR62
2
Thetotalmassydensit4πGρ(z)=∇Φ.Butd2Φρ(z)=mn(z)wheremistheeragevamass.Butfor
2222
motiononlyalongz,ewcanwrite∇Φ≡dz2.Alsooperatordz=z0dy;.Bypoissonsequation,
2222
dΦ2dϕ−ϕdϕ−ϕ(y)2σ
dz2=4πGmn(z);σz2dy2=4πGmn0e;⇒2dy2=eifz0=8πGmn0
0
wNosolvingthisforϕasafunctionofy
2
dϕ−ϕ(y)
2dy2=e
−ϕ(zy)/σ22
Thisentialdifferquationeshouldgiveafunctionϕ(y)suchthatn(z)=ne0=nsech(z/(2z))
000
butI’touldncifndanyasonableersolution
orFlargealuevof|z|
2(z)2(z)
limn0sech=0;limn0sech−=0
z→∞2zz→∞2z
+0−0
Soforlargealuevof|z|theydensitiszero.□
2.5.4.Astellarsysteminhwhicallparticlesareonradialorbitsisdescribedybthedistributionfunction
{−1/2
Aδ(L)(E−E)ifE>E
f(E,L)=00
0otherwise
whereE=ψ−1/2v2iserelativenergyandEandAarets.constan
t0
(a)Bywritingv2=v2+v2,wherevandvaretheradialandersetransvelovcities,andL=rv,
rtrtt
33πdEdX2
evprothattheolumevtelemendv=2πvtdvtdvrymabewrittendv=r2vwhereX=L.
r
(b)Hencewshothattheydensitis
{−2
ρ(r)=Brif(r<r0)
0if(r≥r0)
whereBisatconstanandtheerelativptialotenatrsatisifesψ(r)=E.
000
Solution:
Thebumernydensitisthezerothtmomenofdistributionfunctionwithrespectto.eloyvcitSo
∞∞∞∞
∫3∫∫−1/23∫∫√−1/2πdEdX
n(z)=fdv=Aδ(L)(E−E0)dv=Aδ(X)(E−E0)r2v
r
−∞−∞−∞−∞
∞∞
∫∫√−1/2dE
=Aδ(X)dX(E−E0)r2v
r
−∞−∞
Ifr<rthenE=ψ(r)>E
00
∫∞−1/2
ρ(r)=mn(r)=mA(E−E0)dE
r2v
E0r
[]
1−2m(E−E0)−3/2∞−2
=r23v=Br
rE0
CHAPTER2.CTICGALAOPHYSICSASTR63
Butifr<rthenE=ψ(r)<Eandf(E,L)=0then,
00
∫3
ρ(r)=0dv=0
Thisisaerpwowlaydensitwithydensityingdecaassquareofthedistanceforaifnitespherical
regioninspace.□
2.6orkHomewSix
2.6.1.AssumingtherotationecurvformilkyyawislfatandV(R)=RΩ(R)=200km/sandR0=8kpc.
(a)ComputetheOorttsconstanAandB,andthelocalepicyclicfrequencyκ.(b)IftheSunhasv
x
(radial)=10km/sandvyerse)(transv=5km/s,calculatetheSun’sguidingradiusRgandradial
orbitalamplitudeX.
Solution:
orFlfatrotationecurvv(r)=t.constanso,dv=0..
dr
A=1V(R)=1200km/s=12.50km/s/kpc
2R028kpc
B=−1V(R)=−1200km/s=−12.50km/s/kpc
2R028kpc
2
Thealuevofκisrelatedtotheoorttconstanasκ=−4BΩ
Ω=V(R)/R=200/8=2.5km/s/kpc;κ=√4∗12.50∗2.5=11.18km/s/kpc
Also
v=2BX;⇒X=5km/s=0.2kpc
y2·12.50km/s/kpc
TheguidingtercenisthesumofummaximtdisplacemenXandtheclosesthapproacsoRg=R0+X=
8kpc+0.2kpc=8.20kpc.□
√
2.6.2.wShothat,iftherotationecurvoftheMilkyyaWislfatneartheSun,thenκ=2Ω(R),sothat
locallyκ≈36km/s/kpc.hetcSktheescurvofΩ,Ω±κ/2,andΩ±κ/4inadiskwhereV(R)istconstan
erywhere,evandwshothatthezonewhereo-armedwtspiralesvawcanpersistisalmostfourtimes
largerthanthatforfour-armedspirals.
Solution:
orFalfatrotationecurvV(R)=t,constanso,dV(R)=0.TheoorttconstanBisB=−1V(R)=−Ω.
2dR2R2
Butκ=−4BΩ.Thisesgiv
√Ω√
κ=−4·−2·Ω=2Ω(R)
Thisesgivtheepicyclicfrequencyofthesun.ThegraphforΩ±κandΩ±κareTheestwloand
24κ
highestaluesvofRcanbefoundattheptsoinwhere1ΩcrossesthepatternspeedΩp.TheptoinΩ±2
crossesΩpareRmax=(1±√)RThisesgivtherationofregionas
2
1
Rmax=1+√2=5.8
1
R1−√
min2
κ1
SimilarlyTheptoinΩ±crossesΩareR=(1±√)RThisesgivtherationofregionas
4pmax22
CHAPTER2.CTICGALAOPHYSICSASTR64
1
Rmax1+√
=22=2.09
1
R1−√
min22
Theregionareximatelyapproattheratioof3.0□
2222
2.6.3.(a)enGivthedispersionrelationforagasdisk,(ω−mΩ)=kv−2πGΣ|k|+κ,wShothatthe
s
groupeloyvcitis
2
∂ω|k|vs−πGΣ
vg≡=sign(k).
∂kω−mΩ
R
Solution:
tiatingDifferenbothsidesoftheengivdispersionrelationwithrespecttok,esgiv
2(ω−mΩ)∂ω=2kv2−2πGΣsign(k)
∂ks
orFyanrealbumernkewcanwritek=|k|sign(k)usingthisineabvoexpressioncanberearranged
intheform
∂ω2|k|sign(k)v2−2πGΣsign(k)|k|v2−2πGΣ
=s=sign(k)s
∂k2(ω−mΩ)ω−mΩ
Thisesgivtherequiredgroupeloyvcitasrequired.□
vκ
(b)wShothat,foramirginallystablediskwithQ=s=1thegroupeloyvcitisequaltothe
πGΣ
soundspeedv
s
Solution:
orFQ=1ewevhaπGΣ=vκ.Usingthisintheexpressionofgroupeloyvcitesgiv
s
|k|v2−vκ
v=sign(k)ss
gω−mΩ
CHAPTER2.CTICGALAOPHYSICSASTR65
eWcanuseκ=mΩandk=ω.Ifewdisregardthesignofk(ie,assumekaspe)ositivtheeabvo
v
s
expressionbecomes
ωv−mΩ
vsω−mΩ
v=sign(k)sv=v=v
gω−mΩsω−mΩss
Thiswsshothatthegroupeloyvcitis(withinasignofk)equaltothesoundspeed.□
2.6.4.AsatellitegalaxyofmassMesvmoinacircularorbitofradiusRinasphericallysymmetricgalactic
s
haloofydensitρ(r)=v2/4πGr2,withM≪v2R/G.Thestars(anddarkmatterparticles)inthe
csc
tparengalaxyallevhamasseshucmlessthanMs.
(a)Usetheequationfordynamicalfrictiontowritewndothedragforceonthesatelliteasitorbits.
Solution:
Thedynamicalfrictionisengiv,yb
dv4πG(M+m)
−=snmln(Λ);
dtv2
orFasatellitegalaxyofmassMobitingatvthepassingeloyvcitisV=vthedragforceis
scc
−Mdvc.Notingthatforthegalactichalonm=ρ(r)leadsto.
sdt
✘✿Ms
2✘22
✘
4πG(M+m)vMG
✘
F=−M✘s·cln(Λ)=−sln(Λ)
drgsv24πGr2r2
c
Thisesgivtheexpressionforthedragforceontheorbitinggalaxyinthehalo.□
(b)Thesatellitesinksardwinsowlyslothatitcanbetthoughofasvingmothroughaseriesofcircular
orbits,soitsorbitalspeedatyanradiusrisysaalwequaltothecircularorbitalspeedatr.What
istheangulartummomenL(r)ofthesatelliteatradiusr?
Solution:
Thetaneousinstanspeedatadistancerfromthetercenisv,sothetummomenisP=Mv.
csc
TheangulartummomenisL=r×p
L=r×P=rMsvc(2.22)
SotheangulartummomenofthegalaxyatdistancerisMvr□
sc
(c)ByequatingtherateofhangecofLtothetorqueexertedonthesatelliteybdynamicalfriction,
wshothatthedistancer(t)fromthesatellitetothetercenofthegalaxyobeysthetialdifferen
equation
drGMln(Λ)
=−s
dtvr
c
Solution:
Thetorqueaboutthetercenofthegalactichalohwhicthegalaxyisorbitingisτ=Fr,but
drg
τ=dL,biningcomtheseowtegiv
dt
dLdrM2GdrGMln(Λ)
=Fr;⇒Mv=−sln(Λ)·r;⇒=−s
dtdrgscdtr2dtvr
c
hWhictherequiredtialdifferenequationfortherateofhangecofdistanceoforbitinggalaxyto
tercenofhalo.□
CHAPTER2.CTICGALAOPHYSICSASTR66
(d)eSolvthisequationtoestimatethetimeentakforthesatellitetosinktothetercenofthetparen
.galaxy
Solution:
ThetimetofallttointhetercenofhaloisengivybthetimeforthedistanceofRto0atthe
f0
tercenofhalo.Rearrangingtheeabvotialdifferenequationewget.
t
0f
GMln(Λ)∫∫GMln(Λ)R2GMln(Λ)
rdr=−sdt;⇒rdr=−sdt;⇒−0=st
vv2vf
ccc
R00
R2v
Sothetimetosinkist=0c.□
f2GMln(Λ)
s
10
(e)aluateEvthistimeforayphothetical“MagellanicCloud”withM=2×10Monaninitially
s⊙
circularorbitofradiusR=50kpcaroundour,Galaxywithvc=220km/s.eakTΛ=20.
Solution:
Substutingthesealuesvintheeabvoexpression
323
(50×10)·220×10169
tf=−1110=3.28×10s=1.04×10yr=1.04Gyr
2·6.67×10·2×10M
⊙
Sothesinktimeofthecloudis1.04Gyr□
2.6.5.IftheeeffectivradiusofthesatellitegalaxyinthepreviousproblemisRs=1.5kpc,estimatethe
distancefromthetercenofthetparengalaxyathwhictidaltial)(differenvitationalgraforcesouldw
tlysigniifcanaffectthesatellite’sstructure.
Solution:
Thedistancescaleisengivyb
()1
M3
rt=MRs
s
AssumingM=2×1010MfroompreviousproblemandthemassofgalaxytobethatofMilkyyaw
s⊙
11
M=5.8×10M
⊙
(11)1
5.8×10M3
rt=⊙1500pc=488.2pc
10
2×10M
⊙
Sothedistancefortsigniifcaneffectis488.2pc□
Chapter3
tumQuanhanicsMec
3.1orkHomewOne
′′′′′′
3.1.1.(a)Considerowtetsk′′′|α⟩and|β⟩.Suppose⟨a|α⟩,⟨a|α⟩,···and⟨a|β⟩,⟨a|β⟩,···areallwn,kno
where|a⟩,|a⟩,···formacompletesetofbaseets.kFindthematrixtationrepresenofthe
operator|α⟩⟨β|inthisbasis.
Solution:
eWwknoeryevetkcanbewrittenasthesumofitscomptoneninthe‘direction’ofbaseetk
(completeness)so|γ⟩canbewrittenas∑⟩⟨⟩
ii
|γ⟩=aaγ
i
Lettheoperator|α⟩⟨β|actonanarbitraryetk∑|γ⟩.⟨⟩⟨⟩
ii
|α⟩⟨β|γ⟩=|α⟩βaaγ
i
j⟩

Sothecomptonenofthis|α⟩⟨β|γ⟩inthedirectionofanothereigenetkaisthenengivybthe
j⟩
innerproductofitwitha
⟨j⟩∑⟨j⟩⟨i⟩⟨i⟩

(|α⟩⟨β|γ⟩)j=aα⟨β|γ⟩=|aα{zβa}aγ(3.1)
iN×N
Thisevanoexpressioncanbewrittenasthematrixformas
⟨1⟩⟨1⟩⟨1⟩⟨2⟩⟨1⟩⟨N⟩

(|α⟩⟨β|γ⟩)aαβaaαβa···aαβa(|γ⟩)1
1⟨⟩⟨⟩⟨⟩⟨⟩⟨⟩⟨⟩
(|α⟩⟨β|γ⟩)21222N(|γ⟩)
2=aαβaaαβa···aαβa2
......
......
......
(|α⟩⟨β|γ⟩)⟨N⟩⟨1⟩⟨N⟩⟨2⟩⟨N⟩⟨N⟩(|γ⟩)
Naαβaaαβa···aαβaN
′⟨i⟩

Sinceeryev⟨a|β⟩iswnknoheactelemenβaineabvomatrixcanbewrittenasthecomplex
⟨i⟩∗
conjugateofwnknoaβ.Sothematrixtationrepresenbecomes
⟨⟩⟨⟩⟨⟩⟨⟩⟨⟩⟨⟩
11∗12∗1N∗
aαaβaαaβ···aαaβ
⟨⟩⟨⟩⟨⟩⟨⟩⟨⟩⟨⟩
21∗22∗2N∗
|α⟩⟨β|≡aαaβaαaβ···aαaβ
....
....
⟨⟩.⟨⟩⟨⟩.⟨⟩.⟨⟩.⟨⟩
N1∗N2∗NN∗
aαaβaαaβ···aαaβ
hWhicistherequiredmatrixtationrepresenof|α⟩⟨β|□
67
CHAPTER3.QUANTUMMECHANICS68
(b)Considerofspin1systemandlet|α⟩and|β⟩be|S=ℏ/2⟩and|S=ℏ/2⟩,.respelyectivriteW
2zx
wndoexplicitlythesquarematrixthatcorrespondsto|α⟩⟨β|intheusual(Szdiagonal)basis.
Solution:
1
Thebasisetskare|Sz;+⟩≡|+⟩and|Sz;−⟩≡|−⟩.Thestateetk|Sx;+⟩=√2(|+⟩+|−⟩).So
thefourmatrixtselemenare
⟨α|+⟩=1;⟨α|−⟩=0
1111
⟨β|+⟩=√(1+0)=√⟨β|−⟩=√(0+1)=√
2222
Therequiredmatrixtationrepresenis
[]1[]
⟨+|α⟩⟨β|+⟩⟨+|α⟩⟨β|−⟩=√11
⟨−|α⟩⟨β|+⟩⟨−|α⟩⟨β|−⟩200
hWhicistherequiredmatrixtationrepresenoftheoperatorinthebasis|Sz;+⟩and|Sz;−⟩□
3.1.2.Usingtheyorthonormalitof|+⟩and|−⟩,evpro
[S,S]=iεℏS,{S,S}=(ℏ2)δ,
ijijkkij2ij
Where,S=ℏ(|+⟩⟨−|+|−⟩⟨+|),S=iℏ(−|+⟩⟨−|+|−⟩⟨+|),S=ℏ(|+⟩⟨+|−|−⟩⟨−|)
x2y2z2
Solution:
iℏ2iℏ2
SxSy=4{−|+⟩⟨−|+⟩⟨−|+|+⟩⟨−|−⟩⟨+|−|−⟩⟨+|+⟩⟨−|+|−⟩⟨+|−⟩⟨+|}=4{|+⟩⟨−|−|−⟩⟨−|}
iℏ2iℏ2
SySx=4{|+⟩⟨−|+⟩⟨−|−|+⟩⟨−|−⟩⟨+|+|−⟩⟨+|+⟩⟨−|−|−⟩⟨+|−⟩⟨+|}=−4{|+⟩⟨−|−|−⟩⟨−|}
iℏ2iℏ2iℏ2
[Sx,Sy]=SxSy−SySx=4{|+⟩⟨−|−|−⟩⟨−|}+4{|+⟩⟨−|−|−⟩⟨−|}=2{|+⟩⟨−|−|−⟩⟨−|}=iℏSz
Since[Sx,Sy]=iℏSzitimmediatelywsfollothat[Sy,Sx]=−iℏSzbecause[A,B]=−[A,B].Collecting
alltheseleadsto[Si,Sj]=iεijkSk.
22
{S,S}=SS+SS=iℏ{|+⟩⟨−|−|−⟩⟨−|}−iℏ{|+⟩⟨−|−|−⟩⟨−|}=0
xyxyyx44


{S,S}=SS+SS=2SS=2ℏ2⟨+|+⟩+⟨−|−⟩=ℏ2
xxxxxxxx4|{z}2
ytitIdenoperator
Similarly{S,S}=ℏ2;{S,S}=ℏ2;{S,S}=ℏ2;{S,S}=0;{S,S}=0;{S,S}=0;hwhic
xx2yy2zz2xyyzzx
canbecopactlywrittenas{S,S}=(ℏ2)δforheacoperatorleadstotherequiredrelationofthe
ij2ij
utationcommandtianutationcommrelationoftheengivoperators.□
3.1.3.Thehamiltonianoperatorforao-statewtsystemisengivyb
h=a(|1⟩⟨1|−|2⟩⟨2|+|1⟩⟨2|+|2⟩⟨1|),
whereaisabumernwiththedimensionof.energyifndtheenergyaluesveigenandthecorresponding
energyetseigenk(asalinearbinationscomof|1⟩and|2⟩)
CHAPTER3.QUANTUMMECHANICS69
Solution:
′
lettheenergyeteigenkbe|α⟩=p|1⟩+q|2⟩.letthealueveigenofthisenergyeteigenkbea.operating
thiseteigenkybtheengivhamiltonianoperatorewget.
h|α⟩=a(|1⟩⟨1|−|2⟩⟨2|+|1⟩⟨2|+|2⟩⟨1|)(p|1⟩+q|2⟩)
=a(p|1⟩+p|2⟩−q|2⟩+q|1⟩)
=a[(p+q)|1⟩+(p−q)|2⟩]
′′
sinceewassumeaistheaueveigenofthisetkewustmevhah|α⟩=a|α⟩usth
′
a(p|1⟩+q|2⟩)=a[(p+q)|1⟩+(p−q)|2⟩]
since|1⟩and|2⟩areindeptendenets,kthecoteiffcienofheacetkonlhsandrhsustmequal.comparing
thecotseiffcienewevha
′
′a+a
−ap+(a+a)q=0;→p=aq
′√
′′a+a2′22′
(a−a)p+aq=0;→(a−a)aq+aq=0;→a−a+a=0;a=±2a
√
sotherequiredaluesveigenoftheoperatorare±2a.
thecoteiffcien√
p=a±2aq=(1±√2)q
a
.sinceewevhaafreehoicecofoneoftheparametersewhocosepandqhsucthattheenergyeteigenk
isnormalized.sotherequiredeteigenkis
√1√√1(√)
|α⟩=√2((1±2)|1⟩+|2⟩)=√(1±2)|1⟩+|2⟩
1+(1±2)4±22
√
theeabvoexpression|α⟩esgivtheenergyeteigenkcorrespondingtoalueveigen±2a.□
1
3.1.4.Abeamofspin2atomgoesthroughaseriesofyph-testern-gerlactsmeasuremenasws:follo
(a)theifrsttmeasuremenacceptss=ℏ/2atomsandrejectss=−ℏ/2atoms.
zz
(b)thesecondtmeasuremenacceptss=ℏ/2atomsandrejectss=−ℏ/2atoms,wheresisthe
nnn
alueveigenoftheoperators·nˆwithnˆmakinganangleβinthexz−planewirespecttothez-axis.
(c)thethirdtmeasuremenacceptss=−ℏ/2atomsandrejectss=ℏ/2atoms.
zz
whatistheytensitinoftheifnalsz=−ℏ/2beanwhenthesz=ℏ/2beamsurvivingtheifrstmea-
tsuremenisnormalizedtoy?unitwhoustmewtorienthesecondmeasuringapparatusifewareto
maximizetheytensitinoftheifnals=−ℏ/2beam?
z
Solution:
TheFirsthStern-GerlactmeasuremeninSzisindeptendenofthesecondhStern-Gerlactmeasuremen
innˆtheyprobabilitofatompassingthroughheaccomptonenis1.Duetothistmeasuremenandthe
S=−ℏ/2beingrejectedthesystemtiallyessenforgetstheprevioustmeasuremenandtheatomstill2
n
comeout50%.SothefractionofatomspassingthroughthethirdSGapparatusinSzdirectionisstill
1.SothetotalfractionofatomspassingthourhgthethirdSGapparatusis1×1=1=25%,
2224
IfthesecondSGapparatusistedorienparalleltotheifrstapparatusthenittiallyessenmeasuresthe
|Sz;+⟩stateoftheatomhwhicaswwhatcamefromtheifrstapparatussoitlets100%oftheatom
in|Sz;+⟩state.Andthethirdapparatuswilllethalfofthesecondhwhicis50%oftheatomshwhic
passedthroughtheifrstapparatus.tingOrienthesecondSGapparatusparalleltotheifrstwillletall
oftheatoms,thisistherequiredconditionofmaximizingtheoutputofthird.□
CHAPTER3.QUANTUMMECHANICS70
3.1.5.evProthatifoperatorX=|β⟩⟨α|,thenthehermitianconjugateoftheoperatorisX†=|α⟩⟨β|.
Solution:
ctingAthisoperatorX=|β⟩⟨α|onanarbitraryetk|γ⟩
X|γ⟩=|β⟩⟨α|γ⟩
⇒⟨γ|X†=⟨β|⟨α|γ⟩∗(∵Dualcorrespondence)
⇒⟨γ|X†=⟨β|⟨γ|α⟩(∵⟨γ|α⟩=⟨α|γ⟩∗)
⇒⟨γ|X†=⟨γ|α⟩⟨β|(∵c|δ⟩=|δ⟩c)
⇒⟨γ|X†=⟨γ|(|α⟩⟨β|)(∵Assoeciativpropyert)
⇒X†=|α⟩⟨β|
usThifX=|β⟩⟨α|thenX†=|α⟩⟨β|iswnshoasrequired.□
3.2orkHomewowT
3.2.1.AowtstatesystemisisharacterizedcybaHamiltonianH|1⟩⟨1|+H(|1⟩⟨2|+|2⟩⟨1|)+H|2⟩⟨2|
111222
whereH,H,andHarerealbumersnwiththedimensionof,energyand|1⟩and|2⟩areetseigenk
112212
ofsomeableobserv(̸=H).Findtheenergyetseigenkandthecorrespondingenergyalues.veigen
Solution:
Lettheenergyeteigenkbe|E⟩=p|1⟩+q|2⟩andthealuesveigenbeλ.eratingOPthsistateybthe
engivHamiltonianOperatorewget
H|E⟩=H|1⟩⟨1|+H(|1⟩⟨2|+|2⟩⟨1|)+H|2⟩⟨2|(p|1⟩+q|2⟩)
111222
=Hp⟨1|1⟩|1⟩+Hq⟨1|2⟩|1⟩+Hp⟨1|1⟩|2⟩+Hp⟨2|1⟩|1⟩+Hq⟨1|2⟩|2⟩
1111121212
+Hq⟨2|2⟩|1⟩+Hp⟨2|1⟩|2⟩+Hq⟨2|2⟩|2⟩
122222
=Hp|1⟩+Hp|2⟩+Hq|1⟩+Hq|2⟩
11121222
=(Hp+Hq)|1⟩+(Hp+Hq)|2⟩
11121222
SinceybassumptionλisthealueveigenofthisstateewevhaH|E⟩=λ|E⟩hwhicesgiv
λp|1⟩+λq|2⟩=(Hp+Hq)|1⟩+(Hp+Hq)|2⟩
11121222
Comparingthecoteiffcienofheacindeptendenewget
λp=(Hp+Hq);λq=(Hp+Hq)
11121222
H2
⇒(λ−H)p−Hq=0;p=1q
1112λ−H
()11
H
Hp+(H−λ)q=0;⇒H12q+(H−λ)q=0;
122212λ−H22
11
Solvingthisforλewget
11√222
λ=(H+H)±H−2HH+4H+H
2112221111221222
Thesearetherequiredaluesveigenoftheengivoperator.Thisaluesveigencanbepluggedkbactoin
theengivequationtogetthealuesvofpandq.
H
q=1;p=√12
H−H1
2211±H2−2HH+4H2+H2
221111221222
CHAPTER3.QUANTUMMECHANICS71
Sotherequiredeigenstatesare
H
|E⟩=√12|1⟩+|2⟩
H−H1
2211±H2−2HH+4H2+H2
221111221222
TheeabvoeigenstaecanbenormalizedifrequiredtogettheEnergyet.eigenk□
⟨2⟩⟨2⟩2
3.2.2.(a)Compute(∆S)≡S−⟨S⟩wheretheexpectationaluevisentakfortheS+state.Using
xxxz
ouryresultkheccthegeneralizedyuncertainitrelation
⟨(∆A)2⟩⟨(∆B)2⟩≥1|⟨[A,B]⟩|2
4
withA→Sx,B→Sy.
Solution:
Let|+⟩trepresenthe|Sz;+⟩state.ThentheexpectationaluevofSxfor|Sz;+⟩canbecalculated
as
S=ℏ(|+⟩⟨+|−|−⟩⟨−|);S=iℏ(−|+⟩⟨−|+|−⟩⟨+|);S=ℏ(|+⟩⟨−|+|−⟩⟨+|);
z2y2x2
S|+⟩=ℏ(|+⟩⟨−|+|−⟩|+⟩)|+⟩=ℏ|−⟩;S|−⟩=ℏ(|+⟩⟨−|+|−⟩⟨+|)|−⟩=ℏ|+⟩;
x22x22
S|+⟩=iℏ(−|+⟩⟨−|+|−⟩⟨+|);|+⟩=iℏ|−⟩;S|−⟩=iℏ(−|+⟩⟨−|+|−⟩⟨+|);|−⟩=−iℏ|+⟩;
y22y22
Sotheexpectationaluesvare
⟨S⟩=⟨+|S|+⟩=⟨+|ℏ|−⟩=ℏ⟨+|−⟩=0
xx22
⟨S⟩=⟨+|S|+⟩=⟨+|iℏ|−⟩=−iℏ⟨+|−⟩=0
yy22
⟨⟩ℏℏℏℏ2ℏ2
S2=⟨+|S2|+⟩=⟨+|SS|+⟩=⟨+|S|−⟩=⟨+||+⟩=⟨+|−⟩=
xxxxx22244
⟨⟩22
S2=⟨+|S2|+⟩=⟨+|SS|+⟩=⟨+|Siℏ|−⟩=iℏ⟨+|−iℏ|+⟩=−i2ℏ⟨+|−⟩=ℏ
yyyyy22244
2∗
Since[Sx,Sy]=iℏSzand|⟨[Sx,Sy]⟩|=⟨[Sx,Sy]⟩⟨[Sx,Sy]⟩ewcanwrite
22
⟨[S,S]⟩=⟨iℏS⟩=iℏ⟨+|S|+⟩=iℏ⟨+|ℏ|+⟩=iℏ;⟨[S,S]⟩∗=−iℏ;
xyzz22xy2
ThedispersioninSxandSycanbecalculatedas
⟨⟩⟨⟩22⟨⟩⟨⟩22
222ℏℏ222ℏℏ
(∆S)≡S−⟨S⟩=−0=;(∆S)≡S−⟨S⟩=−0=;
xxx44xxy44
usThifnally
⟨2⟩⟨2⟩12
(∆Sx)(∆Sy)≥4|⟨[Sx,Sy]⟩|
22(2)(2)
ℏ·ℏ≥1iℏ−iℏ
44422
ℏ4ℏ4
16≥16
hWhicistrueasrequired.□
CHAPTER3.QUANTUMMECHANICS72
(b)kChectheyuncertainitrelationwithA→Sx,B→SyfortheSx+State⟨⟩⟨⟩
3.2.3.Findthelinearbinationcomof|+⟩and|−⟩etskthatmaximizestheyunertainitproduct(∆Sx)2(∆Sy)2.
erifyVexplicitlythatthelinearbinationcomouyfound,theytuncertainrelationforSxandSyisnot
violated.
Solution:
LetthelinearbinationcomthatmaximizestheyUncertainitproductbep|+⟩+q|−⟩.Sinceewwkno
thatthecotseiffcienarecomplexingeneralandthattheerallvophaseisimmaterial,ewcanetakpqnd
iδ
qhsucthatp=randq=sewherer,s,δarerealbumers.n
iδ−iδ
|α⟩=r|+⟩+se|−⟩←DC→⟨α|=⟨+|r+⟨−|se
SinceOperatorSx≡ℏ(|+⟩⟨−|+|−⟩⟨+|)andSy≡iℏ(−|+⟩⟨−|+|−⟩⟨+|);ewcanifndtheexpecta-
22
tionaluev
ℏiδℏiδ
Sx|α⟩=2(|+⟩⟨−|+|−⟩⟨+|)(r|+⟩+se|−⟩)=2(se|+⟩+r|−⟩)
[−iδ]ℏiδ
⟨Sx⟩=⟨α|Sx|α⟩=⟨+|r+⟨−|se2(se|+⟩+r|−⟩)
=ℏ{rseiδ+rse−iδ}
2
ℏ{iδ−iδ}
=2rse+e
=ℏrs2cos(δ)=ℏrscosδ
2
AlsoewcancalculatetheexpectationaluevofS2hwhicis
x
⟨2⟩(ℏiδ)
Sx=⟨α|SxSx|α⟩=⟨α|Sx2(se|+⟩+r|−⟩)
[−iδ]ℏ2iδ
=⟨+|r+⟨−|se4(r|+⟩+se|−⟩)
ℏ2ℏ2
=(r2+s2)=(Bynormalizationcondition)
44
hWhiccanbeusetocalculatethedispersionofSxas
⟨2⟩⟨2⟩2ℏ22222ℏ2(222)
(∆Sx)=Sx−⟨Sx⟩=4−ℏrscos(δ)=41−4rscos(δ)
⟨2⟩ℏ2222
Bysimilarprocedureewcancalculate(∆Sy)=4(1−4rssin(δ).Sotheirproductis
⟨⟩⟨⟩2()2()
22ℏ222ℏ222
(∆Sx)(∆Sy)=41−4rscos(δ)·41−4rssin(δ)
ℏ4
2222224422
=16(1−4rssin(δ)−4rscos(δ)+16rssin(δ)cos(δ))
ℏ2
=(1−4r2s2+16r4s44sin2(δ)cos2(δ))
16
ℏ2
22442
=16(1−4rs+4rssin(2δ))
√2
Sincerandsareconstrainedybnormalizationas2s=1−r.Theowtparametersfortheariationv
oftheproductisδandr(ors).Thesincesin(2δ)canattaintheummaximaluevof1hwhhicesgiv
CHAPTER3.QUANTUMMECHANICS73
2ππ
sin(2δ)=1;⇒2δ=2⇒δ=4.Sotheyuncertainitproductreducesto
⟨⟩⟨⟩ℏ2
222244
(∆S)(∆S)=(1−4rs+4rs)
xy16
ℏ2(22)2
=161−2rs
22
Theummaximaluevofthisexpressionoccurswhen2rsistheum,minimhwhicybinspectionis0
atr=0.Usingthisaluevr=0innormalizationconditionr2+s2=1esgivs=±1.Sothelinear
binationcomewstartedreducesto
()⟩
iπ11
|α⟩=0|+⟩±e4|−⟩=√±i√−
22□
′′
3.2.4.wShothateither[A,B]=0or[B,C]=0istsuiffcienfor⟨c|a⟩tobe
Solution:
′′
LetthecommoneteigenkofcompatibleoperatorsA,Bbe|a,b⟩.Sincetheyareableobservtheset
′′′′′′nn
oftheseetseigenkformacompletesetletthembe|a,b⟩,|a,b⟩···|a,b⟩fornstate(dimensional)
system.Intheifrstyawofvividuallyinmeasuringtheoutcomesof⟩Bablesobservthetotalyprobabilit
1
ofobservingcstateisthen
∑
⟨11⟩2⟨iii⟩2⟨ii⟩2

ca=ca,ba,bs
i
□
3.3orkHomewThree
3.3.1.Usingtherulesofetbra-kalgebra,evprooraluateevthewing:follo
(a)tr(XY)=tr(YX),whereXandYareoperators
Solution:∑
′′
Thedeifnitionoftraceofanoperatoristr(A)=⟨a|A|a⟩.Usingthisdeifnitionforoperator
′
a
XYewget
∑′′
tr(XY)=⟨a|XY|a⟩(Deifnition)
′
a
∑∑′′′′′′∑′′′′
=⟨a|X|a⟩⟨a|Y|a⟩(|a⟩⟨a|=1)
′′′′′
aaa
∑∑′′′′′′
=⟨a|Y|a⟩⟨a|X|a⟩(Complexbumernutecomm)
′′′
aa
∑′′′′∑′′
=⟨a|YX|a⟩(|a⟩⟨a|=1)
′′′
aa
=tr(YX)(Bydeifnition)
usThtr(XY)=tr(YX)asrequired□
(b)(XY)†=Y†X†,whereXandYareoperators.
Solution:
CHAPTER3.QUANTUMMECHANICS74
Let|α⟩beyanarbitraryet.k
LetY|α⟩=|γ⟩←DC→⟨γ|Y†=⟨γ|
Usingthisfactandoperatingthearbitrary|α⟩ybtheoperatorXYewget,
XY|α⟩=X|γ⟩(∵Y|α⟩=|γ⟩ybassumption)
⟨α|(XY)†=⟨γ|X†(∵akingTDConbothsides)
⟨α|(XY)†=⟨α|Y†X†(∵⟨γ|=⟨α|Y†)
†††
hWhicimplies(XY)=XY□
(c)exp(if(A))=?inet-brakform,whereAisaHermitianoperatorwhosealuesveigenarewn.kno
Solution:
Xf2(X)f3(X)
Assumingthefunctioncanbewrittenase=1+f(X)+2!+3!+···,whereXisan
operatorintheetkspace.eWevha
if(A)∑if(A)′′(∑′′)
e=e|a⟩⟨a|∵|a⟩⟨a|
′′
aa
′
Here|a⟩aretheetseigenkoftheoperatorif(A)AasitisengivtobeaHermitianoperator.Usingthe
expansionforeewget,
if(A)∑(f2(X)f3(X))′′(∑′′)
e=1+f(X)+2!+3!+···|a⟩⟨a|∵|a⟩⟨a|
a′()a′
∑′′12′′
=|a⟩+f(A)|a⟩+2!f(A)|a⟩+···⟨a|(∵X(|α⟩⟨β|)=(X|α⟩)⟨β|)
a′()
∑′′′12′′′′′′
=|a⟩+f(a)|a⟩+2!f(a)|a⟩+···⟨a|(∵f(X)|a⟩=f(a)|a⟩forHermitianX)
′
a()
∑′12′′′
=1+f(a)+2!f(a)+···|a⟩⟨a|(∵(a|α⟩)⟨β|=a(|α⟩⟨β|))
′
a
∑′
f(a)′′
=e|a⟩⟨a|
a′
f(A)
hWhicistherequiredformfortheoperatore.□
ˆˆ
3.3.2.Aspin1/2systemiswnknotobeinaneigenstateofS·nwithalueveigenℏ/2,wherenisaunitectorv
lyinginthexz-planethatesmakandangleγwiththepeositivz-axis.
(a)SupposeSxismeasured.Whatistheyprobabilitofgettingℏ/2
Solution:
ˆβ
orFaowtstatesystemthegeneralstaeofsystemcanbetedrepresenas|n;+⟩=cos2|+⟩+
iαβ
esin2|−⟩,whereαisthepolarangleandβistheuthalazimangle.orFthisproblemthepolar
angleisα=0anduthalazimangleisβ=γ.Sotheengivsystemand|Sx;+⟩statesare
γγ11
|nˆ;+⟩=sin2|+⟩+cos2|−⟩;|Sx;+⟩=√|+⟩+√|−⟩
22
Sinceybdeifnitiontheyprobablilitofmeasuringyanstatethatiswnknotobein2|beta⟩inastate
|α⟩isengivybˆ|⟨α|β⟩|.Sotheyprobabilitofmeasuring|Sx;+⟩statewhenthesystemiswnkno
tobein|n;+⟩staeis
CHAPTER3.QUANTUMMECHANICS75

()2
211(γγ)
|⟨Sx;+|nˆ;+⟩|=√⟨+|+√⟨−|sin|+⟩+cos|−⟩
2222

2
1γ1γ
=√sin+√cos
2222
12γ1γ1γ12γ
=2sin2+2√sin2√cos2+2cos2
22
=1(1+sinγ)
2
Sotheyprobabilitofmeasuringthe|nˆ⟩statein|Sx;+⟩stateis(1+sinγ)/2.□
⟨⟨2⟩⟩
(b)aluateEvthedispersioninSx–thatis(Sx−Sx)
Solution:
TheSxoperatorisSx=ℏ(|+⟩⟨−|+|−⟩⟨+|).TheresultofSxstateoperatedonthesystemat|nˆ⟩
2
is
S|nˆ⟩=ℏ(|+⟩⟨−|+|−⟩⟨+|)(sinγ|+⟩+cosγ|−⟩)=ℏcosγ|+⟩+ℏsinγ|−⟩
x2222222
ˆγγ
Andthedualcorrespondenceofthestate|n⟩is⟨nˆ|=sin2⟨+|+cos2⟨−|.Sotheexpectation
aluevofSxis
()
(γγ)ℏγℏγℏ(γγ)ℏ
⟨Sx⟩=⟨nˆ|Sx|nˆ⟩=sin2⟨+|+cos2⟨−|2cos2|+⟩+2sin2|−⟩=22sin2cos2=2sinγ
AlsotheexpectationaluevofoperatorS2is
x
()()
⟨2⟩(γγ)ℏℏγℏγ
S=⟨nˆ|SS|nˆ⟩=sin⟨+|+cos⟨−|(|+⟩⟨−|+|−⟩⟨+|)cos|+⟩+sin|−⟩
xxx2222222
()
(γγ)ℏ2(γγ)
=sin2⟨+|+cos2⟨−|4sin2|+⟩+cos2|−⟩
ℏ2(2γ2γ)
=4sin2+cos2
ℏ2
=4
wNothedispersionybdeifnitionis
⟨2⟩⟨2⟩2ℏ2(ℏ)2ℏ2(2)ℏ22
∆Sx≡Sx−(⟨Sx⟩)=4−2sinγ=41−sinγ=4cosγ
hWhicesgivthedispersionintmeasuremenofSxofthesystemin|nˆ⟩.□
3.3.3.ConstructthetransformationmatrixthatconnectstheSzdiagonalbasistothe∑⟩⟨Sxdiagonalbasis.
(r)(r)
wShothatouryresultistconsistenthewithgeneralrelationU=ba
Solution:r
1
Thestates|Sx;±⟩inthe|Sz;±⟩≡|±⟩stateisengivyb|Sx;±⟩=√(|+⟩±|−⟩).Sinceewwknothe
2
transformationmatrixformis
[]1[]1[]
⟨Sx;+|+⟩⟨Sx;+|−⟩=√(⟨+|+⟨−|)|+⟩(⟨+|+⟨−|)|−⟩=√11
⟨Sx;−|+⟩⟨Sx;−|−⟩2(⟨+|−⟨−|)|+⟩(⟨+|−⟨−|)|−⟩21−1
CHAPTER3.QUANTUMMECHANICS76
Let|p⟩=a|+⟩+a|−⟩intheoldSzbasis.hsucthata=⟨+|p⟩andb=⟨−|p⟩.Thisetkistransformed
toin
1[][]11
Mp=√11a≡√(a+b)|+⟩+√(a−b)|−⟩(3.2)
21−1b22
11
=√2(|+⟩+|−⟩)a+√2(|+⟩−|−⟩)b(3.3)
11
=√((|+⟩+|−⟩)⟨+|p⟩+√(|+⟩−|−⟩)⟨−|p⟩)(3.4)
22
=(|S;+⟩⟨+|+|S;−⟩⟨−|)|p⟩(3.5)
xx
∑rr
hWhicisintheformof|b⟩⟨a|.□
3.3.4.evProthat⟨x⟩→⟨x⟩+dx′,⟨p⟩→⟨p⟩underinifnitesimaltranslation.
Solution:
Sinceengiv
[x,T(dx)]=dx;⇒xT(dx)−T(dx)x=dx;xT(dx)=dx+T(dx)x
Letthestateofsystemundertranslationbe|β⟩=T(dx)|α⟩,usth⟨β|=⟨α|T†(dx).wNotheex-
pectationaluevofsystembeforetranslationis⟨x⟩=⟨α|x|α⟩.Theexpectationaluevaftertranslation
is
⟨x⟩=⟨β|x|β⟩
=⟨α|T†(dx)xT(dx)|α⟩
=⟨α|T†(dx)(dx+T(dx)x)|α⟩
=⟨α|T†(dx)+T†(dx)T(dx)x|α⟩
=⟨α|T†(dx)+x|α⟩
=⟨α|T†(dx)|α⟩+⟨α|x|α⟩
=dx+⟨x⟩
Sotheexpectationaluevofpositionaftertranslationis⟨x⟩+dx.
Similarlyfortummomen
|β⟩=T(dx)|α⟩,usth⟨β|=⟨α|T†(dx).wNotheexpectationaluevoftummomenbeforetranslationis
⟨p⟩=⟨α|p|α⟩.Theexpectationaluevaftertranslationis
⟨p⟩=⟨β|p|β⟩
=⟨α|T†(dx)pT(dx)|α⟩
=⟨α|T†(dx)(0+T(dx)p)|α⟩
=⟨α|T†(dx)T(dx)p|α⟩
=⟨α|p|α⟩
Sotheexpectationaluevofsystemaftertranslationisstill⟨p⟩.□
3.4orkHomewourF
′′′
3.4.1.Someauthorsdeifneanatoreroptoberealwheneryevbmemerofitsmatrixtselemen⟨b|A|b⟩isreal
insometation.represenIsthisconcepttationrepresenindept?endenThatis,dothematrixtselemen
CHAPTER3.QUANTUMMECHANICS77
′
remainrealenevifsomebasisotherthan{|b⟩}isused?kChecouryassertionusingxandp.
x
Solution:
′
Letsomeotherbasis|a⟩beusedtotrepresenthematrixthenthenewbasisisrelatedtotheoldbasis
ybthetransformation|a′⟩=U|b′⟩whereUissomeunitaryoperator.
′′′′†′−1
|a⟩=U|b⟩;⇒⟨a|=⟨b|U=⟨b|U
Thematrixtselemeninthisnewbasisthenbecome
′′′′−1′
⟨a|A|a⟩=⟨b|UAU|b⟩
′
Ifthishastoremainrealintheold|b⟩basisthenitustmequaltotheoldmatrixtelemen
′−1′′′′−1
⟨b|UAU|b⟩=⟨b|A|b⟩;⇒UAU=A;⇒AU=UA;⇒[U,A]=0
ButitisnotnecessarythattheoperatorsUandAutecommi.e.,[U,A]=0.usThthematrixtelemen
ofanoperatorymanotremainrealinatdifferenbasisifitisrealinonebasis.
kingChecthisassertionwithxandp.eWwknothatoperatorxishermitianinxbasissothatthe
x
′′′′
aluesveigenofxinposition|x⟩basisarereal.hWhicmeansthethematrixtselemen⟨x|x|x⟩=
′′′′′′′′′′′′
x⟨x|x⟩=xδ(x−x)areallrealbecausexisrealalueveigenofhermitianoperatorofx.
wNothematrixtselemenofxoperatorinpbasisare
′′′∫′′′′′′∫′′′′′′
p||⟩⟨p|x|x⟩⟨x|p⟩dx=x⟨p|x⟩⟨x|p⟩
⟨xp=
∫(′)(′′)∫(′′′′)
1′ipxipx1′(p−p)x′
=2πℏxexp−ℏexpℏdx=2πℏxexpiℏdx
′′′′
makingsubstitutiont=p−pandy=x/ℏ
=1∫ℏyeityℏdy=ℏ∫yeitydy
2πℏ2π
andusingtialdifferenundertegralinsignd∫eitydy=∫iyeitydy⇒∫yeitydy=1d∫eitydyewcan
dtidt
writetheeabvoexpressionas
ℏ1d∫ℏd∫′′′ℏdℏd
′′′ityi(p−p)y′′′′′′
⟨p|x|p⟩=2πidtedy=2πidtedy=2πidt2πδ(p−p)=idtδ(p−p)
Thisaluevisclearlyimaginaryasdeltafunctionispurelyreal.Thiswsshothatalthoughthematrix
tselemenofoperatorxinpositionbasisarerealthetselemenarenolongerrealintummomenbasis.□
′′′
3.4.2.(a)Supposethatf(A)isafunctionofaHermitianoperatorAwiththepropyertA|a⟩=a|a⟩.
′′′′′
aluateEv⟨b|f(A)|b⟩whenthetransformationmatrixfromtheabasistothebbasisiswn.kno
Solution:⟨⟩
(i)(j)
Thematrixtelemenforthetransformationmatrixarebafori,j∈{1,2···N}whereNis
thenoofindeptendenstateofsystem.Theengivexpressioncanbewrittenas
′′′∑⟨′′i⟩⟨i′⟩∑i⟩⟨i

⟨b|f(A)|b⟩=bf(A)aab(∵Insertingaa=1)
ii
∑⟨′′ii⟩⟨i′⟩′′′

=bf(a)aab(∵f(A)|a⟩=f(a)|a⟩)
i
∑i⟨′′i⟩⟨i′⟩

=f(a)baab(∵⟨α|c|β⟩=c⟨α|β⟩)
i
⟨⟩⟨⟩⟨⟩
′′ii′′i∗
Sinceallthematrixtselemenbaandab=baarewnknotheexpressioniscompletely
wn.kno□
CHAPTER3.QUANTUMMECHANICS78
(b)Usingtheuumtinconanalogueoftheresultobtainedin(√3.4.2a),aluateev⟨p′′|F(r)|p′⟩.Simplify
ouryexpressionasfarasouycan.Notethatrisx2+y2+z2,wherex,y,andzareatorserop.
Solution:
Sincethepositionoperatorsx,yandzarecompatibleoperatorseutativ(commi.e.,[x,y]=
′′′′
0,[y,z]=0and[z,x]=0)ewcantrepresenthepositioneteigenkas|x,y,z⟩≡|r⟩.By
problem(3.4.2a)eabvoewget
′′′∫∞′′′′′′3′
⟨p|F(r)|p⟩=−∞F(r)⟨p|r⟩⟨r|p⟩dr
Butewwknotheefunctionvawoftummomeninpositionbasisas
′′′′′
−ip·r′′′−ip·r′′−−ip·r
⟨|⟩eℏp|⟩eℏ⟨|⟩eℏ
pr=⇒⟨r=andrp=
usThtheexpressionbecomes
′′′′
′′′∫∞′−i(p−p)·r3′
⟨p|F(r)|p⟩=F(r)eℏdr
−∞
′
ThistegralinesgivthematrixtelemenofthepositionoperatorF(r)inthetummomenpbasis.□
3.4.3.Thetranslationoperatorforaifnite(spatial)tdisplacemenisengivyb
T(l)=exp(−ip·l),
ℏ
wherepisthetummomenatorerop.
(a)aluateEv[xi,T(l)]
Solution:∑
eWcanwritethedotproductofectorsvpandtdisplacemenlasp·l=pl
iii
[(−ip·l)]∂(−i∑pl)(−i)(ip·l)
[x,T(l)]=x,exp=iℏexpiii=iℏlexp−=lT(l)
iiℏ∂pℏiℏℏi
i
Thisesgivtheexpressionfor[xi,T(l)].□
(b)Using(3.4.3a)(orotherwise),demonstratewhoexpectationaluevof⟨x⟩hangescundertranslation
Solution:
Let|α⟩beyanarbitrarypositionet.kThentheexpectationaluevofforoneofthecomptonenof
positionofthesystem(particle)isengivyb⟨xi⟩=⟨α|xi|α⟩.Letthepositionetkundertranslation†
be|β⟩≡T(l)|α⟩.Thedualcorrespondenceofthisetkis⟨β|=⟨α|T(l).wNotheexpectation
aluevundertranslationis
⟨β|x|β⟩=⟨α|T(l)†xT(l)|α⟩(3.6)
ii
Butybtheutatorcommrelation(3.4.3a)ewevha
[x,T(l)]=lT(l);⇒xT(l)−T(l)x=lT(l)
iiiii
Sinceewwknothatthetranslationoperatoris,UnitaryT(l)†T(l)=1.Operatingonbothsides
†
ofthisexpressionybT(l)ewget
T(l)†{xT(l)−T(l)x}=T(l)†lT(l)
iii
†††
⇒T(l)xT(l)−T(l)T(l)x=lT(l)T(l)
iii
†
⇒T(l)xT(l)=x+l
iii
CHAPTER3.QUANTUMMECHANICS79
Usingthisin(3.6)ewget
⟨β|x|β⟩=⟨α|x+l|α⟩=⟨α|x|α⟩+⟨α|l|α⟩=⟨α|x|α⟩+l
iiiiiii
wNothatewevhafoundtheexpectationaluevoferyevcomptonenofxoperator.Theexpression
forthisoperatorbecomes
T(l)
⟨β|x|β⟩=⟨α|x|α⟩+l;⇒⟨x⟩−−→⟨x⟩+l
iii
old
Thisesgivtheexpectationaluevofpositionoperatorundertranslation.□
3.4.4.orFaGaussianevawet,kpacwhoseevawfunctionispositionspaceisengivyb
′[√1][′x′2]
⟨x|α⟩=√expikx−2
dπ2d
⟨2⟩ℏ222
(a)erifyV⟨p⟩=ℏkandp=2d2+ℏk
Solution:
Theexpectationaluevoftummomenpinthestate|α⟩isengivyb⟨p⟩=⟨α|p|α⟩.Butybcom-
pletenessofthepositionbasisetskewcanwritethestate|α⟩as
∫′′′
⟨p⟩=⟨α|p|α⟩=dx⟨α|x⟩⟨x|p|α⟩
Buttheoperatorytitiden
⟨x′|p|α⟩=−iℏ∂⟨x′|α⟩
∂x′
Enablesustowrite
∫∞′′(∂)′
⟨p⟩=−∞dx⟨α|x⟩−iℏ∂x′⟨x|α⟩
∫∞′[√1][′x′2](∂)[√1][′x′2]
=dx√exp−ikx−2−iℏ′√expikx−2
−∞dπ2d∂xdπ2d
∫∞((′))(′2)
1′xx
=√dx−iℏik−2exp−2
dπ−∞dd
1[∫∞′(x′2)iℏ∫∞′(x′2)]
=√ℏkdxexp−2+2xexp−2
dπ[−∞]dd−∞d
1√iℏ
=√ℏkπd+d20
dπ
=ℏk
CHAPTER3.QUANTUMMECHANICS80
2
Smilarlytheexpectationaluevofoperatorpcanbewrittenas
∫()
⟨⟩∞∂2
2′′′
p=−∞dx⟨α|x⟩−iℏ∂x′⟨x|α⟩
∫∞′[√1][′x′2](2∂2)[√1][′x′2]
=dx√exp−ikx−2−ℏ′2√expikx−2
−∞dπ2d∂xdπ2d
∫(())()
1∞ℏ2x′2x′2
√′2
=dx2−ℏik−2exp−2
dπ−∞ddd
∫∞(222′2)(′2)
1′ℏ222ikℏℏxx
=√dx2+ℏk+2−4exp−2
dπ−∞dddd
[(2)∫∞(′2)2∫∞(′2)2∫∞(′2)]
1ℏ22x′2ikℏ′x′ℏ′2x′
=√2+ℏkexp−2dx+2xexp−2dx−4xexp−2dx
dπd−∞dd−∞dd−∞d
[(2)22(√3)]
1ℏ22√2ikℏℏπd
=√2+ℏkπd+20−4
dπddd2
ℏ2ℏ2
=+ℏ2k2−
22
d2d
ℏ2
22
=2+ℏk
2d
usThtheexpectationaluesvoftheefunctionvawisfoundasrequired.□
2
(b)aluateEvtheexpectationaluevofpandpusigthetum-spacemomenevawfunctionsasell.w
Solution:
orFthetummomenspaceevawfunctionsewcanwrite
∫′′′∫′′2′
⟨p⟩=⟨α|p|p⟩⟨p|α⟩dp=p|⟨p|α⟩|dp
∫[′22]
d′(p−ℏk)d′
=ℏ√πpexp−ℏ2dp
[()()]
d∫′d2′∫′(p−ℏk)2′
=ℏ√πpexp−ℏ2dp+pexpℏ2dp
d[ℏ2k√π]
=ℏ√πd=ℏk
wNofortheexpectationaluevofthesquareoftummomenoperator.
⟨2⟩∫′′′∫′2′2′
p=⟨α|p|p⟩⟨p|α⟩dp=p|⟨p|α⟩|dp
∫[′22]
d′2(p−ℏk)d′
=√pexp−ℏ2dp
ℏπ(√√)
dπℏ3ℏ3k2π
=√3+
ℏπ2dd
ℏ2
22
=2+ℏk
2d
Sotheexpectationaluevoftheoperatorsarethesmaeinthetummomenstateevawfunctionstoo.
□
3.5orkHomeweFiv
3.5.1.(a)evProthewingfollo
CHAPTER3.QUANTUMMECHANICS81
′∂′
i.⟨p|x|α⟩=iℏ∂p′⟨p|α⟩
∫′∗′∂′
ii.⟨β|x|α⟩=dpϕ(p)iℏϕ(p),
β∂p′α
′′′′
whereϕ(p)=⟨p|α⟩andϕ(p)=⟨p|β⟩aretum-spacemomenevawfunctions.
αβ
Solution:
eWwkno
()∞
′′∫′
′′√1ipxi(t−t)x′
⟨x|p⟩=2πℏexpℏ;andedx=2πδ(t−t)
−∞
Withthehelpoftheseowtrelationsewcansimplifytheytitquanewtanwas
′∫′′′′∫′′′
⟨p|x|α⟩=dx⟨p|x|x⟩⟨x|α⟩(∵dx|x⟩⟨x|=1)
∫′′′′′′′′′
=x⟨p|x⟩⟨x|α⟩dx(∵⟨p|x|x⟩=x⟨p|x⟩)
∫′′∫′′′′′′′′′∫′′′′′′
=dpx⟨p|x⟩⟨x|p⟩⟨p|α⟩dx(∵dp|p⟩⟨p|=1)
∫∫(′′)(′′′)
′′′√1ipx√1ipx′′′
=dpx2πℏexpℏ·2πℏexp−ℏ⟨p|α⟩dx
∫∫(′′′′)
1′′′i(p−p)x′′′
=2πℏdpxexpℏ⟨p|α⟩dx
′
eWcanusetegralinundertialdifferensigntoaluateevthedxtegralinas
d∫′′′′′∫′′′′′′
dp′exp(i(p−p)x)dx=xexp(i(p−p)x)dx
′
Usingthsinthedxtegralineabvoewget
∫2∫(′′′′)
1′′ℏ∂i(p−p)x′′′
=2πℏdp−i∂p′expℏ⟨p|α⟩dx
1∫′′ℏ2∂′′′′′
=2πℏdp−i∂p′2πδ(p−p)⟨p|α⟩∫
′′′
=iℏ⟨p|α⟩(∵f(x)δ(x−x)dx=f(x)
Thisesgivustherequiedresult.
∫′′′
⟨β|x|α⟩=dp⟨β|p⟩⟨p|x|α⟩(3.7)
′∂′
Theresulteabvois⟨p|x|α⟩=iℏ∂p′⟨p|α⟩Substutingthisin(3.7)ewget
∫′′∂′
⟨β|x|α⟩=dp⟨β|p⟩iℏ∂p′⟨p|α⟩
′∗′′′
ritingW⟨β|p⟩=ϕβ(p)and⟨p|α⟩=ϕα(p)ewget
∫′∗′∂′
⟨β|x|α⟩=dpϕβ(p)iℏ∂p′ϕα(p)
Thisistherequiedexpression.□
CHAPTER3.QUANTUMMECHANICS82
(b)Whatistheysicalphsigniifcanceof
(ixΞ)
expℏ
wherexisthepositionoperatorandΞissomebumernwiththeunitoftum?momenJustifyoury
er.answ
Solution:()
Inthepositionbasiseigenthepositiontranslationoperatoris⊔(l)=expiplwherelisatconstan
ℏ
ofunitoftlenghandpisthetummomenoperator.
eWevhaheretherolesofoperatorxandphangedcandlandΞhanged.chWhicsuggeststhat
thisoperatorfunctioncanorkswasatummomentranslationoperatorintummomenbasis.eigen□
3.5.2.IftheHamiltonianHisengivas
H=H|1⟩⟨1|+H|2⟩⟨2|+H|1⟩⟨2|
112212
Whatprincipleisviolated?Illustrateouryptoinybexplicitlyattemptingtoesolvthemostgeneraltime-
deptendenproblemusinganillegalHamiltonianofthiskind.(AssumeH=H=0for.)ysimplicit
1122
Solution:
orFaoperatortobeaalidvHamiltonianithastobeaHermitianoperator.eWcankheccifthisisa
Hermitianoperator.
H†=H∗|1⟩⟨1|+H∗|2⟩⟨2|+H∗|1⟩⟨2|=H|1⟩⟨1|+H|2⟩⟨2|+H|1⟩⟨2|
112212112212
SinceH†̸=HtheengivhamiltonianisclearlynotHermitian.Sothisoperatortheenergyetseigenk()
on’twbereal.Also,thetimetranslationoperatorU(t)=exp−iHtwillnotbeunitaryhwhicouldw
ℏ
emakthetimeedolvevstatesnoteconservtheinnerproductso,itviolatestheprincipleofyprobabilit
violation.
SettingH=H=0theHamiltonianbecomesH=H|1⟩⟨2|.Letskhecctheunitarypropyertof
112212
theunitaryoperator
†(iH†t)(iHt)(i(H†−H)t)
U(t)U(t)=expℏ·exp−ℏ=expℏ
orFtheoperatortoremain,unitarytheexptialonenshouldbezerobutsinceH†̸=Htheexptonen
willbenonzeroanditviolatestheprinciplethatthetimeolutionevoperatorsi.unitary□
′′′′′′
3.5.3.Let|a⟩and|a⟩beeigenstatesofaHermitianoperatorAwithaluesveigenaanda,respelyectiv
′′′
(a̸=a).TheHamiltonianoperatorisengivyb
′′′′′′
H=|a⟩δ⟨a|+|a⟩δ⟨a|
whereδisjustarealbumer.n
′′′
(a),Clearly|a⟩and|a⟩arenoteigenstatesoftehHamiltonian.riteWwndotheeigenstatesofthe
Hamiltonian.WHataretheirenergyalues?vEigen
Solution:
′′′
Lettheenergyeteigenkofthishamiltonianoperatorbe|α⟩=p|a⟩+q|a⟩.AndEbetheenergy
eigenalues.vSooperatingybHonthisstateleadsto
′′′′′′′′′
H|α⟩=(|a⟩δ⟨a|+|a⟩δ⟨a|)(p|a⟩+q|a⟩)
′′′
=δq|a⟩+δp|a⟩
CHAPTER3.QUANTUMMECHANICS83
′′′′
IfthisistobetheenergyeigenstatethenitshouldequalE|α⟩=Ep|a⟩+Eq|a⟩.Since|a⟩and
′′
|a⟩areorthogonalstates,thecoteiffciencomparisionleadsto
Ep=δq;⇒p=δq
E
Eq=δp;⇒Eq=δδq;⇒E=±δ
E
Sotheenergyaluesveigenare22E=±δ.Alsosinceewrequiretheeigenstatebenormalizedew
requirep+q=1.Thisresultsin
δ2q211
+q2=1;⇒p=√,q=±√
E222
Sotherequriedenergyetseigenkare
1′′′1′′′
|α⟩=√(|a⟩+|a⟩);|α⟩=√(|a⟩−|a⟩)(3.8)
+2−2
Where|α+⟩istheeteigenkcorrespondingtoalueveigen+δand|α−⟩istheeteigenkcorresponding
toalueveigen−δ□
′
(b)Supposethesystemiswnknotobeinthestate|a⟩att=0.riteWwndothestateectorvof
hroScdingerpicturefort>0.
Solution:()
iHt′
ThetimeolutionevoperatorisU(t)=exp−ℏ.Since|a⟩arenottheeiergyets,eigenkewcan
writethemintermsoftheetseigenkofHamiltonianoperator.romF(3.8)ewcanaddandsubtract
theowtenergyetseigenktoifnd
′1′′1
|a⟩=√(|α⟩+|α⟩)|a⟩=√(|α⟩−|α⟩)
2+−2+−
′
Applicationoftimeolutionevoperatorto|a⟩leadsto
′(iHt)′(iHt)11−iδt1iδt
U(t)|a⟩=exp−|a⟩=exp−√(|α⟩+|α⟩=√eℏ|α⟩+√eℏ|α⟩
ℏℏ2+−2+2−
Againtheapplicationof(3.8)ewcanertvconkbactothebasisstatesengiv
′1−iδt′′′1iδt′′′1−iδtiδt′1−iδtiδt′′
U(t)|a⟩=eℏ(|a⟩+|a⟩)+eℏ(|a⟩−|a⟩)=(eℏ+eℏ)|a⟩+(eℏ−eℏ)|a⟩
222|{zδt}2|{zδt}
2cos(ℏ)2isin(ℏ)
Eulerytitidencanbeusedtoertvconthecomplexexptialsonentosinesandcosies,hwhicegiv
′(δt)′(δt)′′
U(t)|a⟩=cosℏ|a⟩+isinℏ|a⟩(3.9)
′
Thisesgivthetimeolutionevofstate|a⟩underthishamiltonian.□
′′
(c)Whatistheyprobabilitforifndingthesystemin′|a⟩fort>0ifthesystemiswnknotobeinthe
state|a⟩att=0?
Solution:
′
Theyprobabilitoftingifnthesystemknontobein2|a⟩atalatertimet>0isengivyb
′′′
|⟨a|U(t)|a⟩|hwhiccanbeauatedevusing(3.9)
[()()]2()2()
′′′2′′δt′δt′′δt2δt

P=|⟨a|U(t)|a⟩|=⟨a|cos|a⟩+isin|a⟩=isin=sin
′ℏ′′ℏℏℏ
Sotheyprobabilitofifndingthe|a⟩tobeat|a⟩atalatertimeistheoscillatingfunction.The
ysicalphsituationcorrespondingtothisproblemisaNeutrinooscillation.□
CHAPTER3.QUANTUMMECHANICS84
3.5.4.wSho
()∫∞(′′′2)√[′22]
′11′−ipx′xd−(p−ℏk)d
⟨p|α⟩=√1/4√dxexpℏ+ikx−2d2=ℏ√πexp2ℏ2.
2πℏπd−∞
Solution:
Consideringthefactorinsidetheexptialonen
−ip′x′′x′21(′22(ip′x)′)
+ikx−2=−2x−2dik−x
ℏ2d2dℏ
2(ip′x)
Ifewletthetconstantermst=dik−ℏthenintheexptialonenewget
−1(′2′)Completionofsquare−1′22
2x−2tx−−−−−−−−−−−−−−→2((x−t)−t)
2d2d
Withthisthetegralinbecomes
∫()()()∫{()}
∞x′2t2t2∞x′2
′√′
exp−2·exp2dx=exp−2exp−dx
−∞2d2d2d−∞2d
Thistegralinisastandardgammafunctionwhosealuevis
∫{()}∫{()}√
∞x′2∞x′2π√
√′√′
−∞exp−2ddx=20exp−2ddx=2·22d
Usingthisinouroriginalequationewget
′√1(1√)(t2)√
⟨p|α⟩=2πℏπ1/4dexp−2d22πd
eWcansubstitutekbactheariablevtkbactoget
()()(√)(′′)
11t2√1dd4(ik−ipx)2
′√√√ℏ
⟨p|α⟩=1/4exp−22πd=1/4exp−2
2πℏπd2dℏπ2d
√[′22]
d−(p−ℏk)d
=√exp2ℏ2.
ℏπ
hWhicistherequiredsolution□
3.6orkHomewSix
3.6.1.UsingtheHamiltonian
H=−(eB)Sz=ωSz
mc
writethebHeisenergequationofmotionforthetime-deptendenoperatorsSx(t),Sy(t)andSz(t).eSolv
themtoobtainSx,y,zasfunctionsoftime
Solution:
eWwknotheutaioncommrelationforspinoperators[S,S]=iℏεS.Andsincethetimeeativderiv
ijijkk
oftheoperatorinbHeisenergpictureis
d=1[A,H]
dtiℏ
CHAPTER3.QUANTUMMECHANICS85
eWcanwritethetimeeativderivofthespinoperatorsas
dS=1[S,H]=1[S,ωS]=10=0
dtziℏziℏzziℏ
dS=1[S,H]=1[S,ωS]=−ωiℏS=−ωS
dtxiℏxiℏxziℏyy
dS=1[S,H]=1[S,ωS]=ω1iℏS=ωS
dtyiℏyiℏyziℏxx
Bysimilarfashionewcanifndthesecondtimeeativderivoftheoperatorsas
2()
dS=ddS=d=0
dt2zdtdtzdt
2()
dS=ddS=d(−ωS)=−ω2S
dt2xdtdtxdtyx
2()
dS=ddS=d(ωS)=−ω2S
dt2ydtdtydtxy
SincetheifrsttimeeativderivofoperatorSiszero,itistconstanervotime.orF∂2S=−ω2Sforms
z∂t2xx
aOrdinarySecondordertialdifferenequationinoperatorSx.(Assumingesativderivareellwdeifned
foroperators)eWcanwritethesolutionas
Sx=Ae−iωtSy=Be−iωt
WhereAandBarearbitarytconstan(complex)bumers.n□
3.6.2.ConsideraparticleinonedimensionwhoseHamiltonianisengivyb
2
H=p+V(x)
2m
Bycalculating[[H,x],x]evpro
∑2ℏ2
′′′′′′
|⟨a|x|a⟩|(Ea−Ea)=2m,
′
a
′′
where|a⟩isanenergyeteignekwithalueveigenEa
Solution:
SincexisHermitianoperatorandV(x)ispurefunctionofxtheutatorcommofxandV(x)iszero
2[2]
i.e.,[x,V(x)]=0.Bysimilartsargumentheutatorcommofpandpiszeroi.e.,p,p=0Soew
2m2m
cancalculatetheutatorcomm
[2][]
p12p
[H,x]=2m+V(x),x=2mp,x=−iℏm
Alsoewcansimplifytheutatorcommas
[p]iℏℏ2
[[H,x],x]=iℏm,x=−m[p,x]=−m
CHAPTER3.QUANTUMMECHANICS86
theexpectationaluevoftheoperator[[H,x],x]canbecalculatedas
′′′′
⟨[[H,x],x]⟩=⟨a|[[H,x],x]|a⟩
′′′′
=⟨a|[Hx−xH,x]|a⟩
′′22′′
=⟨a|Hx−xHx−xHx+xH|a⟩
′′2′′′′2′′′′′′
=⟨a|Hx|a⟩+⟨a|xH|a⟩−2⟨a|xHx|a⟩
′′′′2′′′′′′2′′′′′′
=E⟨a|x|a⟩+E⟨a|x|a⟩−2⟨a|xHx|a⟩
aa
′′′′2′′′′′′
=2Ea⟨a|x|a⟩−2⟨a|xHx|a⟩
′′2′′
wNotheytitquan⟨a|x|a⟩canbewrittenas
′′2′′′′′′∑′′′′′′∑′′′2
⟨a|x|a⟩=⟨a|xx|a⟩=⟨a|x|a⟩⟨a|x|a⟩=|⟨a|x|a⟩|
′′
aa
′′′′
Similarlyewcanexpress⟨a|xHx|a⟩as
′′′′∑′′′′′′∑′′′′′′′∑′′′′2
⟨a|xHx|a⟩=⟨a|xH|a⟩⟨a|x|a⟩=Ea⟨a|x|a⟩⟨a|x|a⟩=Ea|⟨a|x|a⟩|
′′′
aaa
Finallythesecanbesubstituedtoegiv
∑′′′′′′2
⟨[[H,x],x]⟩=2(E−E)|⟨a|x|a⟩|
aa
′
a
Butsinceewcalcluatedtheepectationaluevtobe−ℏ2ewcanwritetheexpression
m
∑2ℏ2
′′′′′′
(Ea−Ea)|⟨a|x|a⟩|=2m
a′
Thisistherquiredexpression.□
3.7orkHomewenSev
3.7.1.ConsideraparticleinonedimensionwhoseHamiltonianisengivyb
2
H=p+V(x)
2m
Bycalculating[[H,x],x]evpro
∑2ℏ2
′′′′′′
|⟨a|x|a⟩|(Ea−Ea)=2m,
′
a
′′
where|a⟩isanenergyeteignekwithalueveigenEa
Solution:
SincexisHermitianoperatorandV(x)ispurefunctionofxtheutatorcommofxandV(x)iszero
2[2]
i.e.,[x,V(x)]=0.Bysimilartsargumentheutatorcommofpandpiszeroi.e.,p,p=0Soew
2m2m
cancalculatetheutatorcomm
[2][]
p12p
[H,x]=2m+V(x),x=2mp,x=−iℏm
CHAPTER3.QUANTUMMECHANICS87
Alsoewcansimplifytheutatorcommas
[p]iℏℏ2
[[H,x],x]=iℏm,x=−m[p,x]=−m
theexpectationaluevoftheoperator[[H,x],x]canbecalculatedas
′′′′
⟨[[H,x],x]⟩=⟨a|[[H,x],x]|a⟩
′′′′
=⟨a|[Hx−xH,x]|a⟩
′′22′′
=⟨a|Hx−xHx−xHx+xH|a⟩
′′2′′′′2′′′′′′
=⟨a|Hx|a⟩+⟨a|xH|a⟩−2⟨a|xHx|a⟩
′′′′2′′′′′′2′′′′′′
=E⟨a|x|a⟩+E⟨a|x|a⟩−2⟨a|xHx|a⟩
aa
′′′′2′′′′′′
=2Ea⟨a|x|a⟩−2⟨a|xHx|a⟩
′′2′′
wNotheytitquan⟨a|x|a⟩canbewrittenas
′′2′′′′′′∑′′′′′′∑′′′2
⟨a|x|a⟩=⟨a|xx|a⟩=⟨a|x|a⟩⟨a|x|a⟩=|⟨a|x|a⟩|
′′
aa
′′′′
Similarlyewcanexpress⟨a|xHx|a⟩as
′′′′∑′′′′′′∑′′′′′′′∑′′′′2
⟨a|xHx|a⟩=⟨a|xH|a⟩⟨a|x|a⟩=Ea⟨a|x|a⟩⟨a|x|a⟩=Ea|⟨a|x|a⟩|
′′′
aaa
Finallythesecanbesubstituedtoegiv
∑′′′′′′2
⟨[[H,x],x]⟩=2(E−E)|⟨a|x|a⟩|
aa
′
a
Butsinceewcalcluatedtheepectationaluevtobe−ℏ2ewcanwritetheexpression
m
∑2ℏ2
′′′′′′
(Ea−Ea)|⟨a|x|a⟩|=2m
a′
Thisistherquiredexpression.□
3.7.2.Considerafunction,wnknoasthecorrelationfunction,deifnedyb
C(t)=⟨x(t)x(0)⟩
wherex(t)isthepositionoperatorinthebHeisenergpicture.aluateEvthecorrelationfunctionexplicitly
forthegroundstateofaonedimensionalharmonicoscillator.
Solution:
TheoperatorinbHeisenergpicturehangecwithtime.TimeolutionevofoperatorxisbHeisenerg
pictureis
x(t)=x(0)cosωt+sinωtp(0)
mω
wherex(0)andp(0)arepositionandtummomenoperatorattimet=0.usThthecorrelationfunction
becomes
⟨2sinωt⟩
C(t)=⟨x(t)x(0)⟩=x(0)cosωt+mωp(0)(x0)
⟨2⟩sinωt
=x(0)cosωt+⟨p(0)x(0)⟩mω
CHAPTER3.QUANTUMMECHANICS88
Ifewdenotex(0)andp(0)ybjustxandpandthegroundstateofharmonicoscillatoryb|0⟩ewget
2sinωt
C(t)=⟨0|x|0⟩cosωt+⟨0|px|0⟩mω
Theoperatorxandpintermsofcreationandannihilationoperatorare
√ℏ(†)√ℏmω(†)
x=2mωa+ap=i2a−a
2h2†††2ℏ2†††2
x=2mω(a+aa+aa+(a));px=i2(a+aa−aa−(a))
usThthegroundstateexpectationaluevofoperatorsbecome
2ℏ2†††2ℏ†ℏ
⟨0|x|0⟩=2mω⟨0|(a+aa+aa+(a))|0⟩=2mω⟨0|aa|0⟩=2mω
iℏ2†††2iℏ†iℏ
⟨0|px|0⟩=2⟨0|a−aa+aa−(a)|0⟩=2⟨0|aa|0⟩=2
usThthecorrelationfunctionbecomes
C(t)=ℏcosωt+iℏsinωt
2mω2mω
□
3.7.3.wShothatforone-dimensionalsimpleharonicoscillator,
[22]
⟨0|ekx|0⟩=exp−k⟨0|x|0⟩
2
wherexisthepositionatorerop.
Solution:
Theexpectationaluevofx2ingoroundstateis
2ℏ
⟨0|x|0⟩=2mω
[k2ℏ]
SotheRHSofeabvoexpressionbecomesexp−4mωTheLHScanbealuatedevas
ikx∫′ikx′′
⟨0|e|0⟩=dx⟨0|e|x⟩⟨x|0⟩∫
′
′ikx′′
=∫dxe⟨0|x⟩⟨x|0⟩
′2
′ikx′
=dxe|⟨x|0⟩|
Thegroundstateevawfunctionforharmonicoscillatoris
′(1)[1(x′)2]
⟨x|0⟩=1/4√exp−2x
πx00
CHAPTER3.QUANTUMMECHANICS89
Wherex0is√ℏ.usThtheexpectationaluevbecomes
mω
()∫[()]
′2
1′x
ikxikx′
⟨0|e|0⟩=1/4√eexp−xdx
πx00
()∫[()]
′2
=1exp−x+ikx′dx′
1/4√x
πx00
()∫[()]
′222
1xikx0kx0′
=1/4√exp−x−2−4dx
πx00
()(22)∫[(′)2]
1kx0xikx0′
=1/4√exp−4exp−x−2dx
πx00
()
()(22)1/4√
=1exp−kx0πx0
1/4√41
πx0
[k2ℏ]
=exp−4mω
ThiswsshotheLHSandRHSareequalvingprotheproposition.□
Ontheotherhand
ikxik√ℏ(a+a†)
⟨0|e|0⟩=⟨0|e2mω|0⟩
ik√ℏaik√ℏa†
=⟨0|e2mωe2mω|0⟩
ik√ℏaik√ℏ
=⟨0|e2mωe2mω|1⟩
ik√ℏik√ℏa
=e2mω⟨0|e2mω|1⟩
ik√ℏik√ℏ
=e2mω⟨0|e2mω|0⟩
ik√ℏik√ℏ
=e2mωe2mω⟨0|0⟩
2ik√ℏ
=e2mω
I’tdonesewhythisachoappradsletotheongwrsolution?
3.7.4.Let
†ℏ2(††)††
J=ℏaa,J=aa−aa,N=aa+aa,
±±∓z2++−−++−−
†
wherea±andaaretheannihilationandcreationoperatorsofowtendentindepsimpleharmonic
±
oscillatorssatisfyingtheusualsimpleharmonicoscillatorutationcommrelations.evPro
()[]
ℏ2N
[J,J]=±ℏJ,[J2,J]=0,J2=N+1.
z±±z22
Solution:
CHAPTER3.QUANTUMMECHANICS90
Theutatorcomm[J,J]canbecalculatedas
z+
ℏ2[†††]
[J,J]=aa−aa,aa
z+2++−−+−
ℏ2†[†]†[†][††][††]
=aaa,a+aaa,a+aa,aa+aa,aa
2++−++−−−+−++−−+−
ℏ2†[†]†[†]†[†]
=aa,aa−aa,aa+aa,aa
2+++−++−++−−−
()[][]
†2††††2
−a[a,a]−aa,aa−a,aa
++−−+−−+−−
{†}{†}
Sincea.aanda.aareindeptendenoperatorsactingontdifferen(indept)endensystems.
++−−
Thewingfolloutationcommrelationholds
[†][†][†][†]
a,a=1,a,a=0,a,a=0,a,a=1,[a,a]=0
+++−−+−−+−
Usingtheseintheexpressioneabvoewget
2[()]
ℏ††††2†2
[J,J]=a·1·a−a·0·a+a·1·a−a·0−a·0·a−0·a
z+2+−+++−+−−−
ℏ2[†][†]
=2aa=ℏℏaa=ℏJ
2+−+−+
Similarlytheutatorcommof[J,J]canbecalculatedtobe[J,J]=−ℏJ
z−z−−
BydeifnitionJ2=JJ+J2+ℏJ(Sakurai3.5.24).Usingthisewcanwrite
+−zz[]
[J,J]=JJ+J2+ℏJ,J
z+−zzz
=[JJ,J]+[J2,J]+ℏ[J,J]
+−zzzzz
=J[J,J]+[J,J]J+0+0
+−z+zz
=J{ℏJ}+{−ℏJ}J
+−+−
=ℏJJ−ℏJJ
+−+−
=0
romFdeifnition
J2=JJ+J2+ℏJ
+−zz
2††
JJ=ℏaaaa.Similarlytheothertermsinthedeifnitionare
+−+−−+
2ℏ2(††)2
J=aa−aa
z4++−−
ℏ2{††††††††}
=aaaa−aaaa−aaaa+aaaa
4++++++−−−−++−−−−
SoJ2becomes
2ℏ2{††††††††}2††ℏ2(††)
J=aaaa−aaaa−aaaa+aaaa+ℏaaaa+aa−aa
4++++++−−−−++−−−−+−−+2++−−
ℏ2†(††)ℏ2(††)(1†)
=aaaa+aa+aa+aa1+aa
4++++−−2++−−2−−
(††)
ℏ2()aaaa
=a†a+a†a+++1+−−
2++−−22
=ℏ2N(N+1)
22
hWhiccompletestheproof.□
CHAPTER3.QUANTUMMECHANICS91
3.8orkHomewtEigh
3.8.1.ConsideraOne-Dimensionalsimpleharmonicoscillator.
(a)Using
}√mω(ip)}{√
aa|n⟩n|n−1⟩,
†=x±,†=√
a2ℏmωa|n⟩n+1|n+1⟩,
22
aluateev⟨m|x|n⟩,⟨m|p|n⟩,⟨m|{x,p}|n⟩,⟨m|x|n⟩and⟨m|p|n⟩.
Solution:
†
enGivthedeifnitionofaandaewcanexpressoperatorxandpintermsoftheseoperatorsas
√ℏ(†)√ℏmω(†)
x=2mωa+ap=i2a−a
Theresultofoperatorxandponyanstateare
√ℏ(†)√ℏ(†)√ℏ(√√)
x|n⟩=2mωa+a|n⟩=2mωa|n⟩+a|n⟩=2mωn+1|n+1⟩+n|n−1⟩
√ℏmω(†)√ℏmω(†)√ℏmω(√√)
p|n⟩=i2a−a|n⟩=i2a|n⟩−a|n⟩=i2n+1|n+1⟩−n|n−1⟩
eWcancalculatetheseas
√√√√
ℏ(√)ℏ[√]
⟨m|x|n⟩=2mω⟨m|n+1|n+1⟩+n|n−1⟩=2mωn+1δm,n+1+nδm,n−1
√ℏmω(√√)√ℏmω[√√]
⟨m|p|n⟩=i⟨m|n+1|n+1⟩−n|n−1⟩=in+1δ−nδ
22m,n+1m,n−1
Sothisesgivthematrixtelemenofalltheoperatorsen.giv□
3.8.2.Consideragainaone-dimensionalsimpleharmonicoscillator.Dothewingfolloalgebriacally–thatis,
whoutusingevawfunctions.
(a)Constructalinearbinationcomof|0⟩and|1⟩hsucthat⟨x⟩isaslargeaspossible.
Solution:
iδ
Letthelinearbinationcomof|0⟩and|1⟩be|α⟩=r|0⟩+se|1⟩.Whererandsarereal.eWcan
ysaalwhocoserandsrealbecausetheerallvophaseofthestatedoesn’tmatterandδestakcare
ofthephasedifference.Thedualcorrespondenceof|α⟩is⟨α|=⟨0|r+⟨1|e−iδs.Theexpectation
aluevofoperatorxinthisstateis
−iδiδ
⟨α|x|α⟩=(⟨0|r+⟨1|es)x(r|0⟩+se|1⟩)
=r2⟨0|x|0⟩+rseiδ⟨0|x|1⟩+rse−iδ⟨1|x|0⟩+s2⟨1|x|1⟩
=r2·0+rseiδ√ℏ+rse−iδ√ℏ+s2·0
√2mω√2mω
ℏ(iδ−iδ)ℏ
=rs2mωe+e=22mωrscosδ
|{z}
2cosδ
Theummaximaluevofthisexpressioniswhenδ=0.Alsoifewtanwnormalizedstateetkthen
√2√2
s=1−r.Theummaximaluevofrs=max(r1−r).Theummaximaluevcanbeobtained
yb
√2√(22)
d(rs)d(r1−r)22r1−r−r1
dr=dr=1−r−r√2=√2=0;⇒r=√2
21−r1−r
CHAPTER3.QUANTUMMECHANICS92
√21
Substutingthisaluevofrfors=1−resgivs=√2.Sothelinearbinationcomof|0⟩and|1⟩
thatmaximizestheexpectationofxis
11
|α⟩=√(|0⟩+|1⟩);→⟨α|=√(⟨0|+⟨1|)
22
ThealueveigenofoperatorHingeneralnstaeofsimpleharmonicoscillatoris
(1)
H|n⟩=n+2ℏω|n⟩
Inthehroscdingerpicturethetimeolutionevofstate|α⟩is
−iHt1(−iHt−iHt)1(−1/2iωt−3/2iωt)
U(t)|α⟩=eℏ|α⟩=√eℏ|0⟩+eℏ|1⟩=√e|0⟩+e|1⟩
22
Thedualcorrespondenceofthistimeedolvevstateis
†1(1/2iωt3/2iωt)
⟨α|U(t)=√e⟨0|+e⟨1|
2
Againtheoperatorxonstate|0⟩and|1⟩are
√ℏ(†)√ℏ(†)√ℏ
x|0⟩=√2mωa+a|0⟩=√2mωa|0⟩+a|0⟩=√2mω|1⟩
ℏ(†)ℏ(†)ℏ(√)
x|1⟩=2mωa+a|1⟩=2mωa|1⟩+a|1⟩=2mω2|2⟩+|0⟩
usThtheexpectationvlauebecomes
†1(1/2iωt3/2iωt)1(−1/2iωt−3/2iωt)
⟨α|U(t)xU(t)|α⟩=√e⟨0|+e⟨1|x√e|0⟩+e|1⟩
22
1√ℏ(iωt−iωt)
=22mωe+e
=√ℏcosωt
2mω
usThtheummaximexpectationaluevonhroscdingerpictureis⟨x⟩=√ℏcosωt.
2mω
Alsointhebheisenergpicturethetimeolutionevofoperatorxis
x(t)=x(0)cosωt+p(0)sinωt
mω
Sotheexpectationaluevofx(t)isengivyb
⟨x(t)⟩=⟨x(0)⟩cosωt+⟨p(0)⟩sinωt=⟨α|x|α⟩cosωt+⟨α|p(0)|α⟩sinωt=√ℏcosωt
mωmω2mω
⟨2⟩⟨2⟩
Sotheexpectationaluevaresameinbothpictures.oTcalculate⟨2⟩∆xewneedxand⟨x⟩
sinceewalreadywkno⟨x⟩ewcancalculatexas
211−iωt21iωt212
⟨α|x|α⟩=2⟨0|x|0⟩+2e⟨0|x|1⟩+2e⟨1|x|0⟩+2⟨1|x|1⟩
=1√ℏ√ℏ(1+0+0+3)=ℏ
22mω2mωmω
CHAPTER3.QUANTUMMECHANICS93
usTh
⟨2⟩⟨2⟩2ℏℏ2ℏ2ℏ2
∆x=x−⟨x⟩=mω−2mωcosωt=2mω(2−cosωt)=2mω(1+sinωt)
⟨2⟩ℏ
Sotheariancevinthetmeasuremenofpositioninthissatateaftertimetis∆x=2mω(1+
2
sinωt)□
3.8.3.Consideraparticleinonedimensionboundtoaifxedtercenybaδ-functionptialotenoftheform
V(x)=−ν0δ(x),(ν0realandpeositiv).
Findtheevawfunctionandthebindingenergyofthegroundstate.rethereexcitedboundstates?
Solution:
LettheparticlehastotalenergyEandmassm.Theevawfunctionsatisifesthehroscdingerequation
22
ℏdψ(x)+Eψ(x)=V(x)ψ(x)(3.10)
2mdx2
SincethereisadeltafunctionptialotenV(x)=−ν0δ(x)ewcandividetheefunctionvawasowtpart
functionIfx̸=0,V(x)=0.ThenthehroScdingersequationreducesto
2
dψ(x)+2mEψ(x)=0(3.11)
dx2ℏ2
Thesisaellwwnknosecondorderordinarytialdifferenequationwhosesolutionisintheform
√2mE√2mE
2x−2x
ψ(x)=Aeℏ+Beℏ
Thetrequirementhattheefunctionvawshouldbenormalizablerequiresthatlimψ(x)=0.eWcan
x→±∞
aluateevthisinowtparts.
√2mE√2mE
−2x2x
ifx<0Sincethefunctioneℏwsbloupforifx>0Sincethefunctioneℏwsbloupfor
x→−∞itrequiresthatB=0usththesolutionx→∞itrequiresthatA=0usththesolution
intheregionx<0becomesintheregionx<0becomes
√2mE√2mE
2x−2x
ψ(x)=Aeℏ(3.12)ψ(x)=Beℏ(3.13)
−+
Thetrequirementhattheevawfunctionustmbeuoustinconerywhereev(atx=0)requiresthat
limψ(x)=limψ(x),⇒A=B
−+
x→0x→0
tegratingIn(3.10)from−ϵtoϵandtakingthelimitϵ→0ewget
∫ϵ22∫ϵ∫ϵ
ℏdψ(x)dx+Eψ(x)dx=νδ(x)ψ(x)dx
2mdx20
[−ϵ]−ϵ−ϵ
ℏ2()
limψ′(ϵ)−ψ′(−ϵ)+Elim[ψ(ϵ)−ψ(−ϵ)]=−νψ(0)
ϵ→02m+−ϵ→0+−0
Bytheytunitcontrequirementhemiddletermsgoestozerobecauseatx=0ψ(x)=ψ(x)This
−+
simpliifesto
ℏ2[]
ψ′(0)−ψ′(0)=−νψ(0)
2m+−0
CHAPTER3.QUANTUMMECHANICS94
eWcantiatedifferen(3.13)ad(3.12)toobtain
ℏ2[√2mE√2mE]Aℏ2√2mE√2Eℏ2
2m−Aℏ2−Aℏ2=−ν0ψ(0),⇒ψ(0)=mν0ℏ2=Amν2
0
Sothecompletesolutionbecomes
√2mEx
ℏ2

Aeifx<0

√2Eℏ2
ψ(x)=Amν2ifx=0
√0
2mE
−x
ℏ2
Aeifx>0
ThenormalizationconditioncanbeusedtocalculatethealuevofA.
eWcanusetheyuttincontrequirementoaluateevtheedwalloenergy
limψ(x)=limψ(x)=ψ(0)
x→0−−x→0++√
2Eℏ2
⇒A=A=Amν2
√0
2Eℏ2
⇒1=mν2
0
mν2
⇒E=0
2ℏ2
mν2
0
Sothegroundstatebindingenergyis2ℏ2.Sincethisenergyistheonlyonethatsatisifesthescrodinger
equation,thereisnootherboundsate.energy□
Chapter4
MathematicalysicsPhII
4.1orkHomewOne
4.1.1.UsethegeneraldeifnitionandpropertiesofourierFtransformstowshothewingfollo
˜
(a)Iff(x)isperiodicwithperiodathenf(k)=0,unlesska=2πnfortegerinn.
Solution:
eWwknoybdeifnitionoffouriertransform
∫∞−ikt˜
F(f(t))=−∞f(t)edt=f(k)(Deiftion)
∫∞−ikt−ika˜
F(f(t−a))=−∞f(t−a)edt=ef(k)(Shiftingpropy)ert
Sincethefunctionisperiodicf(t)=f(t−a)andhenceF(f(t))=F(f(t−a)).So,
˜−ika˜−ika˜
f(k)=ef(k);⇒(e−1)f(ω)=0;
−ika
Eitherf(k)=0Ore=1;⇒ka=2πn.hWhiccompletestheproof.□
˜
(b)TheourierFtransformoftf(t)isdf(ω)/dω.
Solution:
d(˜)d∫∞−iωt∫∞∂(−iωt)∫∞−iωt
dωf(ω)=dω−∞f(t)edt=−∞∂ωf(t)edt=−∞itf(t)edt=−iF(tf(t))
˜
Sothefouriertransformoftf(t)isF(tf(t))=idf(ω)/dω.□
(c)TheourierFtransformoff(mt+c)is
iωc/m()
e˜ω
mfn
Solution:
p−c1−iωtiωc/m−iω/mp
Makingahangecofariablevmt+c=p;t=m;dt=mdpsoe=ee
∞∞∞
∫∫iωc/m∫iωc/m()
−iωtiωc/miωp1e−iω/mpe˜ω
F(f(mt+c))=f(mt+c)edt=f(p)eemdp=mf(p)edp=mfm
−∞−∞−∞
Sothefouriertransformoff(mt+c)iswnshoasrequired.□
95
CHAPTER4.TICALTHEMAMAPHYSICSII96
˜−1/2
4.1.2.Findthefouriersinetransformf(ω)ofofthefunctionf(t)=tandybtiatingdifferenwithrespect
toωifndthetialdifferenequationsatisifedybit.Hencewshothatthethesinetransformofthisfunction
isthefunctionitself.
Solution:∫
˜∞−1/2
Bydeifnitionofsinetransformf(ω)=0f(t)sin(ωt)dtewevhaforf(t)=t.
d˜d∫∞1∫∞∂(1)∫∞√
dω(f(ω))=dω√sin(ωt)dt=∂ω√sin(ωt)dt=tcos(ωt)dt
0t0t0
tegratingIntheRHSybpartsewget
∫[]
√∞∞√0assumed
✘✿
dsin(ωt)1sin(ωt)1✘✘1
˜√✘✘˜
(f(ω))=t−=lim✘tsin(ωt)−0−f(ω)
✘
dωωωωt→∞2ω
002t✘
Sothetialdifferenequationsatisifedybthesinetransformis
d(˜)1˜
dωf(ω)+2ωf(ω)=0
Thistialdiffernequationcanbeedsolvas:
˜∫˜∫()
df(ω)1˜df(ω)dω˜1˜−1/2
dω=−2ωf(ω);⇒˜=−2ω;⇒lnf(ω)=−2ln(ω)+lnA;⇒f(ω)=Aω
f(ω)
Butsincef(t)=t−1/2thealuevoff(ω)=w−1/2,sofromeabvoexpressionewget.
˜
f(ω)=Af(ω)
˜
Sinceewevhathesinetransformf(ω)=Af(ω)thesinetransformfothisengivfunctionisthefunction
itself.□
4.1.3.evProtheyequalit
∫∞∫∞2
−2at21a
esinatdt=44dω
0π04a+w
Solution:
ItcanbenoticedthattheLHSoftheengivyequalitisthesquaretegralinoffunctionf(t)=e−atsin(at)
from0to∞.Sincetheerwlolimitis0ewcanetakthefouriertransformofthisfunctionu(t)f(t)where
u(t)isthestepfunction
∞∞
˜∫−iωt∫−at−iωta
f(ω)=u(t)f(t)edt=esin(at)edt=22
a+(a+iω)
−∞0
Theabsolutealuevofthefouriertransfomofthefunctionis
2
˜aa
f(ω)==√
2244
a+(a+iω)4a+w
wNoybuseofal’sarsevPtheoremewevha
∞∞
∫2∫˜2
|u(t)f(t)|dt=|f(ω)|dω(al’sarsevPtheorem)
−∞−∞
CHAPTER4.TICALTHEMAMAPHYSICSII97
˜˜
Substitutingf(t)andf(ω)notingthatthefunctionf(ω)isenev
∞∞()∞
∫∫22∫4
−2at21√a1a
esin(at)dt=2π44dω=π4a4+w4dω
4a+w
0−∞0
Thiscompletestheproof.□
4.1.4.Bywritingf(x)asantegralinolvingvintheδ-function,δ(ξ−x)andtakingthelaplacetransformof
bothsideswshothatthetransformofthesolutionoftheequation
d4y
4−y=f(x)
dx
forhwhicyanditsifrstthreeesativderivanishvatx=0canbewrittenas
y˜(s)=∫∞f(ξ)e−sξdξ
4
0s−1
Solution:
Thefunctionf(x)canbewrittenasthetegralinofdeltafunctionsas
f(x)=∫∞δ(ξ−x)f(ξ)dξ
0
SotheLaplacetransformofthefunctionis
˜∫∞{∫∞}−sx∫∞{∫∞−sx}∫∞−sξ
f(s)=00δ(ξ−x)f(ξ)dξedx=00δ(ξ−x)edxf(ξ)dξ=0ef(ξ)dξ
akingTthelaplacetransformoftheengivtialdifferenequationewget
4˜∫∞−sξ∫∞e−sξ
sy˜(s)−y˜(s)=f(s)=0ef(ξ)dξ;⇒y˜(s)=0s4−1f(ξ)dξ
wNoforthesolutionthisfunctioncanbeexpressedastheproductofowtfunctionsas
1∫∞−sξ
y˜(s)=s4−1f(ξ)edξ
|{z}|0{z}
g˜(s)˜
f(s)
˜
Theersevinlaplacetransformoff(s)issimplyf(x)andthefouriertransformofg˜(s)canbeobtained
as
−1(1)−1(1[11])1
g(s)=L4=L2−2=[sinh(x)−sin(x)]
s−12s−1s+12
wNothelaplaceersevinoftheproductofthefunctionistheolutionvconofersesvinso
y(x)=f(x)∗g(x)=∫xf(ξ)g(x−ξ)dξ=1∫xf(ξ)[sinh(x−ξ)−sin(x−ξ)]dξ
020
hWhiccompletestheproof.□
CHAPTER4.TICALTHEMAMAPHYSICSII98
4.2orkHomewowT
4.2.1.eSolvthetialdifferenequationy′′−4y′+y=0;y(0)=0;y′(0)=0usingthelaplacetransformation.
Solution:
LetY(S)bethelaplacetransformationofy(x).akingTthelaplacetransformationofengivtialdifferen
equation
L{y′′−4y′+y}=L{0}
s2Y(s)−y(0)−sy′(0)−4sY(s)+4y(0)+Y(s)=0
Substutingtheengivinitialconditionsy′(0)=0;andy(0)=0esgiv
√
21113
(s−4s+1)Y(s)=0;⇒Y(s)=2=2=√√2
s−4s+1(S−2)−332
(S−2)−3
a
Thelaplacelaplacetransformisintheform22andthelaplaceersevinofthisexpressionis
(s−m)−a
{√}()
−11312t√
y(s)=L√2√2=√esinh3t
3(S−2)−33
Thisistherequiredsolutionforthetialdifferenequation.□
4.2.2.Usingtheolutionvcontheoremestablishthewingfolloresult:
−1{s2}1(1)
L222=tcos(at)+sin(at)
(s+a)2a
Solution:
Theengivexpressioncanbewrittenas
2
s=s·s
2222222
(s+a)s+as+a
ItiseasilyrecognizedthatheacpartistheLaplacetransformofcos(at).SotheersevinLaplace
transformybolutionvcontheoremistheolutionvconofcos(at)withitself
−1∫t
L{·}=cos(at)∗cos(at)=0cos(ax)cos(at−ax)dx
=∫tcos(ax)(cos(at)cos(ax)+sin(at)sin(ax))dx
0∫∫
t1t
=cos(at)cos2(ax)dx+sin(at)sin(2x)dx
020
[][]
2ax+sin(2ax)t1cos(2ax)t
=cos(at)4a0+2sin(at)−2a0
1212cos2(at)11sin(at)
=tcos(at)+sin(at)cos2(at)−sin(at)·+sin(at)+
2(4a)22a4a22a
=1tcos(at)+1sin(at)
2a
Thisistherequiredersevinfouriertransformfortheengivexpression.□
CHAPTER4.TICALTHEMAMAPHYSICSII99
1(2)∫∞cos(u)
4.2.3.wShothatL{Ci(t)}=−2sln1+swhereC(i)=−tudu(Thecosinetegral).in
Solution:
eWwknothetialdifferenundertegralinis
v(t)v(t)
d∫f(t,u)du=f(t,v(t))dv(t)−f(t,u(t))du(t)+∫∂f(u,t)du
dtdtdt∂t
u(t)u(t)
Consideringf(t,u)=cos(u),v(t)=R(asR−→∞)andu(t)=tewget
u
0R()0
∫✟✯
✓✼✟✟
dCi(t)cos(R)dRcos(t)dt∂cos(u)cos(t)
=✓−+lim✟✟du=−
dtRdttdtR→∞∂tut
✓✟✟
t
⇒tCi′(t)=cos(t)
akingTthelaplacetransformofbothsidesandwritingCI(s)≡L{Ci(t)}ewget
L{tCi′(t)}=L{cos(t)}
⇒−dL{Ci′(t)}=s
dss2+1
−d(sCI(S)−Ci(0))=s
dss2+1
−d(sCI(s))=s
dss2+1
Thisexpressionisanordinarytialdifferenequationhwhiccanbeedsolvas
−∫d(sCI(s))=∫ds
2
s+1
1(2)
⇒−sCI(s)=2lns+1
1(2)
CI(s)=−2slns+1
1(2)
ThisistherequiredLaplacetransformofCi(s)≡CI(s)=−2slns+1.□
4.2.4.Byperformingtherationalfractiondecomposition,establishthewingfolloresults:
−1{s+1}
(a)Ls(s2+1)=1+sin(t)−cos(t)
Solution:
Thepartialfractionof
s+1=1−s−1=1−s+1
2222
s(s+1)ss+1ss+1s+1
wNotheervinselaplacetransformis
−1{s+1}−1{1}−1{s}−1{1}
Ls(s2+1)=Ls−Ls2+1+Ls2+1
=1−cos(t)+sin(t)
hWhicistherequiredersevinlaplacetransform□
CHAPTER4.TICALTHEMAMAPHYSICSII100
−1{s+1}
(b)L22
s(s+s+1)
Solution:
Thepartialfractionofthisexpressionis
s+1=−1+1=1−1=1−1√
2222222
s(s+s+1)s+s+1ss(s+1/2)+1−1/4s(s+1/2)+(3/2)
Theersevinlaplacetransformis
−1{s+1}−1{1}−1{2√3/2}
L22=L2−L√√2
s(s+s+1)s3(s+1/2)+(3/2)
(√)
2−t/23
=t−√3esin2t
hWhicistherequiredersevintransform.□
4.3orkHomewThree
4.3.1.Acubemadeofmaterialwhoseyconductivitiskhasitssixfacestheplanesx=±a,y=±aand
z=±a,andtainsconnoternalinheatsources.erifyVthatthetemperaturedistribution()
(πx)(πz)2kπ2t
u(x,y,z)=Acossinexp−2
aaa
obeystheappropiratediffusionequation.crossAhwhicfacesisthereheatw?lfoWHatisthedirection()
3aa22
andtherateofheatwlfoattheptoin4,4,aattimet=a/(κπ)?
Solution:
Sincetheexpressionistheproductofusoidssinandexptials,onentheesativderivareeasytocalculate
andareybinspection
∂2uπ2∂2uπ2∂uκπ2
2=−2u;2=−2u;=−22
∂xa∂ya∂ta
kingChecthisonthediffusionequation,
∂2u∂2uπ21π21∂u
2+2=22=−22=
∂x∂zaκaκ∂t
clearlysatisifesit,wingShothisfunctionobeysthetemperaturediffucsionequation.Thedirectionof
heatwlfoisengivybthetgradienoffunction.tAt=a2;u=Acos(xπ/a)sin(zπ/a)e−2
κπ2
∂uˆ∂uˆπe−2ˆˆ
∇u=∂xi+∂zk=Aa(−sin(xπ/a)sin(zπ/a)i+cos(xπ/a)cos(zπ/a)k)
e−2πˆˆ
=Aa(−sin(π/4)sin(π)i+cos(π/4)cos(π)k)
e−2π(1ˆ)
=Aa−√2k
−2
Aeπˆ
Sotherateofheatwlfois√inthedirectionof−k□
a2
4.3.2.hroScdinger’sequationforannon=reativisticparticleinatconstanptialotenregioncanbeentakas
−ℏ2(∂2u+∂2u+∂2u)=iℏ∂u
2m∂x2∂y2∂z2∂t
CHAPTER4.TICALTHEMAMAPHYSICSII101
(a)Findasolution,separableinthefourindeptendenariables,vthatcanbewrittenintheformofa
planeevaw
()=exp((.−))
ψx,y,z,tAikrωt
UsingtherelationshipsassociatedwithdeBroglie(p=ℏk)andEinstein(E=ℏω),wshothatthe
separationtsconstanustmbehsucthat
222
p+p+p=2mE
xyz
Solution:
Letsassumethesolutionu(x,y,z,t)=XYZTwhereXispurelyfunctionofxonlyandsoon
withTbeingpurefunctionoft.SubstutingthisproductintheengivPDEewget
ℏ2
−2m(X′′YZT+XY′′ZT+XYZ′′T)=iℏXYZT′
WhereX′′andsoonaretotalsecondeativderivoftheironlyparameters,xandsoon.Dividing
thoroughybtheproductXYZTewobtain
−ℏ2(X′′+Y′′+Z′′)=iℏT′
2mXYZT
SinceewassumedthatheacX,Y,Z,andTareindeptendenofheacothertheonlyyawthefunction
ofindeptendenariablesvcanbeequalisiftheyareheacequaltoat.constanLetthetconstan
thatheacsideareequalbeE.Soewget.
−ℏ2(X′′+Y′′+Z′′)=iℏT′=E(SeparationtConstan)
2mXYZT
Solvingtheordinarytialdifferenequanintewget
T′EdTEEiωtE
−
=−i;⇒=−idt;⇒ln(T)=−it;⇒T=Te;Whereω=
TℏTℏℏ0ℏ
AlsotheLHSustmequalsametconstanso
(X′′+Y′′+Z′′)=−2mE
XYZℏ2
TheLHSofthisexpressionissumofthreeindeptendenfunctionsandtheRHSisatconstan
oidvofyanariablesvunderconsiderations.Theonlyyawthatcanhappenisifheacindeptenden
functionisatconstan
X′′=−k2;Y′′=−k2;Z′′=−k2
XxYyZz
Substutingthesekbacinthetialdifferenequationimplythattheyarerelatedybtheexpression
−k2−k2−k2=−2mE.Ifewwritep=ℏk,p=ℏk,andp=ℏk.Thenewget
xyzℏ2xxyyzz
222
p+p+p=2mE(4.1)
xyz
hEacODEinX,YandZareellwwnknoHarmonicoscillatortialdifferenequationsandthesolution
ofheacare
−ikx−iky−ikz
X=Xex;Y=Yey;Z=Zez(4.2)
000
WhereheacofX,YandZarets.constanbiningComalltheseinourifnalsolutionewget
000
−ikx−iky−ikz−iωt
u(x,y,z,t)=XYZT=Xex·Yey·Zez·Te
0000
−ikxx−ikyy−ikzz−iωt
=XYZTe
0000
CHAPTER4.TICALTHEMAMAPHYSICSII102
IfewwriteA=XYZT,k=kxˆ+kyˆ+kzˆandr=xxˆ+yyˆ+xzˆthenthesolutionestak
0000xyx
theform
−i(k·r−ωt)
u(x,y,z,t)=Ae(4.3)
hWhicistherequiredsolutionoftheengivhroScdinger’sequation.□
(b)Obtainatdifferenseparablesolutiondescribingaparticleconifnedtoabxoofsidea(ψstum
anishvattheallswofthebx).owShothattheenergyoftheparticlecanonlyetaktizedquan
aluesv
ℏ2π2()
E=n2+n2+n2
2ma2xyz
wherenx,ny,nzaretegers.in
Solution:
Ifthesolutionanishvattheallwofbxothenheacsolutionengivyb(4.2)shouldanishvatthe
all.wSothisiimplies
−ika−ika−ika
0=Xex0=Yey0=Zez
000
⇒kxa=πnxkya=πny⇒kza=πnz
⇒k=πnxk=πny⇒k=πnz
xayaza
Substutingthesealuesvin(4.1)ewget
()()()22
nxπ2nyπ2nzπ22mEℏπ(222)
−−−=−;⇒n+n+n=E
22xyz
aaaℏ2ma
hWhicistherequiredsolution□
4.3.3.ConsiderpossiblesolutionsofLaplaec’sequationinsideacirculardomainaswsfollo
(a)Findthesollutioninplanepolarcorrdinatesρ,ϕthatestakthealuev+1for0<ϕ<πandthe
aluev−1for−π<ϕ<0whereρ=a.
Solution:
ThegeneralsolutionfortheLaplace’sequationinplanepolarcoordinatesystem,wherethesolution
isifniteatρ=0is
u(ρ,ϕ)=D+∑(Cρn)(Acosnϕ+Bsinnϕ)
nnn
n
Sincetheengivboundaryconditionisanoddfunctionofphi,theenevfunctiontermintheeabvo
generalsolutionustmanishvso,D=0andAn=0.Theremaininggeneralsolutionis
u(ρ,ϕ)=∑ρn(Bnsinnϕ)
n
WhereCnisabsorbedinsideofBn□
222
(b)orFaptoin(x,y)onorinsidethecirclex+y=a,tifyidentheanglesαandβdeifnedyb
α=atan(y);andβ=atan(y)
a+xa−x
wShothatu(x,y)=(2/π)(α+β)isasolutionofLaplace’sequationthatsatisifestheboundary
conditionsengivin(4.3.3a).
CHAPTER4.TICALTHEMAMAPHYSICSII103
Solution:
Usingthetrigonometricytitidenofersevintstangenewget
22((y)(y))2(2y)
u(x,y)=(α+β)=atan+atan=atan222
ππa+xa−xπa−x−y
oTerifyvthatu(x,y)satisifestheLaplace’sequationewevhatowshothat∂2u+∂2u=0.
∂x2∂y2
Calculatingthisexpression
(22)((22)(22))
∂u2y−(a−x)+(a+x)∂u2(a−x)y+(a+x)+(a+x)y+(a−x)
∂x=(22)(22);∂y=(22)(22)
πy+(a−x)y+(a+x)πy+(a−x)y+(a+x)
Similarlythesecondpartialesativderivofheacis
∂2u4ay4ay4xy4xy
=()+()+()−()
∂x222222222
πy2+(a+x)πy2+(a−x)πy2+(a+x)πy2+(a−x)
2(422224224)
∂u=8ay−a−2ax−2ay+3x+2xy−y
∂y222222222
π(a−2ax+x+y)(a+2ax+x+y)
∂2u∂2u
Onadding∂x2and∂y2ewifndthatitisticallyidenzero.Soitsatisifesthelaplace’sequation.
222222222
Onattheboundarya=x+yandinsidetheboundarya>x+ysoa≥x+y.Onthe
boundary
2(2y)2π
u(x,y)=πatana2−x2−y2=πsgn(2y)2=sgn(y)
Wheresgn(x)isthesignfunction.Butonboundaryy=asinϕwhereaistheradiusandϕisthe
uthalazimangle.Thefunctionsinϕispeositivfor0<ϕ<πandenegativfor−π<ϕ<0,so
u(x,y)=sgn(y)=sgn(sinϕ)={10<ϕ<π
−1−π<ϕ<0
usThthefunctionsatisifesLaplace’sequationandalsotheboundarycondition.□
(c)DeduceaourierFseriesexpansionforthefunction
(sinϕ)(sinϕ)
atan1+cosϕ+atan1−cosϕ
Solution:
Againybtrigonometricytitiden
(sinϕ)(sinϕ)(2sinϕ)π{π/20<ϕ<π
f(ϕ)=atan1+cosϕ+atan1−cosϕ=atan22=2sgn(sinϕ)=−π/2−π<ϕ<0
1−sinϕ−cosϕ
Letthefourierseriesofthisfunctionf(ϕ)be
a∑
f(ϕ)=0+acosnϕ+bsinnϕ
2nn
n
Thisisaellwwnknoperiodicsquareevawfunction.Itisanoddfunctionsoan=0whosefourier
seriesisengivyb
πn
a=0;andb=(1−(−1))
nnn
CHAPTER4.TICALTHEMAMAPHYSICSII104
Sotherequiredfouerierseriesofthefunctionis
()()∞
sinϕsinϕ∑πn
f(ϕ)=atan1+cosϕ+atan1−cosϕ=n(1−(−1))sinnϕ
n=1
Thisistherequiredfourierseriesofthefunction.□
4.3.4.Aconductingsphericalshellofradiusacutrounditsequatorandtheowteshalvconnectedtooltagesv
+Vand−V.wShothatanexpressionfortheptialotenattheptoin(r,θ,ϕ)ywhereaninsidetheowt
hemispheresis
∞n()
u(r,θ,ϕ)=V∑(−1)(2n)!(4n+3)r2n+1P(cosθ)
2n+12n+1
n=02n!(n+1)!a
Solution:
orFthesphericalsplitsphericalshelltainedmainatowttialdifferenptials,otenlettheptialotenery-ev
whereinsidethesphericalshellbev.Sinceewwknoelectricifeldisengivyb2E=∇vandsincefor
Electricifeld∇·E=0.eWget∇·E=∇·∇v=∇v=0.SotheptialotensatisifestheLaplace’s
equation.Ifewsupposevasafunctionofr,θ,ϕinsphericalcoordinatesystem,thenthesolutionto
Laplace’sequationinsphericalcoordinatesystemisengivyb
∞
v(r,θ,ϕ)=∑(Arl+Br−(l+1))(Ccosmϕ+Dsinmϕ)(EPm(cosθ)+FQm(cosθ))
ll
l,m
mm
WhereQ(x)andP(x)aresolutiontotheassociatedLegendre’sequations.Andallothertsconstan
ll
aredeterminedybboundarycondition.
Sinceewevhaifniteptialotenatatthetercenofspherer=0,thecoteiffcienB=0.Alsosince
ewevhasphericalsymmetryandtheptialotenissinglealuedvfunctionm=0.Alsoewevhaifnite
m
ptialotenatpolesofspherehhiccwcorrespondtoθ={0,π}andQ(1)erges,divewevhaF=0.Also
l
P0(x)=P(x)whereP(x)arelegendrepolynomials.Owingtotheseboundaryconditionsthemost
lll
generalsolutionis
v(r,θ,ϕ)=∑ArlP(cosθ)(4.4)
ll
l
Sincethereisnoϕdependence,lettheptialotenatsurfacebedenotedybvahwhicisclearlyjutfunction
ofθ.
∑l
v(θ)=v(a,θ,ϕ)=AaP(cosθ)
all
l
IfewultiplymbothsidesybP(cosθ)andandtegrateinwithrespecttod(cosθ)from0to1usingthe
k∫
factthatLegendre’spolynomialsareorthogonal,PP=δewget.
klkl
∫1∫1()
∑l
v(θ)P(cosθ)d(cosθ)=AaP(cosθ)P(cosθ)d(cosθ)
akllk
00l
∑(∫1l)
=AaP(cosθ)P(cosθ)d(cosθ)
llk
l0
∑lk
=Alaδlk=Aka
l
SothecoteiffcienAkisengivyb
A=1∫1v(θ)P(cosθ)d(cosθ)(4.5)
kkak
a0
CHAPTER4.TICALTHEMAMAPHYSICSII105
TherecurrancerelationofLegendrepolynomialscanbeusedtoaluateevthetegralsinas
(2n+1)P=P′(x)−P′(x)(4.6)
nn+1n−1
tegratingIn(4.6)ewget,
∫P=1(P(x)−P(x))+K
n2n+1n+1n−x
SincetialotenPcanevhayanarbitraryreferenceewcanhocosethetegrationintconstantobeK=0.
Usingthisfactin(4.5)ewget
A=1(4.7)
kk
a(2n+1)
Asengivintheproblenontheupperhemispheretheptialotenis+Vandontheewlohemispherethe
ptialotenis−V,Itcanbemathematicallytedrepresenas{
Vif0<θ<π
v(θ)=2
a−Vifπ<θ<π
2
Substutingthisin(4.5)ewgetandwritingx=cosθ
A=1∫1VP(x)dx
kkk
a0
V1(1)
=[Pk+1(x)−Pk−1(x)]
k0
a2k+1
=V1(P(1)−P(1)−P(0)+P(0))
kk+1k−1k+1k−1
a2k+1
=V1(P(0)−P(0))
kk−1k+1
a2k+1
Since
n
(−1)(2n)!,nenev
P(0)=2n2
n2n!
0,otherwise
orFevnaluevofk,bothk−1andk+1areoddandhencePk−1(0)=0andPk−1(0)=0.oeFenevk,
A=V1(0−0−1+1)=0
kk
a2k+1
Butforoddaluevofk,k+1andk−1areen,evhencebothP(1)=P(1)=1andwriting
k=2n+1k−1k+1
(2n2(n+1))
Ak=V1(−1)(2(2n)!−(−1)(2(2(n+1))!
k2(2n)22(2(n+1))2
a4n+32(2n)!2(2(n+1))!
n
=(4n!)=V(−1)(2n)!(4n+3)
k2n+1
dena2n!(n+1)!
Usingthiscoteiffcienin(4.4)ewget
∞n()
v(r,θ,ϕ)=V∑(−1)(2n)!(4n+3)r2k+1P(cosθ)
22n+1n!(n+1)!a2n+1
n=0
hWhicistherequiredptialotenfunctioninsidethesphericalregion.□
CHAPTER4.TICALTHEMAMAPHYSICSII106
4.4orkHomewourF
4.4.1.AsliceofbiologicalmaterialofknessthicLisplacedtoinasolutionofaeradioactivisotopeoftconstan
trationconcenC0,attimet=0.orFalatertimetifndthetrationconcenoferadioactivionsatadepth
xinsideoneofitssurfacesifthediffusiontconstanisκ.
Solution:
Thediffusionequationwithdiffusiontconstanκis
∂2u1∂u
∂x2=κ∂t
Usingtheseparationofariablevhniquetecforthesolutionthesolutioncanbewrittenasu(x,t)=
X(x)T(t)whereXandTarepurefunctionsofxandt.respelyectivSubstitutingthissolutioninthe
solutionewget
X′′=1T′=−λ2
XκT
Thetconstanishosenctobeaenegativbumernsothattheexptialonensolutionisifniteatinifnite
time.Thetimepartofsolutionis
T′2∫dT∫22−κλ2t
T=−κλ;⇒T=−κλdt;⇒lnT=−κλt+K;⇒T(t)=De
X′′2
orFtheotherpartX=−κλhasthesolutionoftheform
(λ)(λ)
Asin√x+Bcos√x
κκ
Thegeneralsolutionthenbecomes
[(λ)(λ)]−λ2κt
u(x,t)=Asin√x+Bcos√xe
κκ
Aftertsuiffcientimehaspassedthetrationconcenthroughouttheslabshouldbethetrationconcenof
isotopesaroundit.Buttheeabvosolutiongoesto0att=∞.Sinceaddingatconstantotheeabvo
solutionisstillthesolutiontothediffusionequation.eWcanaddatconstantoemakitsatisfythis
condition.
Sincethetrationconcenistconstanatalltimesoneithersideoftheslab,u(0,t)=u(L,t)=C0andso
X(0)=X(L)=0.So
2
X(0)=Be−λκt=0;⇒B=0
()√
λλnπκ
X(L)=Asin√L=0;⇒√L=nπ;⇒λ=L
κκ
Usingtheseowtfactsewgetourgeneralsolutiontobe
∞()
∑22
nπ−nπκt
u(x,t)=C0+AnsinxeL2
n=1L
tAt=0thetrationconcenintheslabustmbe0.Sou(x,0)=0
∞()∞()
0=u(x,0)=C0+∑Ansinnπx;⇒−C0=∑Ansinnπx
n=1Ln=1L
CHAPTER4.TICALTHEMAMAPHYSICSII107
AgainthecotseiffcienAncanbecalculatedybusingthefactthat{sin(nx)}formanorthogonalset
n(mπ)
offunctionfortegerinsetofn.tegratingIntheeabvoexpressionybultiplyingmybsinLxonboth
sidesesgiv
∫L(mπ)∫L∑(nπ)(mπ)
0−C0sinLxdx=0AnsinLxsinLxdx
n
=∑A1δ=Am
nLmnL
n{}
∫L(mπ)21+(−1)m
⇒Am=L0−C0sinLxdx=−LC0mπL
Usingthisthegeneralsolutionbecomes
∑m()m2π2κt
u(x,t)=C0−2C01+(−1)sinmπxe−L2
πmmL
Thisesgivthetrationconcenoferadioactivisotopeinsidetheslabatalltimes.□
4.4.2.Determinetheelectrostaticptialoteninaninifnitecyindersplitwiselengthinfourpartsandhargedc
aswn.sho
Solution:
Becausethesidesofcylindricalareconductingtheptialotenistconstanforu(a,ϕ,z)whereaisthe
radiusofcylinder.Itwsfollothatforallz,u(ρ,ϕ)isthesame.Sotheptialotensatisifesplanepolar
formoflaplacesequationhwhichasthegeneralsolution
u(ρ,ϕ)=(C0lnρ+D0)∑(Ancosnϕ+Bnsinnϕ)(Cnρn+Dnρ−n)
n
Sinceewexpectifnitesolutionatρ=0,Dn=0otherwiseitρ−n=∞hwhicon’twsatisfyboundary
condition.BysimilartsargumenCn=0Alsosinceatρ=athesolutionisanoddfunctionhwhic
causesD0=0andAn=0.Thegeneralsolutionthatisleftis
u(ρ,ϕ)=∑Bρnsinnϕ
n
n
AgainthecotseiffcienBcanbecalculatedybusingthefactthat{sinnϕ}formanorthogonalsetof
nn
functionfortegerinsetofn.tegratingIntheeabvoexpressionybultiplyingmybsinmϕonbothsides
esgiv
∫2π∫2π∑n
u(a,ϕ)sinmϕdϕ=Anasinnϕsinmϕdϕ
00n
∑n∫2π
=Aasinnϕsinmϕdϕ
n
n0
∑n2πm
=Aaδ=Aaπ
n2mnm
n∫
12π
⇒Am=πam0u(a,ϕ)sinmϕdϕ
CHAPTER4.TICALTHEMAMAPHYSICSII108
Sinceintheengivproblemu(a,ϕ)hastdifferenaluesvfortdifferenϕewget
π/2π−π/22π
∫∫∫∫
Am=1Vsinmϕdϕ−Vsinmϕdϕ+Vsinmϕdϕ−Vsinmϕdϕ
πam
0π/2−π−π/2
{()m()m()()}
=V−1cosπm+1−(−1)+1cosπm(−1)−1cosπm1cosπm−1
πamm2mmm2mm2m2m
{m()}
=V2(−1)−4cosπm+2
πammm2m
{}
V1(mmm)
=−(−1)2((−1)+1)−(−1)+1
πamm
Sotheifnalsolutionbecomes
{}()
Vmmmρm
u(ρ,ϕ)=1−(−1)2((−1))−(−1)+1sin(mϕ)
mπa
Thisesgivtheptialotenerywhereevinsidethecylinder.□
4.4.3.Aheat-conductingcylindricalrodoflengthListhermallyulatedisnervoitslateralsurfaceandits
endsareeptkatzerotemperature.theinitialtemperatureoftherodisu(x)=u.usingthediffusion
0
equation
∂u∂2u
2
∂t=a∂x2
andtheboundaryconditionsu(0,t)=u(L,t)=0andtheinitialconditionu(x,0)=u,obtainthe
0
solutionu(x,t)oftheeabvoequation.
Solution:
Thegeneralsolutiontothediffusionequationis
22
u(x,t)=(Asin(λx)+Bcos(λx))e−λat
enGivinitialconditionu(0,t)=0
22
u(0,t)=e−λat(Bcos(λx))=0
Sincefunctionhastobe0atalltimestheonlyyawthiscanhappenforalltisB=0Alsotheother
boundaryconditionisu(L,t)=0esgiv
22
u(L,t)=e−λatAsin(λL)=0
SinceA=0willegivusthetrivialsolution0theonlyyawthisfunctioncangotozeroatalltimeis
sin(λL)=0hwhicimplies
sin(λL)=0;⇒λL=nπ;⇒λ=nπ;(n≥1)
L
Sincethesolutioncanbelinearbinationcomofallnsothesolutionis
∞()
∑22nπ
u(x,t)=Ane−λatsinLx
n=1
Butsincetheinitialconditionisthatthetemperatureoftherodisutobeginwith.Theeabvosolution
0
clearlygoestozeroatt=0andx=0.ddingAatconstantoasolutionoftialdifferenequationisstill
CHAPTER4.TICALTHEMAMAPHYSICSII109
aalidvsolution,tosatisfythisconditionewcanaddatconstanu.Thealidvgeneralsolutionthen
0
becomes
∞()
∑22nπ
u(x,t)=u0+Ane−λatsinλx
n=1
tAt=0thethesolutionreducesto
∞()∞()
u(x,0)=u+∑Asinnπx;⇒∑Asinnπx=−u
0nLnL0
n=1n=1
Sinesin(nx)formsanorthogonalsetoffunctionfortegerinsetofn.eWcanifndAnybtegratingin
eabvoexpressionultipliedmwithsinmx
∫l(mπ)∫l∑(nπ)(mπ)
0−u0sinLxdx=0AnsinLxsinLdx
n∫
∑l(nπ)(mπ)
=AnsinLsinLxdx
n0
=∑Alδ=lA
n2nm2m
n∫()
l()m
2umπ2u1−(−1)
⇒A=−0sinxdx=−0l
mlLlm
0
Usingthisinthesolutionewgettheifnalsolutionas
∞(m)()
∑1−(−1)mπ22
−λat
u(x,t)=u−2usine
00mL
m=1
Thisesgivthetemperatureasafunctionofpositionandtimeintheengivcylindrical.body□
4.4.4.Considerthesemi-inifniteheatconductingmediumdeifnedybtheregionx≥0,andarbitraryyandz.
Letitbeinitiallyatat0temperatureandletitssurfacex=0,evhaprescribedariationvoftemperature
u(0,t)=f(t)for(t≥0).wShothatthesolutionoftheeabvodiffusionequationcanbewrittenas
∫−x2
t2
xe4a(t−τ)
u(x,t)=√3/2f(τ)dτ
2aπ0(t−τ)
Solution:
Sincethetemperatureconductionofamaterialsatisifesthediffusionequation,thediffusionequation
canbewrittenas.
∂2u∂u
2
a∂x2=∂t
Sincetheparametersofthisproblemsaret→{0,∞}andx→{0,∞},ewcanetakthelaplacetransform
oftheequationwithrespecttotheariablevthwhicresultsin
∞∞
∫2∂2−st∫∂−st
a∂x2u(x,t)edt=∂tu(x,t)edt
00
∞∞
2∫∫
d−st1∂−st
2u(x,t)edt=2u(x,t)edt
dxa∂t
00
CHAPTER4.TICALTHEMAMAPHYSICSII110
Assumingu(x,t)=g(t),theRHSofeabvoexpressionisthelaplacetransformofeativderivofg(t)
hwhicissG(t)−g(0)hwhiccanbewrittenas
2
dU(x,s)=1(sU(x,s)−u(x,0))
dx2a2
Thetermu(x,0)istheinitialtemperatureofthematerialbodyunderconstruction,sincethebodyis
initiallyat0temperatureu(x,0)=0,usingthisandrearrangingesgiv
2
dU(x,s)−sU(x,s)=0
22
dxa
ThisisaeryvellwwnknosecondorderOrdinarytialDifferenequationwhosesolutionisoftheform
√√
−xs/axs/a
U(x,s)=Ae+Be
Butsincethematerialbodyisinifnitelylonginx≥0thesolutionisifniteatx=∞hwhicimplies
thatB=0.Alsoatthenearendofthematerialx=0thetemperatureu(0,t)=f(t)isen.givThe
laplacetransformofhwhicisU(0,s)=F(s).So
0
U(0,s)=Ae;⇒A=U(0,s)=F(s)
Thisreducesthesolutionintheform
√
U(x,s)=F(s)e−xs/a
tAthisptointhesolutionu(x,t)istheersevinlaplacetransformofU(x,s).Iftheexpressionisentak
√
−xs/a
asproductofF(s)andethesolutionistheolutionvconofersesvinofthese.
Lookingattheresultewexpect,theersevinlaplacetransfromustmtbe
{}−x2
√2
−1−xs/axe4at
Le=√3
2πat2
Iedkheccthisinysympandgotthewingfollo
SotheersevinlaplacetransformofU(x,s)is
∞x2
{√}∫−2
−1−1−xs/axe4a(t−τ)
u(x,t)=L(F(s))∗Le=√3/2f(τ)dτ
2πa(t−τ)
0
Sincethetegrationiniswithrespecttoτtheariablevxistconsanfortegrationinhwhicleadsto
∫−x2
t2
xe4a(t−τ)
u(x,t)=√3/2f(τ)dτ
2aπ0(t−τ)
hWhicistherequiredsolutionoftheheatequation.□
4.5orkHomeweFiv
4.5.1.Astringoflengthlisinitiallyhedstretct,straignitsendsareifxedforallt.tAt=0,itsptsoinare
engivtheeloyvcitv(x)=(∂y)swnshointhediagram.Determinetheshapeofstringattimet,
∂tt=0
thatis,ifndthetdisplacemenasafunctionofxandtintheformofaseries.
CHAPTER4.TICALTHEMAMAPHYSICSII111
v(x)
0
h
ll
2
Solution:
Themotionofthestringisguidedybtheevawequationhwhiccanbewrittenas
∂2y1∂2y
∂x2=c2∂t2
Ifewsupposethesolutiony(x,t)=X(x)T(t)thensubstutingtheseanddividingthourghybXTew
obtain
X′′=1T′′
2
XcT
Theeabvosolutioniscomposedofowtparts,heacfunctionofindeptendenariables,vtheonlyyawthey
canbeequalisiftheyareequaltot,constanletthetconstanthattheyareequalbek2.
X′′=k2;⇒X=Asin(kx)+Bcos(kx)
X
1T′′=k2;⇒T=Dsin(kct)+Ecos(kct)
c2T
Sothesolutiontothetialdifferenequationbecomes,
u(x,t)=[Asin(kx)+Bcos(kx)][Dcos(kcx)+Esin(kcx)]
Butsincethestringisstationaryatbothends.tAx=0andx=L
0=Bcos(kx)[Dcos(kct)+Esin(kct)]
TheonlyyawitcanbezeroforalltisifB=0.Andalsosincethestringhasnotdisplacementobegin
withu(x,0)=0.TheonlyyawthiscanhappensimilarlyisifD=0.Thesolutionthenbecomes
u(x,t)=Asin(kx)sin(kct)
CHAPTER4.TICALTHEMAMAPHYSICSII112
Alsosinceu(l,t)=0forallt,theonlyyawthiscanhappenisifk=nπ.Sinceewevhatdifferen
l
possiblealuesvofnforsolution,thelinearbinationcomofallwillbethemostgeneralsolution
∑(nπ)(nπ)
u(x,t)=Ansinlxsinlct
n
Theeloyvcitofthestringathebeginingis
′∑nπc(nπ)(nπ)(∂y)′∑nπc(nπ)
u(x,t)=Anlsinlxcoslct;⇒∂tt=0=u(x,0)=lsinlx
nn
ThecotseiffcienAcanbefoundybusualourier“Fk”ricTas
n
2∫l′(nπ)
An=nπcu(x,0)sinLxdx
0
SincetheengiveloyvcitfunctionisowtpartfunctionewobtainAnas
2[∫l/2′(nπ)∫l′(nπ)]
A=u(x,0)sinxdx+u(x,0)sinxdx
nnπcll
0l/2
2[∫l/22h(nπ)∫l2h(nπ)]
=nπc0lxsinLxdx+l/2−l(x−l)sinlxdx
=8hlsin(πn)
33
πcn2
Substitingthiskbactointhesolutionewevha
∞()()()
u(x,t)=∑8hlsinπnsinnπxsinnπct
π3cn32ll
n=0
Thisesgivthepositionoferyevptoininthestringasafunctionoftime.□
4.5.2.Considerthesemi-inifniteregiony>0.orFx>0,thesurfacey=0istainedmainatatemperature
Te−x/l,fox<0.Thesurfacey=0isinsulated,sothatnoheadwslfooutorin.Findtheequilibrium
0
temperatureatptoin(−l,0)
Solution:
Thegeneraltialdifferenequationofthetemperaturediffusionis
∂2T∂2T1∂
∂x2+∂y2=κ∂tT
.Butsinceatequilibriumtheterm∂T=0.Thegeneraltialdifferenequationbecomes
∂t
∂2∂2
∂x2T(x,y)+∂y2T(x,y)=0(4.8)
ThetemperatureofthesystemT(x,y)shouldgotozeroas∂y→∞.Alsosincethesurfacey=0is

insulatedforx<0.Theheatwlfoatforx<0is∂tT(x,y)0−=0.Soybytunitconofthefunctionat
y=0,therateofhangecoftemperaturewithyaty=0+shouldequalzero,so∂T(x,0)=0.Sothe
∂y
eeffectivboundaryconditionbecomes
{−x/l{
T0e(x>0)∂?(x>0)
T(x,0)=f(x)=T(x,y)=g(x)=
?(x<0)∂y0(x<0)
y=0
CHAPTER4.TICALTHEMAMAPHYSICSII113
akingTthefouriertransformof(4.8)withrespecttotheariablevxewget
∫∞2∫∞22
∂ikx∂ikx2˜d˜
−∞∂x2T(x,y)edx+−∞∂y2T(x,y)edx;⇒−kT(k,y)+dy2T(k,y)=0
˜
WhereT(K,y)istheourierFtransformofT(x,y)inariablevx.Sincethisisaellwwnknotialdifferen
equationwhosesolutioncanbewrittenas
˜−ky
T(k,y)=Φ(k)e
WhereΦ(k)isanwnunknofunctiontobedeterminedybtheboundaryconditions.Therequired
solutionistheersevinourierFtransformisexpression
−1˜1∫∞ikx−ky
T(x,y)=F(T(k,y))=2π−∞eeΦ(k)dk(4.9)
Sinceewwknotheariousvpartsaty=0substutingtheevanofunctionfory=0esgiv
1∫∞ikx∫∞−ikx
T(x,0)=f(x)=2π−∞eΦ(x)dx;⇒Φ(x)=−∞f(x)edx(4.10)
√22
eWwillsubstituekybk+λsothatoursolutionwillbeinthelimitλ→0..tiatingDifferenthe
functionwithrespecttoyandsettingy=0ewget
∂1∫∞∂√221∫∞√√22
∂yT(x,y)=g(x)=2π∂ye−k+λyΦ(k)dx=2π−k2+λ2e−k+λyΦ(k)dx
−∞−∞
Settingy=0ineabvoexpressionandtakingfourierersevintransformofbothsidesesgiv
√k2+λ2Φ(x)=∫∞g(x)e−ikxdx(4.11)
−∞
eWcanesolv(4.10)and(4.11)withtdifferenpartsofwnknof(x)andg(x).romF(4.10)ewget
∫0−ikx∫∞−ikx∫∞−x/l−ikx
Φ(x)=f(x)edx+f(x)edx=Φ(x)+Teedx(4.12)
−∞0−00
wnUnknofunction|{z}T1
=Φ(x)+0(4.13)
−ik−i/l
Similarlysolving(4.11)ewget
√22
k+λΦ(x)=Ψ+(x)+0(4.14)
ormF(4.12)and(4.14)ewget
√√22
Ψ(x)=k2+λ2Φ+T0k+λ
+−xik−i/l
Dividingyb√k−iλonbothsidesewget
√Ψ(k)T√k+iλ
k+iλΦ(k)−√+=0(4.15)
−k−iλik−i/l
Thiesimpliifcationthisexpressionifnallyesgiv
[−x−iy(√x+iy)]
T(x,y)=TRee1−erf−
0l
CHAPTER4.TICALTHEMAMAPHYSICSII114
Thisisthesolutionforthetemperatureerywhereevintherod.tA(−l,0)ewget
l
T(−l,0)=T0e(1−erf(1))
Thisesgivthetemperatureattherequiredpt.oin□
∫1P(0)−P(0)
4.5.3.(a)DeducetherelationP′−P′=(2l+1)PandwshothatP(x)=l−1l+1;(l≥
l+1l−1ll2l+1
0
1)
Solution:
ThegeneratingpolynomialoftheLegendrepolynomialis
G(x,t)=√1=∑P(x)tn
1−2xt+t2n
n
akingTthepartialeativderivofbothsidesoftheexpressionwithrespecttoariablevxewget
∂(√1)=∂∑Pn(x)tn⇒t=∑P′(x)tn
∂x1−2xt+t2∂x(1−2xt+t2)3/2n
nn
Thiscanbepulatedtoget
∞∞
(1−2xt+t2)∑P′(x)tn−t∑P(x)tn=0
nn
n=0n=0
∞∞∞∞
∑′n∑′n∑n+2∑n+1
P(x)t−2xP(x)t+P(x)t=P(x)t
nnnn
n=0n=0n=0n=0
Comparingthecoteiffcienoftn+1onbothsidesewget
P′(x)−2xP′(x)+P′(x)=P(x)(4.16)
n+1nn−1n
Againifewtiatedifferenthegeneratingfunctionwithrespecttotandcomparecots,eiffcienewget
(2n+1)xP(x)=(n+1)P(x)+nP(x)(4.17)
nn+1n−1
Ifewtiatedifferentherecurrencerelation(4.17)ewget
(2n+1)P(x)+(2n+1)xP′(x)=(n+1)P(x)+nP(x)(4.18)
nnn+1n−1
Alsoifewultiplym(4.16)yb(2n+1)ewget
(2n+1)P′(x)−2(2n+1)xP′(x)+(2n+1)P′(x)=(2n+1)Pn(x)(4.19)
n+1nn−1
Ifewsubtract(4.18)from(4.19)ewget
(2n+1)Pn=P′(x)−P′(x)
n+1n−1
.Thisesgivtherequiredexpression.Thetegralincanbewnowrittenas
∫∫[][]
11P′−P′(x)P−P(x)1P(1)−P(1)−P(0)+P(0)
Pl(x)dx=l+1l−1dx=l+1l−1=l+1l−1l+1l−1
002l+12l+102l+1
SinceP(1)=1foreryevntheexpressionsimpliifesto,andsincethereisP(1)thiswillbe1
nl−1
onlyifl≥1,hwhicwsalloustowrite,
∫1P(0)−P(0)
P(x)=l−1l+1;(l≥1)
l2l+1
0
ThisisterequiredtegralinoftheLegendrepolynomialintheengivrange.□
CHAPTER4.TICALTHEMAMAPHYSICSII115
∫lP(0)
(b)wShothatP()=l−1;(≥.
lxdxl+1l1)
0
Solution:
romF(4.5.3a)ewcanwrite
∫1P(0)−P(0)
P(x)=l−1l+1;(l≥1)
l2l+1
0
eWcantheuse(4.17)toaluateevPl+1(0)hwhicesgiv
(n+1)P(x)=(2n+1)xP(x)−nP;P(0)=−lP(0)
n+1nn−1l+1l+1l−1
Substutingthiskbacewget
∫1P(0)−P(0)1[l]1
P(x)=l−1l+1=P(0)+P=P(0)
0l2l+12n+1l−1l+1l−1l+1l−1
hWhicesgivtherequiredexpressionforthetegral.in□
4.5.4.Ahargec+2qissituatedattheoriginandhargescof−qaresituatedatdistances±afromitalong
thepolaraxis.ByrelatingittothegeneratingfunctionfortheLegendrepolynomials,wshothatthe
electrostaticptialotenΦataptoin(r,θ,ϕ)wither>aisengivyb
∞()
Φ(r,θ,ϕ)=2q∑a2sP(cosθ).
4πϵrr2s
s=1
Solution:
LetPbeaageneralptoinwithcoordinate(r,θ)inaparticularplane.Sincetheptialotenonlydepends
uponrandθandthereisnoϕdependence,ewcancancluateitforaplanepolarcase,hwhicorkswfor
sphericalpolarasell.w
P
r2r
r1
−q2qθ−q
aa
Usingthecosinew,lathetdifferentitiesquanintheengivdiagramcanbewrittenas
()()
222r12aa2
r=r−2racosθ+a;⇒=1−2cosθ+
1rrr
Sincethegeneratingfunctionoflegendrepolynomialis
∞
√12=∑Pn(x)tn
1−2xt+tn=0
Ifewleta=tandcosθ=xewget
r
∞()
1=1√1()2=1∑Pn(cosθ)an(4.20)
r1r1−2acosθ+arn=0r
rr
CHAPTER4.TICALTHEMAMAPHYSICSII116
Similarlyfromrfromthediagram
2
()()
222r12aa2
r2=r−2racos(π−θ)+a;⇒r=1+2rcosθ+r
Similarlyfromeabvoexpressionewget
∞()
1=1√1()2=1∑Pn(−cosθ)an(4.21)
r2r1+2acosθ+arn=0r
rr
TheptialotenyanptoinPthenbecomes
2qqqq[11]
V(r,θ,ϕ)=4πϵr−4πϵr−4πϵr=4πϵr2−r−r
00102012
Substutingrandrfrom(4.20)and(4.21)ewget
12
[∞()∞()]
V(r,θ,ϕ)=q2−∑P(cosθ)an−∑P(−cosθ)an
4πϵrnrnr
0n=0n=0
n
SinceP(x)=(−1)P(−x)ewget
nn
[∞()∞()]
q∑an∑nan
V(r,θ,ϕ)=2−P(cosθ)−(−1)P(cosθ)
4πϵrnrnr
0n=0n=0
hWhiccanbewrittenas
[∞n()]
V(r,θ,ϕ)=2q1−∑1+(−1)Pn(cosθ)an
4πϵ0rn=02r
SinceP(x)=1forallxewcansimplifytheexpression
0
[∞n()]
V(r,θ,ϕ)=2q∑1+(−1)Pn(cosθ)an
4πϵ0rn=12r
n
Sincetheexpression1+(−1)=0foroddnecwcanwrite
2
∞()
V(r,θ,ϕ)=2q∑P(cosθ)a2s
4πϵr2sr
0s=1
hWhicistherequiredexpressionoftheptial.oten□
4.6orkHomewSix
4.6.1.Astringifxedatbothendsandoflengthlhasazeroinitialeloyvcitandaninitial
tdisplacemenaswnshointheifgure.Findthetsubsquesntdisplacemenofthestringy(x)
0
asafunctionofxandt.h
Solution:
Themotionofthestringisguidedybtheevawequationhwhiccanbewrittenaslll
42
∂2y1∂2y
∂x2=c2∂t2
CHAPTER4.TICALTHEMAMAPHYSICSII117
Ifewsupposethesolutiony(x,t)=X(x)T(t)thensubstutingtheseanddividingthourghybXTew
obtain
X′′=1T′′
2
XcT
Theeabvosolutioniscomposedofowtparts,heacfunctionofindeptendenariables,vtheonlyyawthey
canbeequalisiftheyareequaltot,constanletthetconstanthattheyareequalbek2.
X′′=k2;⇒X=Asin(kx)+Bcos(kx)
X
1T′′=k2;⇒T=Dsin(kct)+Ecos(kct)
c2T
Sothesolutiontothetialdifferenequationbecomes,
u(x,t)=[Asin(kx)+Bcos(kx)][Dcos(kcx)+Esin(kcx)](4.22)
Butsincethestringisstationaryatbothends.tAx=0andx=L
0=Bcos(kx)[Dcos(kct)+Esin(kct)]
TheonlyyawitcanbezeroforalltisifB=0.SubstutingB=0in(4.22)andtiatingdifferenwith
respecttot.
∂′
∂tu(x,t)=Asin(kx)[kc(−Dsin(kcx)+Ecos(kcx))];⇒u(x,0)=0=Asin(kx)[Ekc]
TheonlyyawtheeabvoexpressioncanbezeroforallxisifE=0.Alsosinceu(l,t)=0forallt,the
onlyyawthiscanhappenisifk=nπ.Sinceewevhatdifferenpossiblealuesvofnforsolution,the
l
linearbinationcomofallwillbethemostgeneralsolution
∑(nπ)(nπ)
u(x,t)=Ansinlxcoslct
n
Theshapeofthestringathebeginingisengivasapartfunction.Sothefunctionatt=0thenbecomes
∑(nπ)
u(x,0)=Ansinlx
n
ThecotseiffcienAcanbefoundybusualourier“Fk”ricTas
n
2∫l(nπ)
An=lu(x,0)sinlxdx
0
SincetheengiveloyvcitfunctionisowtpartfunctionewobtainAnas
2[∫l/4(nπ)∫l/2′(nπ)∫l(nπ)]
An=lu(x,0)sinlxdx+u(x,0)sinlxdx+u(x,0)sinlxdx
0l/4l/2
2[∫l/42h(nπ)∫l/24h(l)(nπ)]
=l0lxsinLxdx+l/4−lx−2sinlxdx
8h((πn)(πn))
=π2n22sin4−sin2
Substitingthiskbactointhesolutionewevha
∞(()())()()
u(x,t)=∑8h2sinπn−sinπnsinnπxsinnπct
π2n242ll
n=0
CHAPTER4.TICALTHEMAMAPHYSICSII118
Thisesgivthepositionoferyevptoininthestringasafunctionoftime.□
4.6.2.ARLCcircuithasthehargecstoredincapacitorqhwhicsatisifesthetialdifferenequationas
2R
Ldq+Rdq+q=0
dt2dtC
Ifthehargecinthecapacitorattimet=0isq(0)=qifndthehargecasafunctionoftime.CiL
0
Solution:
Letusassumethatthelaplacetransformofq(t)isL{q(t)}=Q(s).akingTlaplacetransformonbothS
sidesewget
{2′}1
LsQ(s)−sq(0)−q(0)+R{sQ(s)−q(0)}+CQ(s)=0
(21)′
Ls+Rs+CQ(s)−(Ls+R)q(0)−Lq(0)=0
(′R)′R
Ls+q(0)+Lsq(0)+
Q(s)=()=()+(L)
2R12R12R1
Ls+Ls+LCs+Ls+LCs+Ls+LC
Sincethereishargecinthecapacitor.Theinitialrateofhargediscofcapacitorisq′(0)=R.The
L
denominatorcanbewrittenasacompletesquaresumandtheexpressionbecomes
q′(0)+R
Q(s)=s+L
()()()()
R21R2R21R2
s+2L+LC−4L2s+2L+LC−4L2
ritingWR=αand(1−R2)=ω2ewget
2
2LLC4L
Q(s)=s+α+αω
(s+α)2+ω2ω(s+α)2+ω2
Theersevinlapalacetransformesgiv
−αt(α)
q(t)=esin(ωt)+ωcos(ωt)
Thisistherequiredhargecasafunctionoftimeinthecapacitor.□
∂2∂−α|x|
4.6.3.eSolvthediffusionequation∂x2q(x,t)=∂tq(x,t)fortheinitialcondition.q(x,0)=n0e
Solution:
∂2q(x,t)∂q(x,t)
∂x2=∂t
akingTfouriertransforminariablevxonbothsides
∫∞∂2q(x,t)=∫∞∂q(x,t);⇒−k2Q(k,t)=dQ(k,t)
−∞∂x2−∞∂tdt
Thisisaifrstordertialdifferenequationinthwhicchasasolution
−k2t
Q(k,t)=A0e
CHAPTER4.TICALTHEMAMAPHYSICSII119
wNoengivtheboundaryconditionq(x,0)=e−α|x|ewcancalculatethettanconAyb
0
Q(k,0)=F(q(x,0))=∫∞e−α|x|e−ikxdx
−∞
∫0αx−ikx∫∞−αx−ikx
=eedx+eedx
−∞0
=∫0e(α−ik)xdx+∫∞e−(α+ik)xdx
−∞0
0∞
nα−ikn0−x(α−ik)
=e−e
α−ikx(α+ik)
nn−∞0
=0+0
α−ikα+ik
=2α
α2+k2
2nα
SothesolutioninkspacebecomesQ(k,t)=0.Thisthencanbeusedtocalculatethesolutionas
α2+k2
(2)∫∞2∫∞ikx−k2t
−12n0α−kt12n0αikx−ktn0αe
q(x,t)=Fα2+k2e=2π−∞α2+k2eedk=π−∞α2+k2dk
Thisesgivthegemeralsolution.□
4.7orkHomewenSev
4.7.1.wShothat
∞l(′l)
1=∑∑r4πYm(θ,ϕ)∗Ym(θ′,ϕ′)
|r′−r|rl+12l+1ll
l=0m=−l
Solution:
Lettheangleeenbwettheectorsvr′andrbeγ.Alsolet|r−r′|=r1.Thenybcosinewlaewwevha
r2=r′2−2r′rcosγ+r2;
1
hWhiccanberearrangedtoget
()∞()
111r′r′−1/21∑r′l
==−2cosγ+1=P(cosγ)
|r−r′|rrrrrlr
1l=0
romFthesphericalharmonicsadditiontheoremewcanwrite
l
Pl(cosγ)=4π∑Ym(θ,ϕ)∗Ym(θ′,ϕ′)
2l+1ll
m=−l
Substutingthisintheeabvoexpressionewget
∞(′)ll
1=1∑r4π∑Ym(θ,ϕ)∗Ym(θ′,ϕ′)
|r−r′|rr2l+1ll
l=0()m=−l
∞l′l
=∑∑r4πYm(θ,ϕ)∗Ym(θ′,ϕ′)
rl+12l+1ll
l=0m=−l
CHAPTER4.TICALTHEMAMAPHYSICSII120
Clearlythisseriesersesvcononlyifr>r′ifinsteadr′>rinthetheexpressioncanberewrittenas
()∞()
1′=1=11−2rcosγ+r−1/2=1∑Pl(cosγ)rl
|r−r|r1r′r′r′r′l=0r′
Usingthesphericalharmonicsadditionrelationleadsto
∞()l
1=1∑rl4π∑Ym(θ,ϕ)∗Ym(θ′,ϕ′)
|r−r′|r′r′2l+1ll
l=0()m=−l
∞ll
=∑∑r4πYm(θ,ϕ)∗Ym(θ′,ϕ′)
r′l+12l+1ll
l=0m=−l
Thesearetherequiredexpressions□
4.7.2.Byhocosingasuitableformforhinthegeneratingfunction
[()]∞
z1∑n
G(z,h)=exph−=J(z)h
2hn
n=−∞
wshothatthetegralintationrepresenofthebesselfunctionsoftheifrstkindareen,givfortegralinm
yb
m∫2π
J(z)=(−1)cos(zcosθ)cos(2mθ)dθ;m≥1,
2m2π
0
m∫2π
J(z)=(−1)sin(zcosθ)cos((2m+1)θ)dθm≥0.
2m+12π
0
Solution:
iθiθ−iθ
Leth=ie.Withthishoicecofhewgeth−1/h=ie+ie=2icosθ.Thissiimpliifesthe
generatingfunctiontegralinto
∞
izcosθ∑(iθ)n
e=J(z)ie
n
n=−∞
∞
cos(zcosθ)+isin(zcosθ)=∑J(z)in(cosθ+isinθ)n
n
n=−∞
∞
∑nn+1
=iJ(z)cosnθ+iJ(z)sinnθ
nn
n=−∞
Sinceinisrealforenevnandin+1isrealforoddn.TherealpartoftheexpressiononRHSis
∞∞
∑2m2m+2∑mm+1
Jicos(2mθ)+Jisin((2m+1)θ)=J(−1)cos(2mθ)+J(−1)sin((2m+1)θ)
2m2m+12m2m+1
m=−∞m=−∞
usThequatingrealpartonbothsidesesgiv
∞
∑mm+1
cos(zcosθ)=J(−1)cos(2mθ)+J(−1)sin((2m+1)θ)
2m2m+1
m=−∞
CHAPTER4.TICALTHEMAMAPHYSICSII121
Sinceewwknothattheset{sinnθ}and{cosnθ}formorthogonalsetoffunctionsewcanifndthe
nn
expressionJ2mybusualourier“Fk”ricTas
2π()
∫∫2π∞
∑mm+1
cos(zcosθ)cos(2rθ)dθ=J(−1)cos(2rθ)+J(−1)sin((2m+1)θ)cos2rθdθ
2m2m+1
0m=−∞
0
2π2π
∞∫∫
∑mm+1
=J(−1)cos(2mθ)cos2rθdθ+J(−1)sin((2m+1)θ)cos2rθdθ
2m2m+1
m=−∞00
∞
∑m
=(−1)J2m2πδmr+0
m=−∞
r
=(−1)J(z)2π
2r
1r
Rearraningtheexpressionesgivsince(−1)r=(−1)
2π
r∫
J2r(z)=(−1)cos(zcosθ)cos(2rθ)dθ
2π
0
Similarlyequatingtheimaginarypartesgiv
∞
∑mm+1
zθJ(−1)sin(2mθ)+J(−1)cos((2m+1)θ)
sin(cos)=2m2m+1
m=−∞
Theusualyorthogonalitesgiv
2π
r∫
J(z)=(−1)sin(zcosθ)cos((2r+1)θ)dθ
2r+12π
0
Theseconcludesthet.rquiremen□
4.7.3.FindtheptialotendistributioninawholloconductingcylinderofradiusRandlengthl.Theowtends
areclosedybconductingplates.Oneendoftheplateandthecylindricalallwareheldatptialoten
Φ=0.TheotherendplateisinsulatedformtheylindreandheldateptialotenΦ=ϕ0
Solution:
Sincethereisnohargecsourceinsidethecylinder,theptialoteninahargelesscregionwsfollothe
2
lapalcesequation∇Φ=0.Usingtheusualculindricalcoordinatesystemfortheproblemthegeneral
solutionoftheLaplacesequationincylindricalsolutionisengivyb
[−kzkz]
Φ(ρ,ϕ,z)=[AJ(kρ)+BY(kρ)][Ccosmϕ+Dsinmϕ]Ee+Fe
mm
Sincetheptialotenisifniteatρ=0attheaxisofcylinder,thecoteiffcienB=0becauseY(0)=−∞.
m
Sincetheptialotenisifniteinthatregionthathastobethecase.Alsosincethereisuthalazimsymmetry
thealuevofm=0.Thegeneralsolutionthenbecomes
[−kz−kz]
Φ(ρ,ϕ,z)=AJ(kρ)Ee+Fe
0
Sincetheptialotenis0atz=0inthebottomendofcylinder.E+F=0;E=−F.Absorbing
2FtoinAewget
Φ(ρ,ϕ,z)=AJ0(kρ)sinh(kz)
Alsoattheallwofthecylinderρ=atheptialoteniszeroso
0=Φ(a,ϕ,z)=AJ(ka)sinh(kz)
0
CHAPTER4.TICALTHEMAMAPHYSICSII122
TheonlyyawthisexpressioncanbezeroforallzisifJ(ka)=0.hWhicmeanskashouldbethezero
0∞
ofbesselfunction.Sincethereareinifnitezerosofbesselfunctionsletthembedenotedyb{α}.
ii=0
Thismeanska=α;⇒k=αiSothegeneralsolutionbecomes
iia
∞()()
∑αα
Φ(ρ,ϕ,z)=AJiρsinhiz
i0aa
i=0
ThecoteiffcienAiisengivyb
2∫a(α)
A=()ρΦ(ρ,ϕ,l)Jiρdρ
iJ2(α)sinhαil0a
1ia0
SinceΦ(ρ,ϕ,l)=ϕthistegralinisbecomes
0
2ϕ∫a(α)
A=0()ρJiρdρ
iJ2(α)sinhαil0a
1ia0
2ϕ[J(α)]
=0()1i
J2(α)sinhαi)lαi
1ia
=2ϕ0()
αJ(α)sinhαil
i1ia
Substutingthiskbacesgivtherequiredgeneralsolution
∞()()
∑2ϕαα
Φ(ρ,ϕ,z)=0()Jiρsinhiz
αJ(α)sinhαil0aa
i=0i1ia
Thisesgivtheptialotenerywhereevinsidethecylinder.□
4.7.4.wShofromitsdeifnition,thattheBesselfunctionofsecondkind,andoftegerinorderνcanbewritten
as
1[∂Jµ(z)ν∂J−µ(z)]
Yν(z)=π∂µ−(−1)∂µµ=ν
UsingtheexplicitseriesexpressionforJ(z),wshothat∂J(z)/∂µcanbewrittenas
µµ
Jν(z)ln(z)+g(ν,z)
2
anddeducethatY(z)canbeexpressedas
ν
Y(z)=2J(z)ln(z)+h(ν,z)
νπν2
Whereh(ν,z)likg(ν,z),isaerpwoseriesinz.
Solution:
Thedeifnitionofthebesselfunctionofsecondkindis
cosµπJ(z)+J(z)
Y(z)=limµ−µ
νµ→νsinµπ
UsingLHopitalsruletoaluateevthislimitewget
′µ′
−πsinµπJµ(z)+cosµπJ(z)−(−1)J(z)
Y(z)=limµ−µ
νµ→µcosµπ
CHAPTER4.TICALTHEMAMAPHYSICSII123
Sinceattegerinaluesvofνthealuevcosνπ=1andsinνπ=0ewget
1[∂Jµ(z)ν∂J−µ(z)]
Yν(z)=π∂µ−(−1)∂µµ=ν
orFtegernon-inνtheerpwoseriestationrepresenoftheBesselfunctionis
∞r()
Jµ(z)=∑(−1)zµ+2r
r=0r!Γ(r+µ+1)2
akingTeativderivwithrespecttoµewget
∞r()()∞r′()
∂Jµ(z)=∑(−1)zµ+2rlnz+∑−(−1)Γ(r+µ+1)zµ+2r
2
∂µr=0r!Γ(r+µ+1)22r=0r!Γ(r+µ+1)2
()∞r()∞r′()
=lnz∑(−1)zµ+2r+∑−(−1)Γ(r+µ+1)zµ+2r
2r!Γ(r+µ+1)2r!Γ2(r+µ+1)2
r=0r=0
|{z}|{z}
()Jµ(z)g(µ,z)
=lnzJ(z)+g(µ,z)
2µ
µ
SinceJ(z)=(−1)J(z).ThisexpressioncanbereusedtocalculatetheeativderivofJ.Multi-
−µµµ−µ
plyingbothsidesofthisexpressionyb(−1)ewget
∂J−µ(z)µ(z)µ
=(−1)lnJ(z)+(−1)g(µ,z)
∂µ2µ
Substutingthiskbacintheexpressionforthebesselfunctionofsecondkindewget
1[(z)µµ(z)µ]
Y(z)=lnJ(z)+g(µ,z)+(−1)(−1)lnJ(z)+(−1)g(µ,z)
νπ2µ2µ
[()()]µ=ν
1zz1ν
=lnJ(z)+lnJ(z)+[g(ν,z)+(−1)g(ν,z)]
π2ν2νπ
|{z}
h(ν,z)
2(z)
=lnJ(z)+h(ν,z)
π2ν
ThisesgivtherequriedexpressionforBesselfunctionofsecondkindfortegerinorder.□
4.8orkHomewtEigh
4.8.1.Consideradampedharmonicoscillatorernedvgoybtheequation
x¨+2λx˙+ω2x=f(t)
0m
Whereλ2−ω2erdampved(ocase).Supposetheexternalforcef(t)iszerofort<0.elopDevthe
0
Green’sfunctioandwritethesolutionx(t)satisfyingconditionsx(0)=x˙(0)=0.
Solution:
LetG(t,ξ)bethesolutiontothetialdifferenequatationwiththeinhomogenouspartreplacedybdelta
functionδ(t−ξ).Thiscanbewrittenas
¨˙2
G(t,ξ)+2λG(t,ξ)+ωG(t,ξ)=δ(t−ξ)(4.23)
0
CHAPTER4.TICALTHEMAMAPHYSICSII124
Ift̸=ξthedeltafunctioniszerosothesolutionforthistialdifferenequationfort<ξis
(√22√22)
−λttλ−ω0−tλ−ω0
G(t,ξ)=eAe+Be(4.24)
−
enGivinitialconditionx(0)=0.ItimpliesthatG(0,ξ)=0,Susbtutingthisintheeabvosolutionew
get
G(0,ξ)=A+B=0;⇒B=−A
−
˙
Alsosincex˙(0)=0,ewustmevhaG(0,ξ)=0.tiatingDifferen(4.24)andsubstutingt=0ewget
(√22√22)(√√22√√22)
˙tλ−ω−tλ−ω−λt22tλ−ω22−tλ−ω−λt
G(t,ξ)=−λAe0−Ae0e+Aλ−ωe0+Aλ−ωe0e
−00
˙√22
⇒G(0,ξ)=2Aλ−ω=0;⇒A=0
−0
Sothesolutionforthecaset<ξistriviallyG(t,ξ)=0.orFt>ξthesolutionsimilarlyis
(√22√22)
G(t,ξ)=e−λtCetλ−ω0+De−tλ−ω0(4.25)
+
BytheytunitcontrequiremenofGreen’sfunctionsolutionattheyvicinitoftieint=ξ±ϵ
limG(ξ−ϵ,ξ)=limG(ξ+ϵ,ξ)
ϵ→0ϵ→0(√√)
−λξξλ2−ω2−ξλ2−ω2
⇒G(t,ξ)=G(t,ξ)=eCe0+De0
−(+√√)
2222
⇒0=Ceξλ−ω0+De−ξλ−ω0
√22
⇒C=−De−2ξλ−ω0(4.26)
Alsotegratingin(4.23)intheyvicinitofti.e.,fort=ξ±ϵinthelimitϵ→0ewget
[∫ξ+ϵ∫ξ+ϵ]∫ξ+ϵ
t=ξ+ϵ
˙˙2
limG(t,ξ)+2λG(t,ξ)dt+ωG(t,ξ)dt=δ(t−ξ)dt
ϵ→0t=ξ−ϵ0
ξ−ϵξ−ϵξ−ϵ
ω2[]
˙˙022
G(ξ,ξ)−G(ξ,ξ)+2λ[G(ξ,ξ)−G(ξ,ξ)]+G(ξ,ξ)−G(ξ,ξ)=1(4.27)
−+−+2+−
SincetheGreensfunctionshouldbeuoustinconattheyvicinitoft=ξ,ewevhaG(ξ,ξ)=G(ξ,ξ)
+−
thisrendersowtmiddledifferenceineabvoexpressiontobezerovingleauswithonlythedifference
ofe.ativderivSinceG(t,ξ)isticallyidenzeroitseativderiviszero.ButtheeativderivofG(t,ξ)at
t=ξis−+
(√22√22)(√√22√√22)
˙ξλ−ω0−ξλ−ω0−λξ22ξλ−ω022−ξλ−ω0−λξ
G(ξ,ξ)=−λAe+Bee+Aλ−ωe−Bλ−ωee
+00
SubstutingCfrom(4.26)ewget
˙√22−λξξ√λ2−ω2
G(ξ,ξ)=2Dλ−ωee0
+0
˙˙
SubstutingG(ξ,ξ)=0andG(ξ,ξ)fromeabvoin(4.27)ewget
−+
√√221√22
22−λξξλ−ω0√−ξ(λ−λ−ω0)
2Dλ−ω0ee=1;⇒D=λ2−ω2e
0
CHAPTER4.TICALTHEMAMAPHYSICSII125
Usingthisin(4.26)ewget
1√22
C=√e−ξ(λ+λ−ω0)
λ2−ω2
0
usThG(t,ξ)becomes
+
e(t−ξ)λ(√22√22)
√(t−ξ)λ−ω0−(t−ξ)λ−ω0
G(t,ξ)=e−e
+λ2−ω2
0
SotherquiredGreen’sfunctionis


0(t−ξ)λ(√√)ift≤ξ
G(t,ξ)=e2222
√(t−ξ)λ−ω0−(t−ξ)λ−ω0
λ2−ω2e−eift>ξ
0
Sothesoutionofthetialdifferenequationbecomes
x(t)=∫tG(t,ξ)f(t)dt
ξm
Thisistherequiredsolutionofthetialdifferenequation.□
4.8.2.eWaretoesolvy′′−k2y=f(x)(0≤x≤L)subjecttotheboundaryconditionsy(0)=y(L)=0.
(a)FindGreen’sfunctionybdirectconstruction.
Solution:
2
dy−k2y=f(x)
2
dx
for0≤x≤l,withy(0)=y(l)=0.
Thegreen’sfunctionsolutiontononhomogenoustialdifferenequationLy=f(x)isasolution
tohomogenouspartofthetialdifferenequationwiththesourcepartreplacedasdeltafunction
Ly=δ(x−ξ).ThetainedonsolutionisG(x,ξ),i.e.,LG(x,ξ)=δ(x−ξ).Thissolutioncorresponds
tothehomogenouspartonlyasitisindeptendenofyansourcetermf(x).
2
dG(x,ξ)−k2y=δ(x−ξ);withG(0,ξ)=0;andG(l,ξ)=0forall0≤ξ≤l(4.28)
2
dx
Sincedeltafunctionδ(x−ξ)iszeroerywhereevexceptx=ξewcanifndsolutionforowtregions
x<ξandx>ξ.orFx<ξletthesolutiontoLy=0bey(x)andforx>ξbey(x)then
12
y′′(x)−k2y(x)=0;forx<ξ;y′′(x)−k2y(x)=0;forx>ξ
1122
TheseareellwwnknosecondorderODESwhosesolutionare
y(x)=Asinhkx+Bcoshkx;y(x)=Csinhkx+Dcoshkx
12
Bytheboundaryconditiony(0)=0andy(l)=0.TheseimmediatelyimplythatB=0.Also
12
sincethesoutiontothetialdifferenequationustmbeuoustincony(ξ)=y(ξ).tegratingInEq.
12
(4.28)intheyvicinitofξewget
0Byytunitcon
ξ+∫✟✯∫
ξ+ξ+
′2✟✟′′
y(x)−kydx=δ(x−ξ)dx;⇒y(ξ)−y(ξ)=1
✟+−
✟✟ξ−ξ−
ξ−
CHAPTER4.TICALTHEMAMAPHYSICSII126
romFthreetdifferenconditions,(i)ytunitconatξ,(ii)y(l)=0and(iii)y′(ξ)−y′(ξ)=1ewget
212
wingfollothreelinearequations.Usingthereparametersewget.
Ckcoshkξ+Dksinhkξ−Aksinhkξ=1
Csinhkξ+Dcoshkξ−Asinhkξ=0
Csinhkl+Dcoshkl=0
hWhiccanbewritteninthematrixformandedsolvas.
sinh(kξ)
kcosh(kξ)ksinh(kξ)−kcosh(kξ)C1Cktanh(kl)
−1sinh(kξ)
sinh(kξ)cosh(kξ)−sinh(kξ)D=0⇒D=k
sinh(kl)cosh(kl)0A0A1(sinh(kξ)−cosh(kξ))
ktanh(kl)
Giving
C=sinh(kξ);D=−1sinh(kξ);A=1(sinh(kξ)−cosh(kξ))
ktanh(kl)kktanh(kl)
Sotherequiredfunctionis
{1(sinh(kξ))sinhkxsinhk(L−ξ)
y(x)=−cosh(kξ)sinhkx=−ifx<ξ
G(x,ξ)=1ktanh(kl)ksinhkl(4.29)
y(x)=sinh(kx)sinh(kξ)−1sinh(kξ)cosh(kx)=sinhkξsinhk(L−x)ifx>ξ
2ktanh(kl)kksinhkl
Eq.(4.29)esgivtheGreen’sfunctionwhcincanbeusedtoifndthesolutiontothetialdifferen
equation
y(x)=∫G(x,ξ)f(ξ)dξ
Thesolutiontotheoriginalinhomogenoustialdifferenequationcanisengivybtheeabvoexpres-
sionintermsofGreen’sfunction.□
′′2
(b)evSolatheequationG−kG=δ(x−ξ)ybtheourierFsineseriesmethod.Canouywshothat
theseriesobtainedforG(x,ξ)istalenequivtothesolutionfoundunder(a).
Solution:
LetsineseriesofsolutionGbe
∑(nπ)′′∑n2π2(nπ)
G=AnsinLx;G=An−L2sinLx
Substutingthesekbacinthetialdifferenequationewget
∑(nπ2)nπ(nπ)
−A−ksinx=δ(x−ξ)
nLLL
Againletthefouriersineseriesofdeltafunctionbe
δ(x,ξ)=∑Bsin(nπx)
nL
ThecoteiffcienBcanbefoundas
n
∫(mπ)∫∑(nπ)(mπ)
δ(x−ξ)sinxdx=Bsinxsinxdx
LnLL
(mπ)∑LL
sinLξ=Bn2δmn=Bm2
()n
⇒B=2sinmπξ
mLL
CHAPTER4.TICALTHEMAMAPHYSICSII127
Substutingthiskbactointhetialdifferenequationewget
∑(nπ2)nπ(nπ)∑2(mπ)(nπ)
−AnL−kLsinLx=δ(x−ξ)=LsinLξsinLx
n
Comparingthecotseiffcienewget
2nπnπ2(nπ)2(nπ)
An(k−L)L=LsinLξ;⇒An=(k2−nπ)nπsinLξ
L
usThtheGreen’sfunctionbecomes
∑(nπ)∑2(nπ)(nπ)
G=AnsinLx=(k2−nπ)nπsinLξsinLx
nnL
hWhicistherequiredsineseriesoftialdifferenequation.Thefourierseriesofsolutioninpart(a)
isexactlythis.□
4.8.3.wShothattheGreen’sfunctiondesignedtoesolvtheDE
d(dy)(2m2)
dxxdx+kx−xy=f(x)(4.30)
subjecttoy(0)<∞andy(a)=0reads
{π[J(kξ)Y(ka)−J(ka)Y(kξ)]
mmmmJ(kx)(x≤ξ)
G(x,ξ)=2Jm(ka)m
π[J(kx)Y(ka)−J(ka)Y(kx)]
mmmmJ(kξ)(x≥ξ)
2J(ka)m
m
AlsoconsiderthecaseJ(ka)=0.wShothatifk̸=0,thenG(x,ξ)doesnotexist,butk=0G(x,ξ)
m
doesexist,althoughtheeabvoformisnotapplicable.aluateEvG(x,ξ)inthiscase.
Solution:
LetG(x,ξ)bethesolutionofthetialdifferenequationwiththeinhomogenousfunctionybadelta
functionδ(x−ξ).Ifx̸=ξthenthedeltafunctioniszeroandthetialdifferenequationisabessel
tialdifferenequation.Sothesolutionofbesseltialdifferenequationofordermareis
G(x,ξ)=AJ(kx)+BY(kx).
mm
eWcandividethesolutiontoinowtpartswithx<ξandx>ξ.
enGivtheboundaryconditionthesolutionisifniteforx=0,thecoteiffcienofbesselfunctionofsecond
kindY(kx)iszerobecauseitwsbloupatx=0.
m
G(x,ξ)=y(x)=AJ(kx);(x<ξ)
−1m
orFx>ξ,thesolutiony2(x)canis
G(x,ξ)=y(x)=CJ(kx)+DY(kx);(x≥ξ)
−2mm
Sincetheengivinitialconditionisy2(a)=0ewget
J(ka)
D=−Cm
Ym(ka)
J(ka)
Sotheernelkofsolutionisbeginy(x)=J(kx)+qY(kx)whereq=−m.
2mmYm(ka)
CHAPTER4.TICALTHEMAMAPHYSICSII128
Thegreensfunctionsolutionis
{y(x)y(ξ)
12(x≤ξ)
G(x,ξ)=p(ξ)W(ξ)(4.31)
y(ξ)y(x)
12(x≥ξ)
p(ξ)W(ξ)
WhereW(ξ)=[y(x)y′(x)−y′(x)y(x)]isthewronskianofowtindeptendensolutiontoowtparts
1212x=ξ
intherange[a,b]dividedybtheptoinξ∈[a,b]
wNoewevhatocalculatethewronskianW(y(x),y(x))hwhiccanbealuatedevas
12
W(y(x),y(x))=W(J(kx),J(kx)+qY(kx))
12mmm
=W(Jm(kx),Jm(kx))+qW(Jm(kx),Ym(kx))
=0+q2
πx
usThW(ξ)=2q.eWevhaourp(ξ)=ξ.us,Th
πξ
y(x)y(ξ)J(kx)[J(kξ)+qY(kξ)]π[J(kξ)Y(ka)−J(ka)Y(kξ)]
12=mmm=J(kx)mmmm
p(ξ)W(ξ)ξ·2qm2J(ka)
πξm
y(x)y(ξ)[J(kx)+qY(kx)]J(kξ)π[J(kx)Y(ka)−J(ka)Y(kx)]
21=mmm=mmmmJ(kξ)
p(ξ)W(ξ)ξ·2q2J(ka)m
πξm
Usingthisingreenfunctionsolutionin(4.31)ewobtain.
{π[J(kξ)Y(ka)−J(ka)Y(kξ)]
mmmmJ(kx)(x≤ξ)
G(x,ξ)=2Jm(ka)m
π[J(kx)Y(ka)−J(ka)Y(kx)]
mmmmJ(kξ)(x≥ξ)
2J(ka)m
m
hWhicistherequiredGreen’sfunctionsolution.□
4.8.4.Considerthalueeboundary-vproblem
2
dy=f(x)y(0)=0dy=0(0≤x≤L).
2
dxdx
(a)Findthenormalizedeigenfunctionsoftheoepratorfortehengivbybodaryconditions
Solution:
Letg(x)betheeigenfunctionoftheoperatorwith−λ2aseigenaluevusth
2
dg(x)=−λ2g(x)
dx2
eWwknothesolutionofthistialdifferenequationare
g(x)=sin(λx);g′(x)=λcos(λx);g′(L)=0=λcos(λL);⇒λL=(2n+1)π
2
usThtheeigenfunctionsbecome
((2n+1)π)2((2n+1)π)2
g(x)=sin2Lx;−λ=2L
Thisisisthealueveigenandeigenfunctionoftheengivoperatorsubjecttoengivboundarycondi-
tion□
CHAPTER4.TICALTHEMAMAPHYSICSII129
(b)riteWbilinearulaformforGreen’sfunction
Solution:
Thebilinearformis
((2n+1)π)((2n+1)π)
∑g(x)g∗(ξ)∑sin2Lxsin2Lξ
G(x,ξ)=nn=()
λn(2n+1)π2
nn2L
Thisistherequiredbilinearform□
Chapter5
TheStandardModel
5.1orkHomewOne
5.1.1.(SMIN1.1)Consideraectorvifeldinthree-dimensionalcartesianspace:
ixy
2
u=x+2
3
i
(a)Computethecomptsonenof∂u.
j
i
(b)Compute∂u
i
ji
(c)Compute∂∂u.
j
Solution:
Sinceforthree-dimensionalcartesianspacetheindicesrunfrom1through3,
123
∂u∂u∂uy2x0
111
i123
∂u=∂u∂u∂u=x00
j222
123
∂u∂u∂u000
333
i123
∂u=∂u+∂u+∂u=y+0+0=y
i123
Sinceforcartesianspacethemetrictselemenare
∂
()100∂x
∂∂∂ijjij∂
∂=,g=g=010⇒∂=g∂=
i∂x∂y∂ziji∂y
001∂
∂z

∂2∂2∂2xy2
j2ji22
⇒∂∂=++≡∇⇒∂∂u=∇x+2=0
j∂x2∂y2∂z2j
30
hEacofthecomptsonenareusthcalculated.□
n
5.1.2.(SMIN1.4)Expressthewingfollotitiesquaninnauralunits,intheform(#V)Ge.
−263
(a)Thetcurrenenergyydensitoferseuniv:∼1×10kg/m.
(b)1angstrom
(c)1nanosecond
130
CHAPTER5.THEARDANDSTMODEL131
(d)1gigaparsec≃3×1025m
26
(e)Theyluminositofthesun≃4×10W
Solution:
()()()()
1kgc2Jc2J14c5/4ℏ3/44
=()=()=c5ℏ3J4=c5ℏ3GeV=GeV
31311−1399
mcscℏJe×10e×10
()
−26−26/45/43/44
1×10kg1×10cℏ−114
3=9GeV=(1.01×10GeV)
me×10
()
1m=1s=1J−1=(cℏJ)−1=cℏGeV−1
9
ccℏe×10
()
10−1
−101×10cℏ−5−1
1Å=1×10m=9GeV=(1.24×10GeV)
(e×10)
251×10−25cℏ−1−41−1
1Gpc≃3×10m=9GeV=(4.13×10GeV)
3·e×10
()()−1
1−1ℏ−1−91×109ℏ−15−1
1s=J=9GeV⇒1ns=1×10=9GeV=(4.41×10GeV)
ℏe×10e×10
1W=1J=11J=ℏJ2=(ℏ1/29GeV)2
1sℏJ−1e×10
()
261/21/22
L⊙≃4×1026W=(4×10)ℏGeV=(3.21×106GeV)2
9
e×10
□
5.1.3.(SMIN1.6)Consideraector4-v

2

µ3
A=

0
0
(a)ComputeA·A=AµAµ.
Solution:
orFwskiominkspacethemetricis

1000
()
µν0−100ν
g=g=⇒A=gA=2−3000
µνµµν
00−10
000−1
A·A=AµA=2·2+(3)·(−3)=−5
µ
usThthedotproductis−5.□
CHAPTER5.THEARDANDSTMODEL132
(b)WhatarethecomptsonenofAµ¯ifouyrotatethecoordinateframearoundthez-axisthroughan
π
angleθ=/?
3
Solution:
Thetransformationmatrixforrotationaroundz-axisatanangleθ=πis
3

100010002
√
ππ133/2
µ¯0cos/sin/00//0µ¯µ¯ν
3322√
Λ=ππ=√A=ΛA=
µ31ν
0−sin/cos/00−//0−33/2
3322
000100010
Thesearetherequiredtransformedcompts.onen□
(c)orFouryeranswinpart(5.1.3b),erifyvthatAµ¯Aµ¯isthesameasinpart(5.1.3a).
Solution:
10002
µ¯ν¯µ¯ν¯0−100µ¯ν¯ν¯−3/2
g=ΛΛg=g=gA=gA=√
µνµν00−10µ¯ν¯µ¯µ¯ν¯
33/2
000−10
(√)√927
µ¯3333
Aµ¯A=2·2+/·(−/)+−3/+3/=4−−=−5
222244
Theinnerproductis−5asrequired.□
µ¯
(d)WhatarethecomptsonenAifouyboosttheframe(from(5.1.3a))aspeedv=0.6inx-direction?
Solution:√
2
orFv=0.6=3/5the“gammafactor”isγ=1/1−.6=1.25=5/4andusthvγ=0.75=3/4
Thetransformationmatrixunderthisboostis

γvγ005/43/40019/4

µ¯vγγ003/45/400µ¯µ¯ν21/4
Λ==A=ΛA=
µν
001000100
000100010
hwhicaretherequiredcomptsonenunderboost.□
(e)orFouryeranswinpart(5.1.3d),erifyvthatAµ¯Aµ¯isthesameasinpart(5.1.3a).
Solution:
SincethewskioMinkmetricistarianvinunderboost,thetransformedmetricisg=g
µ¯ν¯µν

100019/4
()
0−100ν¯−21/4µ¯19192121
g=,A=gA=,AA=/·/+/·−/=−5
µ¯ν¯µ¯µ¯ν¯µ¯4444
00−100
000−10
Theinnerproductis−5asin(5.1.3a).□
5.1.4.(SMIN1.10)Considerascalarifeld
ϕ(x)=2t2−3x2
(a)Computethecomptsonenof∂ϕ.
µ
(b)Computethecomptsonenof∂µϕ.
(c)Compute∂∂µϕ.(Thisoperationisthebd’Alemertianoperator).
µ
CHAPTER5.THEARDANDSTMODEL133
Solution:
∂2t∂2t
∂t∂t
∂∂
∂x−6xµµν−∂xµ6x
∂ϕ(x)=ϕ(x)=∂=g∂=,⇒∂ϕ(x)=
µ∂0ν−∂0
∂y∂y
∂0−∂0
∂z∂z
Theoperator
∂∂
∂t∂t()
∂−∂∂2∂2∂2∂2∂2
µ∂x∂x2
∂∂==−−−≡−∇
µ∂−∂∂t2∂x2∂y2∂z2∂t2
∂y∂y
∂−∂
∂z∂z
∂∂µϕ(x)=(∂2−∇2)ϕ(x)=4+6=10
µ∂t2
usThthebd’Alemertianoperatorontheengivscalarfunctionϕ(x)is10.□
5.1.5.(SMIN1.13)Anexcitedydrogenhatomemitsa10.2eVLyman−αphoton.
(a)Whatisthetummomenofthephoton?(Expressinnaturalunits.)
(b)AsNewton’sthirdwlaremainsinforce,whatisthekineticenergyoftherecoilinggroundstate
ydrogenhatom?
(c)Whatistherecoilspeedofproton.
Solution:
orFaphotonE=psothetummomenofthephotonis10.2eV.
IfNewton’swlaremaininsforce,thentherecoilinggroundstateatomhasthesametummomenasthe
outgoingphoton.Sotherecoilingtummomenofgroundstateatomis10.2eV.
Massofprotonis9.38×108eV,sincethemassofelectronisnegligiblecoomparedtoprotonletus
assumetheprotoncarriesallthetummomenso,
√vp√1√1−8
p=γmv⇒2=⇒v=()=()2=1.08×10≡3.26m/s
1−vm28
1+m1+9.38×10
p10.2
−8
Sotherecoilspeedofprotonis1.08×10≡3.26m/s.□
5.2orkHomewowT
5.2.1.(SMIN2.3)Considerowtparticlesofequalmassmconnectedybaspringoftconstankandconifned
evtomoinonedimension.Thetireenssytemesvmowithoutfriction.tAequilibriumthespringhas
lengthL.
(a)riteWwndotheLagrangianofthissystemasafunctionofx1andx2andtheires.ativderivAssume
x>x.
21
Solution:
Thekineticenergyofheacmassis1mx˙2andforthesecondmassis1mx˙2.Thetotalcompression
2122
CHAPTER5.THEARDANDSTMODEL134
inthespringisx−x−LsothetotalptialotenenergyinthespringisV=1k(x−x−L)2.
21221
Sothelagrangianofthesystembecomes
L=1mx˙2+1mx˙2−1k(x−x−L)2
2122221
ThisistherequiredLagrngian.□
(b)riteWtheEuler-Lagrangeequationforthissystem.
Solution:
d(∂L)=∂L⇒mx¨=k(x−x−L)
dt∂x˙∂x121
(1)1
d∂L=∂L⇒mx¨=−k(x−x−L)
dt∂x˙∂x221
22
ThesearetherequiredEuler-Lagrangeequationofthesystem.□
(c)eMakthehangecofariablesv
∆≡x−x−LX=1(x+x)
21222
riteWtheLagrangianinthesenewariables.v
Solution:
Eliminatingx1andx2eenbwettheowttransformationariablesv∆andXewget
x−x=∆+Lx+x=2X
2111
1˙1˙
2x=2X+∆+L⇒x=X+(∆+L)x˙=X+∆
22222
1˙1˙
2x=2X−∆−L⇒x=X−(∆+L)x˙=X−∆
12212
Usingtheseariablesvthelagrangianbecomes
1(˙1˙)21(˙1˙)212
L=2mX+2∆+2mX−2∆+2k∆
1(˙2˙2)12
=2m2X+∆+2k∆
Thisisthelagrangianinthetransformedcoordinatesystem.□
µ∗2∗
5.2.2.(SMIN2.7)wShothatthecomplexLagrangianL=∂ϕ∂ϕ−mϕϕisalgebraicallyticalidento
µ
1µ1221µ122
L=∂ϕ∂ϕ−mϕ+∂ϕ∂ϕ−mϕ
2µ112112µ22222
ifm=m1=mand
12
ϕ=(ϕ1+ϕ2)+i(ϕ1−ϕ2)
22
Solution:
CHAPTER5.THEARDANDSTMODEL135
Assumingthescalarifeldsϕ1andϕ2arerealaluedvfunction.Thecomplexifeldanditsconjugateare
(ϕ+ϕ)(ϕ−ϕ)(ϕ+ϕ)(ϕ−ϕ)
ϕ=12+i12ϕ∗=12−i12
2222
∗1[22]1(2222)
⇒ϕϕ=(ϕ+ϕ)+(ϕ−ϕ)=ϕ+ϕ+2ϕϕ+ϕ+ϕ−2ϕϕ
41212412121212
=1(ϕ2+ϕ2)
212
∂ϕ∂µϕ∗=∂(ϕ1+ϕ2+iϕ1+ϕ2)∂µ(ϕ1+ϕ2−iϕ1+ϕ2)
µµ2222
=1(∂ϕ+∂ϕ+i∂ϕ+i∂ϕ)1(∂µϕ+∂µϕ−i∂µϕ−i∂µϕ)
2µ1µ2µ1µ221212
=1(∂ϕ∂µϕ+∂ϕ∂µϕ−i∂ϕ∂µϕ−i∂ϕ∂µϕ
4µ11µ12µ11µ12
+∂ϕ∂ϕ+∂ϕ∂µϕ−i∂ϕ∂µϕ−i∂ϕ∂µϕ
µ2µ1µ22µ21µ22
+i∂ϕ∂µϕ+i∂ϕ∂µϕ+∂ϕ∂µϕ+∂ϕ∂µϕ
µ11µ12µ11µ12
+i∂ϕ∂µϕ+i∂ϕ∂µϕ+∂ϕ∂µϕ+∂ϕ∂µϕ)
µ21µ22µ21µ22
=1(∂ϕ∂µϕ+∂ϕ∂µϕ)
2µ11µ22
SubstutingthesekbacintothecomplexLagrngianewget
µ∗2∗
L=∂ϕ∂ϕ−mϕϕ
µ
1µµ2122
=(∂ϕ∂ϕ+∂ϕ∂ϕ)−m(ϕ+ϕ)
2µ11µ22212
1µ1221µ122
=∂ϕ∂ϕ−mϕ+∂ϕ∂ϕ−mϕ
2µ112112µ22222
ThiswsshotheowtLagrangianaret.alenequiv□
5.2.3.(SMIN2.9)Consideralagrangianofaluedreal-vscalarifeld:
1µ12213
L=∂ϕ∂ϕ−mϕ−cϕ.
2µ263
(a)IsthisLagrangiantzLorent?arianvinItistarianvinunderC,P,andTtransformationsindividu-
ally?
Solution:
SinceeryevterminthelagrangianisascalaritistriviallytzLorent.arianvinAsϕisrealaluedv
∗ˆ∗
scalaritscomplexconjugateisitselfϕ=ϕsincetheCtransformationtransformsϕtoϕhwhic
ˆ
areticalidensotheLagrngianistarianvinunderCtransformation.
ˆˆ
ItisnottarianvinunderPandTtransformation.□
(b)Whatisthedinemsionofc?
3
Solution:
4
SincethedimensionofLagrangianydensitis[E]andtheydimensionalitfoϕis[E]thedimen-
ysionalitofc3is[E]3□
(c)WhatisEuler-Lagrangeequationforifeld?
Solution:
∂(∂L)=∂L
µ∂(∂ϕ)∂ϕ
µ
µ2c32
∂(∂ϕ)=−mϕ−ϕ
µ2
CHAPTER5.THEARDANDSTMODEL136
ThisistherequiredEuler-LagrangeequationfortheengivLagrangian.ydensit□
(d)Ignoringthec3ribution,tconafree-ifeldsolutionymabewritten
−ip·x∗ip·x
ϕ(x)=Ae+Ae
0
foracomplexcoteiffcienA.Consideraest-orderwlotributionconforϕ1≪Atoapeturbation
hsucthatϕ(x)=ϕ0+ϕ1.eDerivadynamicalequtionforϕ1.
Solution:
c
µ232
∂∂(ϕ(x))+mϕ(x)+ϕ=0
µ2
Substutingϕ=ϕ+ϕ
01
µ212
⇒∂∂(ϕ(x)+ϕ(x))+m(ϕ(x)+ϕ(x))+c(ϕ+ϕ)=0
µ01012301
()
1ϕ2
µ221µ2
⇒∂∂ϕ(x)+mϕ(x)+cϕ1+=−∂∂ϕ−mϕ
µ11230ϕµ00
(0)
1ϕ
µ221µ2
⇒∂∂ϕ(x)+mϕ(x)+cϕ1+2=−∂∂ϕ−mϕ
µ11230ϕµ00
0
TheifrstterminRHSofeabvoexpressionis
∂∂µϕ=gµν∂(∂ϕ)
µ0ν(µ0)
µν−ip·x∗ip·x
=g∂(−ip)Ae+(ip)Ae
(νµµ)
ν−ip·xν∗ip·xµν
=∂(−ip)Ae+(ip)Ae(Distributingg)
ν
ν−ip·xν∗ip·x
=(−pp)Ae+(−pp)Ae
νν
22
=−(E−|p|)ϕ0
□
5.3orkHomewThree
5.3.1.(SMIN3.3)AparticleofmassmandhargecqinanelectromagneticifeldhasaLagrangian
1˙2˙
L=2mr−q(ϕ−r·A),
whereϕisthescalarptial,otenandAistheectorvptial.oten
(a)Suppose(justforthet)momenthattheptialotenifeldsarenotexplicitfunctionsofx.Use
Noether’stheoremtocomputetheedconservytitquanoftheelectromagneticLagrangian.
Solution:
ritingWtheLagrangianincartesiancoordinatesystemewget
1(222)
L=mx˙+y˙+z˙−q(ϕ(y,z)−xA˙−y˙A−z˙A)
2xyz
Sincethelagrangianistarianvinundertranslationinx→x+ϵtheedconservytitquanis
∂Ldx=mx˙+qAx
∂x˙dϵ
Sotheedconservytitquaniftheptialotenifeldsareindeptendenofxismx˙+qAx.□
CHAPTER5.THEARDANDSTMODEL137
(b)Moregenerallyassumethattheptialotenifeldsaryvinspaceandtime.WhataretheEuler-
LagrangeequationsforthisLagrangiancorrespondingtoparticlepositionxi?
Solution:
IftheptialotenifeldsdependuponspaceandtimetheLagrangianbecomes
1(ii)(ii)
L=2mx˙x˙−qϕ(x)−x˙Ai
TheEuler-Lagrangeequationsare
d(∂L)=∂L
dt∂x˙i∂xi
d(i)(∂ϕj∂Aj)
⇒dtmx˙+qAi=−q∂xi−x˙∂xi
idAi(∂ϕj∂Aj)
⇒mx¨+qdt=−q∂xi−x˙∂xi
ThesearetherequiredEuler-Lagrangeequations.□
(c)eSolvtheprevioussolutionexplicitlyformx¨.ExpressouryifnaleranswasabinationcomofE
andBifelds.
Solution:
Speciifcallyforxi=xtheeabvoexpressionbecomes
mx¨+qdAx=−q(∂ϕ−x˙∂Ax−y˙∂Ay−z˙∂Az)
dt∂x∂x∂x∂x
(∂Axdx∂Axdy∂Axdz)(∂Ax∂Ay∂Az)
mx¨+q++=−q−E−x˙−y˙−z˙
∂xdt∂ydt∂zdtx∂x∂x∂x
mx¨+q(∂Axx˙+∂Axy˙+∂Axz˙)=qEx+q(∂Axx˙+∂Ayy˙+∂Azz˙)
∂x∂y∂z∂x∂x∂x
mx¨=qEx+q(y˙(∂Ay−∂Ax)−z˙(∂Ax−∂Az))
∂x∂y∂z∂x
mx¨=qE+q(y˙B−z˙B)
xzy
mx¨=qEx+q(r×B)x
ThisistherequiredequationofmotionforthexcoordinateunderengivLagrangian.□
5.3.2.(SMIN3.6)ee’vWseenthatarealaluedvscalarifeldymabeexpandedasaevaplane-wsolution:
∫3[]
dp1−ip·x∗ip·x
ϕ(x)=(2π)3√cpe+cpe
Ep
Computerthetotalanisotropicstress∫d3xTijwherei̸=j,foraaluedreal-vifeldybtegratinginervo
thestress-energytensor.
Solution:
Thestress-energytensoris
Tµν=∂L∂νϕ−gµνL
∂∂µϕ
orFi̸=jgij=0sotheensorTreducesto
Tij=∂L∂jϕ=∂ϕ∂jϕ
∂(∂ϕ)i
i
CHAPTER5.THEARDANDSTMODEL138
orFacomplexscalarifeldengivewevha
∫3[()()]∫3i[]
dp1−ip·x∗ip·xdpp−ip·x∗ip·x
∂ϕ(x)=√c∂e+c∂e=i√ce+ce
i3pipi3pp
(2π)Ep(2π)Ep
Andsimilarlythe
∫3[()()]∫3[]
dp1dpp
j√j−ip·x∗jip·x√j−ip·x∗ip·x
∂ϕ(x)=(2π)3Ecp∂e+cp∂e=−i(2π)3Ecpe+cpe
pp
usThtheproductis
∫3i[]∫3[]
dppdpp
ij√−ip·x∗ip·x√j−ip·x∗ip·x
T=icpe+ce·−icpe+ce
(2π)32Ep(2π)32Ep
∫pp
33i[][]
dpdqpq
√√j−ip·x∗ip·x−iq·x∗iq·x
=ce+cece+ce
33ppqq
(2π)(2π)2E2E
∫pq
33i[]
dpdq√pqj−i(p+q)·x∗−i(p−q)·x∗i(p−q)·x∗∗i(p+q)·x
=cce+cce+cce+cce
33pqpqpqpq
(2π)(2π)4EpEq
tegratingInthisytitquanervotheolumevyields
∫∫33i[]
dpdqpq
ij33√j−i(p+q)·x∗−i(p−q)·x∗i(p−q)·x∗∗i(p+q)·x
Tdx=dxcce+cce+cce+cce
(2π)3(2π)34EEpqpqpqpq
pq
(5.1)
Sincethetegrationinoperatoriseutativcommforindeptendenariablesvtheolumevtegralinreduces
thecomplextegralintoDiracdeltafunctions
∫i(p−q)·x33(3)
edx=(2π)δ(p−q)
Soifewperformqtegralinyanthetegralinisnonzeroonlywhentheqaluevisequaltopas
∫3
dqqp
√jccδ(3)(p−q)(2π)3=√jcc
3pqpp
(2π)2E2E
qp
So(5.1)reducesto
∫∫3i[]
dppp
ij3√j∗∗∗∗
Tdx=cc+cc+ccp+cc
(2π)34E2p−pppppp
p
∫3i[]
=dpppjcc+cc∗+c∗c+c∗c∗
(2π)32Epp−ppppppp
Thisistherequiredanisotropicstressrequired.□
5.3.3.(SMIN3.8)eWtmighsuppose,thataectorvifeldhastarianvtz-inLorenLagrangian
νµ2µ
L=∂A∂A−mAA
µνµ
(a)ComputetheEuler-LagrangeequationsofforthisLagrangian.
Solution:
TheEuler-Lagrangeequationare
∂(∂L)=∂L
µνν
∂(∂A)∂A
µ
µ2
∂∂A=−mA
µνµ
µµ2µ
∂∂A=−mA
ν
ThesearetherequiredEuler-Lagrangeequations.□
CHAPTER5.THEARDANDSTMODEL139
(b)Assumeaevaplane-wsolutionfortheectorvifeld
∫3[]
µdpµ√1−ip·x∗ip·x
A=ϵae+ae
3pp
(2π)2Ep
µ
whereewen’tvhaspeciifedpolarizationstate(s)e.explicitly
elopDevanexplicitrelationshipeenbwetpolarization,thetummomenoftheifeld,andthemass.
Whatconditiondoesthisimposeforamasslessectorvparticle?
Solution:
Thestress-energytensoris
µν∂Lναµν
T=∂(∂Aα)∂A−gL
µ()
µναµναα2α
=∂A∂A−g∂A∂A−mAA
αµνα
UsingtheengivAµectorvwitheryevinthisexpressionewget
∫33[][]
dpµαdq√1−ip·x∗ip·x−iq·x∗iq·x
ϵϵae+aeae+ae
(2π)3(2π)34EEppqq
pq
Usingsmiilartelopmendevin(5.1)ewget
∫µν3∫d3pµν√1[αα2∗]
Tdx=ϵϵ(pp+pp−m)aa
3ααpp
(2π)4EE
pp
∫3222
dpµν2Ep−p−m∗
=ϵϵaa
3pp
(2π)2E
p
□
(c)Whatistheenergyydensitoftheectorvifeld?
Solution:
Thestress-energytensorisintheformorFenergyydensitµ=0,ν=0
∫3222
00dp002Ep−2p−m∗
T=ϵϵaa
(2π)32Epp
p
Thisesgivtherequiredenergy.ydensit□
5.3.4.(SMIN3.9)eWwilloftendescribeultipletsmofscalarifelds,
Φ=(ϕ1),
ϕ2
whereϕandϕish,eacinthiscase,aaluedreal-vscalarifeldforexample
12
L1∂ΦT∂µΦ−1m2ΦTΦ
2µ2
isacompactyawofdescribingowtfreescalarifeldswithticalidenmasses.ThisLagrangianissymmetric
underthetransformation
Φ→(I−iθX)Φ
whereXissomewnunkno2×2matrix,andθisassumedtobesmall.
CHAPTER5.THEARDANDSTMODEL140
TT
(a)WhatisthetransformationΦ?wShothatΦΦremainstarianvinunderthistransformation.
Solution:
akingTthetransposeofΦewget
ΦT→ΦT(I−iθXT)
TheytitquanΦTΦaftertransformationis
ΦTΦ→ΦT(I−iθXT)·(I−iθX)Φ
(TTT)
=Φ−iθΦX(Φ−iθXΦ)
=ΦTΦ−iθΦTXΦ−iθΦTXTΦ−θ2ΦTXXΦ
=ΦTΦ−iθΦT(X+XT)Φ−O(θ2)
ButthistransformationespreservtheproductΦTΦonlyifXT=−Xsothatthemiddleterm
anishes.v
ΦT→ΦT(1+iθX)
Ineitherofthesecase
TT2T
ΦΦ→ΦΦ−O(θ)≈ΦΦ
ThiswsshothatthistransformationespreservΦTΦ.□
(b)Whatistheedconservtcurreninthissystem?
Solution:
ritingWouttheLagrangianintermsofϕandϕewget
12
1()(∂µϕ)1()(ϕ)
∂ϕ∂ϕ12ϕϕ1
L=2µ1µ2∂ϕ−2m12ϕ
µ22
1µµ12(22)
=(∂ϕ∂ϕ+∂ϕ∂ϕ)−mϕ+ϕ
2µ11µ22212
orFthistransformationthetransformedscalarifeldtselemenofthematrixare
(′)()()()
ϕ1−iθX−iθXϕ(1−iθX)ϕ−iθXϕ
1=00011=001012
′
ϕ−iθX1−iθXϕ−iθXϕ(1−iθX)ϕ
210112101112
′′
usThtheeativderivofϕandϕwithθbecome
12
′′
dϕdϕ
1=−iXϕ−iXϕ2=−iXϕ−iXϕ
dθ001012dθ101112
Theedconservtcurrenwnobecomes
∂Ldϕ′∂Ldϕ′
1+2
∂(∂ϕ)dθ∂(∂ϕ)dθ
µ1µ2
=1(∂µϕ(−iXϕ−iXϕ)+∂µϕ(−iXϕ−iXϕ))(5.2)
210010122011112
Thisesgivexplicitexpressionforedconservtcurrenintermsofmatrixtselemenofwnunknomatrix
X.□
(c)Asewwillsee,fortheparticularcasedescribedtinhisproblem,thetselemenofXare
X=(0i).
−i0
CHAPTER5.THEARDANDSTMODEL141
Computetheedconservtcurrenintermsofϕandϕ.explicitly
12
Solution:
IfthismatrixisentakthenX=0,X=0,X=iandX=−i.Substutingthesein(5.2)
00110110
ewget,
Jµ=1[(∂µϕ)ϕ−(∂µϕ)ϕ]
21221
Thisesgivtheexplicitexpressionofedconservtcurrenforthisparticulartransformationmatrix.□
5.4orkHomewourF
5.4.1.(SMIN4.1)Considerarectangle.
(a)Listallthepossibleuniquetransformationsthatcanbeperformedthatwillevleaitlookingthe
sameasitdid.initially
Solution:
Thepossibletransformationsthatevleatherectanglelookingthesameare
i.vingLeawhereitis(I).
ii.Rotationthrough180◦(R).
iii.FlippingalongtheerticalvaxisthroughmidptsoinofA&BandC&D(F).
y
iv.FlippingalongthetalhorizonaxisthroughmidptsoinofA&CandB&D(F).
x
□
(b)Constructtheultiplicationmtableforourysetoftransformations.
Solution:
Theultiplicationmtableforthetransformationsis
◦IRFF
xy
IIRFF
xy
RRIFF
yx
FFFIR
xxy
FFFRI
yyx□
(c)Doesthissetevhathepropertiesofagroup?
Solution:
romFtheultiplicationmtableitisclearthatthetelemensatisfyclosure.ThetelemenIactsas
the.ytitidenhEactselemenaretheersesvinofes.themselvAndassoyciativitistlyevidened.wfollo
Thisesvprothatthetselemenformagroup.□
5.4.2.(SMIN4.2)Quaternionsareasetofobjectsthatareanextensionofimaginarybumersnexceptthat
therearethreeofthemi,jandk,withtherelations
i2=j2=k2=ijk=−1
.
(a)Constructthesmallestgrouppossiblethattainsconallthequarternions.
Solution:
Closureofthegrouprequiresthatatleast,i,j,kand−1tobethebmemersofthegroup.Since
2
i=i◦i=−1,ican’tbetheytitidenofthegroup.Similarlyjandkcan’tbeytitidenofthe
group.Thatesvlea−1astheonlycandidatefortheytitidenofthegroup.Ifewcansatisfyother
trequiremenofgroup,theni,j,kand−1willformagroupwith−1asthe.ytitiden
CHAPTER5.THEARDANDSTMODEL142
Ifewdeifne−1◦−1=−1,hwhicdoesn’tviolateyanoftheengivts,requiremen−1,orkswasthe
ytitident.elemen
Sincei2=i◦i=−1and−1is,ytitideniybdeifnitionbecomestheersevinofitself.Similarlyj
andkareersesvinofes.themselvSothegroupis
G({−1,i,j,k},◦)
□
(b)Computetheutationcommrelation[j,i].
Solution:
Theutatorcommofagroupisdeifnedas
[j,i]=j−1i−1ji
Wherei−1andj−1aretheersesvinofiandj.respelyectivAlsosinceijk=−1.Multiplyingyb
−1−1
iontheleftesgivjk=iandultiplyingmybkonthetrighesgivij=k.romF(5.4.2a)ew
evhai−1=iandj−1=j
[j,i]=j−1i−1ji=jiji=j(ij)i=j(k)i=(jk)i=ii=−1
Sincetheutatorcommisytitidentelemenofthegroup,thisgroupisabeliansothatthetselemen
ute.comm□
(c)Constructaultiplicationmtableforthequarternions.
Solution:
Theultiplicationmtablebecomes
◦−1ijk
−1−1ijk
ii−1kj
jjk−1i
Thisistherequiredultiplicationmtable.kkji−1□
5.4.3.(SMIN4.6)Expandtheseriese−iθσ2explicitlyandreducetocommontrigonometric,algebraicor
yphergeometricfunctions.
Solution:
TheSU(2)rotationmatrixwithgeneratorσisM(θ)=e−iθσ2.ExpandingitoutasayloraTseries
2
esgiv
θ2θ3θ4
−iθσ2234
e=1−iθσ2−σ+iσ+σ−...
22!23!24!
SincefortheauliPmatricesσ2=1hwhicimpliesthatforodderspwotheauliPmatricesarethe
i
matricesesthemselvandforeneverpwotheyreduceto,ytitidenusthewcanwrite
θ2θ3θ4
e−iθσ2=1−iθσ2−2!+iσ23!+4!−...
θ2θ4θ3
=1−2!+4!+...−iθσ2+iσ23!−...
(θ2θ4)(θ3)
=1−2!+4!+...−iσ2θ−3!+...
=cosθ−iσ2sinθ
CHAPTER5.THEARDANDSTMODEL143
()()
ritingWouttheexplicitmatrixformforytitiden10andσ=0−iewget
012i0
()()
−iθσ2100−i
M=e=cosθ01−ii0sinθ
()
=cosθ−sinθ
sinθcosθ
−iθσ2
Thisistherequired2×2matrixtationrepresenofe.□
5.4.4.(SMIN4.10)Consideraerseunivconsistingofacomplexifelddeifnedybowtcomptsonen
()
Φ=ϕ1
ϕ2
TheLagrangianestaktheform
µ†2†
L=∂Φ∂Φ−mΦΦ.
µ
∗∗
Insomesense,therearefourifeldsatorkwhere,ϕ,ϕ,ϕandϕ.Butforthepurposeofthisproblem,
2211
ouyshouldgenerallythinkΦandΦ†astingrepresentheowttdifferenifelds.Sinceheacisa2−D
ector,vtherearestillfourdegreesoffreedom.
(a)ConsiderarotationinSU(2)inθ1direction(σ).ExpandMasinifniteseries,andexpressasa
x
2×2matrixofonlytrigonometricfunctionsofθ1.
Solution:
−iθσx
TheSU(2)rotationmatrixwithgeneratorσxisM(θ)=e.ExpandingitoutasayloraT
seriesesgiv
θ2θ3θ4
−iθσx234
e=1−iθσx−σ+iσ+σ−...
x2!x3!x4!
SincefortheauliPmatricesσ2=1hwhicimpliesthatforodderspwotheauliPmatricesarethe
i
matricesesthemselvandforeneverpwotheyreduceto,ytitidenusthewcanwrite
θ2θ3θ4
−iθσx
e=1−iθσx−2!+iσx3!+4!−...
θ2θ4θ3
=1−2!+4!+...−iθσx+iσx3!−...
=(1−θ2+θ4+...)−iσx(θ−θ3+...)
2!4!3!
=cosθ−iσxsinθ
()()
ritingWouttheexplicitmatrixformforytitiden10andσ=01ewget
01x10
()()
M=e−iθσx=cosθ10−i01sinθ
0110
=(cosθ−isinθ)
−isinθcosθ
Thisistherequired2×2matrixtationrepresenofSU(2)tingrepresenrotationinθ1direction.□
CHAPTER5.THEARDANDSTMODEL144
(b)erifyVumericallynthatourymatrix(i)isunitaryand(ii)hasatdeterminanof1.
Solution:
kingChecforyUnitarit
MM†=(cosθ−isinθ)·(cosθisinθ)
−isinθcosθisinθcosθ
=(cos2θ+(−isinθ)(isinθ)(icosθsinθ)+(−icosθsinθ))
(−icosθsinθ)+(icosθsinθ)cos2θ+(−isinθ)(isinθ)
(22)
=sinθ+cosθ0
22
0sinθ+cosθ
()
=10=I
01
Thiswsshothematrixis.unitarykingChecfortdeterminan

cosθ−isinθ22
det{M}==cosθcosθ−(−isinθ)(−isinθ)=cosθ+sinθ=1

−isinθcosθ
Thetdeterminanofthematrixisalso1.□
(c)Computeageneralexpressionforthetcurrenassociatedwiththerotationsinθ1.
Solution:
ThisLagrangianisclearlytarianvinunderthetransformationΦ→MΦ.Thegeneratorofhwhic
isσusththeedconservtcurrenis
2
µν†2†µ†2†
L=g∂Φ∂Φ−mΦΦL=∂Φ∂Φ−mΦΦ
νµµ
⇒∂L=gµν∂Φ=∂νΦ⇒∂L=∂µΦ†
∂(∂Φ†)µ∂(∂Φ)
νµ
Jµ=∂LdΦ+dΦ†∂L
∂(∂Φ)dϵdϵ∂(∂Φ†)
µµ
=∂µΦ†(−iσ2Φ)+(iσ2Φ†)∂µΦ
Thisesgivtheexpressionforedconservt.curren=iσ2(−(∂µΦ†)Φ+Φ†∂µΦ)□
5.5orkHomeweFiv
5.5.1.(SMIN5.1)aluateEv
(a){γ0,γ0}
Solution:
{γ0,γ0}=γ0γ0+γ0γ0=2γ0γ0=2(0I)(0I)=2(I0)=2I
I0I00I4×4
Theifnalmatrixisthe4×4yittidenmatrix□
(b)γ2γ0γ2
Solution:
(0σ)(0I)(0σ)(0σ)(−σ0)(0I)
γ2γ0γ2=22=22==γ0
−σ0I0−σ0−σ00σI0
2222
CHAPTER5.THEARDANDSTMODEL145
□
(c)[γ1,γ2]
Solution:
[γ1,γ2]=γ1γ2−γ2γ1
(0σ)(0σ)(0σ)(0σ)
=12−21
−σ0−σ0−σ0−σ0
1221
()()
−σσ0−σσ0
=12−21
0−σσ0−σσ
(12)(21)
[σ,σ]0−2iσ0
=21=3
0[σ,σ]0−2iσ
213
□
5.5.2.(SMIN5.3a)Computetheariousvtracesofthebinationscomogγ-matricesexplicitly
(a)rT(γ0γ0)
Solution:
()()()()
γ0γ0=0I0I=I0=I⇒rTγ0γ0=4
I0I00I4×4
□
(b)rT(γ1γ1)
Solution:
(0σ)(0σ)(−I0)()
γ1γ1=11=⇒rTγ1γ1=−4
−σ0−σ00−I
11
□
(c)rT(γ1γ0)
Solution:
()()()()
0σ0Iσ0
γ1γ0=1=1⇒rTγ0γ1=0
−σ0I00−σ
11
□
5.5.3.(SMIN5.7)Intumquanifeldtheorycalculations,ewwilloftenifnditusefultocmputetheproducts
elik
[u¯(1)γµu(2)],
where1correspondstospin,massand4−tummomenofaparticlestate,and2correspondstosimilar
titiesquanforsecondparticle.orFparticle1.m=m1;p=0,ands=+1/2andforparticle2,m=0;
ˆ
p=pkands=+1/2
z
(a)Calculatetheectorvaluesvof[u¯(1)γµu(2)]forthestateslisted.
Solution:√
22
orFparticle1m=m,p=0⇒E=p+m=mandforparticle2m=0,|p|=p⇒E=
√111z
22µ†0µ
p+m=p,And[u¯(1)γu(2)]=u(1)γγu(2)soewewevha
1
CHAPTER5.THEARDANDSTMODEL146
1
m10†√()
u(1)=√E⇒u(1)=m11010
E+0m
1
0
1√0
m/E+p
m000

u(2)=√E+p=√limu(2)=√
E+pmE+pm→02E
000
Alostheariousvproductofgammamatricesare
(0I)(0I)(I0)(0I)(0σ)(−σ0)
γ0γ0==γ0γ1=1=1
I0I00II0−σ00σ
()()()()(1)(1)
0I0σ−σ00I0σ−σ0
γ0γ2=2=2γ0γ3=3=3
I0−σ00σI0−σ00σ
2233
Usingthesetocalculatetheectorsvewget
theariousvcomptsonenare
†00√†02
u(1)γγu(2)=2m1Eu(1)γγu(2)=0
†03√†01
u(1)γγu(2)=2m1Eu(1)γγu(2)=0
Sotherequiredmatrixis

√
2mE
1
µ0
[u¯(1)γu(2)]=
0
√
2mE
1
□
(b)Dothesameforspinwndostates.
Solution:
Similarlyforthespinwndostatesewget
0
m11†√()
u(1)=√⇒u(1)=m10101
E−00
E
m
1
000
m1√√
m/E−p2E
u(2)=√=limu(2)=
E−p0√0m→00
E−pE−p0
m
Similarlyewget

√
2mE
1
µ0
[u¯(1)γu(2)]=
0
√
2mE
1
□
CHAPTER5.THEARDANDSTMODEL147
(c)Calculatetheectorvofaluesvfors2=−1/2
5.5.4.(SMIN5.12)orFthesingle-particleDiracequationHamiltonian
ˆi
H=−iγ∂+m
i
(a)ComputetheutatorcommofHamiltonianoperatorwiththezcomptonenoftheangulartummomen[]
ˆˆ
operatorH,L2,where
ˆ
L≡r×p
Solution:
ritingW−i∂=pewget
ii
ˆii
H=−iγ∂+m=γp
ii
SincemisscalaritutescommwiththeLoperatorsoewget
[ˆˆ][i][123]
H,L=γp,L=γp+γp+γp,xP−yP
zizxyzxx
Butusingtheutationcommrelations[x,p]=iδand[p,p]=0ewget
ijijij
[1][1][1]11
γp,xP−yP=γp,xp−γp,yp=γ(−ip)=−iγp
[xyx][xy][xx]yy
γ2p,xP−yP=γ2p,xp−γ2p,yp=γ2(ip)=iγ2p
[yyx]yyyxxx
γ3p,xP−yP=0
zyx
usThtheutationcommbecomes
[ˆˆ]12
H,L=−iγp+iγP
zyx
hWhicistherequiredutationcommrelationofHamiltonianandthezcomptonenofL.□
(b)wnoconsiderthespinoperator
1()
ˆσ0
S=20σ.
ˆz
ComputethezcomptonenofSu(p)
−
Solution:
ˆz
TheoperatorSandthestateu(p)are
−
0
1000
1m1
ˆz0−100
S=u(p)=√
−0
20010E−p
000−1E−p
m
1
Applyyingtheoperatorsimplyscalesyb2andlfipsthesignofseconandlastcomptonenyielding
ˆz1
Su(p)=−u(p)
−2−
Thisistherequiredstateafteroperation.□
CHAPTER5.THEARDANDSTMODEL148
[ˆˆ]
(c)ComputeH,S2.
Solution:
ritingWHamiltonianas
ˆii
H=−iγ∂+m=γp
ii
Sincetheoperatorputecommwiththe4×4matricesSandγ
i
[i][123][1][2][3]
γp,S=γp+γp+γp,S=pγ,S+pγ,S+pγ,S
izxyzzxzyzzz
Usingtheutationcommrelations[S,γ1]=iγ2and[S,γ2]=−iγ1ewobtain
zz
[ˆˆ]21
H,S=−iγp+iγp
zxy
hWhicistherequiredutationcommrelation.□
(d)Cotomparinouryers,answederivaedconservytitquanforthefreefermions.
Solution:[]
ˆ
ClearlyfromowtpartseabvoH,L+S=0usththeedconservoperatorisL+S.orFfree
fermionofstateψ(p)theedconservytitquanis
(L+S)ψ(p)
Thealueveigenofthisoperatoresgivtheedconserv.ytitquan□
5.6orkHomewSix
5.6.1.(SMIN6.2)Suppose,trarycontoourorkwinthishapter,cthatthephotonhadaeryvsmallmass,
−4
10V.eWhatouldwtheeeffectivrangeoftheelectromagneticforcebe?Expressouryeranswin
meters.ximatelyApprowhotligh(inkilograms)ouldwthephotonneedtobehsucthatearth-scale
magneticifeldsouldwstillbemeasurable?
Solution:
Theteractioninifeldisximatelyapproengivyb
E≈e−mr
tin4πr
−4−40
orFameasuralbleifeldE≊1sowithM=10Ve≈1.8×10kgewevha
−mr
e≊1⇒r≈0.0795m
4πr
−96
ThemagneticifeldofearthisB=25×10Tforthistobemeasurableinearthscaler≈6.4×10m
ewagainesolvformintheequation
B≈e−mr
tin4mπr
6
−m6.4×10
25×10−9≈e⇒m≈1.9×10−7kg
6
4mπ6.4×10
Themassofphotonhastobeeryvwloinorderforthistobemeasured.□
5.6.2.(SMIN6.6)Inclassicalelectrodynamics,radiationispropagatedalongthetingynoPector,v
S=E×B,
CHAPTER5.THEARDANDSTMODEL149
anordinaryector.3-vExpressthecomptsonenofSiintermsofcomptsonenofFµνinassimpliifed
formaspossible.
Solution:
Inindexnotationthecrossproductsofowtectorvis
Si=εEjBk
ijk
SincethemagneticifeldandelectricifeldcomptsonenintermsofthearadyFtensortselemenare
F0i=EiFij=Bk
ThetingynoPectorvbecomes
i0jijF02F12−F03F31
S=εFF⇒S=F03F23−F01F12
ijk
F01F31−F02F23
ThisistherequiredtingynoPectorvintermsofthecomptsonenofaradyFtensor.□
5.6.3.(SMIN6.9)Inelopingdevtheowtpolarization-statesmodelforthephotonewlieduponU(1)gauge
ariance,vinhwhicinturndependsonamasslessphoton.eWwknothataspin-1particlearesupposed
toevhathreespinstates,butewclaimedthatthethirdstateaswedwalloswybthebCoulomgauge
condition.Lets’happroacthequestionofthreestatesybassumingthatthephotondoesevhamassand
obeyslagrangian
1µν12µ
L=−FF+MAA
4µν2µ
(a)riteWtheEuler-Lagrangeequationfortheemassivphotonifeld.
Solution:
SinceybdeifnitionthearadyFtensoristhetisymmetricantensorformedybariousvesativderivof
thecomptsonenofAµ.
F=∂A−∂AFµν=∂µAν−∂νAµ
µνµννµ
Theproductterminthelagrangianis:
FµνF=(∂µAν−∂νAµ)(∂A−∂A)
µνµννµ
=∂µAν∂∂−∂µAν∂A−∂νAµ∂A+∂νAµ∂A
µννµµννµ
µνµν
=2(∂A∂A−∂A∂A)
µννµ
ritingWoutthelagrangianintermsofthesecomptsonenewget
1µνµν12µ
L=−(∂A∂A−∂A∂A)+MAA
2µννµ2µ
usThtheEuler-Lagrangeequationsbecome
∂(∂L)=∂L
µ∂(∂A)∂A
µνν
1µννµ12ν
−∂(∂A−∂A)=MA
2µ2
−∂Fµν=M2Aν
µ
ThesearetherequiredEuler-Lagrangeequations.□
CHAPTER5.THEARDANDSTMODEL150
(b)Letthephotonifeldetaktheformofasingleplanee:vaw
µµ−ip·x
A=εe.
ExpresstheEuler-Lagrangeequationsasdotproductsofpandεwithesthemselvandwithheac
other.wShothattheersetransvevawconditiondropsoutofthedispersionrelationregardlessof
whethertheifeldhasmass.
Solution:
orFthisifeldthearadyFtensorbecomes
µνµννµν−ip·xµ−ip·xνµ−ip·x
F=∂A−∂A=−ipεe+ipεe=−i(pε−pε)e
µνµν
SotheEuler-Lagrangeequationsbecome
−∂Fµν=M2Aν
[µ]
νµµ−ip·x2ν−ip·x
−−i(pε−pε)(−ip)e=Mee
µν
µνµµ2ν
(ppε−pεp)=Mε
µν
ν2ν
(p·pε−pε·p)=Mε
ν
RegardlessofthemassthecoteiffcienofpontheLHSustmbe0sothedotproductε·p=0.□
ν
(c)Whatisthethirdpossiblepolarization-stateforaemassivphotonpropagatinginthez-direction?
Solution:
orFthisectorvifeld,p·p=M2andε·p=0.orFaparticlevingmoinzdirectionwithmotumen
µ(E00p)T
pandEnergyEthetummomenector4-visp=z.Thelinearlyindeptendenε
z
ectorvsatisfyingtheserelationsapartfromtheonesengivis

ppE
zz
000
ε=asε·p=·=pE−Ep=0
30300zz
EEp
z
222
Sincetheinnerproductofεwithitselfisp−E=−M,ewcouldhocosenormalizationfactor
i/Mforε.□
(d)Whataretheelectricandmagneticifeldsoftheemassivphotonifeldinthisthirdpolarization
state?Whathappenstotheoseifeldsform=0?
Solution:
wNotheElectricandmagneticifeldsaresimplythecomptsonenofaradyFtensor
i0i(i0)ip·x
E=F=−ipε−pεe
(0i)
0110ip·x
E=F=−ipε−pεe=0
x(01)
0220ip·x
E=F=−ipε−pεe=0
y(02)
0330ip·x22−ip·x2−ip·x
E=F=−ipε−pεe=−i(E−p)e=−iMe
z03
ij(jj)ip·x
B=F=−ipε−pεe
k(ii)
2332ip·x
B=F=−ipε−pεe=0
x(23)
3114ip·x
B=F=−ipε−pεe=0
y(31)
1221ip·x
B=F=−ipε−pεe=0
z12
[2−ip·x]
SoE=−iMeˆzandB=0.IfM=0thentheElectricifeldanishesvasell,wsoboththe
ifeldsanish.v□
CHAPTER5.THEARDANDSTMODEL151
5.6.4.(SMIN6.10)Consideranelectroninaspinstate
()
ϕ=a
b
inamagneticifeldB0tedorienalongthez-axis.eWwillcalculatethearmorLquencyerFybhwhicthe
electronprecesses.
(a)urnTtheteractioninHamiltoniantoinaifrstordertialdifferenequationintime.
Solution:
Theengivteractioninhamiltonianis
qB()
ˆe010
H=−
tin2m0−1
ritingWthehamiltonianasi∂=i∂
0∂t
∂()qB()
ia=−e0a
∂tb2m−b
Sothetialdifferenequationsare
∂aqB∂bqB
i=−e0ai=e0b
∂t2m∂t2m
Thesearetherequiredtialdifferenequations.□
(b)eSolvthetialdifferenequationinparta.WhatistherequencyFofoscillationofthephaseeencdiffer
eenbwettehowtcompts?onen
Solution:
Thesolutionsare
iqeBiqeB
0t−0t
a=ae2mb=be2m
00
Thephasedifferenceis
(iqBt)(iqBt)iqBt
φ=e0−−e0=e0
2m2mm
Thefrequencyofoscillationis
iqB
e0
m
Thisistherequired.frequency□
Chapter6
StatisticalhanicsMec
6.1orkHomewOne
6.1.1.Aparticularsystemobeysowtequationsofstate
23
T=3As(thermalequationofstate),P=Ashanical(mecequationofstate).
vv2
WhereAisat.constan
(a)Findµasafunctionofsandv,andthenifndthetalfundamenequation.
Solution:
enGivPandTthetialdifferenofheacofthemcanbecalculatedas
223
dT=6Asds−3Asdv⇒sdT=6Asds−3Asdv
vv2vv2
2323
dP=3Asds−2Asdv⇒vdP=3Asds−2Asdv
v2v3vv2
TheGibbs-Duhemrelationinenergytationrepresenwsallotocalculatethealuevofµ.
dµ=vdP−sdT
2323
=3Asds−2Asdv−6Asds+3Asdv
vv2vv2
=−[3As2ds−As3dv]=−d(As3)
vv2v
As3
Thiscanbetiifedidenasthetotaleativderivofvsotherel
(3)3
dµ=−dAs⇒µ=−As+k
vv
Wherekisarbitraryt.constaneWcanplugthiskbactoEulerrelationtoifndthetalfundamen
equationas.
u=Ts−Pv+µ
3333
=3As−As−As+k=As+k
vvvv
3
SothetalfundamenequationofthesystemisAs+k.□
v
152
CHAPTER6.TISTICALASTMECHANICS153
(b)Findthetalfundamenequationofthissystemybdirecttegrationinofthemolarformofthe
equation.
Solution:
Thetialdifferenformofternalinenergyis
du=Tds−Pdv
23
=3Asds−Asdv
vv2
3
As
Asbeforethisisjustthetotaltialdifferenofvsotherelationleadsto
(3)3
du=dAs⇒u=As+k
v2v
Thiskshouldbethesamearbitrarytconstanthatewgotinthepreviousproblem.□
6.1.2.ThetalfundamenequationofsystemAis
S=C(NVE)1/3,
andsimilarlyforsystemB.Theowtsystemareseparatedybrigid,impermeable,adiabaticall.wSystem
−63−63
Ahasaolumevof9×10mandabumermolenof3moles.SystemBhasolumevof4×10mand
amolebumernof2moles.Thetotalenergyofthecompositesystemis80J.
(a)PlottheytropenasafunctionofE/(E+E).
AAB
Solution:
Sincethetotalenergyofthesystemis80JthesumEA+EB=80J.Thetotalytropenofsystem
canbewrittenas
[{()}1{()}1]
EA3EA3
S=CNV·80·+NV·80·1−
11E+E22E+E
ABAB
ThegraphofytropEnSvstheenergyfractioniswnshoinFigure6.1.□
0.16
0.14
0.12
Units)0.1
(Arbitrary
En
trop
y
00.10.20.30.40.50.60.70.80.91
E/(E+E)
AAB
Figure6.1:PlotofytropEnvsenergyfraction.
(b)Iftheternalinallwiswnomadediathermalandthesystemisedwallotocometoequilibrium,
whataretheternalinenergiesofheacoftheindividualsystems?
CHAPTER6.TISTICALASTMECHANICS154
Solution:
Iftheallwismadediathermalandtheenergycanwlfothetotalenergyoftheremainstconstan
E=E+E.akingTtialdifferenonbothsidesewgetdE=dE+dE=0.Sincethereisno
ABAB
hangecinolumevorthebumernofmoleculesdV=0anddN=0.usThthetialdifferenrelation
ofytropenreducestodS=1dE.Theeadditivpropyertwsalloustowrite
T
11dEdE11
dS=dS+dS=dE+dE⇒A=−B⇒=
ABTATBTTTT
ABABAB
ThetitiesquanTandTforheacsystemscanbefromthetalfundamenequationusth
AB
()()()()
1∂SCNV1/31∂SCNV1/3
=A=AA=B=BB
T∂E3E2T∂E3E2
AAABBB
TheseexpressionscanbesimpliifedwndotogetandnotingEA+EB=80Jewevhaowtlinear
expressions
√NV
E=BBEE+E=80⇒E=51.93JE=28.07J
BNVAABAB
AA
SotheafterequilibriumtheternalinenergyofsystemAisE=51.93JandforsystemBitis
A
EB=28.07J.□
(c)tCommenontherelationeenbwettheseowtresults.
Solution:
ThegraphofSvsEA/(EA+EB)isedewsktothetrighanditsummaximisatEA/(EA+EB)=
0.64.TheifnalenergyofsystemAis51.93hwhicis0.64·80.usThtheifnalifnalenergiesarehsuc
thatthetotalifnalytropenisum.maxim□
6.1.3.Animpermeable,diathermal,andrigidpartitiondividesatainercontoinowtolumes,vsubofolumev
nVandmV.Theolumesvsubtaincon,respelyectivnmolesofHandmmolesofNe,heactobe
002
consideredasasimpleidealgas.ThesystemistainedmainatatconstantemperatureT.Thepartition
issuddenlyrupturedandequilibriumisedwallotore-establish.Findthehangecinytropenofthe
system.wHoistheresultrelatedtotheytrop“enofmixing”?
Solution:
ThetalfundamenequationofidealgascanbewrittenasU=cNRTandtlyalenequivasPV=NRT
usththetitiesquan
U=cNRT⇒1=cRN=cR
TUu
PV=NRT⇒P=RN=R
TVv
Sineitistrueforheacofthesesystemsewcanwrite
1PcRR(u)(v)
ds=du+dv⇒ds=du+dv⇒s=s+cRlnf+Rlnf
TTuv0uv
ii
Theinitialandifnalmolarolumevforheacofthegasesis
v=nV0=Vv=mV0=V
ihn0inm0
(m+n)V(m)(m+n)V(n)
v=0=1+Vv=0=1+V
fhnn0fnmm0
CHAPTER6.TISTICALASTMECHANICS155
Alsosincethetemperatureofsystemistconstanandthatnoheatwslfoinoroutofthecomposite
systemthehangecinternalinenergyiszerousthu=uforbothusththetotalifnalytropenbecome
if
(u)(v)(m)
s=s+cRlnfh+Rlnfh=s+Rln1+
h0huv0hn
ihih
SimilarlyforNetheifnalytropenofsystemis
(n)
s=s+Rln1+
n0nm
Thetotalhangecinytropenis
∆S=ms+ns−(ms+ns)
n(h)0n0h()
=mRln1+n+ms+nRln1+m+ns−(ms+ns)
mn0nh00n0h
(m)(n)
=nln1+n+mln1+m
ThisisexactlyequaltotheytropEnofmixing.□
6.1.4.Theytropenofkbblacodyradiationisengivybtheulaform
S=4σV1/4E3/4,
3
whereσisat.constan
(a)Determinethetemperatureandthepressureoftheradiation.
Solution:
TheytropEnrelationcanbeertedvintoget
E=(81S4)1/3
256σ4V
tiatingDifferenthiswithrepsecttoVtogetthepressureesgiv
()√4
∂E36S3
P=−∂V=44
8V3σ3
Thetemperaturesimilarlyis
()√√
∂E363S
T==√4
∂S3
2Vσ3
usThthetemperatureandpressurearedetermined.□
(b)evProthat
PV=E
3
Solution:
SubstutingS=4σV1/4E3/4tointhepressureexpression
3
√4√()4
3341/43/43
6S363σVE,EE
P=44=44=3V⇒PV=3
8V3σ38V3σ3
usThPV=Eisedvproasrequired.□
3
CHAPTER6.TISTICALASTMECHANICS156
6.1.5.orFaparticularsystem,itisfoundthate=(3/2)PvandP=AvT4.FindthemolarGibbsptialoten
andmolarHelmholtzptialotenforthesystem.
Solution:
Sincethereareowtequationsofstateewcanmodifythemtoexpresstheetensivinparametersas
P=2eT=(P)1/4=(2u2)1/4
3vAv3Av
ThesecanbeusedinytropEntialdifferenequationtoget
ds=1de+Pdv
TT
()()
21/431/4
=3Avde+8Aedv
2e27v2
Theeabvoexpressioncanberecocnizedasthetotaltialdifferenof(128Av2e3)1/4
27
()()
128Av2e31/4128Av2e31/4
ds=d27⇒s=27+s0
MultiplyingthourghybNtogetthenonmolartitiesquanewget
()
128AV2E31/4
S=27N+S0
Thiseabvorelationcanbeertedvintogetthetalfundamenenergytation.represenSoewget
[]
27N1/3
E=(S−S)4
128AV20
ThisesservasthetalfundamenEnergyrelationhwhiccanbeusedtoifndtheGibbsandHelmholtz
ptial.oten
eWcanwnoifndtheetensivinparametersTandPintermsoftheeextensivparametersas
()[]()
∂E∂27N1/31N1/3
T==(S−S)4=√(S−S)1/3
∂S∂S128AV2032AV20
eWcanertvintoifndSasafunctionofTso
2AV2
S=S+T3
0N
SimilarlyewcanifndtheetensivinparameterPas
()()
∂E1S−S1/3
P=−=√N1/30
∂V232AV5
ThiscanagainbeertedvintogetVasafunctionofP
[]
1N(S−S)41/5
V=0
16AP3
EquippedwiththesefunctionsewcanwnoifndtheGibbsptialotenas
G=E−TS+PV
[]()[]
27N1/32AV21N(S−S)41/5
=(S−S)4−T·S+T3+P·0
128AV200N16AP3
=(A3P2T12V8)1/5−AT4V2
N32N
CHAPTER6.TISTICALASTMECHANICS157
ThisesgivtheGibbstialotenPwnotheHelmholtzptialotencanbesimilarlyfoundas
F=E−TS
[]()
27N1/32AV2
43
=(S−S)−T·S+T
128AV200N
AT4V2
=−2N
usThtheHelmholtzptialotenis−AT4V2.□
2N
6.2orkHomewowT
∑
6.2.1.wShothatforaengivNwithNp=1,theytuncertainfunctionS({p}),estakitsummaximaluev
riii
whenp=1foralli,thatisS({p})=A(N)
iNi
Solution:∑
TheytuncertainfunctionisS({p})=−Cplnp.eWtanwtomaximizethisfunctionsubjectto
∑iiii
thetconstrainp=1.UsingLagrange’sultipliermmethodtoifndtheumextremoffunction,ew
ii∑
candeifneanewfunctionS−λ(p−1)
ii
′[()]
∂S=∂−C∑plnp−λ∑p−1
∂p∂piii
jjii
()()
=−C∑δlnp+1pδ−λ∑δij
ijipiij
iji
=−C(lnp+1)−λ(6.1)
j
Butforumextremconditionofthisfunctionthepartialeativderivwithrespecttoeryevpjshould
anish.vusThewget
λ[λ]
lnp=−−1⇒p=exp−−1
jCjC
TheRHSofeabvoexpressionisat,constanletscallthattconstanMsop=MforsometconstanM
i
butsinceyprobabilithastoadd1ewget
∑p=1;⇒∑M=1⇒MN=1⇒M=1
jN
jj
Substutingthiskbacewget
p=1
jM
usThtheytuncertainfunctionestakitummaximaluevwhenp=1/Nforallp□
ii
6.2.2.Consideraurnproblemdiscussedinclass:Anurnisiflledwithballs,heacbumerednn=0,1,2....
Theeragevaaluevofnis⟨n⟩=2/7.Calculatetheprobabilitiesp,pandphwhicyieldtheummaxim
01⟨⟩2
.ytuncertainFindtheexpectationalue,vbasedontheseprobabilitiesn3−2⟨n⟩.
Solution:
Theexpectationaluevofnisengivyb
⟨n⟩=p·0+p·1+p·2⇒p+2p=2/7
01212
CHAPTER6.TISTICALASTMECHANICS158
Thisisoneofthetsconstrainformaximizingtheytuncertainfunction,theothertconstrainequationis
p+p+p=1.Usingtheseasewcalculatedin(6.2)ewevha
012
S′=S−α(p+2p−2/7)−β(p+p+p−1)
C12012
akingTeativderivwithrespecttoαandβandequatingtozeroesgiv
lnp+1−β=0
0
lnp+1−α−β=0
1
lnp+1−2α−β=0
2
Thesethreeequationsalongwithowttconstrainequationformeifvequationineifvwnonunkp,p,p,α,β.
012
eWcanesolvthisequationtogettheumericnaluevoftheparameters.Solvingfortheparametersew
get
p=15p=4p=1
021121221
wNotherequriedfunctionis
⟨3⟩33
n−2⟨n⟩=p·0+p·1+p·2−⟨n⟩
012
=p+8p−22
127
=4+81−4
21217
=0
Therequriedaluevis0
□
6.2.3.Assumingthe,ytropenSandthebumernofmicrostates,Ωofaysicalphsystemarerelatedthrough
anarbitraryfunctionalformS=f(Ω),wshothattheeadditivharactercofSe(extensivparameter)
andtheeultiplicativmparameterΩmeaningΩ=Ω1,Ω2,...,isthebumernofmicroscopicstatesfora
subsystemnecessarilyrequirethatthefunctionF(ω)isoftheform
S=kln(Ω)
wherekisaersal)(univt.constanTheformaswifrstwrittenwndoybMaxPlank.
Solution:
enGivtheeultiplicativmparameterΩ=Ω1·Ω2...Ωr.Theeextensivparameterasafunctionofthis
parameterhwhicisaeadditivfunctionbeS.usThewevha
S(Ω·Ω...Ω)=S(Ω)+S(Ω)+...+S(Ω)
12r11r
r
S(Ω)=∑S(Ω)
j
j
tiatingDifferenwithrespecttoΩionbothsides
r
dS(Ω)=d∑S(Ω)
dΩdΩj
iij
r
dS(Ω)dΩ=∑dS(Ωj)δ
dΩdΩdΩij
iji
CHAPTER6.TISTICALASTMECHANICS159
ButsincetheeativderivofproductΩ=ΠjΩjwithrespecttoΩiisjusttheproductwithoutthat
parameterdΩ=Πj̸=iΩj.MultiplyingbothsidesybΩiewget
dΩ
i
Ω(ΠΩ)dS(Ω)=ΩdS(Ωi)⇒ΩdS(Ω)=ΩdS(Ωi)
ij̸=ijdΩidΩidΩidΩi
Buttheexpression1dx≡d(ln(x))recognizingsimilarexpressioninbothsidesoftheyequaliteabvo
x
ewget
dS(Ω)=dS(Ωi)
d(lnΩ)d(lnΩi)
TheexpressioninRHSisindeptendenofexpressionont.righSincetheproductoftheparameterscan
beariedvwhilestilleepingkoneoftheparametersΩit.constanSotheexpressioncanonlybeequalto
heacotheriftheyareequaltoat.constan
dS(Ω)=k⇒dS(Ω)=kd(lnΩ)
d(lnΩ)
tegratingInthisexpressionewget
S(Ω)=klnΩ
hWhicistherequiredexpression.□
6.2.4.wShothatinlnx≤x−1,ifforallrealpeositivx.Theyequalitholdsforx=1.
Solution:
Rearrangingtheequationlnx−x≤−1.Letusdeifneafunctiong(x)=lnx−x.tiatingDifferenthis
functionwithrespecttoxewget
g′(x)=1−1=1−x=−x−1
xxx
Sinceforallpeositivaluesvofxi.e.,∀x>0ewevha
x−1<x⇒x−1<1⇒g′(x)=−x−1<−1
xx
letf(x)=ln(1+x)−xsothatf(0)=0.
Clearlyx
f′(x)=−1+x
andhenceg′(x)>0if−1<x<0andf′(x)<0ifx>0.Itwsfollothatthatf(x)inincreasingin
(−1,0]anddecreasingin[0,∞).usThewevhaf(x)<f(0)if−1<x<0andf(x)<f(0)ifx>0.
Itusthwsfollothatf(x)≤f(0)=0forallx>−1andthereisyequalitonlywhenx=0.Soewcan
write
ln(1+x)≤x∀x≥−1
Sincexisjustaydummariablevewcantransformx→x−1toget
ln(x)≤x−1∀x≥0
Thiscompletestheproof.□
CHAPTER6.TISTICALASTMECHANICS160
6.2.5.evProthatlogX=logX.terpretInthemeaningof
2log2
S=−∑plog(p)
i2i
i
.
Solution:
Lety=log2X.Raisingbothsidesto2aesgivus
ylog2xy
2=2⇒2=x
akingTlogarithmonbothsidewithrespecttobase110ewget
ylogX
logX=log(2)⇒logX=ylog2⇒y=log2
Butybourassumptiony=log2Xusthewevha
logX=logX
2log2
Indigitalelectronicsandininformationtheorywheretheytrepresenthesignalinformationin,binary
thelogarithmofabumernwithrespectto2esgivthetotalbumernofbitsrequiredtotrepresenthe
bumer.nMultiplyingthebumernofbitslog2Nybtheyprobabilitofthebumernesgivthetotalerageva
bumernofbitsrequired.
SothentheytropenfunctionS=−∑plog(p)tsrepresentheinfromationttenconofthebinary
ii2i
signal.□
6.3orkHomewThree
∑
6.3.1.wShothatforaengivNwithNp=1,theytuncertainfunctionS({p}),estakitsummaximaluev
riii
whenp=1foralli,thatisS({p})=A(N)
iNi
Solution:∑
TheytuncertainfunctionisS({p})=−Cplnp.eWtanwtomaximizethisfunctionsubjectto
∑iiii
thetconstrainp=1.UsingLagrange’sultipliermmethodtoifndtheumextremoffunction,ew
ii∑
candeifneanewfunctionS−λ(p−1)
ii
′[()]
∂S=∂−C∑plnp−λ∑p−1
∂p∂piii
jjii
()()
=−C∑δlnp+1pδ−λ∑δ
ijipiijij
iji
=−C(lnp+1)−λ(6.2)
j
Butforumextremconditionofthisfunctionthepartialeativderivwithrespecttoeryevpjshould
anish.vusThewget
λ[λ]
lnp=−−1⇒p=exp−−1
jCjC
TheRHSofeabvoexpressionisat,constanletscallthattconstanMsop=MforsometconstanM
i
butsinceyprobabilithastoadd1ewget
∑p=1;⇒∑M=1⇒MN=1⇒M=1
jN
jj
CHAPTER6.TISTICALASTMECHANICS161
Substutingthiskbacewget
p=1
jM
usThtheytuncertainfunctionestakitummaximaluevwhenp=1/Nforallp□
ii
6.3.2.Consideraurnproblemdiscussedinclass:Anurnisiflledwithballs,heacbumerednn=0,1,2....
Theeragevaaluevofnis⟨n⟩=2/7.Calculatetheprobabilitiesp,pandphwhicyieldtheummaxim
01⟨⟩2
.ytuncertainFindtheexpectationalue,vbasedontheseprobabilitiesn3−2⟨n⟩.
Solution:
Theexpectationaluevofnisengivyb
⟨n⟩=p·0+p·1+p·2⇒p+2p=2/7
01212
Thisisoneofthetsconstrainformaximizingtheytuncertainfunction,theothertconstrainequationis
p+p+p=1.Usingtheseasewcalculatedin(6.2)ewevha
012
S′=S−α(p+2p−2/7)−β(p+p+p−1)
C12012
akingTeativderivwithrespecttoαandβandequatingtozeroesgiv
lnp+1−β=0
0
lnp+1−α−β=0
1
lnp+1−2α−β=0
2
Thesethreeequationsalongwithowttconstrainequationformeifvequationineifvwnonunkp,p,p,α,β.
012
eWcanesolvthisequationtogettheumericnaluevoftheparameters.Solvingfortheparametersew
get
p=15p=4p=1
021121221
wNotherequriedfunctionis
⟨3⟩33
n−2⟨n⟩=p·0+p·1+p·2−⟨n⟩
012
=p+8p−22
127
=4+81−4
21217
=0
Therequriedaluevis0
□
6.3.3.Assumingthe,ytropenSandthebumernofmicrostates,Ωofaysicalphsystemarerelatedthrough
anarbitraryfunctionalformS=f(Ω),wshothattheeadditivharactercofSe(extensivparameter)
andtheeultiplicativmparameterΩmeaningΩ=Ω1,Ω2,...,isthebumernofmicroscopicstatesfora
subsystemnecessarilyrequirethatthefunctionF(ω)isoftheform
S=kln(Ω)
wherekisaersal)(univt.constanTheformaswifrstwrittenwndoybMaxPlank.
Solution:
CHAPTER6.TISTICALASTMECHANICS162
enGivtheeultiplicativmparameterΩ=Ω1·Ω2...Ωr.Theeextensivparameterasafunctionofthis
parameterhwhicisaeadditivfunctionbeS.usThewevha
S(Ω·Ω...Ω)=S(Ω)+S(Ω)+...+S(Ω)
12r11r
r
S(Ω)=∑S(Ω)
j
j
tiatingDifferenwithrespecttoΩionbothsides
r
dS(Ω)=d∑S(Ω)
dΩdΩj
iij
r
dS(Ω)dΩ=∑dS(Ωj)δ
dΩdΩdΩij
iji
ButsincetheeativderivofproductΩ=ΠjΩjwithrespecttoΩiisjusttheproductwithoutthat
parameterdΩ=Πj̸=iΩj.MultiplyingbothsidesybΩiewget
dΩ
i
Ω(ΠΩ)dS(Ω)=ΩdS(Ωi)⇒ΩdS(Ω)=ΩdS(Ωi)
ij̸=ijdΩidΩidΩidΩi
Buttheexpression1dx≡d(ln(x))recognizingsimilarexpressioninbothsidesoftheyequaliteabvo
x
ewget
dS(Ω)=dS(Ωi)
d(lnΩ)d(lnΩi)
TheexpressioninRHSisindeptendenofexpressionont.righSincetheproductoftheparameterscan
beariedvwhilestilleepingkoneoftheparametersΩit.constanSotheexpressioncanonlybeequalto
heacotheriftheyareequaltoat.constan
dS(Ω)=k⇒dS(Ω)=kd(lnΩ)
d(lnΩ)
tegratingInthisexpressionewget
S(Ω)=klnΩ
hWhicistherequiredexpression.□
6.3.4.wShothatinlnx≤x−1,ifforallrealpeositivx.Theyequalitholdsforx=1.
Solution:
Rearrangingtheequationlnx−x≤−1.Letusdeifneafunctiong(x)=lnx−x.tiatingDifferenthis
functionwithrespecttoxewget
g′(x)=1−1=1−x=−x−1
xxx
Sinceforallpeositivaluesvofxi.e.,∀x>0ewevha
x−1<x⇒x−1<1⇒g′(x)=−x−1<−1
xx
letf(x)=ln(1+x)−xsothatf(0)=0.
Clearlyx
f′(x)=−1+x
CHAPTER6.TISTICALASTMECHANICS163
andhenceg′(x)>0if−1<x<0andf′(x)<0ifx>0.Itwsfollothatthatf(x)inincreasingin
(−1,0]anddecreasingin[0,∞).usThewevhaf(x)<f(0)if−1<x<0andf(x)<f(0)ifx>0.
Itusthwsfollothatf(x)≤f(0)=0forallx>−1andthereisyequalitonlywhenx=0.Soewcan
write
ln(1+x)≤x∀x≥−1
Sincexisjustaydummariablevewcantransformx→x−1toget
ln(x)≤x−1∀x≥0
Thiscompletestheproof.□
6.3.5.evProthatlogX=logX.terpretInthemeaningof
2log2
S=−∑plog(p)
i2i
i
.
Solution:
Lety=log2X.Raisingbothsidesto2aesgivus
ylog2xy
2=2⇒2=x
akingTlogarithmonbothsidewithrespecttobase110ewget
ylogX
logX=log(2)⇒logX=ylog2⇒y=log2
Butybourassumptiony=log2Xusthewevha
logX=logX
2log2
Indigitalelectronicsandininformationtheorywheretheytrepresenthesignalinformationin,binary
thelogarithmofabumernwithrespectto2esgivthetotalbumernofbitsrequiredtotrepresenthe
bumer.nMultiplyingthebumernofbitslog2Nybtheyprobabilitofthebumernesgivthetotalerageva
bumernofbitsrequired.
SothentheytropenfunctionS=−∑plog(p)tsrepresentheinfromationttenconofthebinary
ii2i
signal.□
6.4orkHomewourF
6.4.1.ConsideranN−dimensionalsphere.
(a)IfaptoinishosencatrandominanN−dimensionalunitsphere,whatistheyprobabilitofit
fallinginsidethesphereofradius0.99999999?
Solution:
TheyprobabilitofaptoinfallinginsideaolumevofradiusrwithinasphereofradiusRisengiv
yb
p=V(r)(6.3)
V(R)
CHAPTER6.TISTICALASTMECHANICS164
100
n=2
n=5
80
n=24
n=54
60
40
V
olume
of20
P
ercen
tage
0
00.10.20.30.40.50.60.70.80.91
Radius
whereV(x)istheolumevofsphereofradiusx.TheolumevofNdimensionalsphereofradiusx
is
πn/2
V(x)=()xn
Γn+1
2
Theprogressionofolumevfortdifferenradius.
Usingthisin(6.3)ewobtain
p=(r)n(6.4)
R
ThisesgivtheyprobabilitofaparticlefallingwithinaradiusrinaNdimensionalsphereofradius
R.□
(b)aluateEvouryeranswforN=3andN=NA(theogadrovAbNumer)
Solution:
23
orFr=0.999999andN=3andN=N=6.023×10ewget
A
()()23
0.99999930.9999996.023×10
p==0.999997000003p==0.0000000000000
3N
1A1
Theyprobabilitofaparticlefallingwithintheradiusnearly1inhighero-dimensionalwtsphereis
anishninglyvsmall.□
(c)Whatdotheseresultsysaaboutthealenceequivofthedeifnitionsofytropenintermsofeitherof
thetotalphasespaceolumevoftheolumevofoutermostenergyshell?
Solution:
ConsideringaphasespaceolumevboundedybE+∆where∆≪E.Theytropenofsystem
boundedybtheE+∆andtheoutermostshellΣ(E+∆)−Σ(E),
SE=kln(Σ(E+∆)),S∆=kln(Σ(E+∆)−Σ(E))
3N3N
hh
Subtractingtoseethediffernceewget
SE−S∆=kln(1−Σ(E))≤−Σ(E)
Σ(E+∆)Σ(E+∆)
Butforlargedimension,therationΣ(E)≪0.Soewobtain
Σ(E+∆)
S−S≊0S≊S
E∆e∆
CHAPTER6.TISTICALASTMECHANICS165
Thiswsshothattheytropentermsinofoutrmostshellolumevandthetireenolumevarealmost
thesame.□
6.4.2.AharmonicoscillatorhasaHamiltonianenergyHrelatedtoitstummomenPanditstdisplacemenq
ybtheequation
p2+(Mωq)2=2MH
WhenH=U,atconstan,energyhetcskthepathofthesystemino-dimensionalwtphasespace.
Solution:
Thephasespacetrajectorycanberearrangedtoin
22
p+q=1
()(√)
√22
2MH12H
ωM
Thistsrepresenanellipseinthephasespacewithsemimajoraxisa=√2MHandthesemiminoraxis
b=1√2H.
ωM
Figure6.2:Phaseplotofthesystem.
Theolumevofthisolume’‘vinphasespacefortconstanenergyH=Uistheareaofellipsehwhicis
√1√2U2πU
V=πab=π2MU·ωM=ω
Thisesgivtherequiredphasespace.olume’‘v□
Whatolumevofphasespacedoesitenclose?InthecaseofNsimilaroscillators,hwhicevhathetotal
energyUengivyb
NN
∑2∑2
p+(Mωq)=2MU
j=1j=1
withadditionalcouplingterms,toosmalltobeincludedbutlargeenoughtoensureequipartitionof
,energywhatisthenatureofthepathersedvtraybthesystempt?oin
(a)wShothattheolumevofthephasespace“enclosed”ybthispathis1(2πU)N.
N!ω
Solution:
Letsassumethatthephasespaceolumevofnharmonicoscillatorshwhicforma2ndimensional
nn
ellipsoidbeCnabThecoteiffciencanbefoundybusualmethodtobe
πn
Cn=Γ(n+1)
CHAPTER6.TISTICALASTMECHANICS166
Notingthatforthisproblema=√2MUandb=1√2U.Thephasespaceolumevbecomes
ωM
(√)
n()n()n
π√n12U12πU
Σ(U)=Γ(n+1)2MUωM=n!ω
Thisesgivtherequiredphasespaceolume.v□
(b)Usetheifnalresultof(6.4.2a)towshothattheytropenofNdistinguishableharmonicoscillators,
accordingtomicrocannonicalbleensemis
[(kT)]
S=Nk1+lnℏω
Solution:
Theytropenofsystemybdeifnitionis
S=kln(Σ(U))=kln(1(2πU)n)=kln(1)+nkln(U)
n
hn!hωn!ℏω
UsingSterling’sximationapproforewget
ln(1)=−nlnn+n
n!
Susbtutingthiskbacintheytropenequationesgiv
(U)[(U)]
S=nk−nklnn+nklnℏω=nk1+lnnℏω
ButforthesimpileharmonicoscillatortheenergyU=nkTusingthisesgiv
[(kT)]
S=nk1+lnℏω
Thisistherequireexpressionfortheytropenofthesystem.□
6.4.3.ConsiderasystemfoNparticlesinhwhictheenergyofheacparticlecanassumeowtandonlyowt
distinctaluesv0andE(>0).Denoteybnandntheoccupationbumersnofenergyellev0andE,
01
.respelyectivThetotalenergyofthesystemisU.
(a)Findtheytropenofhsucasystem.
Solution:
SincethereareN=n0+n1particlesthetotalysawinhwhicn0particlecangotoin0energy
ellevisengivyb
Ω=NC=N!
n0n!n!
01
Sotheytropenofsystemis
S=klnΩ=kln(N!)=klnN!−klnn!−klnn!
n!n!01
01
orFlargeNthiscanbesimpliifedybusingSterling’sximationapproas
S=k(NlnN−N+n0−n0lnn0+n1−n1lnn1)=kN[ln(N)+n1ln(n0)]
n0Nn1
CHAPTER6.TISTICALASTMECHANICS167
Thiscanberearrangedtoobtain
[(n0)(n1)]
S=−kn0lnN+n1lnN
Thisistherequiredytropenofthesystem.□
(b)Findthemostprobablealuevofthenandnandifndthemeansquarelfuctuationsofthese
01
tities.quan
Solution:
orFthissystem,theenergytconstrainis
n·0+n·E=U
01
Andthetotalbumerntconstrainis
N=n0+n1
eWevhatomaximizethefunction
S′=S−α(n+n−N)+β(n·0+n·E−U)
k0101
tiatingDifferenwithrespecttoheacoccupationbumernnandnandαandβ.eWget
01
lnn1+α=0
lnn1+α+E=0
Solvingthesetheonlypossiblealuevofn1is
n=Un=N−n=N−U
1E01E
Thesearethepossiblealuesvofnandntheoccupationbumers.n
01
□
(c)Whathappenswhenasystemofenegativtemperatureisedwallotohangeexcheatwithasystem
ofpeositivtemperature?
Solution:
Whenthesystemofenegativtemperatureisedwallotohangeexcenergywiththesystemofpos-
eitivenergytheenergywslfofromthesystemofenegativtemperaturetothesystemofpeositiv
temperature.□
6.4.4.(Huang6.4)Usingthecorrectedytropenula,formorkwouttheytropenofmixingforthecaseof
tdifferengasesforthecaseofticalidengases,usthwingshoexplicitlythatthereisnoGibbsxparado
yanmore.Findalsoternalin,energyU,andhemicalcptial,otenµ,usingthecorrectedytropenulaform
andcorrectedytropenula.formThelatteriscalledetrodekur-T‘Sac.equation’
Solution:
ByusinggibbscorrectionthephasespaceolumevshouldbedividedybN!
()()()
1VN1VN2π3N/2√
Σ(E)=CR3N=()(2ME)3N
33N33N
N!hN!h3NΓ2
sotheytropenfunctionreallybecomes
S=kln(Σ(E))=−NlnN+N+Nkln[VR3]+klnC3N
3
h
CHAPTER6.TISTICALASTMECHANICS168
SinceNisaeryvlargebumernewcanemakthebetterximationapprooftheSterlingximationappro
(πn/2)n(2πe)
lnCn=lnΓ(n+1)≈2lnn
2
hWhicyieldsus
S=k[3Nln(2πe)+Nln(V3+3Nln(2mE))]
23NNh2
=Nk[ln(4πmE)3/2+ln(V3)]+3Nk
3Nh2
(V3/2)3[5(4πm)]
=Nklnu+Nk+ln2
N233h
Thisisthetalfundamenequationofthesystemhshiccanbeysaalwertedvintoifndouretensivin
parameters.Sotheternalinenergybecomes
U=(3Nℏ2)3/2exp(2S−5)
4πmV3Nk
□
6.5orkHomeweFiv
6.5.1.(a)Asystemiscomposedofowtharmonicosicllators,heacofnaturalfrequencyωandheacvingha
()0
permissibleenergiesn+1ℏω,wherenisyanenon-negativteger.inwHoymanmicrostatesare
20
ailabelvatothesystem?Whatistheynetropofthesystem.
Solution:
Lettheifrstoscillatorbeinn1andthesecondbeinn2state.Thetotalenergyofthesystemthen
isthesumoftheenergiesofheacone
(1)(1)′′
n+ℏω+n+ℏω=nℏω⇒n+n+1=n
120220012
Theifrstoftheseoscillatorscangotoyanoneofn′states,butthesecondoneisconstrainedto
beinn2=n′−n1−1state.Sothereisfreedomofonlyonehoicecamongn′states.Sothetotal
bumernofmicrostatesisjust
Ω=n′C=n′!=n′
11!(n′−1)!
Sotheytropenofthesystemis
S=kln(n′)
IntermsofenergyofthesystemE′=n′ℏω0,theytropenbecomes
S=kln(E′)
ℏω0
Thisistherequiredytropenofthesystem.□
CHAPTER6.TISTICALASTMECHANICS169
(b)Asecondsystemisalsocomposedofowtharmonicoscillators,hacwofnaturalfrequencey2ω.
0
′′′′′′
ThetotalenergyfothesystemisE=nℏω0,wherenisenevteger.inwHoymanmicrostates
areailablevainthesystem?Whatistheytropenofthesystem?
Solution:
Lettheifrstoscillatorbeinn1andthesecondbeinn2state.Thetotalenergyofthesystemthen
isthesumoftheenergiesofheacone
(1)(1)′′′′
n+2ℏω+n+2ℏω=nℏω⇒n+n+1=n/2
120220012
Sincen′′isenevtegerinthebumernm=n′′/2isanotherteger.inTheifrstoftheseoscillatorscan
gotoyanoneofmstates,butthesecondoneisconstrainedtobeinn2=m−n1−1state.So
thetotalbumernofmicrostatesisjust
Ω=mC=m!=m=n′′
11!(m−1)!2
Sotheytropenofthesystemis
S=kln(n′′)
2
IntermsofenergyofthesystemE′′=n′′ℏω0,theytropenbecomes
S=kln(E′′)
2ℏω
0
Thisistherequiredytropenofthesystem.□
(c)Whatistheytropenfothesytemcomposedofteowtpreceedingsybsystems(separatedand
enclosedaybtotallyerestrictivall)?wExpresstheytropenasafunctionofE′′andE′.
Solution:
Thetotalytropenofthesystemisjustthesumofindividualtropiesenso
S=S1+S2=kln(E)+kln(E′′)=kln(E′E′′)
22
ℏω2ℏω2ℏω
000
Thisesgivthetotalytropenofthesystemcomposedofowtengivsubsystems.□
6.5.2.Asystemconsistsofthreedistinguishablemoleculesatrest,heacofhwhichasatizedquanmagnetic
t,momenhwhiccanevhaitsz-comptonen+M,0and−M.wShothatthereare27tdifferenpossible
statesofthesystem;listthemall,givingthetotalz−comptonenMofthemagnetictmomenforh.eac
∑i
ComputetheytropenS=−kifilnfiofthesystemforthewingfolloprioriporbabilities:
A:-M-M-M-M-M-M-M-M-MMMMMMMMMMOOOOOOOOO
B:-M-M-MMMMOOO-M-M-MMMMOOO-M-M-MMMMOOO
C:-MMO-MMO-MMO-MMO-MMO-MMO-MMO-MMO-MMO
Sum:-3M-M-2M-MM0-2M0-M-MM0M3M2M02MM-2M0-M02MM-MM0
(a)All27statesareequally.elylik
Solution:
Ifallstatesareequallyelylikthentheyprobabilitofheacstateisf=1.Sothetotalytropenof
i27
systemis
27()
S=−k∑filnfi=∑1ln1=kln27
ii=12727
□
CHAPTER6.TISTICALASTMECHANICS170
(b)hEacstateisequallyelylikforhwhicthez-comptonenMofthetotalmagnetictmomeniszero;
z
f=0forallotherstates.
i
Solution:
Therearesixstateswherethetotaltmomeniszero.Sotheifallthemareequallyelylikandthe
resthasyprobabilitf=0thenewevha
i
6()
S=−k∑flnf=∑1ln1=kln6
ii66
ii=1
Thisesgivtherequired.ytropen□
(c)hEacstateisequallyelylikforhwhicM=M;f=0froallotherstates.
zi
Solution:
ThereareensevstateswherethetotaltmomenisM.Sotheifallthemareequallyelylikandthe
resthasyprobabilitf=0thenewevha
i
7()
S=−k∑flnf=∑1ln1=kln7
ii77
ii=1
Thisesgivtherequired.ytropen□
(d)hEacstateisequallyelylikforhwhicM=3Mf=0forallotherstates.
2i
Solution:
Thereisjustonestatewherethetotaltmomenis3M.Sotheifallthemareequallyelylikandthe
resthasyprobabilitf=0thenewevha
i
1()
S=−k∑flnf=∑1ln1=kln1=0
ii11
ii=1
Thisesgivtherequired.ytropen□
(e)Thedistributionforhwhic∑Sisummaximsubjecttothetrequirementhat∑fi=1andthemean
comptonenifiMi=γM.wShothatforthisdistribution
f=e(3M−Mi)α
i(1+x+x2)3
αM
wherex=e(αbeginLagrangeultiplier)mandwherethealuevfoxisdeterminedybequation
3(1−x)2
γ=1+x+x2.ComputexandSforγ=3andcompareouryers.answ
Solution:∑
TheytropenofthesystemisS=−kflnf.eWevhatomaximizethisfunctionsubjecttothe
∑∑ii
tsconstrainf=1andfM=γM.Usinglagrangesultipliermhniquetecthefunctionto
iiiii
S
maximizethefunctionkis
()()
F=∑flnf−α′∑f−1−β∑fM−γM
iiiii
iii
CHAPTER6.TISTICALASTMECHANICS171
tiatingDifferenwithrespecttofandsettingequalto0ewget
j
∂F=∑[∂filnf+f∂lnfi]−α′∑∂fi−β∑M∂fi
∂f∂fii∂f∂fi∂f
jijjijj
∑[1]′∑∑
=δlnf+fδ−αδ−βMδ
ijiifijijiij
ijii
=lnfj+1−α′−βMj
|{z}
α
orFummaximtheeativderivhastoanish,vsettingthiseativderivequaltozeroewevha
−α+βMi
lnfi+α−βMj=0⇒fi=e(6.5)
Thesumofyprobabilittconstrainandtheeragevatconstrainare
∑∑−α+βMiα∑βMi
f=e=1⇒e=e
i
iii
ThelastexpressiononthetrighcanbewrittenasthesumervoallthetotaltsmomenMwith
i
yultiplicitmg(M)as
i
α∑βMi
e=g(M)e
i
Mi
Lookingattheconifgurationtableewevhathattheyultiplicitmforheacstatesis
g(3M)=g(−3M)=1g(−2M)=g(2M)=3g(−M)=g(M)=6g(0)=7
βM
Denotingx=eewevha
α−3−2−1023
e=g(−3M)x+g(−2M)x+g(−M)x+g(0)x+g(M)x+g(2M)x+g(3M)x
−3−2−123
=x+3x+6x+7+6x+3x+x
−3(23456)
=x1+3x+6x+7x+6x+3x+x
=(1+x+x2)3(6.6)
x3
Substutingthiskbactoin(6.5)ewget
3βMi3βM+βMi(3M+Mi)β
f=xe=e=e
i232323
(1+x+x)(1+x+x)(1+x+x)
wNookingvintheeragevatmomentconstrainewget
∑∑3(Mi)β
fM=γM⇒xeM=γM
ii(1+x+x2)3i
ii
Usingeα=∑eβMitheexpressionbecomes
i[]
x3∂∑x3∂eα
eβMi=γM⇒=γM
23∂β23∂β
(1+x+x)i(1+x+x)
βMdxα
Sinceewevhax=e,thiscanbetiateddifferentoget,dβ=Mx.Andsubstutingefrom(6.6)
theeabvoexpressionbecomes
x33∂[(1+x+x2)3]∂x=γM
23
(1+x+x)∂xx∂β
3[3(x2−1)(1+x+x2)2]
xMx=γM
234
(1+x+x)x
3(x2−1)
1+x+x2=γ
CHAPTER6.TISTICALASTMECHANICS172
Solvingthisequationforariousvaluesvofγewgetaquadraticequation.
√2
(γ−3)x2+γx+(γ+3)=0⇒x=γ±3(12−γ)
2(3−γ)
Speciifcallyfor3≥γ≥0ewevhatohocosethepeositivsignso
√2
x=γ+3(12−γ)
2(3−γ)
orFtheariousvaluesvcomputedare
γxS
0.01.0-3.29584
1.01.69-3.04037
3.0-2.0∞
Thisesgivtheariousvaluesvofytropenfortheengivaluesvofγ.□
6.5.3.evProthatforasystemincannonicalbleensem
⟨3⟩2[4(∂Cv)3]
∆E=kT+2TC
∂Tv
V
inparticular,foridealgas
⟨⟩⟨⟩
()()
∆E22∆E38
U=3NandU=9N2
Solution:
TheexpectationaluevofcubeoflfuctationofEfrommeanaluevcanbewrittenas
⟨3⟩⟨3⟩⟨3223⟩
∆E=(E−⟨E⟩)=E−3E⟨E⟩+3E⟨E⟩−⟨E⟩
⟨2⟩⟨2⟩⟨2⟩3
=E−3E⟨E⟩+3⟨E⟩E−⟨E⟩
⟨3⟩⟨2⟩⟨2⟩3
∆E=E−3E⟨E⟩+2⟨E⟩(6.7)
Intlighof(6.7)Theeragevaenergyofthesystemcanbewrittenas
∑−βE
Eiei
U=⟨E⟩=∑−βE(6.8)
ei
tiatingDifferen(6.7)withrespecttoβewget
∑2−βEi[∑∑−βEi]
∂UEeEe
=−∑i+−Ee−βEi∑i
−βEii−βE2
∂βe−(ei)
∑[∑]
2−βE−βE2
EeiEei
=−∑i+∑i
∑e−βEie−βEi
E2e−βEi⟨⟩2
=−∑i+U2≡E2−⟨E⟩(6.9)
e−βEi
CHAPTER6.TISTICALASTMECHANICS173
tiatingDifferen(6.9)againwithrespecttoβewget
2∑3−βEi[∑2−βEi∑]
∂UEeEe∂U
ii−βE
=∑−∑·−Eiei+2U
2−βEi−βEi2
∂βe−(e)∂β
∑3−βE[∑2−βE∑−βE][]
EeiEeiEiei⟨⟩2
=∑i−∑i·∑+2⟨E⟩−E2+⟨E⟩
−βEi−βEi−βEi
eee
⟨3⟩[⟨2⟩]⟨2⟩3
=E−E⟨E⟩−2⟨E⟩E+2⟨E⟩
=⟨E3⟩−3⟨E2⟩⟨E⟩+2⟨E⟩3(6.10)
Comparing(6.7)and(6.10)ewget
⟨∆E3⟩=∂2U=∂(∂U∂T)=∂(∂U∂T)∂T
∂β2∂β∂T∂β∂T∂T∂β∂β
Sinceβ=1theeativderiv∂T=−kT2andrecocnigingthat∂U=Cewget
kT∂β∂Tv
⟨3⟩∂(2)2(2∂Cv)223(∂Cv)
∆E=−kTC(−kT)=kT+2kTC(kT)=kTT+2C
∂Tv∂Tv∂Tv
⟨3⟩2222
Thisistherequiredexpressionfor∆E.UsingU=3NkT;U=9NKTandwithCv=
3NK,∂Cv=0andsubstutigkbacintheexpressionewget
∂T
⟨⟩⟨⟩
()()
∆E22∆E38
U=3NandU=9N2
Thesearetherequiredaluesvforidealgas.□
6.5.4.erifyVthat,foridealgas,
S(Q)(∂lnQ)
=ln1+T1
NkN∂TP
Solution:
orFanidealgas,ewassumethatheacmoleculeisfreeandsotheytdonexertforceonheacother,so
theptialoteniszero.Alsotheyevhasametummomeminalldirectionshwhicleadstothehamiltonian
∑2222
pppp
H=i=x+y+z
i2m2m2m2m
Thepartitionfunctionforasinglemoleculeis
[]∞
1∫−βH331∫3∫−βH3
Q=edqdp=dq·edp
133
hhV
−∞
Thetegrationinofthespacecoordinatesqijustesmakesgivtheolumevofthesystemasitisindeptenden
ofthetummomencoordinates
∞[()]
∫222
V1ppp
Q=exp−x+y+zdpdpdp
13xyz
hβ2m2m2m
∞
Sincethetummomeninheacdirectioncanbeconsideredtobethesameandtheparameterβ=1ew
kT
get
∞[()]∞()3
∫222∫2
V1pppVp
Q=exp−++dpdpdp=2exp−dp
133
hkT2m2m2mh2mkT
∞0
CHAPTER6.TISTICALASTMECHANICS174
Thistegralinisjustthegammafunctionandthetegraliniseasilycomputedtobe√mkT·√π.Sothe
2
partitiionfunctionbecomes
()
[]3/2
V√√32πmkT
Q1=h32mkT·π=Vh2
AlsoforidealgastherelationPV=NKTtakigtheariousvesativderivofthepartitionfunctionew
get
(Q)[KT(2πmkT)3/2]
ln1=ln
NPh2
(∂lnQ)∂[3(2πmkT)][131]5
1=ln(NKT)−lnP+ln=−0+=
∂T∂T2h2T2T2T
PP
biningComtheseowtewget
(Q)(∂lnQ)[kT(2πmkT)3/2]5
ln1+T1=ln+
N∂TPh22
TheexpressiononthetrighisjustS□
Nk
Chapter7
tumQuanhanicsMecII
7.1orkHomewOne
′′′
7.1.1.(Sakurai2.33)Thepropagatorintummomenspaceisengivyb⟨p,t|p,t0⟩.eDerivanexplicit
expressionfor⟨p′′,t|p′,t0⟩forthefreeparticlecase.
Solution:
orFafreeparticletheHamiltonianis
p2
H=2m
Sothetimeolutionevoperatorforyanstateintummomenspaceisengivyb
iHt[2]
U(t)=eℏ⇒expipt
2mℏ
Thebaseetskeolvevervotimeas
′†′′′′[ip2t]
|p,t⟩=U(t)|p,0⟩→⟨p,t|=⟨p,0|U(t)=⟨p,t0|exp2mℏ
Sothepropagatorbecomes
[′′2][′2]
′′′′′iptipt0′
⟨p,t|p,t0⟩=⟨p,0|exp2mℏexp−2mℏ|p,0⟩
[]
i(′′2′2)′′′
=exp2mℏpt−pt0⟨p,0|p,0⟩
[]
i(′′2′2)′′′
=exp2mℏpt−pt0δ(p−p)
Thisesgivexplicitexpressionforthepropagatorofthefreeparticle.□
7.1.2.(Skurai2.37)
()2[()]
(a)erifyV[Π,Π]=iℏeεB.andmdx=dΠ=eE+1dx×B−B×dx
ijcijkkdt2dt2cdtdt
Solution:
175
CHAPTER7.QUANTUMMECHANICSII176
ThekinematicaltummomenforelectromagneticifeldisdeifnedasΠ≡mdx=p−eAwhereA
dtc
istheectorvmagneticptialotenisafunctionofoperatorx.Theutatorcommthenis
[Π,Π]=[p−eA,p−eA]
ijicijcj
=[p,p]−[p,eA]−[eA,p]+[eA,eA]
ijicjcijcicj
e(∂Aj)e(∂Ai)
=0−c−iℏ∂xi−ciℏ∂xj+0
=iℏe(∂Aj−∂Ai)
c∂xi∂xj
=iℏeB
ck
repeatingthissameprocessforallthecomptsonenofthiskinematicaltummomenoperatorewget
[Π,Π]=iℏeεB(7.1)
ijcijkk
TheHamiltonianforelectromagneticifeldidH=Π2+eϕ.orFthetzLorenforceulaformewevha
2m
2
mdx≡Πtiatingdifferenthiswithtimeesgivmdx=dΠybusingbHeisenergequationofmotion
dtdt2dt
ewcanwrite
2
dxdΠ1
mi=i=[Π,H]
dt2dtiℏi
1[Π2]
=iℏΠi,2m+eϕ
[2][]
=1Π,Π+1p+eA,eϕ
iℏi2miℏicx
=1∑[Πi,Π2]+1[pi,eϕ]
2miℏjiℏ
j
[2]
ButtheutatorcommofΠ,Π=Π[Π,Π]+[Π,Π]Πhwhicybuseof(7.1)reducesto
ijrijijr
[2]iℏeiℏe
Π,Π=ΠεB+εBΠ
ijjcijkkcijkkj
Andalso1[p,eϕ]=1(−iℏ)∂eϕ=−eE
iℏiiℏ∂xi
Usingtheseowtfactskbacinintheoriginalutatorcommleadsto
2∑
mdxi=1εpBiℏe+εBiℏep−eE
dt22miℏijkjkcijkkcji
j
=eE+1∑(dxjB−Bdxk)
2cdtkjdt
j
Theeabvoexpressioncanbeobtainedforheaccomptsonenijandktoobtaintherequiredrelation
in3D
2[()]
mdx=dΠ=eE+1dx×B−B×dx
dt2dt2cdtdt
Thisistherequiredtzlorenforcerelation.□
CHAPTER7.QUANTUMMECHANICSII177
∂ρ′(ℏ∗′(e)2)
(b)erifyV∂t+∇·j=0withjengivybj=mIm(ψ∇ψ)−mcA|ψ|,
Solution:
Bydeifnitiontheyprobabilitydensitfunctionistheabsolutealuevsquareofaefunction.wThe
Hamiltonianforelectromagneticifeldforarbitraryefunctionvawψisengivyb
2()2
H=Π+eϕ=2p−eA+eϕ
2m2mc
Thetummomenoperatorinpositionspaceefunctionvawcanbewrittenas∂−iℏ∇.Usingthe
hroscdingerequationHψ=EψwhereoperatorEisengivybE=iℏ∂tewget
Hψ=iℏ∂ψ
∂t
[()]
∂ψ=11−iℏ∇−eA2+eϕ
∂tiℏ[2mc]
22
1−ℏ2ee2
=∇+iℏ(∇·A+A·∇)+2A+eϕψ
iℏ[2m2mc2mc]
22
1−ℏ2ee2
=∇ψ+iℏ(∇·(Aψ)+A·∇ψ)+2Aψ+(eϕ)ψ
iℏ2m2mc2mc()
2
iℏ2eee−ie2
=∇ψ+(∇·A)ψ+A·∇ψ+A·∇ψ+2A+eϕψ
2m2mc2mc2mcℏ2mc
(2)
iℏ2ee−ie2
=∇ψ+(∇·A)ψ+A·∇ψ+2A+eϕψ
2m2mcmcℏ2mc
akingTtheconjugateofthisexpressionleadsto
∂ψ∗−iℏ2∗e∗ei(e22)∗
=∇ψ+(∇·A)ψ+iℏA·∇ψ+2A+eϕψ(7.2)
∂t2m2mcmcℏ2mc
akingTthetimeeativderivoftheyprobabilitydensitfunctionewget
∂ρ=∂=∂=ψ∗∂ψ+∂ψ∗ψ
∂t∂t∂t∂t∂t
orFaergencedivfreemagneticectorvptialotenh(whicewcanysaalwhocose),Multiplying(7.2)
ybψanditsconjugateybψ∗andaddingewget
∂ρ∗iℏ2∗e−iℏ2∗e∗
∂t=ψ2m∇ψ+ψmcA·(∇ψ)+ψ2m∇ψ+ψmcA·(∇ψ)
iℏ[∗22∗]e∗∗
=2mψ∇ψ−ψ∇ψ+mc(ψA·(∇ψ)+ψA·(∇ψ))
=iℏ(2i∇·Im(ψ∗∇ψ)+e(∇·(Aψ∗ψ))
2mmc
=−ℏ∇·(Im(ψ∗∇ψ))+e∇·(A|ψ|2)
m(mc)
ℏ∗e2
=−∇·mIm(ψ∇ψ)−mcA|ψ|
=−∇·j
Thiscompletestheproof.□
7.1.3.(Sakurai2.38)ConsideraHamiltonianofthespinlessparticleofhargece.Inpresenceofastatic
magneticifeld,theteractionintermscanbegeneratedyb
P→P−eA,
operatoroperatorc
CHAPTER7.QUANTUMMECHANICSII178
whereAistheappropirateectorvptial.otenSuppose,for,ysimplicitthemagneticifeldBisuniformin
thepeositivz−direction.evProthattheeabvoprescipritionindeedleadstothecorrectexpressionfor
theteractioninoftheorbitalmagnetictmomen(e/2mc)LwiththemagneticifeldB.wShothatthere
isalsoanextratermproportionaltoB2(x2+y2),andtcommenbrielfyonitsysicalphsigniifcance.
Solution:
Sincetheelectricifeldiszeroewcanassignascalarptialotenastconstanandthetconstancanysaalw
behosenc0usthϕ=0.TheectorvmagneticptialotenforuniformmagneticifeldisA=1x×Bˆz.
2
Sincethereisafreehoicecofectorvmagneticptialotenaslongasitscurlisergencedivfree,ewhosec
thisptialotenhwhicisalsoergencedivfree.usThforthiscase∇·A=0.
romF(7.2)ewevhathehamiltonianofthesystemis
ℏ2iℏeiℏee2
H=−∇2+A·∇+∇·A+A2
2
2mmc2mc2mc
Since∇·A=0ybourhoicectheteractioninoperatortermstroinducedduetothepresenceofmagnetic
ptialotenis
22
iℏee2ee2
A·∇+2A=−A·(−iℏ∇)+2A
mc2mcmc2mc
212(22)
Buttheoperator−iℏ∇isthetummomenoperatorpandA=4Bx+yThisenablesustowrite
theteractionintermsas
e1()e2B2(22)
ˆˆ
−B−yi+xj·p+2x+y
mc28mc
ˆˆ
eWcanrecocnizetheterm(−yi+xj)·p=−yP+xP=LSubstutingthisintheeabvoexpression
xyz
ewget
ee2B2()
BL+x2+y2
2mcz8mc2
Sotheifnalhamiltonianbecomes
[222()]
ℏ2eeB22
H=−∇+BLz+2x+y
2m2mc8mc
Sotheteractionintermstroinducedintheabsenceofscalarptialotenbutthepresenceofmagneticpo-()
tialtenhasoperatorfororbitalangulartummomeneBLandatermproportionaltoB2x2+y2□
2mcz
7.1.4.(Sakurai2.39)Anelectronesvmoinhepresenceofauniformmagneticifeldinthez-direction(B=
Bˆz)
eAeA
(a)aluateEv[Π,Π]whereΠ≡p−x,Π≡p−y.
xyxxcyyc
Solution:
[Π,Π]=[p−eA,p−eA]
xyxcxycy
=[p,p]−[p,eA]−[eA,p]+[eA,eA]
xyxcycijcicj
e(∂Aj)e(∂Ax)
=0−c−iℏ∂x−ciℏ∂xy+0
=iℏe(∂Ay−∂Ax)
c∂x∂y
=iℏeB
cz
CHAPTER7.QUANTUMMECHANICSII179
hWhicistherequiredexpressionfortheutator.com□
(b)BycomparingtheHamiltonianandtheutationcommrelationobtainedin7.1.4withthoseofthe
one-dimensionaloscillatorproblem,wshowhoewcanimmediatelywritetheenergyaluesveigenas
23()()
Ek,n=ℏk+|eB|ℏn+1
2mmc2
Solution:
Sincethehargedcparticleisonlyinthemagneticifeld,theelectricifeldist,absenhwhicmeansthe
electricptialotenisatconstanhwhiicewymaassumetobe0.Sothehamiltonianofthesystemis
Π2Π2Π2Π2
H==z+y+x
2m2m2m2m
Theenergyalueveigenequationfortheageneralefunctionvawψ(x′)ewevha
α
[222]
ΠΠΠ
′zyx′
Hψ(x)=++ψ(x)
α2m2m2mα
SincethemagneticifeldiscompletelyinˆztheectorvmagneticptialotencanbewrittenasA(x)=
1x×BˆzsothatA=0.Thissimpliifesthealueveigenequationto
2z
[222]
pΠΠ
′zyx′
Hψ(x)=++ψ(x)
α2m2m2mα
Theifrstofthesethreeexpressionphaswnknoalueveigenℏkengivintheproblem.Thesecond
z
owttermscanbealuatedevusingtheOnedimensionalsimpleharmonicoscillator.Sincethe[]
utatorcom[Π,Π]=iℏeBewcanscaleΠybctoemakΠ,cΠ=iℏ.LetY=cΠ
xycyeBxeByeBy
Usingthistheexpressionbecomes
[p2Π21e2B2]
Hψ(x′)=z+x+mY2ψ(x′)
α22α
2m2m2mc
†
eWcanagaintrytheraisingaoperatoranderingwlooperatorsaoutofthelastowtexpression.
√eB(ic)†√eB(ic)
a=2ℏcY+eBΠxa=2ℏcY−eBΠx
†mciHmc1†
Andsinceaa=ℏeBH+2ℏ[Y,Πx]=ℏeB−2.IncompleteanalogytoSHOewifndaaorksw
†
asultaneoussimoperatorwithHamiltonianH,i.e.aautescommwithH,andsoactsonenergy
eigenstatestoegivtegerinnasitsalue.veigenSothealueveigenbecome
[p2][Π21e2B2]
′z′x2′
Hψ(x)=ψ(x)++mYψ(x)
α2mα2m2m2c2α
′ℏ2k2′[(1)|eB|]′
Hψ(x)=ψ(x)+n+ℏψ(x)
α2mα2mcα
SothealueveigenoftheoperatorHhwhicaretheenergyaluesvare
E=ℏ2k2+(n+1)ℏ|eB|
n2m2mc
Thisesgivtheedwalloenergyofthehargedcparticle.□
CHAPTER7.QUANTUMMECHANICSII180
7.2orkHomewowT
()()
7.2.1.(Sakurai3.1)Findtheectorsveigenofσ=0−i.Supposeanelectronisinspinstateα.If
yi0β
Syismeasured,whatistheyprobabilitoftheresultℏ/2?
Solution:
Supposethealueseignevofthematrixareλ.Theharactersticscequationforthematrixis
(0−λ)(0−λ)−(−i·i)=0⇒λ=±1
()
Lettheectorveigenbex.Thentheectorveigencorrespondingtoλ=1ewevha
y
()()()
0−ix=λx⇒−iy=x⇒x=1
i0yyix=yy=i
√22√
Normalizingthisectorveneivewevhathenormalizationfactor1+1=2.Sotherequirednor-
malizedectorveigencorrespondingtoλ=1is
1()
√1
2i
Thentheectorveigencorrespondingtoλ=−1ewevha
()()()
0−ix=λx⇒−iy=−x⇒x=1
i0yyix=−yy=−i
√22√
Normalizingthisectorveneivewevhathenormalizationfactor1+1=2.Sotherequirednor-
malizedectorveigencorrespondingtoλ=−1is()
11
√−i
2
Sotheectorsveigencorrespondingtoheacaluesveigenare
1()1()
λ=1→√1λ=−1→√1
2i2−i
()()
Letthearbitraryspinstatebe|γ⟩=αhsucthatitsdualcorrespondenceis⟨γ|=α∗β∗.Since
β
thematrixtationrepresenoftheSoperatorisℏσ.Theyprobabilitthatthestatebemeasuretobe
y2y
inSywithalueveigenℏis
2
()ℏ()()ℏ()()iℏ
⟨γ|ℏ/2σ|γ⟩=α∗β∗0−iα=α∗β∗−iβ=(−βα∗+β∗α)
22i0β2iα2
Sotheyprobabilitofmeasuringtheengivstatein|Sy;+⟩stateisiℏ(αβ∗−α∗β).□
2
7.2.2.(Sakurai3.2)Find,ybexplicitconstructionusingauliPmatrices,thealuesveigenforHamiltonian
H=−2µS·B
ℏ
foraspin1particleinthepresenceamagneticB=Bxˆ+Byˆ+Bˆz.
2xyz
Solution:
Thehamiltonianoperatorintheengivmagneticifeldas
H=−2µ(SB+SB+SB)
ℏxxyyzz
CHAPTER7.QUANTUMMECHANICSII181
SincethespinoperatorsSx,SyandSzarethepaulimatriceswithafactorofℏ/2ewcanwritethe
eabvoexpressionas
2µℏ[()()()]
H=−01B+0−iB+10B
ℏ210xi0y0−1z
(BB−iB)
=−µzxy
B+iB−B
xyz
Theharactersticscequationforthethismatrixis
((B−λ)(−B−λ)−(B−iB)(B+iB))=0⇒λ2−B2−(B2+B2)=0⇒λ=±|B|
zzxyxyzxy
SothealueveigenoftheHamiltonianhwhicis−µtimesthematrixis−µ·λ=∓µ|B|.□
7.2.3.(Sakurai3.3)Consider2×2matrixdeifnedyb
a+iσ·a
U=0
a−iσ·a
0
whereaisarealbumernandaisathree-dimensionalectorvwithrealcompts.onen
0
(a)evProthatUisunitaryandunimodular.
Solution:
enGivmatrixUandhermitianconjugatecanbewrittenas
a+i∑aσa−i∑aσ†
0jjj†0jjj
U=a−i∑aσU=∑†
0jjja0−ijajσj
Multiplyingtheseowttokheccforyunitarit
a−i∑aσ†a+i∑aσ
†0jjj0jjj
UU=∑†·a−i∑aσ
a−iaσ0jjj
0jjj
2∑∑†∑∑†
a+iaσa−iaσa+σaσa
=00jjj0jjjjkjjkk
2∑∑†∑∑†
a−iaσa+iaσa+σaσa
00jjj0jjjjkjjkk
SinceheacpaulimatricesareHermitian,forheaciewevhaσ†=σ.Thisesmaktheumeratorn
ii
theexactsameasthedenominator.usThtheycancelout
2∑∑∑∑
a+iaσa−iaσa+σaσa
†00jjj0jjjjkjjkk
UU=2∑∑∑∑=1
a−iaσa+iaσa+σaσa
00jjj0jjjjkjjkk
Thiswsshothatthismatrixis.unitaryExpandingoutthematrixintermsofthepaulimatrices
ewget


a+iaia+a
0312
2222
ia−aa−ia(a+ia)(a−ia)−(ia+a)(ia−a)a+a+a+a
detU=1203=03031212=0123=1
2222
a−ia−ia+a(a−ia)(a+ia)−(−ia+a)(−ia−a)a+a+a+a
0312030312120123

−ia−aa+ia
1203
Thiswsshothatthematrixisunimodular.□
(b)Ingeneral,a2×2unitaryunimodularmatrixtsrepresenarotationinthreedimensions.Findthe
axisandtheangleofrotationappropriateforUintermsofa,a,aanda.
0123
CHAPTER7.QUANTUMMECHANICSII182
Solution:
Thematrixcanberewrittenas
1(2)
a−a+2iaa2aa+2iaa
U=0030201
222
a+a−2aa+2iaaa−a−2iaa
00201003
Sincethemostgeneralunimodularmatrixoftheform(ab)trepresenarotaionthroughan
∗∗
angleϕthroughthedirectionnˆ=nxˆ+nyˆ+nˆzrelatedas−ba
xyz
Re(a)=cos(ϕ),Im(a)=−nzsin(ϕ)(7.3)
22
(ϕ)(ϕ)
Re(b)=−nysin2,Im(b)=−nxsin2(7.4)
Makingthesecomparisioninthismatrixewget
()22(22)
ϕa−aa−a
cos=0⇒ϕ=2acos0
2222
2a+aa+a
00
Andsimilarlyewget
aaa
n=−1n=−2n=−3
x|a|y|a|z|a|
Thisesgivtherotationangleandthedirectionofrotationforthisengivunimodularmatrix.□
7.2.4.(Sakurai3.9)Considerasequenceofrotationstedrepresenyb
(1/2)(−iσ3α)(−iσ2β)(−iσ3γ)(e−i(α+γ)/2cosβ−e−i(α+γ)/2cosβ)
D(α,β,γ)=expexpexp=22
222e−i(α−γ)/2sinβei(α+γ)/2cosβ
22
Solution:
Againthisifnalmatrixcanbewrittenasacoplexformas
()()
((α+γ)(α+γ))β((α+γ)(α+γ))β
D1/2(α,β,γ)=cos2+isin2cos(2)−cos2−isin2cos(2)
((α+γ)(α+γ))β((α+γ)(α+γ))β
cos2+isin2cos2−cos2−isin2cos2
Letϕbetheangleofrotationtedrepresenybthisifnalrotationmatrix.Usingagaintheequations(7.3)
ewget
(ϕ)(α+γ)(β)−1[(α+γ)(β)]
cos2=cos2cos2⇒ϕ=2coscos2cos2
Thisesgivtheangularrotationaluevforthismatrix.Thedirectionofrotationcansimilarlybefound
ybusing(7.3)tocalculatethedirections.□
7.2.5.(Sakurai3.15a)LetJbeangulartum.momenUsingthefactthatJ,J,JandJ≡J±Jsatisfy
xyz±xy
theusualtumangular-momenutationcommrelations,evpro
J2=J2+JJ−ℏJ
z+−z
.
Solution:
CHAPTER7.QUANTUMMECHANICSII183
MultiplyingoutJandJewget
+−
JJ=(J+iJ)(J−iJ)
+−xyxy
=J2−iJJ+iJJ+J2
xxyyxy
=J2−i[J,J]+J2
xxyy
=J2+J2−i(iℏJ)
xyz
=J2−J2+ℏJ
zz
RearrangingeabvoexpressionesgivJ2=J2+JJ−ℏJhwhiccompletestheproof.□
z+−z
7.3orkHomewThree
7.3.1.Expandthematrix
(j)−i(m′α+mγ)′(−iJyβ)
D′(α,β,γ)=e⟨j,m|exp|j,m⟩.
mmℏ
Solution:
Clearlytheorderofmatrixdependsuponthealuevofj.Therangeofaluesvformareconstrainedyb
thealuevofj.Soforj=1,thematrixbecomes
111
2(1+cosβ)−√2sinβ2(1−cosβ)
(1)11
d(β)=√sinβcosβ−√sinβ
22
111
2(1−cosβ)√2sinβ2(1+cosβ)
□
7.3.2.(Sakurai3.13)Antumangular-momeneigenstate|j,m=m=j⟩isrotatedybaninifnitesimal
max
(j))
angleεabouty-axis.Withougusingtheexplicitformofthed′function,obtainanexpressionfor
mm
2
theyprobabilitforanewrotatedstatetobefoundintheoriginalstateuptotermsoforderε
Solution:
Lettheengivstatebe|α⟩=|j,j⟩Therotationoperatorthorugh()Yaxisis
22
D(ε)=exp−iJyε=1−iJyε−Jyε+...
y2
ℏℏ2ℏ
ritingWJ=1(J−J)andexpandingouttheexpressionewget
y2i+−
ε2
D(ε)=1−JJ
y8ℏ2+−
Sotherotetaedstateis
2
|α⟩=D(ε)|α⟩=1−εJJ|jj⟩
Ry8ℏ2+−
Theyprobabilitofifndingtherotatedstateintheoriginalstateisengivyb|⟨α|α⟩|2calculatingthi
R

2222
2εε
|⟨α|α⟩|=⟨jj|1−JJ|jj⟩=⟨jj|jj⟩−⟨jj|JJ|jj⟩
R2+−2+−
8ℏ8ℏ

222
√√2εjεj

=1−2jℏ2jℏ=1−≈1−
42
CHAPTER7.QUANTUMMECHANICSII184
2
Thisistherequiredyprobabilitintheorderofε.□
7.3.3.(Sakurai3.16)wShothattheorbitaltumangular-momenoperatorLutescommwithboththeopera-
torsp2andx2
Solution:
2
TheutatorcommofheaccomptonenofLwithpare
[2][222]
Lz,p=xpy−ypx,p+p+p
[][xy]z
=xp,p2−yp,p2
yxxy
(∂2)(∂2)
=iℏppy−iℏppx
∂px∂py
xy
=2iℏ[p,p]
xy
=0
[2]
SimilarlyewcanwshothatthisistrueforeryevcomptonenoftheLhenceitisedvproforL,p.
wNofortheutationcommofx2twithheoperatorL
[2][222]
Lz,x=xpy−ypx,p+p+p
[][xy]z
=xp,p2−yp,p2
yyxx
(∂2)(∂2)
=x−iℏ∂yy−y−iℏ∂xx
=−2iℏ[x,y]
=0
Sinccethisistrueforthe[]Lzcomptonenitisalsotrueforeryevtoehrtcomopnensothattheectorv
utatiorcommL,x2
□
7.3.4.(Sakuraieq3.6.11)evProthewingfollo
(a)()
′∂∂′
⟨x|Lx|α⟩=−iℏ−sinϕ∂θ−cotθcosϕ∂ϕ⟨x|α⟩
(b)()
⟨x′|L|α⟩=−iℏcosϕ∂−cotθsinϕ∂⟨x′|α⟩
y∂θ∂ϕ
(c)[()]
1∂21∂∂
′22′
⟨x|L|α⟩=−ℏsin2θ∂ϕ2+sinθ∂θsin∂θ⟨x|α⟩
Solution:
Theangulartummomenoperatorisdeifnedas
L=r×P=r×(−iℏ∇)=(−iℏ)r×∇
Theseectorsvinspericalcoordinatesystemare
∂1∂1∂
ˆˆˆˆ
r=rˆr+θθ+ϕϕ∇=ˆrθ∂r+θr∂θ+ϕsinθ∂ϕ
CHAPTER7.QUANTUMMECHANICSII185
Sotehcorssproductis
ˆˆ
ˆrθϕ()
∂1∂
rθϕˆˆ
L=(−iℏ)r×∇=∂1∂1∂=(−ℏ)ϕ∂θ−θsinθ∂ϕ
r∂rr∂θrsinθ∂ϕ
wNothecartesianunitectorsvinthesphericalunitectorsvare
ˆˆ
xˆ=ˆrsinθcosϕ+θcosθcosϕ−ϕsinϕ
ˆˆ
yˆ=ˆrsinθsinϕ+θcosθsinϕ+ϕsinϕ
usThtheangulartummomenoperatorintheLdirectionbecomes
x
(∂∂)
L=xˆ·L=(−iℏ)−sinϕ−cosϕcotθ
x∂θ∂ϕ
usTh
(∂∂)
⟨x|Lx|α⟩=−(−iℏ)−sinϕ∂θ−cosϕcotθ∂ϕ⟨x|α⟩
SimilarlytheoperatorLis
y
Ly=yˆ·L=(−iℏ)(cosϕ∂−sinϕcotθ∂)
∂θ∂ϕ
usTh
⟨x|Ly|α⟩=−(−iℏ)(−cosϕ∂−sinϕcotθ∂)⟨x|α⟩
∂θ∂ϕ
Alsotheangulartummomensquaredopeatorbecomes
2[(ˆ∂ˆ1∂)][(ˆ∂ˆ1∂)]
L=L·L=(−ℏ)ϕ∂θ−θsinθ∂ϕ·(−ℏ)ϕ∂θ−θsinθ∂ϕ
=−ℏ2[∂2+cotθ∂+1∂2]
∂θ2∂θ2∂ϕ2
sinθ
2[1∂(∂)1∂2]
=−ℏsinθ∂θsinθ∂θ+2∂ϕ2
sinθ
usThewcanwrite
22[1∂(∂)1∂2]
⟨x|L|α⟩=−ℏsinθ∂θsinθ∂θ+sin2θ∂ϕ2⟨x|alpha⟩
Thesearetherequiredoperatortationrepreseninsphericalcoordinatesystem.□
7.4orkHomewourF
2
7.4.1.(Sakurai3.18)AparticleinasphericallysymmetricalptialoteniswnknotobeinaneigenstatofL
andLzwithaluesveigenℏ2l(l+1)andmℏ,.respelyectivevProthattheexpectationaluesveenbwet
|lm⟩statessatisfy[]
⟨⟩⟨⟩222
⟨L⟩=⟨L⟩=0,L2=L2=l(l+1)ℏ−mℏ
xyxy2
CHAPTER7.QUANTUMMECHANICSII186
trepretIntheresult.semiclassically
Solution:
SincethedeifnitionoftheoperatorsL±=Lx±Lytheserelationscanberearrangedintotheoprators
theoperators
L=L++L−L=L+−L−
x2y2i
TheexpectationaluevofoperatorLxis
⟨L⟩=⟨lm|L|lm⟩=⟨lm|L++L−|lm⟩
xx2
=1⟨lm|L|lm⟩+1⟨lm|L|lm⟩
2+2−
=1⟨lm|C|lm+1⟩+1⟨lm|C|lm+1⟩
2+2−
=0+0=0
SimilarlyforLytheexpectationalueviszero.TheL2opeartorcanbeexpandedtoin
[][x]
L2=L++L−L++L−
x22
=1(L2+LL+LL+L2)
4++−−+−
ButtheexpectattionaluevofL2andL2arebothzerobecausetheyraiseanderwlothestateetkwicet
+−
hwhicareothognoaltoheacother.
wNotheexpectationaluevreducesto
⟨2⟩1
L=⟨LL+LL⟩
x4+−−+
But
22222222
LL+LL=L−iLL+iLL+L+L+iLL−iLL+L=2(L+L)=2(L−L)
+−−+xxyyxyxxyyxyxxz
UsingthistoifndtheexpectationaluevofL2ewget
x
⟨2⟩11⟨22⟩1(222)
L=⟨L+L−+L−L+⟩=L−L=ℏl(l+1)+ℏm
x4222
22
SimilarlytheexpectatinaluevofLissameasforLandtheyarearequal.□
yx
7.4.2.(Sakurai3.19)Supposeategerhalf-inlalue,vysa1,erewedwallofororbitalangulartum.momen
2
romF
L+Y1/2,1/2(θ,ϕ)=0
ewymadeduce,asusual
iϕ/2√
Y(θ,ϕ)∝esinθ
1/2,1/2
wNotrytoconstructY(θ,ϕ)yb(a)applyingLtoY;and(b)usingLY(θ,ϕ)=
0.wShothattheowtproceduresleadtotradictoryconresult.1/2,−1/2−1/2,1/2(θ,ϕ)−1/2,−1/2
Solution:
ApplyingL−ontheengivstateY1/2,1/2ewget
−iϕ(∂)iϕ/2√
Y(θ,ϕ)=−iℏei−cotθesinθ
1/2,−1/2∂θ
−iϕ−iϕ/21cosθiiϕ/2√
=iℏe(−1)e2√+iℏcotθ2esinθ
sinθ
−iϕ/2cosθ
=−ℏe√
sinθ
CHAPTER7.QUANTUMMECHANICSII187
kinghecctoseeifLY(θ,ϕ)=0
−1/2,−1/2
iϕ(∂∂)−iϕ/2cosθ
L−Y(θ,ϕ)=−iℏe−i−cotθe√(−ℏ)
1/2,−1/2∂θ∂ϕsinθ
2−iϕ((sinθ1cos2θ)−iϕ/2(1)−iϕ/2√)
=iℏe−i−√sinθ−2√3e−cotθ−i2esinθ
([sinθ])
2−3iϕ/2√1221
=ℏe3−2sinθ−cosθ+2sin2θ
sinθ
Thelastexpressionisnotzerohwhictradictsconourpropositionthattereexistsahalftegerinl−alue.v
□
7.4.3.(Sakurai3.20)Consideranorbitalangular-motumemneigenstate|l=2,m=0⟩.Supposethissate
isrotatedybanangleβabouty−axis.Findtheyprobabilitforthenewstatetobefoundinm=0,±1
and±2.(Thesphericalharmonicsforl=0,1and2ymabeuseful).
Solution:
Letthearbitrarystatebe|P⟩=|l=2;m=0⟩thestatekaetintherotatedsystemis|P⟩R=
D(0,β,0)|P⟩Thisreotaedstatecanbecalculatedas
∑′′
DR(0,β,0)|P⟩=|l=2;m⟩⟨l=2;m|DR(0,β,0)|l=2,m=0⟩
m′√
∑′(2)∑′4πm′∗
=|l=2;m⟩Dm′,0(0,β,0)=|l=2;m⟩5Y2(β,0)
m′m′
usThtheyprobabilitofifndingtherotatedstatesameastheoriginalstaeis

√2√2
∑4π′4π
2′m∗m∗
|⟨P|DR|P⟩|=⟨l=2,m=0|l=2;m⟩Y(β,0)=Y(β,0)
5252
m′
Thisistherequiredyprobabilitofifndingtherotatedstateinoriginalstate.
√52122
wNoform=0ewevhaY=(3cosβ−1)thisesgivtheyprobabilit(3cosβ−1).
2,016π4
√15322
orFm=±1ewevhaY2,±1=8π(sinβcosβ)thisesgivtheyprobabilit4sinβcosβ.
√15234
orFm=±2ewevhaY2,±2=32π(sinβ)thisesgivtheyprobabilit8sinβ.□
Chapter8
StatisticalhanicsMecII
8.1orkHomewOne
8.1.1.aluateEvtheydensitmatrixϱofanelectroninamagneticifeldinthetationrepresenthatesmakσˆ
diagonal.Next,wshothatthealuevof⟨σ⟩,resultingfromthistation,represenispreciselythesameas
theoneobtainedinclass.
Solution:
Thepaulispinoperatorσisdiagonalinthetationrepresenwherethebasisstatesareenstateseivof
x
Sxoperator.InSztationrepresentheSxstatesareengivyb
1
|Sx;±⟩=√(|+⟩±|−⟩)
2
ThetransformationoperatorthatestakfromSztationrepresentoSxtationrepresenisengivyboperator
U=|S;+⟩⟨+|+|S;−⟩⟨−|
xx
Sothematrixtationrepresenofthisoperatoris
[]1[]
U=⟨Sx;+|+⟩⟨Sx;+|−⟩=√11
⟨Sx;−|+⟩⟨Sx;−|−⟩21−1
Theoperatorinthenewbasiscanbeobtainedfromtheoldbasiswiththetransformation.
1[][][]1[]
σ′=U†σzU=111011=0−1
z2−110−11−12−10
TheHamiltonianofthesysteminnewbasisis
H′=µB·σ′=−µBzσ′
z
Theydensitoperatorincannonicalbleensemisengivyb
′e−βH
ϱˆ=−βH(8.1)
Tr(e)
Carryingouttheylortaexpansionoftheumeratornintheydensitoperator
′′βµBzσ′(βµBzσ′)2(βµBzσ′)3(βµBzσ′)4
−βHβµBσzzzz
e=ezz=1+++++...
[1!2!]3![4!]
(βµBz)2(βµBz)4βµBz(βµBz)3
=1+++...+σ′++...
2!4!z1!3!
=cosh(βµBz)+σzsinh(βµBz)(8.2)
188
CHAPTER8.TISTICALASTMECHANICSII189
whereewevhausedthefactthatσ′2n=1;σ′2n+1=σ′forallnin{0,1,...}Alsoewevha
zzz
Tr(1)=2Tr(σ′)=0
z
SotakingtraceofEq.(??)ewget
′
Tr(e−βH)=Tr(cosh(βµB)+σsinh(βµB))=cosh(βµB)Tr(1)+sinh(βµB)Tr(σ′)=2cosh(βµB)
zzzzzzz
Sotheydensitoperator(8.1)becomes
cosh(βµB)+σsinh(βµB)11
′zzz′
ϱˆ=2cosh(βµB)=2+2σztanh(βµBz)
z
wNotheexpectationaluevofoperatorσzforthe
⟨σ′⟩=Tr(ϱσˆ′)=Tr(1σ+1σ2tanh(βµB))=Tr(1σ+1tanh(βµB))=tanh(βµB)
zz2z2zz2z2zz
Thisesgivtheexpectationaluevoftheoperator.Thisexpressionisthesameastheoneewobtained
usingthebasisstateswhereσaswdiagonalinsteadofσthatewevhahere.□
zx
8.1.2.eDerivtheties,uncertain∆x,∆pand∆E,ofafreeparticlein3Dbxousingtheydensitmatrixexpression
inthecoordinatetation.represenThencalculatetheytuncertainproduct∆x·∆p.
Solution:
orFaparticleinabxothetheydensitmatrixisengivyb
′1[m′2]
⟨r|ϱˆ|r⟩=Vexp−2βℏ2|r−r|
Theeragevapositionoftheparticleisengivyb
∫[]
1m′233
⟨r⟩=Tr(rϱˆ)=exp−|r−r|rdr=R
V2βℏ2′4
r=r
Theeragevasquaredpositionisengivyb
∫[]
⟨2⟩21m′22332

r=Tr(rϱˆ)=exp−|r−r|rdr=R
V2βℏ2′5
r=r
Sotheyuncertainitinthepositionofparticleisengivyb
∆r=√⟨r2⟩−⟨r⟩2=1√3R
45
wNotheeragevaaluevoftummomenisengivyb

∫[]∫
−iℏ∂m′23ℏ3
⟨p⟩=Tr(pϱˆ)=exp−|r−r|dr=−i0dr=0
V∂r2βℏ2′V
r=r
Theeragevatummomensquaredis
2∫2[]
⟨2⟩2ℏ∂m′23
p=Tr(pϱˆ)=−exp−|r−r|dr=3mkT
22
V∂r2βℏr=r′
CHAPTER8.TISTICALASTMECHANICSII190
Againtheyuncertainitintummomenisengivyb
√22√
∆p=⟨p⟩−⟨p⟩=3mkT
Sotheyuncertainitproductis
∆r·∆p=3√mkTR
45
Thisesgivtheyuncertainitproductinpositionandtum.momen□
8.1.3.evProthat
−βH′[(∂)]′
⟨q|e|q⟩=exp−βH−iℏ∂q,qδ(q−q),
where
H(−iℏ∂,q)
∂q
istheHamiltonianofthesysteminthetation,q-represenhwhicformallyoperatesupontheDiracdelta
function,δ(q−q′).riteWδ-functionisasuitableform;applythisresulttoafreeparticle.
Solution:
letψ(q)=⟨n|q⟩beenergyeigenfunctionwithalueveigenEinconifgurationspaceq.Thenyb
nn
hroscdingersequationewevha
H(−iℏ∂,q)ψ(q′)=Eψ(q′)
∂qnnn
SinceewwknothatforoperatorsAψ(x)=λϕ(x)=⇒f(A)ϕ(x)=f(λ)ϕ(x)
−βH(−iℏ∂,q)′−βE′
e∂qψn(q)=enψn(q)
Thiscanbeusedtowrite
−βH′∑−βH′(∑)
⟨q|e|q⟩=⟨q|n⟩⟨n|e|q⟩Inserting|n⟩⟨n|
nn
∑−βE∗′
=ψ(q)enψ(q)
nn
n∑
H(−iℏ∂,q)∗′
=e∂qψ(q)ψ(q)
nn
n
ButsincethetheeigenfunctionsoftheHamiltonianareorthogonaltoheacotherewget∑ψ∗(q′)ψ(q)=
′n
δ(q−q)ewget
−βH′H(−iℏ∂,q)′
⟨q|e|q⟩=e∂qδ(q−q)(8.3)
Thisistherequiredexpressionforthematrixtelemenoftheydnesitoperatore−βH.
eWcanalsowritetheδ–functionusingthefouriertransformtaationrepresenofδ–functionas
()∞
′13∫ik(q−q′)
δ(q−q)=2πedk(8.4)
−∞
CHAPTER8.TISTICALASTMECHANICSII191
orFafreeparticletheHamiltoniancanbewrittenas
∂p2ℏ2∂2
H(−iℏ,q)==−2(8.5)
∂q2m2m∂q′
wNousing(8.5)and(8.4)thisin(8.3)ewget
()∞
−βH′13∫H(−iℏ∂,q)ik(q−q′)
⟨q|e|q⟩=e∂qedk
2π
−∞
()∞
13∫−βℏ2+ik(q−q′)
=e2mdk
2π
−∞
Thiscanbeedsolvybcompletingthesquareintheexptialonenandusingthegammafunctiontheifnal
resultis
()3
m2−m(q−q′)2
⟨q|e−βH|q′⟩=2e2βℏ2
2πβℏ
Thisisthematrixtelemenoftheydensitoperatorforthefreeparticleinabx.o□
8.1.4.eDerivtheydensitmatrixρforafreeparticleinthetummomentationrepresenandstudyitsmain
properties,hsucastheerageva,energytum.momen
Solution:
TheHamiltonianofthefreeparticleinmotumementationrepresenis
2
H=pˆ
2m
Let|ψ⟩bethetummomenefunctionvawoftheparticlethentheexpressionforthetummomene-vaw
k
functionis
1ik·r
ψ(r)=√e
kV
Sincethetummomeneigenfunctionsemakcompletesetofstatestheyareorthonormal
⟨ψ|ψ′⟩=δ′
kkk,k
wNothecannonicalpartitionfunctionofthesystemis
β
−H
Q(V,T)=Tre
∑−βH
=⟨ψ|e|ψ⟩
kk
k
∑2
−βℏk2
=e2m
k
Sincethestatesareeryvcloseintummomensapceewcanreplacethesumybtegralin
∫2
V−βℏk2
Q(V,T)=dKe2m
3
(2π)
()3
V2mπ2
=32
(2π)βℏ
=V
λ3
CHAPTER8.TISTICALASTMECHANICSII192
Thematrixtelemenofthisoperatorwnobecome
3βℏ22
′λ−k
⟨ψk|ϱˆ|ψ⟩=e2mδk,k′
kV
Thsistherequriedydensitmatrixtationrepresenintummomensapce.□
8.1.5.eWedwshoinclassthatlinearlypolarizedtlighcorrespondstoapurestateandnon-polarizedtlighisin
amixedstate.Whatisthecircularlypolarized,amixedstateorapurestate?erifyVourytstatemen
Solution:
olarizedPtlighustmbepurestatebecause,atyanengivtimeitonlyhascomptsonenTheowtplane[][]
polarizedcomptsonenxbetedrepresenybA1aandyplanepolarizedbetedrepresenyb0.The
01
mostgeneralpolarizationofthetlighcanbewrittenasthelinearbinationcomoftheseowtplane
polarizedconptsonenas
[][]
1iθ10iθ2
P=ae+e
gen01
whereaandbingeneralarecomplexbumers.norFacircularlypolarize.Iftheowtplanepolarized
comptsonenevhaatotalphasedifferenceofnπthenthetlighisplanepolarized.Butforthephase
differenceδ=θ−θ=(2n+1)πthetlighiscircularlypolarized.Letthephaseθ=0andθ=π/2
21212
hsucthephasedifferenceisπ/2ewget
1[]i[]
P=√1+√0
circular2021
wNoforthistation,representheydensitmatrixcanbeobtainedeasilyas
[∗∗][]
ϱˆ=aaab=1/2−i/2
∗∗
babbi/21/2
Thistsprepresenapurestateas
[][][]
21/2−i/21/2−i/21/2−i/2
ϱˆ=i/21/2×i/21/2=i/21/2=ϱˆ
Thiseriifesvthatthecircularlypolarizedtlighispurestate.□
8.2orkHomewowT
8.2.1.eWtionedmeninclassthatincalculatingthematrixofe−βH,⟨1,2,3,N|e−βH|1,2,3,N⟩,putationerm
goththeparticlecoordinatesintheifrstevawfunctionandenergystatesinthesecondyieldsaresult
hwhicisN!oftheresultforaifxedsetof{k,}statesthatis,withoutputingermtheenergystates.Do
itexplicitlyofowtparticleandowtstatecasestartingwithu(1)u(2).
ab
Solution:
ThegeneralmatrixtelemenforNparticlenstatesystemfromathriaPeq(5.5.12)is
[][]
−βH′′1∑−βℏ2k2∑∑{∗∗}
⟨1,...,N|e|1,...,N⟩=e2mδ{u(p)...u}...δu(p)...u
pk1knpk1kn
N!11
kpp
orFowtparticleandowtsateewget
∑22
−βH′′1−βℏk∗∗∗∗
⟨1,2|e|1,2⟩=e2m[u(1)u(2)±u(2)u(1)][u(1)u(2)±u(2)u(1)]
2!abababab
k
CHAPTER8.TISTICALASTMECHANICSII193
Multiplyingtheefunctionsvawewget
−βH1∑−βℏ2k2∗∗∗∗
⟨1,2|e|1′,2′⟩=e2m[u(1)u(2)u(1)u(2)±u(1)u(2)u(1)u(2)
2!abababab
k
∗∗∗∗
+u(1)u(2)u(1)u(2)±u(1)u(2)u(1)u(2)]
abababab
orFthecaseofifxed{k},i.e.,ifonlytheparticlesareputederm
i
−βH′′1∑−βℏ2k2∗∗∗∗
⟨1,2|e|1,2⟩=e2m[u(1)u(2)u(1)u(2)±u(1)u(2)u(1)u(2)]
2!abababab
k
Butsincetheydensitoperatorishermition,thematrixtselemenareequaltothecomplexconjugateof
itselfwiththecoordinatehangedexc
−βH′′−βH′′∗
⟨1,2|e|1,2⟩=⟨1,2|e|2,1⟩
Thisouldwtiallyessenmean
∗′∗′∗′∗′
u(1)u(1)u(2)u(2)=u(2)u(2)u(1)u(1)
aabbaabb
∗′∗′∗′∗′
u(2)u(1)u(1)u(2)=u(1)u(2)u(1)u(2)
aabbaabb
Usingthisinthesumewget
∑22
−βH′′1−βℏk∗∗∗∗
⟨1,2|e|1,2⟩=e2m[u(1)u(2)±u(2)u(1)][u(1)u(2)±u(2)u(1)]
2!abababab
k
∑22
−βℏk∗∗∗∗
=e2m[u(1)u(2)u(1)u(2)±u(1)u(2)u(1)u(2)]
abababab
k
Herethelastexpressionisexactlywicettheexpressionforifxed{ki}case.Where2isequaltothe
factorialofitself2!=2usththerusultisN!timestheexpressionforifxed{k}case.□
i
8.2.2.Studytheydensitmatrixandthepartitionfunctionofasystemoffreeparticles,usingunsymmetrized
evawfunctioninsteadofsymmetriedevawfunction.wShothat,wingfollothetextprocedure,on
1
tersencounterneighthGibbs’correctionfactorN!noraspatialcorrelationamongtheparticles.
Solution:
Ifewusedunsymmetrizedevawfunctionratherthansymmetrizedevawfunctionewget
−βH∑−βℏ2k2∗′∗′
⟨1,2,...,N|e|1,2,...,N⟩=e2m(u(1)...u(N))(u(1)...u(N))
kkkk
1n1N
k
∑2k2+...+k2
βℏ1N∗′∗′
=e2m(u(1)...u(N))(u(1)...u(N))
kkkk
1n1N
k,...k
1N
Thesummationintheexptialonencanwnobehangedctoinproductoftheexptialonenandtheex-
pressionbecomes
N[{}]
∏−βℏ2/2m∗′
=eu(i)u(j)
kikj
i=1
Sincethestatesaredenseewcanhangecthesummationervokiybthetegrationin
()3N()
m2m()
−βH′2′2
⟨1,2,...,N|e|1,2,...,N⟩=exp−|r−r|+...|r−r|
2πβℏ22βℏ212NN
CHAPTER8.TISTICALASTMECHANICSII194
romFthisexpressionitseasytocalculatethediagonalts,elemenbecausefordiagonaltselemenewevha
r=r′.Thisesmaktheexptialonenticallyidenequaltozeroandewgetthematrixtelemen
ii
()3N
m2
−βH
⟨1,2,...,N|e|1,2,...,N⟩=2πβℏ2
Usingtheelengthvawparameter
λ=√m2
2πβℏ
ewgettheMatrixtelemenas
()
13N
⟨1,2,...,N|e−βH|1,2,...,N⟩=λ
wNothecannonicalpartitionfunctionisjustthetraceofthisexpression
∫()()
13NVN
−βH3N
QN(T,V)=Tr(e)=λdr=λ3
Thisexpressionhasneigherthegibbscorrectionfactor1northespatialcorrectionfactor.□
N!
3
8.2.3.Determinethealuesvofthedegeneracytdiscriminannλforydrogen,hheliumandxygenoat.NTP
eMakanestimateoftherespeyectivtemperaturerangeswherethemagnituesofthisytitquanbecomes
blecoparamtoyunitandhencetumquaneffectsbecomeimpt.ortan
Solution:
Theytitquannλ3canbewrittenintermsoftemperatureandboltzmantconstanas
333
nλ3=nh=Nh=hP(8.6)
(2πmkT)3/2V(2πmkT)3/2(2πm)3/2(kT)5/2
orFstandardtemperatureandpressure
T=293KandP=1.01×105
UsingthemassofHiydrogen,HeliumandOxygenewget
−345
36.63×101.01×10−5
H:nλ==2.86×10
2−273/2−235/2
2π(1.67×10)(1.38×10×293)
−345
36.63×101.01×10−6
He2:nλ=−273/2−235/2=3.61×10
2π(6.64×10)(1.38×10×293)
−345
36.63×101.01×10−7
O:nλ==4.78×10
2−273/2−235/2
2π(25.6×10)(1.38×10×293)
ertingINvtherelation(8.6)andsettingnλ3≃1ewget
(62)1/5
T=1hP
K(2πm)3
SoforthetdifferenmassesofH2,He2andO2ewget
H:T=4.46K
2
He2:T=1.95K
O:T=0.868K
2
CHAPTER8.TISTICALASTMECHANICSII195
Thisegivthetemperatureinhwhicthetdiscriminaniscloseto1.□
8.2.4.Asystemconsistsofthreeparticles,heacofhwhichasthreepossibletumquanstates,hwhicenergy0
,2E,or5E.respelyectivriteWoutthecompleteexpressionofthecannonicalpartitionfunctionQfor
thissystem:
(a)ifthearticlesobeyells-BoltzmanMaxwstatistics.
Solution:
Thesingleparticlecannonicalparitionfunctionfor
∑−βEn−2β−5β
Q(V,T)=e=1+e+e
1
n
ThecannonicalpartitionfunctionforNdistinguishableparticlesisobtainedybQ(V,T)=
N
1[Q(V,T)]NSoforthreeparticlesewget
N!1
1[−2β−5β]3
Q3(V,T)=3!1+e+e
Thefreeenergyofthesystemis
(1[])()
−2βE−5βE3−2βE−5βE
F=kTlnQ=kTln61+e+e=−kTln6+3kTln1+e+e
Theytropenisengivyb
S=−(∂F)
∂TN,V
(−2E−5E)()
Tk6EeTk+15EeTk−2E−5E3
T2kT2k1+eTk+eTk
=−2E−5E+kln6
1+eTk+eTk
Thisesgivtheexpressionfortheytropenoftheparticles.□
(b)iftheyobeyBose-Einsteinstatistics,
Solution:
orFboseeinsteincase,theparticlesaretedcounindistinguishable.Soheacofthethreeparticle
canbelongtowingfolloenergystateSothetotalpartitionfunctionofthesystembecomes
n0,n1,n25,0,25,5,25,5,05,2,20,2,25,0,00,2,05,5,52,2,20,0,0
otalTEnergy7E12E10E9E4E5E2E15E6E0
−2Eβ−4Eβ−5Eβ−6Eβ−7Eβ−9Eβ−10Eβ−12Eβ−15Eβ
QN(T,V)=1+e+e+e+e+e+e+e+e+e
SimilarlythefreeenergyisengivybF=kTlnQ(V,T)andtheytropenisengivybS=−∂F
N∂T
Thisesgivtheexpressionfortheytropenoftheparticles.□
(c)iftheyobeyermi-DiracFstatistics,
Solution:
orFtheparticlesatisfyingermi-DiracFstatisticsnoowtparticlescanoyccupthesameenergyelslev
soheachastositonitswnoenergyenlevhwhicesgivthepartitionfunction
[−2βE−5βE]
QN(V,T)=1+e+e
CHAPTER8.TISTICALASTMECHANICSII196
Thefreeenergyofthesystemis
(−2E−5E)
F=Tklog1+eTk+eTk
Sotheytropenbecomes
(3Eβ)()
∂FE2e+5−2Eβ−5Eβ
S=−=−5Eβ3Eβ−klog1+e+e
∂TT(e+e+1)
Thisesgivtheytropenofparticlesforermi-DiracFstatistics.□
8.3orkHomewThree
8.3.1.athria(PandBeale6.1)wShothattheytropenofanidealgasinthermalequilibriumisengivyb
theulaform
S=k∑[⟨n+1⟩ln⟨n+1⟩−⟨n⟩ln⟨n⟩]
εεεε
ε
inthecaseofbosonsandybtheulaform
S=k∑[−⟨1−n⟩ln⟨1−n⟩−⟨n⟩ln⟨n⟩]
εεεε
ε
inthecaseoffermions.erifyVthattheseresultsaretconsistenwiththegeneralulaform
S=−k∑{∑p(n)lnp(n)},
εε
εn
wherep(n)istheyprobabilitthatthereareexactlynparticlesintheenergystareε.
ε
Solution:
Thegeneralformofytropenofofthesystemisengivyb
∑[()()(∗)]
ggns
S=Kn∗lni+n∗−iln1−ai
in∗iag
iii
∗
where,nisthesetconifrmingtomostprobabledistributionamongthecells.Withthedegeneracy
i
factorg=1,ewgetni=n∗Alsotheeragevanisengivyb
igiiε
⟨n⟩=z(∂q)=1=n∗
ε−1−βϵi
∂zV,Tze+a
Substitutingn∗=⟨nε⟩ewget
i
∑[(1)]
S=k−⟨n⟩ln⟨n⟩+⟨n⟩−ln(1−a⟨n⟩)
εεεaε
ε
wNoforbosonsa=−1,ewget
S=k∑[−⟨n⟩ln⟨n⟩+(⟨n⟩+1)ln(1+⟨n⟩)]
εεεε
ε
hWhicistherequiredexpressionforthebosons.wNoforfermionsewsubstitutea=1andobtainoT
wshothatthegeneralexpression
CHAPTER8.TISTICALASTMECHANICSII197
S=−k∑{∑p(n)lnp(n)},
εε
εn
orkswfortheytropenewifrstnoticethattheexpressioncanbemodiifedrewrittenas
S=−k∑⟨lnpε(n)⟩
ε
Alsoforbosonstheyprobabilitofvinghaexactlynparticleinthestatewithenergyεisengivyb
⟨n⟩n
p(n)=ε(8.7)
εn+1
(⟨nε⟩+1)
lnp(n)=nln⟨n⟩−(1+n)ln(⟨n⟩+1)(8.8)
εεε
wNosubstitutingthistothegeneralexpressionofytropentheinnersummervoallnbecomes
S=−k∑⟨nln⟨n⟩−(1+n)ln(⟨n⟩+1)⟩
εε
ε
=−k∑⟨nε⟩ln⟨nε⟩−(1+⟨nε⟩)ln(⟨nε⟩+1)
ε
=k∑[−⟨n⟩ln⟨n⟩+(⟨n⟩+1)ln(1+⟨n⟩)]
εεεε
ε
hwhicwsshothatthegeneralexpressionistrueforbosons.
Substitutinga=1forfermionsewget
S=k∑[−⟨n⟩ln⟨n⟩+(⟨n⟩−1)ln(1−⟨n⟩)]
εεεε
ε
hWhicistherequiredexpressionforthefermions.ytropen
Theyprobabilitofvingfaexactlyn={0,1}particlesinthecellforfermoinsisengivyb
{
1−⟨n⟩ifn=0
p(n)=ε
ε⟨n⟩ifn=1
ε
Thisesgivonlyowttermsintheinnersumofthegeneralexpressionso
S=−k∑[⟨nε⟩ln⟨nε⟩+(1−⟨nε⟩ln(1−⟨nε⟩))]
ε
hWhicwsshothatthegeneralexpressionholdsforfermionstoo.□
8.3.2.athria(PandBeale6.2)eDerivforallthreestatistics,thetanrelevexpressionsfortheytitquan
⟨⟩(∂⟨n⟩)
n2−⟨n⟩2=kTε
εε∂µ
T
Comparewiththepreviousresultsthatewedwshoinclass,
⟨2⟩2(∂⟨n⟩)
n−⟨n⟩=kT∂µT
forasystembemeddedinagrandcanonicalble.ensem
Solution:
CHAPTER8.TISTICALASTMECHANICSII198
Thisproblemistheifndtheifrstandsecondtsmomenofnεandtheirdifference.Onceewwknothe
yprobabilitmassfunction(pmf)oftheariablevifndingtmomenquitegenerallyis
⟨f(x)⟩=∑f(x)p(x)
x
wherep(x)isthepmf.wNoforthebosons,(8.9)canbetlyslighrewrittenas
()()
⟨n⟩n1⟨n⟩n⟨n⟩⟨n⟩n
p(n)=ε=ε=1−εε(8.9)
ε(⟨n⟩+1)n+1⟨n⟩+1(⟨n⟩+1)n1+⟨n⟩⟨n⟩+1
εεεεε
Withsubstitution⟨nε⟩=tewget
1+⟨nε⟩
p(n)=(1−t)tn
wNotheifrsttmomenofthispmfis
∞∞
⟨n⟩=∑n(1−t)tn=(1−t)t2=t∵12=∑ntn−1
n=0(1−t)1−t(1−t)n=0
Similarlythesecondtmomenis
∞
⟨n2⟩=∑n2(1−t)tn=(1−t)t(1+t)=t(1+t)∵
(1−t)3(1−t)2
n=0
usThtheariancevis
⟨⟩2
n2−⟨n⟩2=t(1+t)−t=t
εε(1−t)2(1−t)2(1−t)2
wNosubstutingkbacthealuevoftewget
⟨2⟩22
nε−⟨nε⟩=⟨nε⟩+⟨nε⟩
orFtheermionsFewget
1
⟨2⟩∑2
n=np(n)=p(1)=⟨n⟩
εεεε
n=0
Thisimpliestththeariancevis
⟨2⟩22
nε−⟨nε⟩=⟨nε⟩−⟨nε⟩
orFBoltzmannparticlethepmfisapoissondistribution
⟨n⟩ne−⟨nε⟩
p(n)=ε
εn!
orFpoissondistributionisitcanbeeasilywnshothatthemeanandariancevisjusttheparameter⟨nε⟩
usThewevha
⟨2⟩2
n−⟨n⟩=⟨n⟩
εεε
Lookingatheacofthesethreeariancesvewseethatitisofthegeneralfrom
CHAPTER8.TISTICALASTMECHANICSII199
⟨2⟩22
nε−⟨nε⟩=⟨nε⟩−a⟨nε⟩
Alsotheexpectationaluev⟨nε⟩isengivyb
⟨n⟩=1
εz−1eβε+a
tiatingDifferenthiswithrespecttoµattconstantemperatureewget
[∂⟨nε⟩]⟨nε⟩2[1]
∂µT=kT⟨nε⟩−a
Rearrangingewget
[∂⟨n⟩]
KTε=−⟨n⟩−a⟨n⟩2
∂µεε
T
wNothecomparisionofthisexpressionforallthestatisticsleadsto
⟨n2⟩−⟨nε⟩2=KT[∂⟨nε⟩]
ε∂µ
T
Thisexpressionistrueingeneralforallstatistics.□
8.3.3.(K.Huang8.6)Whatistheequilibriumratioofortho-toydrogenpara-hatatemperatureof300K?
Whatisthisratiointhelimitofhightemperature?Assumethatthedistanceeenbwettheprotonsin
themoleculeis0.74Angstrom.
Solution:
Theequilibriumratioisengivyb
∑−βℏ2/2Il(l+1)
N(2n+1)e
ortho=3∑n=odd
N(2n+1)e−βℏ2/2Il(l+1)
paran=even
aluatingEvthissumexplicitlywithseriesmethodewgetorFlargealuesvofntheratiogotoone
becauseforlargentheowttitiesquaninNumeratoranddenomenatoraretiallyessenthesame.Soew
get
N
ortho=3
N
para
Thisesgivtheequilibriumratiooforthoandparaydrogenhinthetemperaturerequired.□
8.3.4.Considerthethermalpropertiesofconductingelectronsinametalandtreatelectronsasteractingnon-in
particles,whenparticleydensitishigh.AssumingheacCuatomdonatesanelectrontotheconducting
electrongas,calculatethehemicalcptial,otenortheermiF,energyofcopper,forhwhicthemassydensit
9g
is3.ExpressouryeranswinKelvin.
cm
Solution:
Thefermienergyisengivyb
2()2
h3N3
Ef=8mπV
CHAPTER8.TISTICALASTMECHANICSII200
orFCutheydensitofatomsis
28−3
n=8.5×10m
usThewgetthefermienergyequalto
()2()2
23−342−3
h3N(6.6×10)328−18
Ef=8mπV=8×9.1×10−31π8.5×10=1.1×10=6.7eV
4
Inelvinkthisistalenequivto6.7eV=6.4×10K□
8.4orkHomewourF
8.4.1.eDerivthevirialexpansionoftheidealBosegasybertingvintherelationnλ3=g(z)seriesto
3/2
expresszintermsofnλcbandthesubstituteitintheP/KTequation.Usingthisexpressionederiv
theexpansionforCv/Nkalidvathightemperaturelimit.
Solution:
orFhightemperatureN≪Ntherelationcanbewrittenas
0
z2z3
nλ3=g(z)=z+++...
3/23/23/2
23
oTertvintheserieswithusualhniquetecewwritethezasaerpwoseriesinnλ3as
z=c(nλ3)+c(nλ3)2+c(nλ3)3+...
123
substutingthealuevofztointheifrstseriesewget
[][(c(nλ3)+c(nλ3)2)2]
nλ3=c(nλ3)+...+12+...
13/2
2
3
Comparingthecotseiffcienofelikerspwoofλinbothsidesewget
c22ccc3
c=1;c+1=0c+12+1=0
12−3/23−3/2−3/2
223
ritingWsimilarlyewget
c1=1;c2=−1c3=1−1
−3/24−3/2
23
wNotheexpressionlnQbecomes
PV=13(z+z2+z3+...)
NkTnλ−5/2−5/2
23
Substutingthealuevofzfromtheseriesinnλ3withtheariousvcotseiffceinc,c...ewget
12
∞()
PV∑λ3l−1
=a
NkTlv
l=1
Thisistherequiredvirialexpansionoftheexpression.wNoforthespeciifcheatattconstanolumev
∂U
ewevhatoifndout∂T,thiscanbesimpliifedas
Cv=1(∂U)=3[∂(PV)]
NkNk∂TN,V2∂TNkv
CHAPTER8.TISTICALASTMECHANICSII201
Insimilarfashionfortheexpansionofg5/2(z)ewget
∞()
C∑35−3lλ3l−1
v=a
Nk22lv
l=1
Substutingallthecoteiffcienewget
C3[(λ3)(λ3)2]
v=1+c+c+...
Nk21v2v
wheretehcotseiffcienarec1=0.088,c2=0.0065,...Thisistheexpressionofspeciifcheatofbosegas
correctathightemperature.□
8.4.2.athria(P&Beale,7.3)biningComequation7.1.24and7.1.26,andmakinguseoftheifrstowterms
ofulaform(D.9)inAppendixD,wshothat,asThesapproacT,fromeabvotheparameterα(=lnz)of
theidealbosegasassumestheform
()()
13ζ(3/2)2T−T2
α=c
π4T
Solution:
eWevhafrompreviousproblemnλ3=g(z).Butatλ=λcewevhaz=1.Butforz=1
3/2
g(1)=1+1+1+...=ζ(3/2)
3/23/23/2
23
Substutingthisintheexpressionforthecriticaltemperatureandtakingtheratio
()()2
Tλ2g3/2(z)−3
Tc≡λc=ζ(3/2)
Theexpressionforg(z)canbeexpandedintermfosseriestheseriesexpansionfromappendixD.9
3/2
canbeusedtoobtain
(√)2
Tζ3/2−2πα+...−3
T=ζ(3/2)
c
Sinceewevhaα≪1ewcanemakuseofbinomialexpansionoftheseries
(1+x)n≈1+nx;x≪1
Usingjusttheifrstowttermsewget
√
T≈1+4πα
T3ζ(3/2)
c
w,Nothisexpressioncanbesimpliifedfurthertoget
√(T−T)
4πα=3ζ(3/2)c
T
c
Squaringbothsidesledsto
()
13ζ(3/2)(T−T)2
α=c
π4T
Thisistherequiredexpression.□
CHAPTER8.TISTICALASTMECHANICSII202
8.4.3.eDerivindetailedstepsthewingfolloexpressionforanidealBosegas.
C15g(z)9g(z)
v=5/2−3/2
Nk4g(z)4g(z)
3/23/2
Solution:
orFidealbosegasfrom7.1.7and7.1.8ewget
P=1g(z)
kTλ35/2
N−N1
0=g(z)
Vλ33/2
tAhightemperatureewcanassumethatz≪1iseryvsmallandewcansafelyignoreN.eWcanetak
0
theratiooftheseowttitiesquantoget
PV=g5/2(z)
NkTg3/2(z)
Alsotheternalinenergycanbecalculatedas
(∂)2(∂(PV))3V
U≡−lnQ=kT=kTg(z)
∂β∂TKT2λ35/2
z,Vz,v
wNotheexpressionforthespeciifcheatis
Cv=∂U=[∂(3Tg5/2(z))]
∂T∂T2g3/2(z)v
wNoewcanusetherecurrancerelationforthefunctiong(z)as
z∂g(z)=g(z)
∂zνν−1
Alsosincethefunctiong3/2(z)isproportionaltocuberootofthesquareofthetemperatureewget
[∂]3
∂Tg3/2(z)v=−2Tg3/2(z)
biningComtheseowtexpressionsewget
1(∂z)=−3g3/2(z)
z∂Tv2Tg1/2(z)
wNocarryingoutthetiationdifferenoftheexpressionCewget
v
3g(z)∂(g(z))∂z
C=Nk5/2+Nk5/2
v2g(z)∂Tg(z)∂T
3/23/2
∂z
Usingthepreviousexpressionfor∂Tandusingtheproductruleinthetiationdifferenewget
Cv=3[5g5/2(z)−3g3/2(z)]
Nk22g3/2(z)2g1/2(z)
Simplifyingtheexpressionesgiv
C15g(z)9g(z)
v=5/2−3/2
Nk4g(z)4g(z)
3/21/2
Thisistherequredexpressionforthespeciifcheatofbosonsinhightemperaturelimit.□
CHAPTER8.TISTICALASTMECHANICSII203
8.4.4.evProthewingfolloforandatomicBosegaswithspinS
(a)Itsydensitofstateisengivy:b
()
2m3/2
g(E)=2πV(2S+1)E1/2
2
h
Solution:
Ifewconsideratomicteratingnon-inatomicgaswithspinS,thenforheactummomenstate,there
are2S+1spinstates.Thenthegrandpartitionfunctionbecomes
∏2S+1
Q=Q
i
i
Thegrandptialotenbecomes
∑(−β(ε−µ))
Φ=−kTlnQ=kT(2S+1)ln1±e
i
ximatingApprothesumwiththetegrationinewget
∞
∫(−β(ε−mu))
Φ=kT(2S+1)ln1+eg(E)dE
0
Hereg(E)istheydensitofstateshwhiccnbesimpliifedforuniformalydistributedparticlesas
4πk2(2S+1)dk(2S+1)VK2dk
g(k)dk=(2π/L)3=2π2
22
WitholumevV=L3andE=hkewget
2m
√()
(2S+1)VEdE2m3/2
g(E)dE=(2π)2ℏ2
Usingℏ=handwritingtheydensitofstatesewget
2π
()
√2m3/2
gEπVS
()=2(2+1)Eh
hWhicistherequiredydensitofstates.□
(b)ThenwshothatitsBose-Enisteintemperatureisengivyb
[]
h2n2/3
Tc=2πmk2.612(2S+1)
Solution:
wNothetotalbumernoparticlesNcanbeobtainedas
∞
N=∫g(E)dE
β(ε−µ)
e+1
0
Substutingtheydensitfunctionewget
[()]∞√
N=2πV2m3/2∫EdE
2−1βE
h0ze−1
CHAPTER8.TISTICALASTMECHANICSII204
Thetegrandincanberecocnizedastheeinsteinfunctiong(z).soewget
3/2
N=(2S+1)Vg(z)
λ33/2
orFT=Tewcanconsiderzcomparableto,yunitus,thewevhaz=1,substutingthisewget
C
g(1)=ζ(3/2)=2.612
3/2
nλ3
2S+1=ζ(3/2)=2.612
[s]3/2
Makingthissubstutionandrecocnizingλ=hRearraingingewget
2πmkT
2[]
T=h2n
cπmk2.612(2S+1)
whereewevhamadeuseofn=N.ThisesgivtheexpressionforthecriticaltemperatureofBose
gas.V□
Chapter9
articlePysicsPh
9.1orkHomewOne
9.1.1.(Griiffth1.2)Themassofa’swukaYmesoncanbeestimatedasws.folloWhenowtprotonsina
ucleusnhangeexcameson(massm),theyustmtemporarilyviolatetheationconservofenergyyban
2
tamounmc(therestenergyofthemeson).ThebHeisenergytuncertainprincipleyssathatouyyma
’bw’orroanenergy∆E,videdproouyy’paitk’bacinatime∆tengivyb∆E∆t=ℏ(whereℏ=h/2π).
22
Inthiscase,ewneedtobworro∆E=mclongenoughforthemesontoemakitfromoneprotonto
theother.Ithastocrosstheucleusn(sizer0),anditels,vtra,presumablyatsometialsubstanfraction
ofthepeedoft,lighso,roughlyspeaking,∆t=r0.Puttingallthistogether,ewevha
c
m=ℏ
2r0c
−13MeV
Usingr0=1×10cm,calculatethemassofa’swukuYmeson.Expressouryeranswinc2,and
comparetheedobservmassofπon.
Solution:
−15−228
enGivr0=1×10m,M=6.58×10MeVs;c=3×10sewcansubstitutetoifndthetotalmass
m=ℏ=(ℏc1)=98.7MeV
22
2rc2rcc
00
Sothepredictedmassis98.7MeV,buttherealmassofa’swukuYmesonis138Mevhwhicisoffyba
factorofabout1.4.□
−23
9.1.2.(a)bMemersofonbarydecupletypicallytydecaafter1×10secondstoinaterlighonbary(from++
theonbaryoctet)andameson(fromthepseudo-scalarmesonoctet).usThforexample,∆→
++−∗+∗−
p+π.Listallydecamethodsofthisformforthe∆,ΣandΞ.bRememerthatthese
ysdecaustmeconservhargecandstrangeness(theyarestrongteractions).in
Solution:
Theydecahastosatisfythehargecationconservandstrangenessation.conservThepossibleydeca
forheacoftheseare:
−−−0
∆→n+πandΣ+K
∗+¯0+0+00++0
Σ→p+k;Σ+π;Σ+η;π+Σ;Λ+π;K+Ξ
∗−0−−0−¯0−0−−
Ξ→Σ+K;Ξ+π;Σ+K;Λ+K;Ξ+π;Ξ+η
Theseareallthepossibleydecahemesscthatepreservhargecandstrangeness.□
205
CHAPTER9.TICLEARPPHYSICS206
(b)Inyan,ydecathereustmbetsuiffcienmassintheoriginalparticletoervcothemassesoftheydeca
products.(Thereymabemorethanenough;theextrawillbeed’soakup’intheformofkinetic
energytintheifnalstate.)kChecheacoftheydecaouyproposedinpart(9.1.2)toseehwhicones
meetthiscriterion.Theothersarekinematicallyforbidden.
Solution:
hEacoftheseysdecaareowtbodyysdecaoftheformA→B+C,thethresholdenergiesinheac
canbecalculatedwith
222
M−m−m
E=Bc
2M
A
Usingthemassaluevofheacoftheseproductsewifndthattheonlyedwalloysdecaare
∆−→π−+n
∗++0+0−
Σ→Σ+π;Λ+π;Σ+π
∗−0−−0
Ξ→Σ+π;Ξ+π
Thesearetheonlyedwalloys.deca□
9.1.3.(Griiffth2.5)
(a)hWhicydecadoouythinkouldwbemore,elylik
−−−−
Ξ→Λ+πorΞ→n+π
Solution:
−−−−
AlthoughtheydecaΞ→n+πisoredvfakinematicallyervotheydecaΞ→Λ+πstrangeness
uu
π−π−
W−dW−d
susWu
∗−ss0−sudn
ΞddΛΞdd
(a)Ξ∗−→π−+Λ(b)Ξ∗−→π−+n
Figure9.1:eynmanFdiagramforowttdifferenys.deca
ationconservorsvfathesecondone.Sincetheowtsquarksevhatobeedconserv(strangenesscon-
ation);servanextraW−isrequires.Thismeansthereareowtextraeakwertices.vHigherbumern
oferticesvouldwemaktheprocesshucmmoreless.elylik□
(b)hWhicydecaofD0(cu¯)mesonismostelylik
D0→K−+π+orD0→π−+π+,orD0→K++π−
hWhicisleastely?likwDratheeynmanFdiagrams,explainouryeranswandkhecctheexperi-
talmendata.
Solution:
TheeynmanFdiagramforD0→K−+π+is□
Thesecondydecaismoreoredvfabecausethereisnogenerationcrosservointheparticle.ydeca
Whenthereisagenerationcrosservointheydecaprocessitislessoredvfaintheydecaalthough
itisedwallo.kinematicallySothemostoredvfaydecaprocessisD0→π−+π+.
CHAPTER9.TICLEARPPHYSICS207
dds
π+π+K+
WuWuWu
0cs−D0cd−D0cd−
DuuKuuπuuπ
(a)D0→π++K−(b)D0→π−+π+(c)D0→π−+K+
Figure9.2:eynmanFdiagramforthreetdifferenydecahemesscforD0.
µ
9.1.4.(Griiffth3.13)Ispe,timelike,spacelikoretliklighfora(real)particleofmassm?wHoabouta
masslessparticle?wHoaboutavirtualparticle?
Solution:
2µ
oTdeterminethenatureoftheparticlesewifndthetzLorenscalarforh.eacFindingp=p·p=ppµ
ewget
222
p=mc
2
orFarealparticlewithmassmtheytitquanp>0sotheparticleise.timelikorFamasslessparticle
2
γthescalarp=0sothisise.tliklighAndforvirtualparticlethenaturedependsuponthemassas
therecouldbemasslessandemassivvirtualparticles.□
9.1.5.(Griiffth3.16)articlePA(EnergyE)hitsparticleB(atrest),producingC1,C2,...A+B→
C1+C2+...Cn.Calculatethethreshold(i.e.,umminimE)forthisreaction,intermsofariousv
particlemasses.
Solution:
InthelabframeletsconsiderparticleAwithmassmandtummomenpwithenergyEesstrika
AA
µ
stationarytargetparticleBwithmassm.ThefourtummomenofAisp=(E,p)andthefour
BAA
µ
tummomenofBisp=(m,0).ThetarianvintzLorenscalarinthelabframeis
Bb
2µµ22222
p=(p+p)=(E+m,p)=E+m+2Em−|p|
AbBABBA
222
ButforparticleAewevhaE−|pA|=mAsubstitutingthisineabvoexpressionewget
222
p=m+m+2EmB
AB
2
SincethistzLorenscalaristarianvininyanreferenceframeewevhatoevhathesamealuevforthep
fortheifnalproducts.orFthresholdconditiontheterdaughparticlesarejustcreatedsoythdonot
carryyantum.momenhWhicimpliesforheacparticlestheirtummomenmn=Ensoforheacofthem
µ
thefourtummomenisp=(m,0).ThetzLorenscalarfortheifnalyqtis
nn
2µµµ2222
p=(p+p+p)=(m+m+...+m,0)=(m+m+...+m)−0=My)(sa
12n12n12n
wherethebsymolsMisusedtomeanthetotalsumofmassesofallterdaughparticles.Equatingthe
tzLorenscalarewget
222
222M−m+m
M=m+m+2Em=⇒E=AB
ABb2m
B
Thisesgivthethresholdenergyinlabframeoftheincomingparticle.□
9.1.6.(Griiffth3.22)articlePA,atrest,ysdecatointhreeormoreparticles:A→B+C+D+...
(a)DeterminetheummaximandumminimenergiesthatBcanevhainhsuca,ydecaintermsofthe
ariousvmasses.
CHAPTER9.TICLEARPPHYSICS208
Solution:
Theummimimenergyfortheoutgoingparticleisequaltoitsmasswhentheproducedparticle
isjustcreatedandhasnospatialtummomenandallotherenergyiscarriedyawaybtheother
outgoingparticles.
Emin=mB
TheummaximenergyiscarriedybparticleBwhentheparticleAysdecainhsucayawthat
particleBesvmoinonedirectionandallotherparticlesevmoinotherdirectioninunison.Since
ewouldwgetummaximenergywhentheotherparticlesdonotevmoerelativtoheacothergiving
ummaxim,energythisimpliesthatallotherparticleevmoasasingleunitoftotalmasswiththe
sumoftheirmasses.Soewcanrewritethedecyas
A→B+(C+D...)≡B+N
wheretheparticleNisasititsasingleparticlewiththemassequaltosumofmassesofheacof
therestofterdaughparticles.
m=m+m+...
NcD
Thisproblemiswnoelikasingleparticleyingdecatoinowtwithequalandoppositetum.momen
22
IntheCMframethealuevoftzLorenscalarp=M
A
µµµ
p=p+p≡(m,0)=(E,p)+(E,−p)
ABNABBNB
=⇒(EN,−p)=(mA,0)−(EB,p)
BB
Squaringbothsidesandequating
222
(E,−p)=(m,0)+(E,−p)−2(m,0)·(E,p)
NBABBABB
22222
E−|pB|=m+E−|pB|−2mAEB
NAB
Sinceewevham2=E2−|p|2ewget
222
m=m+m−2mE
NABAB
222
2mE=m+m−m
ABABN
222
m+m−(m+m+...)
E=ABCD
B2m
A
ThisesgivtheummaximenergyoftheparticleB.□
−−
(b)Findtheummaximandumminimelectronenergiesinuonm,ydecaµ→e+ν¯+ν.
eµ
Solution:
Theumminimenergyoftheelectronisthemassofelectronitself(innaturalunitsofcourse)so
Emin=me=511keV
Byeabvodiscussiontheummaximenergyis
222
m+m−−(m+m)
µµµ¯
E=eee
max2m
µ
Sincetheneutrinosevhaeryvytinmass(almostmassless)ewignoretheirmasseswsoewevha
1052−0.5112
Emax≈2=52.50MeV
2×105
Thisesgivtheummaximmassoftheoutgoinguon.m□
CHAPTER9.TICLEARPPHYSICS209
9.2orkHomewowT
9.2.1.DiscussthepossibleydecamodesoftheΩ−edwalloybationconservws,laandwshowhoeakwydeais
theonlyremaininghoice.c
Solution:
TherearethreeydecaodesofΩ−.Theyare
−0−
Ω→Ξ+π
sssussud¯
−0−
Ω→Λ+K
sssudsus¯
−−0
Ω→Ξ+π
sssdssuu¯
Alloftheseprocessesviolatethestrangenessation.conservSotheycan’tproceedviastrongteraction,in
soeakwteractioninistheonlyhoice.c□
9.2.2.Determinehwhicisospinstatesthewingfollobinationcomofparticlescanexistin
(a)π0π−π0
Solution:
UsingtheclebshGordancotseiffcientowritethestatecompositionewget.
+−0⟩

πππ=|11⟩|1−1⟩|10⟩
111
|11⟩|1−1⟩=√|20⟩+√|10⟩+√|00⟩
623
|20⟩|10⟩=√3|30⟩−√2|10⟩
√5√5
|10⟩|10⟩=2|20⟩−1|00⟩
33
|00⟩|10⟩=|10⟩
SothepossibleisospintaionsbincomareI={0,1,2,3}□
(b)π0π0π0
0−0⟩

πππ=|10⟩|10⟩|10⟩
|10⟩|10⟩=√2|20⟩−√1|00⟩
33
|10⟩|00⟩=|10⟩
|20⟩|10⟩=√3|30⟩−√2|10⟩
55
SothepossibleisospintaionsbincomareI={1,3}
9.3orkHomewThree
0¯
9.3.1.(Griiffth6.6)Theπisacompositeobject(uu¯anddd),andsoequation6.23doesnotreally.apply
Butletspretendthattheπ0isatruetaryelemenparticleandseewhocloseewcame.,Unfortunatelyew
don’twknotheamplitudeM;erevwhoitustmevhathedimensionsofmasstimes,eloyvcitandthereis
onlyonemassandoneeloyvcitailable.vaer,vMoreotheemissionofheacphotontroinducesafactorof
CHAPTER9.TICLEARPPHYSICS210
√
α(theifnestructuret)constantoin0M,sotheamplitudeustmbeproportionaltoα.Onthisbasis,
estimatethelifetimeofπ.Comparetheexptalerimenalue.v
Solution:
Theydecarateforaparticleydecaisengivyb
Γ=S|p||M|2
2
8πℏm
π
AssumingtheydecaamplitudeisM=αmπewget
1pα2
Γ=(αm)2=|p|
2π
28πℏm16πℏ
π
Thethresholdenergyofheacoutgoingphotonis
E=1m
γ2π


eWcanusethefactthatforphotonpγ=Eγ,sotheoutgoingtummomencanbewrittenas
1

p=m
γ2π
Usingthisintheydecarateexpression
α2m
Γ=π
32πℏ
Sothelifetimeisengivybthereciprocalofydecarate
Lifetime(τ)=32πℏ
α2m
π
Substutingα=1andmassofπonis135MeVewget
137
τ=32π(6.58×10−22)=9.2×10−18s
135·1
1372
Sotheestimatedlifetimeis8.4×10−18s.Themeanlifetimefromtheparticledatagrouplisting1is
(8.30±0.19)×10−17s,hwhicisoffybaboutanordermagnitude.
□
9.3.2.(Griiffth6.8)considerthcaseofelasticscattering,A+B→A+B,inthelabframe,(Binitiallyat
rest)assumingthetargetissovyheam≫Ethatitsrecoilisnegligible.Determinethetialdifferen
BA
scatteringcrosssection.
Solution:
IntheCMframe,forowtbodyscatteringewevhathetialdifferenscatteringcrosssectionisengivyb
()22

dσℏcS|M|pf
dΩ=8π(E+E)2|p|
ABi
Sincethetargetparticleiseryv,vyheaitistiallyessenatrestafterthescattering.Sotheexpressionis
sameforCMframeandlabframe.Theenergyandtummomenoftheincomingparticleandoutgoing
particleistiallyessenthesameasthetargetparticledoesn’tetakyanappreciable.energy


pf=|pi|
1
http://pdg.lbl.gov/2018/listings/rpp2018-list-pi-zero.pdf
CHAPTER9.TICLEARPPHYSICS211
orFthevyheaparticle,sinceitistiallyessenatrest,theenergyisengivyb
222
E=|pB|+m
BB
hWhicfor|pB|≈0esgiv
E=m
BB
SinceengivthatEA≪mBewcanximateappro
E+E=E+m≈m
ABABB
AlsotheparticlesarenotticalidensothefactorS=1usThtheifnalexpressionforthescatteringis
engivyb
dσ=(ℏc)2|M|2
2
dΩ8πm
B
Thisistherequiredexpressionforthetialdifferencrosssectionofrecoil.□
9.3.3.(Griiffth6.9)Considerthecollision1+2→3+4inthelabframe(2atrest),withparticles3and4
massless.Obtaintheulaformfortialdifferencrosssection.
Solution:
Theexpressionforthetialdifferencrosssectionforthecollision1+2→3+4+...+nisengivyb2
2n3
Sℏ∏1dp
dσ=√|M|2(2π)4δ4(p+p−p−...−p)×√i
4(p·p)2−(mm)2123n22(2π)3
12122p+m
i=3ii
Inthelabframe,withparticle2atrest,ewevha
2222
p=|p2|+m=m
222
Alsofortheexpressionunderthesquarerootis,
p=(E,p)p=(m,0),⇒p·p=Em
111221212
Thisesgiv
√22√2222√222
(p·p)−(mm)=Em−mm=m(E−m)=m|p|
1212121221121
Substutingtheseforn=4ewget,
()∫
2233
Sℏ1dpdp
dσ=|M|2δ(E+E−|p|−|p|)×δ3(p+p−p−p)34
4pm4π12341234|p||p|
1234
thedeltafunctionemaktheptegralintrivialas
4
2∫3
dσ=Sℏ|M|2δ(E+m−|p|−|p−p|)dp3
64π2|p|m12313|p||p−p|
12313
Assumming3particlescattersoffatanangleθerelativtothetincidenparticle1ewget
222
(p−p)=|p|+|p|−2|p||p|cosθ
131313
3
Alsotheolumevtelemeninthephasespacedp3canbewrittenas
d3p=|p|2d|p|dΩ
333
2
Griifftheq.6.38
CHAPTER9.TICLEARPPHYSICS212
WhoreΩisthesolidangle.Thisenablesustowrite
∞()
dσSℏ2∫2√22|p|2d|p|
=|M|δE+m−|p|−|p|+|p|−2|p||p|cosθ√33
dΩ64π2|p|m1231313
12|p||p|2+|p|2−2|p||p|cosθ
031313
tAthisptoinallthetummomenparethespatialtummomenectorsvofheacparticles(nottoconfuse
withearliernotationptomeanfourtum),momenwritingforheac|p|=pandewevhathistegralin
iii
inpwherepisindeptendenofp
331
∞()
2∫√2
dσSℏ2pdp3
22√3
=|M|δE+m−p−p+p−2ppcosθ
dΩ64π2pm123131322
12pp+p−2ppcosθ
031313
√22
Thisisnoeasytegralintoorkwout,butlets,trysupposex=p+p+p−2ppcosθtiatingDifferen
31313
thiswithrespecttop3ewget
√22
dxp−pcosθp+p−2ppcosθ+p−pcosθx−pcosθ
311313311
dp=1+√22=√22=√22
3p+p−2ppcosθp+p−2ppcosθp+p−2ppcosθ
131313131313
Thisesgiv
dpdx
√3=
22x−pcosθ
p+p−2ppcosθ1
1313
Usingthisinthetegralinewget
∞
dσSℏ2∫pdx
=|M|2δ(E+m−x)3
dΩ64π2pm12x−pcosθ
121
0
Thistegralinerevwhoistrivialbecauseofthedeltafunction,asitonlykspicupthetermsforx=E+m
12
usthewget
dσSℏ2p
=|M|23
dΩ64π2pmE+m−pcosθ
12121
Thiscanbesimpliifedtowrite
dσ(ℏ)2S|M|2p
=3
dΩ8πmp(E+m−pcosθ)
21121
Thisistheexpressionofthescatteringcrosssection.□
9.4orkHomewourF
(1)(2)(1)†(2)
9.4.1.(Griiffth7.4)wShothatuuaregonalortho,inasensethatuu=0.ewise,Likwshothat
(3)(4)(1)(3)
uanduareorthogonal.Areuanduorthogonal?
Solution:
(1)(2)
Thebispinorsuanduare
10
(1)0(2)1
u=pu=p−ip
zxy
E+mE+m
p−ipp
xy−z
E+mE+m
CHAPTER9.TICLEARPPHYSICS213
(1)†(2)
kingChecforyorthogonalitwithuuewget
()0
(1)†(2)pp−ip1
uu=10zxyp−ip
E+mE+mxy
E+m
−pz
E+m
p(p−ip)p(p−ip)
=0+0+zxy−zxy
(E+m)2(E+m)2
=0
(1)†(2)(3)(4)
Sincetheproductuu=0theowtbispinorsareorthogonal.Similarlythebispinorsuandu
are
p+ipp
xyz
E+mE+m
pp+ip
−zxy
(3)E+m(4)E+m
u=0u=−1
10
(3)†(4)
kingChecforyorthogonalitwithuuewget
pz
()E+m
px+ipy
(3)†(4)p+ipp
uu=−xy−z01E+m
E+mE+m1
0
p(p+ip)p(p+ip)
=zxy−zxy+0+0
(E+m)2(E+m)2
=0
(3)†(4))
Sincetheproductuu=0theowtbispinorsareorthogonal.
(1)(3)
wNokingheccfortheyorthogonalitofuanduewget
px−ipy
()E+m
p−ip−pz
(1)†(3)pxy
uu=10zE+m
E+mE+m0
1
p−ipp−ip
=xy+0+0+xy
E+mE+m
2p
=x
E+m
(1)†(3))
Sincetheproductuu̸=0theowtbispinorsarenotorthogonal.□
9.4.2.(Griiffth7.17)
(a)Expressγµγνasalinearbinationcomof1,γ5,γµ,γµγ5andσµν.
Solution:
Theytitquanσµνisdeifneds
σµν=i(γµγν−γνγµ)(9.1)
2
Alsoewwknofromtheutationti-commanrelationofthegammamatricesybdeifnition
{γµ,γν}=2gµν
=⇒γµγν+γνγν=2gµν(9.2)
CHAPTER9.TICLEARPPHYSICS214
ddingA(9.1)and(9.2)ewget
2γµγν=2(gµν−iσµν)
γµγν=gµν−iσµν
HeregµνisthewskioMankmetricandiscompletelycomposedofbumersn1,−1and0.Sothisis
therequiredexpression.□
(b)Constructthematricesσ12,σandσ23andrelatethemtoΣ1,Σ2,andΣ3.
Solution:
Bydeifnition
σµν=i(γµγν−γνγµ)σ12=i(γ1γ2−γ2γ1)(9.3)
22
[γ1,γ2]=γ1γ2−γ2γ1
(0σ)(0σ)(0σ)(0σ)
=12−21
−σ0−σ0−σ0−σ0
1221
()()
−σσ0−σσ0
=12−21
0−σσ0−σσ
(12)(21)
[σ,σ]0−2iσ0
=21=3
0[σ,σ]0−2iσ
213
usTh
i[]()
σ0
σ12=γ1,γ2=3=Σ3
20σ
3
Similarly
σ13=i(γ1γ3−γ3γ1)
2
[γ1,γ3]=γ1γ3−γ3γ1
(0σ)(0σ)(0σ)(0σ)
=13−31
−σ0−σ0−σ0−σ0
1331
()()
−σσ0−σσ0
=13−31
0−σσ0−σσ
1331
()()
[σ,σ]02iσ0
=31=2
0[σ,σ]02iσ
312
usTh
i[](σ0)
σ13=γ1,γ3=−2=−Σ2
20σ
2
Similarly
σ23=i(γ2γ3−γ3γ2)
2
CHAPTER9.TICLEARPPHYSICS215
[γ2,γ3]=γ2γ3−γ3γ2
(0σ)(0σ)(0σ)(0σ)
=23−32
−σ0−σ0−σ0−σ0
2332
()()
−σσ0−σσ0
=23−32
0−σσ0−σσ
(23)(32)
[σ,σ]0−2iσ0
=32=2
0[σ,σ]0−2iσ
322
usTh
i[]()
σ0
σ23=γ2,γ3=1=Σ1
20σ
1
HeretheutationcommrelationfortheauliPmatrices[σ,σ]=ε2iσhasbeenused.Thisesgiv
ijijkk
ustherequiredrelationship.□
¯
9.4.3.(Griiffth11.4)AsitstandsDiracLagrangiantreatsψandψ.asymmetricallySomepeoplepreferto
dealwiththemonanequalfooting,usingthemodiifedLagrangian
iℏc[¯µ¯µ]2¯
L=ψγ(∂ψ)−(∂ψ)γψ−(mc)ψψ
2µµ
ApplytheEuler-LagrangeequationstothisL,andwshothatouygettheDiracequationsandits
t.adjoin
Solution:
TheEulerLagrangeequationisfortheLagrangianydensitL(∂ϕ,∂ϕ,...,ϕ,ϕ,...)is
µ1µ212
∂(∂L)=∂L
µ∂(∂ϕ)∂ϕ
µii
orFthismodiifedLagrangianewget
∂(∂L)=∂L
µ¯¯
∂(∂ψ)∂ψ
(µ)
∂iℏc[−γµψ]=iℏc[γµ∂ψ]−mc2ψ
µ22µ
iℏc[−γµ∂ψ]=iℏc[γµ∂ψ]−mc2ψ
2µ2µ
iℏ(γµ∂ψ)−mcψ=0(9.4)
µ
¯
Similarlyewgettheotheronewithψ
∂(∂L)=∂L
µ∂(∂ψ)∂ψ
(µ)
iℏc[¯µ]iℏc[¯µ]2¯
∂ψγ=−∂ψγ−mcψ
µ22µ
iℏc[¯µ]iℏc[¯µ]2¯
∂ψγ=−∂ψγ−mcψ
2µ2µ
(¯µ)¯
iℏ∂ψγ+mcψ=0(9.5)
µ
eWifndoutthat(9.4)and(9.5)aretheDiracequationsanditst.adjoinusThthisLagrangianalso
esgivthesameDiracequations.□
CHAPTER9.TICLEARPPHYSICS216
9.4.4.(Griiffth11.20)ConstructtheLagrangianforABC.theory
Solution:
SincetheABCmodelofparticlesareheacscalarparticlewithspin0,infreeform,heaccanbedescribed
withaKlein-GordanLagrangian.SoewcanobtainthetotalLagrangianwithfreeformpartofKlein-
Gordanandteractioninterm.ThefreefromLagrangianisforheacparticle,ifewassumethescalar
ifeldϕ,ϕandϕ,respelyectiv
ABC
1µ122
L=∂ϕ∂ϕ−mϕ
A2µAA2AA
L=1∂ϕ∂µϕ−1m2ϕ2
B2µBB2BB
1µ122
L=∂ϕ∂ϕ−mϕ
C2µCC2CC
Theteractionintermsasinthemodelhasthestrengthof−ig.Sotheteractionintermis
L=−igϕϕϕ
tinABC
SotheifnalLagrangianis
1µ1221µ122
L=∂ϕ∂ϕ−mϕ+∂ϕ∂ϕ−mϕ
2µAA2AA2µBB2BB
+1∂ϕ∂µϕ−1m2ϕ2−igϕϕϕ
2µCC2CCABC
ThisistherequiredLagrangiandensitufortheABCytomodel.□
Chapter10
ClassicalElectrodynamics
10.1orkHomewOne
10.1.1.kson(Jac1.1)UseGauss’stheoremtoevprothewing:follo
(a)yAnexcesshargecplacedonaconductormoustlietirelyenonitssurface.(Aconductorybdeif-
nitiontainsconhargesccapableofvingmofreelyundertheactionofappliedelectricifelds.)
Solution:
Letsassumethatthehargecliesinsidetheolumevoftheconductor.Makingagaussiansurface
thatlieswithinaolumevofconductorandenclosesthisassumedhargecouldwimplythereisifnite
lfuxthroughthissufraceandhenceelectricifeld.Butelectricifeldinsideaconductorisnotpos-
siblebecauseotherwisethehargescouldwevmoandewouldwnolongerevhastaticequilibrium.
usThybtradiction,contherecanbenohargecinsidetheolumevofconductor.□
(b)Aclosed,wholloconductorshieldsitsteriorinfromifeldsduetohargescoutside,butdoesnot
shielditsexteriorfromtheifeldsduetothcargesplacedinsideit.
Solution:
Letsconsiderowtcases,whenthereishargecinsidetheconductorandwhenthereishargecoutside
theconductor.Intheifrstcaseifewetakagaussiansurfacethatcompletelyenclosesthewhollo
conductor,ybgauss’swlaewgetifniteelectricifeldatyanarbitraryptoinoutsidethewhollo
conductor.usThtheconductordoesn’tshieldtheoutsidefromelectricifeld.
Inthesecondcase,whenhargecisoutside.Thelfuxthroughthegaussiansurfaceenclosingthe
conductoriszeroasthereisnohargecinside.Sincetheelectricifeldoutsideonlyinducesthe
hargeconthesurfaceoftheconductor.Therecan’tbeifeldinisdethewholloconductor.□
σ
(c)Theelectricifeldatthesurfaceofaconductorisnormaltothesurfaceandhasamagnitudeϵ
0
whereσisthehargecydensitperunitareaonthesurface.
Solution:
Letusassumeanarbitrarygaussiansurfaceparallelanderyvclosetothesurfaceofconductor.
Inhsucacasetheifeldateryevptoinonthesurfaceisequalandnormaltothetheplaneofthis
surface.Usinggaussianwlaforthis
∫E·dA=q
ϵ
A0
EA=σA
ϵ
0
=⇒E=σ
ϵ
0
Thisshousthattheelectricifeldnearthesurfaceofconductorisnormaltothesurfaceandhas
217
CHAPTER10.CLASSICALYNAMICSODELECTR218
magnitudeofσ/ϵ0.
□
10.1.2.kson(Jac1.3)UsingtheDiracdeltafunctionsintheappropriatecorrdinates,expressthewingfollo
hargecdistributionsasthree-dimesionalhargecdensitiesρ(x)
(a)Insphericalcoordinates,ahargecQuniformalydistributedervoasphericalshellofradiousR.
Solution:
SincethetotalhargecQisuniformlydistributedervothesurfaceofshell,andthetotalsurface
areaofshellis4πR2ewe,vhatotalsurfaceydensitengivyb
Q
4πR2
wNointhetireenspace,theonlyplacethissurfacehargeccanbefoundisatthesurfaceofsphere
ofradiousRusththetotalhargecydensitervoallspacebecomes
ρ(x)=ρ(r,θ,ϕ)=Qδ(r−R)
4πR2
ThisesgivthetotalhargecydensitervoallspaceifthehargecisinsphericalshellofradiusR.□
(b)Incylindricalcoordinates,ahargecλperunittghlenuniformlydistributedervoaculindrical
surfaceofradiousb.
Solution:
Letusconsideraarbitrarylengthofthecylindricalsurfacel,withradiusb.w,Nothetotalsurface
areaofthisarbitrarycylindricalsectionisV=2πbl.Thetotalsurfaceydensitofhargecisfor
somehargecQis
Q
2πbl
Theonlyplacethishargeccanbefoundinallofspaceisforlocationswherer=b(incylindrical
coordinate).usThthetotalhargecydensitervoallaspacebecomes
ρ(x)=ρ(r,ϕ,z)=Qδ(b−r)=λδ(b−r)
2πbl2πb
Thisesgivthetotalhargecydensitervoallspaceifthehargecisincylindricalsurfaceofradius
bofradiusbvidedprothelinearhargecydensitisλ.□
(c)Incylindricalcoordinates,ahargecQspreaduniformlyervoalfatcirculardiscofnegligible
knessthicandradiusR.
Solution:
ThetotalareaofthecirculardiscisπR2.Thesurfacehargecydensitofforthisdiskis
Q
πR2
.wNosincethediskisnegligibleknessthicthetotalonlyplacewherethishargecresidesintireen
spaceiswherez=0andr≤R.usThthetotalolumevhargecydensitervotireenspacevbecomes
ρ(x)=ρ(r,ϕ,z)=Qδ(z)H(r−R)
πR2
whereH(x)=1ifx<0,0otherwise.□
CHAPTER10.CLASSICALYNAMICSODELECTR219
(d)Thesameas(10.1.2c),butusingsphericalcoordinates.
Solution:
romF(10.1.2c),ewcanemakthehangecofcoordinateasz=rcos(θ).Substutinthatintaden
function,andusingthepropyertfotadenfunction
δ(αx)=1δ(x)
|α|
thetotalydensitbecomes
ρ(x)=ρ(r,ϕ,θ)=Qδ(rcosθ)H(r−R)=Q1δ(cosθ)H(r−R)
πR2πR2r
whereH(x)=1ifx<0,0otherwise.□
10.2orkHomewowT
10.2.1.kson(Jac1.6)Asimplecapacitorisadeviceformedybowtinsulatedconductorstadjacentoheac
other.Ifequalandoppositehargescareplacedontheconductors,therewillbeacertaindifferenceof
ptialoteneenbwetthem.Theratioofthemagnitudeofthehargecononeconductortothemagnitude
oftheptialotendifferenceiscalledthecapacitance.UsingGauss’w,lacalculatethecapacitanceof
(a)owtlarge,lfat,conductingsheetsofareaA,separatedybsmalldistanced
Solution:
MakingaarbitraryGaussiansurfaceofareaSnearthesurfaceandparalleltothesurfaceofthe
largesheetewifndthattheelectricifeldnearthesurfaceis
E=σ
2ϵ
0
Sinceeenbwettheplacesbothplatesevhathesameifeldtheyadduptowicetthealue.vSince
theifeldisuniformeenbwettheplates,theptialotendifferenceissimplytheproductoftheifeld
andtheseparationusthewget
V=Ed=2×σ×d=σd
2ϵϵ
00
Alsothetotalhargecinthetireensurfaceissimplytheproductthehargecydensitanditsarea
usthewget
qdqAϵ
V==⇒C==0
AϵVd
0
Thisesgivthecapacitanceoftheowtlargelfatplates.□
(b)owttricconcenconductingsphereswithradiia,b(b>a);
Solution:
orFowtconductingsphere,theelectricifeldduetooutersphereinthespaceeenbwetowtspheres
iszero.Theonlyifeldisduetothehargeconinnerconductor.ConstructingasphericalGaussian
surfaceenclosingtheinnersphereewifndthetotalifeldintheregioneenbwettheowtspheresis
Qˆ
E=4πϵr2r
0
wherea<r<bisthedistancefromthetercenofthespheres.wNotheptialotendifference
eenbwetthespheresistheorkwdoneonunithargecvingmofrominnerspheretotheoutsphere
usthewevha
b[]
V=∫Q2dr=Q1−1=Qb−a
4πϵ0r4πϵ0ab4πϵ0ab
a
CHAPTER10.CLASSICALYNAMICSODELECTR220
ThecapacitanceiswnosimplytheratioofQandVhwhicis
C=Q=4πϵab
V0b−a
Thisesgivthecapacitanceofsphericalcapacitor.□
(c)owttricconcenconductingcylindersoflengthL,largecomparedtotheirradiia,b(b>a).
Solution:
Similartopart(10.2.1b)ewgetnoifeldinsidetheinnercylinderandoutsidetheoutercylinder.
Inthespaceeenbwettheo,wtonlytheinnercylindertributescontotheelectricifeld.Again
withacylindricalGaussiansurfaceboundingtheinnercylinderewifndthattheifeldinthespace
eenbwetthoseis
IE·dA=q=⇒E2πrL=Q=⇒E=Q
ϵϵ2ϵπLr
000
wherea<r<bistheradialdistancefromthetercenofthecylinders.Theptialotendifference
wnoisagaintheorkwdoneonunithargechwhicis
b()
V=∫Qdr=Qlnb
2ϵ0πLr2πϵ0La
a
ThecapacitanceisybdeifnitionfoundybtheratioofQtoV;
Q2πϵ0L
C=V=ln(b)
a
Thisesgivthecapacitanceofcylindricalcapacitor.□
(d)Whatistheinnerdiameteroftheouterconductorinanairiflledcoaxialcablewhosetercen
−11−12
conductorisacylindricalwireofdiameter1andwhosecapacitanceis3×10F/m?3×10
F/m.
10.2.2.kson(Jac2.1)Aptoinhargecofqistbroughtoapositionadistancedyawafromaninifniteplane
conductorheldatzeroptialoten.Usingthemethodofimages,ifnd:
(a)thehargesurface-cydensitinducedontheplane,andplotit;
Solution:
Theimagehargecforaptoinhargecneartheinifniteconductorisbehindtheplaneataequal
distanceandthehargecisofequalmagnitudeandoppositesign.usThthetotalptialotendueto
theimagehargecandtheptoinhargec(inpolarcoordinatesystem)isUsingthecosinew,lathe
P
r2r
r1
−qθq
dd
CHAPTER10.CLASSICALYNAMICSODELECTR221
tdifferentitiesquanintheengivdiagramcanbewrittenas
222√2
r=r−2rdcosθ+d;=⇒r1=1−2drcosθ+d
1√
2222
r=r−2rdcos(π−θ)+d;=⇒r2=1+2drcosθ+d
2
TheptialotenatyangeneralptoinP(r,θ)isengivyb
ϕ(r,θ)=q[1−1]
4πϵ0r1r2
Sinceybgauss’swlanearaconductorthesurfacehargecydensitandthenormalcomptonenof
theifeldarerelatedybtheequation
E=σ=⇒σ=ϵE
ϵ0
0
ewcalculatethetgradienoftheptialotenandaluateevatitssurface.Thetgradienis
∂ϕˆ1∂ϕˆ
E=∂rr+r∂θθ
aluatingEvtheradialanduthalazimcomptonenandaluatingevatthesurfacehwhiccorresponds
toθ=π/2ewget,
Er=q[−−dcos(θ)−r3+dcos(θ)−r3]=0
4πϵ0222222
ddrθrd2cos()+
(+2cos()+)(−drθr)θ=π
[]2
E=q1−drsin(θ)−drsin(θ)=−2qd
θ4πϵr33223/2
02222224πϵ0(d+r)
(d+2drcos(θ)+r)(d−2drcos(θ)+r)θ=π
2
usThthetotalhargecydensitisengivyb
σ=ϵE=−qd
0223/2
2π(d+r)
Thisesgivtherequiredsurfacehargec.ydensit□
(b)theforceeenbwetthehargecanditsimage
Solution:
Sincetheimagehargecandtheptoinhargecareequalandoppositeandmagnitudeandareatotal
distance2dapartewgettheforceybcolumns’wlaas
q2q2
F=−2=−2
4πϵ0(2d)16πϵ0d
Thisistherequiredforce.□
σ2
(c)thetotalforceactingontheplaneybtegratingin2ϵervothewholeplane;
0
Solution:
Letsassumeasmallcircularareatelemenatadistancerfromthetercenofthecirclethenthe
areatelemenisda=2πrdrusththetotalareategralinervothewholeareais
∞∞
F=∫σ22πrdr=qd2∫2r23dr=qd2·14=q22
2ϵ4πϵ(r+d)4πϵ4d16πϵd
0000
r=00
Thisesgivthesameforceasinthepreviouspart.□
CHAPTER10.CLASSICALYNAMICSODELECTR222
(d)theorkwnecessarytoevremothehargecqfromitspositiontoy;inifnit
Solution:
Withtheimagehargecatdfromthesurfaceewevhatoevmothehargecfromdeabvothesurface
to,yinifnitthetotalorkwdoneisengivyb
∞[]
W=∫Fdz=∫−q22dr=−q2−1∞=−q2
4πϵ0(d+z)4πϵ0d+zd8πϵ0d
z=d
Thisistherequiredorkwforthealvremoofhargecto.yinifnit□
(e)theptialotenenergyeenbwetthehargecqanditsimage.
Solution:
Thetotalptialoteneenbwetthehargecandimageissimplytheelectricptialotenofowtequaland
oppositeptoinhargecqatadistance2dusthewget
q2−q2
V=−4πϵ0(2d)=8πϵ0d
Thisistheptialoteneenbwettheharge.cAsrequired,thisisexactlythesameasewgotin
(10.2.2d).□
(f)Findtheeranswtopart(10.2.2d)inelectronoltsvforanelectronoriginallyoneangstromfrom
thesurface.
Solution:
−10−12
orFd=1×10andq=1e−19andϵ=8.85×10ewget
0
q2
−18
V=−8πϵd=1.15×10J=7.19eV
usThtheptialotenenergyeenbwetthehargescis7.19eV.□
10.2.3.kson(Jac2.7)Consideraptialotenprobleminthehalf-spacedeifnedybz≥0,withhletDiricboundary
conditionsontheplanez=0(andaty),inifnit
(a)riteWwndotheappropriateGreenfunctionG(x,x′),
Solution:
′′′′
Lettherebeaptoinhargecqx=(ρ,ϕ,z).orFtheptialotentobezeroatplanez=0ew
′′′
assumeaimagehargec−qat(ρ,ϕ,−z).Thegreen’sfunctionissimplytheptialotendueto
theseptoinhargecatagenerallocationx=(ρ,ϕ,z)
G(x,x′)=1−1
r1r2
whereristhedistanceofgeneralptointotheptoinhargecandristhedistancefromimage
12
hargectothegeneralpt.oineWcancalculatethedistancesas
√′2′2′2√2′2′′′2
r=(ρcosϕ−ρcosϕ)+(z−z)+(ρsinϕ−ρsinϕ)=ρ+ρ−2ρρcos(ϕ−ϕ)+(z−z)
1
Similarly
√2′2′′′2
r=ρ+ρ−2ρρcos(ϕ−ϕ)+(z+z)
2
Sincethehoicecofcoordinatesystemisarbitraryduetouthalazim,symmetryewcanhocoseϕ′=0
withoutlossof.ygeneralit
′11[√1√1]
G(x,x)=r−r=2′2′′2−2′2′′2
12ρ+ρ−2ρρcosϕ+(z−z)ρ+ρ−2ρρcosϕ+(z+z)
Thisisthegreens’function.□
CHAPTER10.CLASSICALYNAMICSODELECTR223
(b)iftheptialotenontheplacez=0isspeciifedtobeΦ=Vinsideacircleofradio’sateredcenat
theorigin,andΦ=0outsidethatcircle,Findtegralinexpressionfortheptialotentanheptoin
Pspeciifedintermsofcylindricalcoordinates(ρ,ϕ,z).
Solution:
Thetegralinequationtoesolvfortheptialotenis
Φ(x)=1∫G(x,x′)ρ(x′)dV−1IΦ(x′)∂G(x,x′)da′
4πϵ0V4πs∂n
Sinceewdon’tevhahargecydensitinsidetheolumevboundedybthecylinderρ(x′)=0usththe
onlyremainingtermisthesecondterm.Theardwoutnormalonthesurfaceofthecylindercanbe
calculated.Butsincetheptialotenatupperinifniteplaneiszeroithasnotribution.conSimilarly
theallsidewofthecylinderdonottributecontothetegralinbecausethecylindricalallwevhaa
surfaceareayinifnitandthetegralingoestozero.usThtheonlytributionconcomesfrombaseof
cylinderwithradiusa.Onthisfacez′=0soewget
[′′]
∂G12(z−z)12(z+z)
=+
∂n22′2′′23/222′2′′23/2
z′=0(ρ+ρ−2ρρcosϕ+(z−z))(ρ+ρ−2ρρcosϕ+(z+z))z=0
=2z
2′2′23/2
(ρ+ρ−2ρρcosϕ+z)
SincetheptialotenatthatsurfaceisΦ(x′)=Vewget
a2π
V∫∫2z′′′
Φ=4π2′2′23/2ρdϕdρ
ρ′=0′(ρ+ρ−2ρρcosϕ+z)
ϕ=0
Thisistherequiredtegralinexpression.□
(c)wShothat,alongtheaxisofthecircle(ρ=0),theptialotenisengivyb
Φ=V(1−√z)
a2+z2
Solution:
Solvingatρ=0ewget
a[]a[][]
∫′
2Vzρ′√11√1√z
Φ(x)=4π·2π′223/2dρ=Vz′22=Vzz−22=V1−22
(ρ+z)ρ+zz+az+a
00
hWhicistherequiredexpression.□
10.3orkHomewThree
10.3.1.kson(Jac2.13)
(a)owTeshalvofalongwholloconductingcylinderofinnerradiusbandseparatedybasmalllength-
wisegapsonheacside,andareeptkattdifferenptialsotenV1andV2.wShothattheptialoten
insideisengivyb
V+VV−V(2bρ)
1212−1
Φ(ρ,ϕ)=+tan22cosϕ
2πb−ρ
whereϕismeasuredfromaplaneperpendiculartotheplanethroughthegap.
Solution:
CHAPTER10.CLASSICALYNAMICSODELECTR224
2ρ
Theelectricptialotenwsfollooisson’sPequation∆ϕ=ϵsince,inthisparticularproblemthere
0
2
isnohargecinthespace,itreducestoLaplace’sequation∆ϕ=0.Sincetheproblemtailsen
cylindricalboundaryconditionsewlookforsolutionofLaplace’sequationincylindricalcoordinate
system.Also,sincethecylinderislong,theptialotenhasnozdependence,ewcantiallyessen
esolvtheptialotenatthebottomplaneofthecylinderz=0andthissolutionorkswforeryevz.
SothegeneralsolutionofLaplace’sequationinpolarcoordinatesystemis
∑n−n
u(ρ,ϕ)=(Clnρ+D)+(Acosnϕ+Bsinnϕ)(Cρ+Dρ)
00nnnn
n
−n
Sinceewexpectifnitesolutionatρ=0,D=0otherwiseitρ=∞hwhicon’twsatisfy
n
boundarycondition.BysimilartsargumenC0=0asthesolutionhastobeifniteatρ=0but
lnρergesdivatρ=0.Sothesolutionreducesto,(absorbingCtoinAandB)
nnn
u(ρ,ϕ)=D+∑[Acosnϕ+Bsinnϕ]ρn
0nn
n
Theboundaryconditionare,attheedgeofthecylinderρ=b,letshocoseourcoordinatesystem
hsucthatthetrighhalfofthecylinderfromϕ=−πtoϕ=πisatptialotenVandthelefthalf
221
ϕ=πtoϕ=3πisatptialotenV
222
{V1if−π<ϕ<π
u(b,ϕ)=22
Vifπ<ϕ<3π
222
wNoattheedgeofthe
∞
∑n
u(b,ϕ)=D+b[Acosnϕ+Bsinnϕ]
0nn
n=1
wNothetconstancoteiffcienDcanbeeasilycalculatedybtegratinginbothsidesas
0
3π/23π/23π/2
∫u(b,ϕ)dϕ=∫Ddϕ+∑bn∫[Acosnϕ+Bsinnϕ]dϕ
0nn
−π/2−π/2n−π/2
0
0

✚❃
✚❃
✚
π/23π/23π/23π/2✚3π/2
∫∫∫∞∫∫
∑n✚✚
✚✚
u(b,ϕ)dϕ+u(b,ϕ)dϕ=Ddϕ+bAcosnϕdϕ+Bsinnϕdϕ
0n✚n✚
n=1✚
✚
−π/2π/2−π/2−π/2−π/2
V1π+V2π=D02π+0✚✚
D=V1+V2
02
AgainthecotseiffcienBandAcanbecalculatedybusingthefactthat{sinϕ}and{cosϕ}
nnnn
formanorthogonalsetoffunctionfortegerinsetofn.tegratingIntheeabvoexpressionyb
ultiplyingmybsinmϕonbothsidesesgiv
3π/23π/203π/2
✘✿
∫∫✘∫
∑✘✘
n✘✘
✘
u(b,ϕ)sinmϕdϕ=bAcosnϕsinmϕdϕ+Bsinnϕsinmϕdϕ
n✘n
✘✘
✘
n✘
−π/2∑−π/2−π/2
n2πm
=Bbδ=Bbπ
n2mnm
n
3π/2
⇒B=1∫u(b,ϕ)sinmϕdϕ
mπbm
−π/2
CHAPTER10.CLASSICALYNAMICSODELECTR225
SimilarlythecotseiffcienAmcanbecalculatedas
3π/2
Am=1∫u(b,ϕ)cosmϕdϕ
πbm
−π/2
Sinceintheengivproblemu(a,ϕ)hastdifferenaluesvfortdifferenϕewget
π/23π/2
A=1∫u(b,ϕ)cosmϕdϕ+∫u(b,ϕ)cosmϕdϕ
mπbm
−π/2π/2
π/23π/2
=1V∫cosmϕdϕ+V∫cosmϕdϕ
πbm12
−π/2π/2
[(m)(m)]
=1V1−(−1)+V(−1)−1
πbm1m2m
[(m)]
=1(V1−V2)1−(−1)
πbmm
WirkingoutthetegralinforBleadstoB=0forallm.Sotheifnalsolutionbecomes
mm
∞n
V1+V2∑n11−(−1)
u(ρ,ϕ)=2+ρπbn(V2−V2)ncosnϕ
n=1
∞()n
=V1+V2+V1−V2∑ρn1−(−1)cosnϕ
2πn=1bn
clearlythesumtermiszeroforenevn,foroddntheexpressionisjust2/n.Theclosedformof
thesumesgivtherequiredexpression
u(ρ,ϕ)=V1+V2+V1−V2arctan(2bρcosϕ)
2πb2−ρ2
Thisesgivtheptialotenerywhereevinsidethecylinder.□
(b)Calculatethehargesurface-cydensitofheachalfofthecylinder.
Solution:
Thehargecydensitcanbesimplyfoundybifndingthenormalcomptonenofelectricifeldatthe
surface.
()
∂u(ρ,ϕ)V1−V2∂2ρcosϕ
σ(ϕ)=ϵ=ϵarctan
0022
∂ρρ=bπ∂ρb−ρ
Thiseativderivaswaluatedevybusingysymptoobtain
V1−V24b2·2b·cosϕV1−V22cosϕV1−V2
σ(ϕ)=ϵ=ϵ=ϵ
04400
π2b+2bcos2ϕπb(1+cos2ϕ)πbcosϕ
orFheaceshalvewevhatheconditionforϕ.Subsistingthealuevofϕforheaceshalvesgivthe
hargecydensitofheachalf.□
10.3.2.kson(Jac2.15)
CHAPTER10.CLASSICALYNAMICSODELECTR226
(a)wShothatthegreenfunctionG(x,y,x′,y′)appropriateforhletDiricboundaryconditionsfora
squareo-dimensionalwtregion,0≤x≤1,0≤y≤1,hasanexpansion
∞
G(x,y,x′,y′)=2∑g(y,y′)sin(nπx)sin(nπx′)
n
n=1
whereg(y,y′)satisifes
n
(∂2−n2π2)g(y,y′)=−4πδ(y′−y)andg(y,0)=g(y,1)=0
2nnn
∂y′
Solution:
Thegreen’sfunctionsolutiontononhomogeneoustialdifferenequationLh(x)=f(t)isasolution
tohomogeneouspartofthetialdifferenequationwiththesourcepartreplacedasdeltafunction
Lh(x)=δ(t−ξ).TheobtainedsolutionisG(t,ξ),i.e.,LG(t,ξ)=δ(t−ξ).Thissolution
correspondstothehomogeneouspartonlyasitisindeptendenofyansourcetermf(t).Let
G(t,ξ)bethesolutiontothetialdifferenequitationwiththeinhomogeneouspartreplacedyb
deltafunctionδ(t−ξ).Thegreen’sfunctionsolutiontoLaplace’sequationisthen:
(∂2+∂y)G(x,y;x′,y′)=−4πδ(x′−x)δ(y′−y)
∂x2∂yy
SinceewevhaboundaryconditionthatG(x′=0)=0andG(x′=1)=0ewetakoddfunction
fourierexpansionoftheGreen’sfunction
∞
G(x,y;x′,y′)=∑f(x,y;y′)sin(nπx′)(10.1)
n
n=1
UsingthisexpressionintheLaplace’sequationewobtain
∞(2)
∑∂2−n2π2fn(x,y,y′)sin(nπx′)=−4πδ(x′−x)δ(y′−y)(10.2)
n=1∂y
Completenessoftheorthogonalfunctionssin(nπx)wsalloustowritethedeltafunctionas
∞
δ(x−x′)=∑sin(nπx)sin(nπx′)
n=1
Replacingthisexpressionin(10.2)ewobtain
∞(2)∞
∑∂2−n2π2fn(x,y;y′)sin(nπx′)=−4πδ(y′−y)∑sin(nπx)sin(nπx′)(10.3)
n=1∂y′n=1
ComparingthefunctionbviorehaofparameterxonLHSandRHSof(10.3)ewobtainthatthe
functionfisusoidal.sinSeparatingouttheypartoftheexpressiontoinotherfunctiongewget
nn
fn(x,y;y′)=gn(y,y′)sin(4πx)
wNoewcansubstitutethiskbactoinourgreen’sfunctionGin(10.1)ewget
∞
G(x,y;x′,y′)=∑gn(y,y′)sin(nπx)sin(nπx′)
n=1
Substitutingthisexpressionkbacto(10.2)ewobtain
(∂2−n2π2)g(y,y′)=−4πδ(y′−y)(10.4)
2n
∂y′
ThisexpressiongalsohastosatisfytheboundaryconditionsasthecompletegreensfunctionG
n
soewevhag(y,0)=0andg(y,1)=1asrequired.□
nn
CHAPTER10.CLASSICALYNAMICSODELECTR227
(b)akingTforg(y,y′)appropriatelinearbinationscomofsinh(nπy′)andcosh(nπy′)intheowtregions
n
y′<yandy′>y,inaccordwiththeboundaryconditionsandtheyuittindisconinsloperequired
ybthesourcedeltafunction,wshothattheexplicitformofGis
∞
G(x,y;x′,y)=0∑1sin(nπx)sinh(nπx′)sinh(nπy)sinh(nπ(1−y))
nsinh(nπ)<>
n=1
wherey(y)isthesmaller(larger)ofyandy′.
<>
Solution:
wNothatewevhathegeneralexpressionforthegreen’sfunction(10.4)ewcandividetheregion
toinpartswithx′>xandx′<x,sinceinheacofthesecases,thesourceterminthetialdifferen′′
equationiszeroasthedeltafunctioniszerothere{δ(x−x)=0ifx̸=xsoewget
g≡asinh(nπy′)+bcosh(nπy′)ify′<y
g(y,y′)=<<<
ng≡asinh(nπy′)+bcosh(nπy′)ify′>y
>>>
Findingthisfunctionis,wndotoifndingthewnunknocotseiffciena<,a>,b<,b>.Applyingthe
boundaryconditiong(y,0)=0=g(y,1)ewget
nn
g=g∂g=∂g+4πify′>y
><∂y′<∂y′>
wNotheboundaryconditionhsucthatg>(y′=1)=0andg<(y′=0)=0suggestssinhfunctions
suittheboundaryconditionthansin.usThewget
{
gn(y,y′)=a<sinh(nπy′)ify′<y
a>[sinh(nπy′)−tanh(nπ)cosh(nπy′)]ify′>y
yuittinConrequiresthatthefunctionhmatcaty=y′soewevha
a<sinh(nπy′)=a>[sinh(nπy′)−tanh(nπy′)](10.5)
andthejumpyuittindisconofgreensfunctionrequire
∂g(y)−∂g(y)=1(10.6)
∂y′n<∂y′n>
Theequations(10.5)and(10.6)egivsystemofequationhwhiccanbeedsolvas
(sinh(nπy)−sinh(nπy)+tanh(nπ)cosh(nπy))(a)(0)
<==
cosh(nπy)−cosh(nπy)+tanh(nπ)cosh(nπy)a4
>n
Againsolvingthiswithsimplyesgiv
()4()
acosh(nπ)sinh(nπy)−sinh(nπ)cosh(nπy)
<=−
ansinh(nπ)cosh(nπ)sinh(nπy)
>
Substitutingthecotseiffcienewget
{
gn(y,y′)=4×sinh(nπy′)[sinh(nπ)cosh(nπy)−cosh(nπ)sinh(nπy)]ify′<y
nsinh(nπ)sinh(nπy)[sinh(nπ)cosh(nπy′)−cosh(nπ)sinh(nπy′)]ify′>y
Asrequiredthegreensfunctionissymmetricinitsparameters.Thesymmetryishsucthatthe
expressionlooksexactlysameiftheparameterarehanged.excIfewdenoteytobetheumminim
<
ofyandy′andsimilarlyforyewcanwritetheeabvoexpressioninacompactyawas
>
g(y,y′)=4sinh(nπy)sinh(nπ(1−y))
nnsinh(nπ)<>
CHAPTER10.CLASSICALYNAMICSODELECTR228
Subsistingthistothegreensfunctionsolutionewget
∞
G(x,y,x′,y′)=∑8sin(nπx)sin(nπx′)sinh(nπy)sinh(nπ(1−y))
nsinh(nπ)<>
n=1
Thisistherequiredexpressionforthegreen’sfunction.□
10.4orkHomewourF
10.4.1.kson(Jac3.1)owTtricconcenspheresevharadiia,b(b>a)andheacisdividedtoinowthemispheres
ybthesametalhorizonplane.Theupperhemisphereoftheinnersphereandteherwlohemisphereof
theoutersphereaaretainedmainatptialotenV.Theotherhemispheresareatzeroptial.oten
teDetermintheptialotentinehregiona≤r≤bastheseriesinLegendrepolynomials.Includeterms
atleastupttol=4.kChecourysolutionagainstwnknoresultsinthelimitinghasescb→∞,and
a→0.
Solution:
ThegeneralsolutiontoLaplace’sequationinsphericalcoordinatesystemis
u(r,θ,ϕ)=[Arl+Br−(l+1)][Ccosmϕ+Dsinmϕ][EPm(cosθ)+FQm(cosθ)]
ll
Sincethereisuthalazimsymmetrythealuevofm=0.Theptialotenisifniteatboththepoles,butthe
m
associatedLegendrefunctionofsecondkindQ(x)ergesdivatx=±1,hwhiccorrespondstopoles,so
l
ewrequireF=0.AbsorbingtconstanCandFtoinAkandBk,thegeneralsolutionreducesto
∞[]
u(r,θ,ϕ)=∑Arl+Br−(l+1)P(cosθ)(10.7)
lll
l=0
HerethefunctionP0(x)=P(x)istheLegendrepolynomial.
ll
MultiplyingbothsidesybPk(cosθ)andtegratinginwithrespecttodcosθfrom−1to1ewget
11
∫∫∞
u(r,θ,ϕ)P(cosθ)dcosθ=∑[Arl+Br−(l+1)]P(cosθ)P(cosθ)dcosθ
kkklk
−1−1l=0
∞[]
=∑Arl+Br−(l+1)2δ
kk2l+1lk
l=0
=[Akrk+Bkr−(k+1)]2
2k+1
wNoaluatingevthetegralinforr=aandr=b.respelyectiv
CHAPTER10.CLASSICALYNAMICSODELECTR229
orFr=a
1
2k−(k+1)∫
[Aa+Ba]=u(a,θ,ϕ)P(cosθ)dcosθ
2k+1kkk
−1
01
=∫0·Pk(cosθ)dcosθ+∫VPk(cosθ)dcosθ
−10
1
=∫VPk(x)dx
0[()()]
VΓ1Γ1
22
=2k+1Γ(−k+1)Γ(k+1)−Γ(−k)Γ(k+3)
222222
hWhicimplies
k−(k+1)V[Γ(1)Γ(1)]
22
Aa+Ba=()()−()()=βysa(10.8)
kk2Γ−k+1Γk+1Γ−kΓk+3
222222
Againdoingthisforr=bewget
1
2k−(k+1)∫
[Ab+Bb]=u(b,θ,ϕ)P(cosθ)dcosθ
2k+1kkk
−1
01
=∫V·P(cosθ)dcosθ+∫0·P(cosθ)dcosθ
kk
−10
1
=∫VPk(−x)dx
0
1
∫k
=V(−1)P(x)dx
k
0[()()]
kΓ1Γ1
V(−1)22
=2k+1Γ(−k+1)Γ(k+1)−Γ(−k)Γ(k+3)
222222
hWhicimplies
k[Γ(1)Γ(1)]
k−(k+1)V(−1)22k
()()()()−β
Ab+Bb=−=(1)
kk2Γ−k+1Γk+1Γ−kΓk+3
222222
Sotheowtlinearequationsare
k−(k+1)
Aa+Ba=β
kk
k−(k+1)k
Ab+Bb=(−1)β
kk
eWcancasttheseowtequationofwnsunknoAandBtoinmatrixequationas
kk
[k−(k+1)][][β]
aaA
k=(10.9)
k−(k+1)Bk
bbk(−1)β
CHAPTER10.CLASSICALYNAMICSODELECTR230
(a)Eletricifeldlineswithb/a=(b)Eletricifeldlineswithb/a=2.5(c)Eletricifeldlineswithb/a=10
1.25
Solvingthematrixequationewgetthematrix
[A]β(ak+1−(−1)kbk+1)
k2k+12k+1
=a−b
k+1kkk
Bkβ(ab)(a−(−1)b)
2k+12k+1
a−b
Substutingthealuevofβfrom(10.8)ewget
1k+1k+1kk3kk1
Γa+(−b)Γ−Γ+−Γ−+1Γ+
[](2)()((2)(22)(2)(22))
AV2k+12k+1kkk1k3
(a−b)Γ−Γ−+1Γ+Γ+
k=(2)(2)(22)(22)
Γ1(ab)k+1bk−(−a)kΓ−kΓk+3−Γ−k+1Γk+1
B2(2)()((2)(22)(2)(22))
k−
(a2k+1−b2k+1)Γ−kΓ−k+1Γk+1Γk+3
(2)(2)(22)(22)
Thecotseiffcienareallzerosforallenevk≥2.
[][]
A2m=0∀m∈Z;
B0+
2m
Theifrstfewoddofthiscoteiffcienare
[][][][22][][44][][66]
13(a+b)−7(a+b)11(a+b)
AA33A77A1111
0=V2;1=V4(a−b);3=V16(a−b);5=V32(a−b)
2244336655
B0B−3ab(a+b)B7ab(a+b)B−11ab(a+b)
013337751111
4(a−b)16(a−b)32(a−b)
Substutingthesecotseiffcienin(10.7)ewobtaintheptialoteninwiththisboundarycondition.
[(2222)(3444433)
u(r,θ,ϕ)=V1+3r(a+b)−ab(a+b)P(cosθ)+7−r(a+b)+ab(a+b)P(cosθ)
332331774773
24a−br(a−b)16a−br(a−b)
(5666655)(7888877)]
+11r(a+b)−ab(a+b)P(cosθ)+75−r(a+b)+ab(a+b)P(cosθ)+···
11116111151515815157
32a−br(a−b)256a−br(a−b)
Thisistherequiredptialotenintheregion[a≤r≤b.Inthelimitb→∞ewevha]
()()()()
13a27a411a675a8
u(r,θ,ϕ)=V+P(cosθ)−P(cosθ)+P(cosθ)−P(cosθ)+...
24r116r332r5256r7
Inthelimitb→∞theproblemistheyptialotenofsplattedsphereerywhereevoutsidethesphere.
Andtheeabvoexpressionhesmatctheexpectedresult.Inthelimita=0,hwhiccorrespondstothe
ptialoteninsidethesphereinsidethesplittedptial.oten
[()()()]
13r7r311r575r7
u(r,θ,ϕ)=V−P(cosθ)+P(cosθ)−P(cosθ)+P(cosθ)+...
24b116b332b5256b7
hWhicalsohesmatcourexpectation.□
CHAPTER10.CLASSICALYNAMICSODELECTR231
10.5orkHomeweFiv
10.5.1.kson(Jac3.6)owTptoinhargescqand−qarelocatedonthezaxisatz=aandz=−arespelyectiv
(a)Findtheelectrostaticptialotenasanexpansioninsphericalharmonicsanderspwoofrforboth
r>aandr<a.
Solution:
Lethepositionectorvofptoinhargesc+qand−qber1(a,0,ϕ)andr2(−a,π,ϕ).respelyectivyAn
ptoinwithpositionectorvrwillevhaptialotenengivyb
Φ=q[1−1]
4πϵ0|r−r1||r−r2|
Iftheangleeenbwetowtpositionectorsvrandr′isγ,afunctionofthisform,withthehelpof
cosinew,lacanbewrittenas
√1′
ifr≥r∞()
11r′1+(r)2−2(r)cosγ∑rn
=√=r′r′=<Pn(cosγ)
|r−r′|r2+r′2−2rr′cosγ√1ifr′<rrn+1
r1+(r′)2−2(r′)cosγn=0>
rr
Here,r=max(r,r′)andr=min(r,r′).Alsothegeneratingfunctionexpansionoflegendre
><
polynomialshasbeenused
∞
∀t<1:√1=∑tnP(x)
1+t2−2txn
n=0
Byusingtheadditiontheoremforthelegendrepolynomialsewcanwrite
l
P(cosγ)=4π∑Ym(θ,ϕ)∗Ym(θ,ϕ)
l2l+1l11l
m=−l
Soewcanwritetheexpression
∞nl
1=∑r4π∑Ym(θ1,ϕ1)∗Ym(θ,ϕ)
|r−r|rn+12l+1ll
1l=01m=−l
Sinceewevha|r|=|r|(=a),ewcangeneralizer=max(r,a)andr=min(r,a).Sothe
12><
potnetialexpressionbecomes
∞ll
Φ=q∑∑4πr<{Ym(θ,ϕ)∗Ym(θ,ϕ)−Ym(θ,ϕ)∗Ym(θ,ϕ)}
4πϵ2l+1rl+1l11ll22l
0l=om=−l>
∞ll[]
=q∑∑4πr<Ym(0,ϕ)∗−Ym(π,ϕ)∗Ym(θ,ϕ)
4πϵ2l+1rl+1lll
0l=om=−l>
orFtheengivproblemθ=0,θ=π.But
12
∀m̸=0:Ym(0,ϕ)=0∧Y0(0,ϕ)=√2l+1P(1)⇒Ym(0,ϕ)=√2l+1δ
ll4πll4πm,0
m0√2l+1m√2l+1l
∀m̸=0:Y(π,ϕ)=0∧Y(π,ϕ)=P(−1)⇒Y(π,ϕ)=(−1)δ
ll4πll4πm,0
CHAPTER10.CLASSICALYNAMICSODELECTR232
Substutingtheseewget
∞l√l[]
q∑∑4πr<lm
Φ=(1−(−1))δY(θ,ϕ)
4πϵ2l+1rl+1m,0l
0l=0m=−l>
∞√l
q∑4πr<l0
=4πϵ02l+1rl+1(1−(−1))Yl(θ,ϕ)
l=0>
0√2l+12k2k+1
SinceY(θ,ϕ)=P(cosθ)and∀k∈N:(1−(−1)=0)∧(1−(−1)=2),ewget
l4πl
∞()
()q∑r2k+1
∞P(cosθ)ifr≤a
2q∑r2k+12πϵ0a2k+22k+1
∀k∈N:Φ=<P(cosθ)=k=0()
2k+1∞
4πϵ0r2k+2q∑a2k+1
k=0>P(cosθ)ifr>a
Thisistherequiredexpressionfortehptialotenduetothisdipole.2πϵ0k=0r2k+22k+1□
(b)Keepingtheproductqa=p/2t,constanetakthelimitofa→0andifndtheptialotenforr̸=0.
Thisisybdeifnitionadipolealongthezaxisanditsptial.oten
Solution:
Inthelimita→0ewevhar>asoewget
∞(2k+1)
Φ=limq∑aP(cosθ)
a→02πϵr2k+22k+1
0k=0
(2)
=limqa1P(cosθ)+aP(cosθ)+...
a→02πϵr21r33
0
=pcosθ
4πϵ0r2
Thisistherequiredexpressionforptialotenduetoadipole.□
(c)supposewnothatthedipolein(10.5.1b)issurroundedybagroundedsphericalshellofradiusb
tricconcenwiththeorigin.Bylinearsuperpositionifndtheptialotenerywhereevinsidetheshell.
Solution:
Sincethegroundedsphereattainshargecduetoinductionofthedipoleinsideit.Itcreatesits
wnoelectricptialoteninsidethespherehwhicwsfolloLaplace’sequation.Thegeneralsolutionto
Laplace’sequationinsphericalcoordinatesystemis
u(r,θ,ϕ)=[Arl+Br−(l+1)][Ccosmϕ+Dsinmϕ][EPm(cosθ)+FQm(cosθ)]
ll
Sincethereisuthalazimsymmetrythealuevofm=0.Theptialotenisifniteatboththepoles,but
m
theassociatedLegendrefunctionofsecondkindQ(x)ergesdivatx=±1,hwhiccorrespondsto
l
poles,soewrequireF=0.Alsosincetheptialotenisifniteatr=0ewrequireB=0Absorbing
tconstanEtoinAk,thegeneralsolutionreducesto
∞
u(r,θ,ϕ)=∑ArlP(cosθ)(10.10)
ll
l=0
HerethefunctionP0(x)=Pl(x)istheLegendrepolynomial.Bysuperpositionprinciplethe
l
tialtoteninsidethesphereofradiusbustmbeptialotenduetotheinducedhargecinsphereand
theptialotenybdipole.Soptialotenerywhereevinsidethesphereis
∞∞
∑pP(cosθ)∑
Φ′=Φ+ArlP(cosθ)=1+ArlP(cosθ)
ll4πϵr2l
l=00l=0
CHAPTER10.CLASSICALYNAMICSODELECTR233
ButewrequireΦ′=0atr=b.
∞
∑qP(cosθ)
l1
AbP(cosθ)=−
ll4πϵb2
l=00
Since{P(x);l∈N}formasetoforthogonalfunctionsthecoteiffcienofP(x)oneithersideof
ll
equationustmbeequalforthisequationtobe,ytitidenusthewget
q1q1l
Ab=−=⇒A=−;Ab=0,=⇒A=0;∀l̸=1
14πϵb214πϵb3ll
00
UsingthealuevofAin(10.10)ewget
l
Φ′=pcosθ−qrcosθ=1[p−r]cosθ
2323
4πϵ0r4πϵ0b4πϵ0rb
Thisistherequiredptialotenerywhereevinsidethesphere□
10.5.2.kson(Jac4.1)ryTtoobtainresultsforthenonanishingvtsmomenalidvoralll,butinheaccaseifnd
theifrstowtsetsofnonanishingvtsmomenattheeryvleast.Calculatetheultipmoletsmomenqlmof
thehargecdistributionswnsho
(a)
Solution:
Thehargecydensitcanbewrittenas
q[(π)(3π)]
ρ(x)=r2δ(r−a)δ(cosθ)δ(ϕ)+δϕ+2−δ(ϕ−π)δϕ+2
Sinceallthehargescareinplaneθ=πsocosθ=0.Theultipmoletsmomenareengivyb
2
∫lm3
qlm=rYl(θ,ϕ)ρ(x)dx
√2l+1(l−m)!lm[−imπ/2−imπ−im3π/2]
=qaP(0)1+e−e−e
4π(l+m)!l
SicnePm(0)=0forallenevmewcanwritem=2k+1;k∈N
l
2k+1l[k]√2l+1(l−(2k+1))!2k+1
ql=2qa1−i(−1)4π(l+(2k+1))!Pl(0)
l[k]2k+1(π)
=2qa1−i(−1)Yl2,0
Thisanishesvforallenevlusththealuesvforoddlandmare
∗√3
q=−q=−2qa(1−i)
1,11,−18π
q=−q∗=2qa3(1+i)√35
3,33,−34π
q=−q∗=2qa3(1−i)1√21
3,13,−144π
Thesearetheifrstfewnonanishingvts.momen□
CHAPTER10.CLASSICALYNAMICSODELECTR234
(b)
Solution:
Thehargecyensitis
ρ(x)=q[δ(r−a)δ(1−cosθ)+δ(r−a)δ(1+cosθ)−δ(r)]
2πr2
Theultipmoletsmomenareengivyb
q=∫rlYm(θ,ϕ)ρ(x)d3x
lml
lmm∗m∗
=qaP(0)[Y(0,0)+Y(π,0)]
lll
forl>0andq00=0.Byuthalazim,symmetryonlythem=0stmomenarenonanishing.vusTh
ewget
l√2l+1[]
ql=qaPl(1)+P−1)
04π(
l[l]√2l+1
=qa1+(−1)4πl>0
So,thisleadsto
q=√5qa2;q=0
2,0√π2,m̸=0
q=9qa4;q=0
4,0π4,m̸=0
Thesearethets.momen□
(c)orFthehargecdistributionofthesecondsetbwritewndotheultipmoleexpansionfortheptial.oten
Keepingonlytheest-orderwlotermintheexpansion,plottheptialoteninthex0yplaneasa
functionofdistancefromtheoriginforthedistancesgraterthana.
Solution:
Theexpansionoftheptialotenintermsofultipmolecotseiffcienis
∞l
1∑∑4πY(θ,ϕ)
Φ=qlm
4πϵ2l+1lmrl+1
0l=0m=−l
Sinceewonlyevhanon-zerocotseiffcienform=0andlenevewevha
1∑4πY0(θ,ϕ)
Φ=ql
4πϵ2l+1l0rl+1
0l=2,2,4
q∑4π√2l+1√2l+1P(cosθ)
=qall
4πϵ02l+1ππrl+1
l=2,4...
l
=q2aP(cosθ)
4πϵrl+1l
0
Theestwloordertermisl=2.Andinthex−yplaneθ=πsoewget
2
()
qa3
Φ=−4πϵ0ar
ThisistheersevincubicfunctionwhosegraphiswnshoinFig.(10.2)lookse.lik□
CHAPTER10.CLASSICALYNAMICSODELECTR235
estwLotermintheultipmoleexpansion
3
2
Φ
1
0
0.20.40.60.811.21.41.61.82
r
Figure10.2:Firsttermofultipmoleexpansion.
(d)Calculatedirectlyfromb’sColoumwlatheexactptialotenforbinthex−yplane.Plotitasa
functionofdistanceandcomparewithetheresultfoundinpartc.
Solution:
orFthehargescengivewevhainthecartesiancoordinatesystem,inx−yplane,ifthedistance
fromtheorigintoyanptoinontheplaneisrewget
Φ=q(√212−1+√212)
4πϵ0r+arr+a
PlottingthisasafunctionrewgettheplotinFig.(10.3)□
Exactsolution
2
1.5
Φ1
0.5
0
0.20.40.60.811.21.41.61.82
r
Figure10.3:Exactsolution
10.5.3.kson(Jac4.9)Aptoinhargecqislocatedinfreespaceadistancedfromthetercenofadielectric
sphereofradiusa(a<d)anddielectrictconstanϵ/ϵ0
CHAPTER10.CLASSICALYNAMICSODELECTR236
(a)Findtheptialotenatallptsoininspaceasanexpansioninsphericalharmonics.
Solution:
Thehargecatducesinhargecinthesphere.Theinducedhargecproducestheifeldinsidethe
sphere.Again,usingthegeneralsolutionofLapalace’sequationinsphericalsystemwithuthalazim
symetryewget
∞
Φ=q∑ArlP(cosθ)(10.11)
in4πϵll
l=0
eWcanhosecthecoordinatesystemhsucthattheZaxisofourcoordinatesystempassesthrough
thehargecandthetercenofsphere.WIhthis.Outsidethespheretheptialotenduetothehagec
isengivyb
∞l
Φ=1q+q∑BaP(cosθ)(10.12)
out4πϵ|r−r′|4πϵlrl+1l
00l=0
∞[l(l)]
=q∑r<+BaP(cosθ)(10.13)
4πϵrl+1lrl+1l
0l=0>
Thecomptonenofelectriifeldparalleltothesurfaceofthesphereis
∞[l]∞
in1∂Φinq∑r′q∑1′
E=−=AlP(cosθ)sinθ=AlP(cosθ)sinθ
θl+1ll
r∂θr=a4πϵ0l=0ar=a4πϵ0l=0a
(10.14)
Similarlythecomptonenoutsidethesphereis
∞[()]∞[]
∑rll∑l
Eout=q<+BaP′(cosθ)sinθ=qa+BlP′(cosθ)sinθ
θl+1ll+1ll+1l
4πϵ0rr4πϵ0da
l=0>r=al=0
(10.15)
Equating(10.14)and(10.15)ewget
[l][l+1]
qAqaBϵa
l=+l=⇒A=+B(10.16)
l+1ll+1l
4πϵa4πϵdaϵd
00
∞[]∞[]
∂Φq∑lrl−1q∑l
Ein=−ϵin=AP(cosθ)=AP(cosθ)(10.17)
rll+1ll2l
∂rr=a4πl=0a4πl=0a
r=a
Similarlyfortheradialcomptonenofifeldoutsiethesphereis
∞[][]
∑l−1l+1l−1
∂Φqla(l+1)aqla(l+1)
out
Eout=−ϵ=−BP(cosθ)=A−BP(cosθ)
r0l+1ll+2lll+1l2l
∂rr=a4πl=0drr=a4πd(10.18)a
Equating(10.17)and(10.18)ewget
[l−1]l+1
qAll=qla−Bl+1=A=⇒a−Bl+1(10.19)
2l+1l2ll+1l
4πa4πdadl
SolvingowtlinearequationsinAlandBlfrom(10.19)and(10.16)ewget
(ϵ0−1)lal+1
B=ϵ(10.20)
lϵ0l+1
l+(l+1)ϵd
2l+1al+1
Al=ϵ(10.21)
0l+1
l+(l+1)ϵd
CHAPTER10.CLASSICALYNAMICSODELECTR237
Substutingthecoteiffcienin(10.20)an(10.12)ewget
∞l
Φin=q∑2l+1ϵ0rPl(cosθ)
l+1
4πϵl=0l+(l+1)ϵd
Andsimilarly
∞[lϵ02l+1]
Φ=q∑r+(ϵ−1)laP(cosθ)
outl+1ϵl+1l
4πϵ0dl+(1+l)ϵ(rd)
l=00
Thesearetheexpressionfortheelectriifeldinsideandoutsidethesphere.□
(b)Calculatetherectangularcomptsonenoftheelectricifeldnearthetercenofthesphere.
Solution:
Insidethesphere,theifrstfewtermsare
q[13r5r23]
Φ=P(cosθ)+P(cosθ)+P(cosθ)+O(r)
inϵ00ϵ01ϵ022
4πϵ0ϵ1+2ϵd2+3ϵd
Theradialradialcomptonenoftheifeldis
∂Φinˆq[31]ˆq[3cosθ]ˆ
Er=−∂rr=−4πϵ0+1+2ϵ0dP1(cosθ)+O(r)r=−4πϵd1+2ϵ0+O(r)r
ϵϵ
Inthelimitr→0ewget[]
q3cosθˆ
Er=−4πϵd1+eϵ0r
ϵ
Similarlythetialtangen(θ)comptonenofifeldis
1∂Φinˆ−1q[−3sinθr]ˆq[3sinθ]ˆ
Eθ=−r∂θθ=r4πϵ0+1+eϵ0d+O(r)θ=4πϵd1+2ϵ0+O(r)θ
ϵϵ
Inthelimitr→0ewget[]
q3sinθˆ
Eθ=ϵθ
4πϵd1+20
ϵ
Sincetheϕcomptonenoftheifeldis0astheptialotenisindeptendenofϕewget
q3[ˆˆ]q3ˆ
E=ϵ−cosθr+sinθθ=ϵk
4πϵd1+204πϵd1+20
ϵϵ
ˆ
Wherekistheunitectorvalongz−axis.□
(c)erifyVthat,inthislimitϵ/ϵ0→∞,hourresultisthesameasthatforconductingsphere
Solution:
Inthelimitϵ/ϵ→∞ewevha
0
Φ=q
in4πϵd
0
and[]
∞l∞2l+1
Φ=q∑r<−∑aP(cosθ)
out4πϵrl+1(rd)l+1l
0l=0>l=1
eWcaneokvinthesphericalharmonicsexpansioninreerseandwritetheexpressionas
q[q/d+1′−a2ˆ]
4πϵ0r|r−r||dr−ar|
Whihisindeedtheptialotenofasphereoutsidethesphere□
CHAPTER10.CLASSICALYNAMICSODELECTR238
−35
1.04·10
1.04
mol1.04
γ
1.04
1.03
2.052.12.152.22.252.32.352.42.452.52.55
1−3
T·10
10.6orkHomewSix
10.6.1.kson(Jac4.12)aterWaporvisapolargaswhosedielectricotcnstanexhibitsandappreciabletem-
peraturedependence.Thewingfollotableesgivexptalerimendataonthiseffect.Assumingthataterw
aporvobeystheidealgasw,lacalculatethemolecularpyolarizabilitasafunctionofersevinntempera-
tureandplotit.romFthesolopeofthee,curvdeduceaaluevfortheptermanendiptolemomenofthe
HOmolecule.
2
ϵ5
T(K)Pressure(cmHg)(ϵ0−1)×10
39356.49400.2
42360.93371.7
45365.34348.8
48369.75328.7
Solution:
Withtheidealgasequationewevha
PV=NkT=⇒n=N=P
VkT
ByClausis-Mossettiequationewevhathemolecularpyolarizabilitisengivyb
3(ϵ/ϵ−1)3kT(ϵ/ϵ−1)
γ=0=0
nϵ/ϵ−2Pϵ/ϵ−2
00
1−35
PlottingthisasafunctionofTesgivTheslopeis8.9×10
□
10.6.2.kson(Jac4.13)owTlong,coaxial,cylindricalconductinsurfacesofradiiaandbareeredwloerticallyv
toinaliquiddielectric.Iftheliquidraisesaneragevatheighheenbwettheelectrodeswhenaptialoten
differenceVisestablishedeenbwetthem,wshothatthyesusceptibilitoftheliquidis
22
χe=b−aρghln(b/a)
ϵV2
0
whereρistheydensitoftheliquid,gistheacclerationdueto,yvitgraandtheysusceptibilitofthair
isneglected.
CHAPTER10.CLASSICALYNAMICSODELECTR239
Solution:
ThetotalenergyinthecapacitorofcapacitanceCisengivyb
E=1CV2
2
Thecapacitanceofcoaxialcylinderperunitlengthisengivyb
C=2πϵ0
ln(b/a)
Letlisthelengthofthecylindricalconductorseabvotheliquid,ofhwhictheliquidraisesuptoh.The
sectionwithl−hisairiflledandthesectionwiththeighhevahotheliquidsurfaceisdielectriciflled.
Sothecapacitanceofheacsectionesgiv
C=2πϵ0(l−h);C=2πϵh
airln(b/a)liquidln(b/a)
Thetotalardwupforceontheraisedliquidisusth
F=dE=1V2dC=1V2d(2πϵ0(l−h)+2πϵh)
dh2dh2dhln(b/a)ln(b/a)
=π[−ϵ+ϵ]
ln(b/a)0
Butewevhaϵ=ϵ+χϵ,soewget
0e0
πχϵ
F=e0
ln(b/a)
Thisforceisbalancedybthevitationalgraforceinquilibriumhwhicisengivyb
F=mg=ρVrg
TheolumevofraisedliquiedVris
22
Vr=π(b−a)h
usTh
22
F=ρπ(b−a)hg
Equatingtheforces
πχϵ
e022
ln(b/a)=ρπ(b−a)hg
22
χe=(b−a)ρghln(b/a)
ϵV2
0
Thisistherequiredexpression.□
Chapter11
ClassicalElectrodynamicsII
11.1orkHomewOne
11.1.1.kson(Jac6.1)Inthreedimensionsthesolutiontotheevawequation(6.32)foraptoinsourceinspace
andtime(atlighlfashatt’=0,x’=0)isasphericalshelldisturbanceofradiusR=ct,namely
(+)
theGreenfunctionG.Itymabeinitiallysurprisingthatinoneorowtdimensions,thedisturbance
possessesae”,ak“wenevthoughthesourceisa“pt”oininspaceandtime.Thesolutionsforerfew
dimensionsthanthreecanbefoundybsuperpositioninthesuperlfuousdimension(s),toeliminate
dependenceonhsucariable(s).vorFexample,alfashinglinesourceofuniformamplitudeistalenequiv
toaptoinsourceinowtdimensions.
(a)Startingwiththeretardedsolutiontothethree-dimensionalevawequation,wshothatthesource
′′′′
f(x,t)=δ(x)δ(y)δ(t),talenequivtoat=0ptoinsourceattheorigininowtspatialdimensions,
producesao-dimensionalwte,vaw
2cΘ(ct−ρ)
Ψ(x,y,t)=√222
ct−ρ
whereρ2=x2+y2andΘ(ξ)istheunitstepfunction[Θ(ξ)=0(1)ifξ<(>)0]
Solution:
Theretardedsolutionis
∫[f(x′,t′)]3′
Ψ(x,y,z,t)=retdx(11.1)
|x−x′|
Substutingthesourcefunctionwiththeengivdeltafunctionsewget
∫x−x′
δ(x′)δ(y′)δ(t−|c|)′′′
Ψ=Rdxdydz
∞
∫δ(t−R)′
=cdz
R
−∞
Sinceewevhacylindricalcoordinatesystemewget
′√2′2′′
R=|x−x|=ρ+(z−z)wherex=y=0
Thistegralincanbedonewithsubstitution.Supposingu=z′+z,ewgetdz′=duandthelimit
240
CHAPTER11.CLASSICALYNAMICSODELECTRII241
ystathesame
∞√
∫22
δ(t−ρ+u/c)
Ψ(ρ,t)=√22du(11.2)
ρ+u
−∞
wNothistegralinisoftheform
Ψ(a)=∫δ(f(x,a))dx
g(x)
makingsubstitutionofariablevf(x)=βewgetdβ=f′(x)dxsothatewget
Ψ(a)=∫δ(β)′1dβ
g(x)f(x)
Itisclearthatthedeltafunctiononlykspicupaluesvofxforhwhicβ=f(x)=0.Sothedelta
functionreducesthetegralintothesumofifnitealuesvforhwhicβ=f(x)=0,letthesolutions
ofβ=f(x)=0beαi,thises,mak
Ψ(a)=∑1′
ig(αi)f(αi)
√22
forthisproblemewevhaf(u)=t−ρ+uwhosezerosare
c
√22√
t−ρ+αi=0=⇒α=±c2t2−ρ2ifct>ρ
ci
therearenorootsifct<ρandthedeltafunctioniszeroandthetegralinisticallyidenzero.Also
theeativderivattherootis
√222
f′(u)=√u=⇒f′(αi)=±ct−ρ
22cct
cρ+u
Substitutingthisinthetegralin(11.2),wingknothatthereareowtaluesvofαiewget
{2
√2ct1ifct≥ρ
222ct
Ψ(ρ,t)=ct−ρ
0ifct≤ρ
theowtcasescanbebinedcomybusingvisideheafunction
2cΘ(ct−ρ)2cΘ(ct−√x2−y2)
Ψ(x,y,t)=√222=√2222
ct−ρct−x−y
Thisistherequiredformoftheevaw
□
(b)wShothata“sheet”source,talenequivtoaptoinpulsesourceattheorigininonespacedimension
producesaonedimensionalevawproportionalto
Ψ(x,t)=2πcΘ(ct−|x|)
Solution:
orFthesheetsourceewexpectaplanepropagationofthee.vawThesourcefunctionforthesheet
CHAPTER11.CLASSICALYNAMICSODELECTRII242
sourceatsomeparticulartimet′=0,letthex′=0planebethesource,soewcanwritethe
sourcefunctionas
f(t′,x′)=δ(x′)δ(t′)
Usingthissourcefunctiontogettheretardedtimesolutionandsubstutingin(11.1)ewget
∞
∫′′
δ(x)δ(t)
ret′′′
Ψ(x,y,z,t)=Rdxdydz
−∞
√′2′2′2
AgainewgetR=(x−x)+(y−y)+(z−z).Againsimilartothepreviousrpoblem
haningcofariablesvwithu=y−y′,v=z−z′andrecognizingthatthedeltafunctiontegralin
simplykspicupx′=0ewget
∞√
∫222
δ(t−x+u+v)
Ψ(x,y,z,t)=√cdudv
222
x+u+v
−∞
√22
Sincethetegralinhascylindricalsymmetrywhenewevhaρ=u+vewcanemakcylindrical
ariablevsubstitutiontoget
∫√22
δt−ρ+x/c
Ψ(ρ,ϕ,z)=√22ρdρdϕ
ρ+x
Duetocylindricalindependencethephitegralinis2πandewareleftwithdeltafunctiontegralin
similartopreviousproblem
∫√22
δ(t−ρ+x/c)
Ψ(x,t)=√22ρdρ
ρ+x
Thisagainhasadeltafunctioninsidethetegral,inandisnon-zeroonlyforthedeltafunction
equaltozero,thezerosoftheexpressioninsidethedeltafunction,onlyegivnonzeroaluesvand
thetegralintrunstoasumervotheseifnitealuesvofsolution,thezerosofthedeltaare
√22√222
t−ρ+x/c=0=⇒ρ=±ct−xifct>x
√22
Alsosupposingβ=f(ρ)=t−ρ+x/cewget
′√2ρ√22
dβ=f(ρ)dρdβ=22=⇒ρdρ=cρ+xdβ
2cρ+x
Substutingthese
∫√δ(β)√22
Ψ(x,t)=cρ+xdβ
22
ρ+x
Sincethereareowtaluesvofzerosofthetionfunewevhaowttermsinsumandewget
Ψ(x,t)=c+c
Bysimilartsargumenasinthepreviousoneewgetnonzerotegralinonlyifct>xewcanwrite
thisusingthevisideHeafunction
Ψ(x,t)=2cΘ(ct−x)
Thisistherequiredfunction.□
CHAPTER11.CLASSICALYNAMICSODELECTRII243
11.1.2.kson(Jac6.4)AuniformlymagnetizedandconductingsphereofradiusRandtotalmagnetictmomen
m=4πMR3/3rotatesaboutitsmagnetizationaxiswithangularspeedω.INthesteadystateno
tcurrenwslfointheconductor.Themotionisnonrelativistic;thespherehasnotexcesshargeconit.
(a)ByconsideringOhm’swlainthevingmoconductor,wshothatthemotioninducesandelectric
ifeldandauniformolumevhargecydensitintheconductorρ=mω/πc2R3
Solution:
Themagnetictmomenofsphereisengivybm=MVwhereV=2πR3istheolumevofsphere.
3
ˆ
ComparingittotheengivmagnetictmomenewgetthatM=Mz.Themagneticlfuxydensit
insidethesphereisengivyb
2µ0mˆ
B=3µ0M=2πR3z
Byohm’swlathetcurreninthevingmoconductoris
J=σ(E+v×B)
SincethereisnotcurrenJ=0hwhicimplies
E=−v×B
Sincethespherehasangularfrequencyˆω,thetranslationaleloyvcitatrisengivybv=r×ω=
ωr×zusthewget
µ0mˆˆˆˆ
E=r×ω×B=2πR3[z(z·r)−r(z·z)]
Thissimpliifesto
µ0mωˆˆ
E=2πR3(z(z·r)−r)
Thisistheprojectionofectorvrtoonthetalhorizonaxis,hwhicincylindricalsystemis
E=−µ0mωρ
ρ2πR3
wNothatewevhatheifeldewcanapplygauss’wlatocalculatethehargecydensit
∇·E=ρ
ϵ0
Sincourifeldonlyhascomptonenalongρewevha
ρ=ϵ∂Eρ=−µ0ωmρ
0∂ρ2πR3
Thisistherequiredolumevhargec.ydensit□
(b)Becausethesphereiselectricallyneutral,thereisnomonopoleelectricifeldoutside.Usesymmetry
tsargumentoowshthattheestwlopossibleelectricultipymolaritisquadrupole.wShothatonly
quadrupoleifeldexistsoutsidethatthequadrupuletmomentensorhasnonanishingvcomptsonen
22−Q
Q=−4mωR/3c,Q=Q=33.
3311222
Solution:
Sincethereisnohargecinsidethespheretheexteriorcanbedescribedastheultipmoleexpansion.
Sincethereisnoharge,cthemonopoletmomenhwhicisthetmomenoftotalhargeciszero.The
electrostaticptialotencanbeobtainedas
Φ(ρ)=−∫Edl=−∫Edρ=Φ+µ0mωρ2
ρ04πR3
CHAPTER11.CLASSICALYNAMICSODELECTRII244
Thiscanbesimpliifedybusingthecartesiancoordinateulationformas
Φ(r,θ)=Φ+µ0mωr2sin2θ.
04πR3
tingrinWsin2θintermsoflegendrepolynomialsewget
Φ(r,θ)=Φ+µ0mωr2[P(cosθ)−p(cosθ)]
06πR302
thissimliifesto
(µ0mω2)µ0mω2
Φ(r,θ)=Φ+rP(cosθ)−rP(cosθ)
06πR306πR32
tAthesurfaceofthespherer=Rewgettheptialotenas
(µ0mω2)µ0mω2
Φ(r,θ)=Φ+rP(cosθ)−rP(cosθ)
06πR306πR32
Sincetheptialotenisuthallyazimsymmetric,ewcanwritetheexternalptialotenasalegendre
polynomialseries
V(θ)=∑AP(cosθ)
ll
i
onthesurface,andoutsidethesurfacetheptialotenis
Φ(,θ)=∑A(R)l+1P(cosθ)
lrl
l
Sincethereisnohargecthemonopoletermforl=0anishesvsoewget
Φ=−µ0mω
06πR
Andtheexpressionbecomes.
µ0mωR2
Φ(r,θ)=−P(cosθ)
6πr32
wNotathewhaeteexteriorptialotencanbeertedvcontoexpressionwithsphericalharmonics
√4πµmωR2Y(θ,ϕ)
Φ=−020
56πr2
Thestandardultipmoleexpansionexpressionis
∞l
Φ=1∑∑2πqYlm(θ,ϕ)
4πϵ2l+1lmrl+1
0l=−∞m=−l
compariosionesgiv
√5µ0mωR2√52mωR3
q=−4πϵ=−
2004π6π4π3c2
Thetmomenexpressionincartesiancoordinatesystemisengivyb
Q=2√4πq=−4mωR2,Q=Q=−1Q
33202112233
53c2
thisistehrequiredexpression.□
CHAPTER11.CLASSICALYNAMICSODELECTRII245
(c)Byconsideringtheradialelectricifeldsinsideandoutsidethesphere,wshothatthenecessary
surfacehargecydensitσ(θ)is
14mω[5]
σ(θ)=··1−P(cosθ)
4πR23c222
Solution:
thesurfacehargeccanbecomputedybusingtehnormalcomptonenasesativderivofptial.oten
Inthesphericalcoordinatesewget
µmωR2
0
Eout=−P(cosθ)
r2πr42
µ0mωr
Ein=−[p(cosθ)−P(cosθ)]
r3πR3o2
thesurfacehargecisusth
()µϵmω[3]
00
σ=ϵEout−Ein=P(cosθ)−(P(cosθ)−P(cosθ))
0rrr=R3πR22202
mω[5]
=3πc2R3P0(cosθ)−2P2(cosθ)
Thisesgivtherequiredexpressionfortehsurfacehargec.ydensit□
(d)Therotatingsphereesservaasaunipolarinductiondevieifastationarycircuitishedattacyba
slipringtothepoleandslidingtactcontotheequator.wShothattehlinetegralinoftheelectric
ifeldfromtheequatortatcontothepoletactconisE=µmω/4πR
0
Solution:
Thelinetegralinis
pol
E=∫Edl=Φequator−Φpol=Φ(θ=π/2)−Φ(θ=0)
equator
Substutingthealuevofthetaintheexpressionfortheptialotenewget
E=−µ0mω[P(0)−P(1)]=µ0mω
6πR224πR
Thisesgivtherequiredexpressionforthetegral.in
□
11.2orkHomewowT
11.2.1.kson(Jac6.11)Aersetransvplaneevawistincidennormallyinacuumvonaperfectlyabsorbinglfat
screen
(a)romFawlaofationconservoflineartummomen,wshothatthepressuerexertedontehscreenis
equaltotheifeldenergyperunitolumevinthee.vaw
Solution:
eWcanhocoseourcoordinatesystemhsucthatthezaxisliesalongthedirectionthattheplane
evawels.vtraSinceelectricamdmagneticifeldsareperpndiculartoheacotherandtothedirection
ofpropagationtheelectricifeldandmagneticifeldbecome
ˆˆ
E=EiH=Hj
CHAPTER11.CLASSICALYNAMICSODELECTRII246
Thetummomenationconservequationforjthcomptonenofthetummomenis
d(P+P)=I∑Tnda(11.3)
dtifeldshmecjiji
i
ˆ
romFtheyawewhosecourcoordinatesystemnonlyhascomptonenalongthekdirection,the
indexforhwhicis3soewcanreplacethesummationybasingleterm
∑Tn=T
iji3j
i
ThestressenergytensorTisengivyb
ij
(12)(12)
T=ϵEE−Eδ+µHH−Hδ
ij0ij2ij0ij2ij
Calculatingthethecomptonenofthetensorintherquireddirecitonewget
(12)(12)
T=ϵEE−Eδ+µHH−Hδ
j303j23j03j23j
=1(ϵE2+µH2)δ
2003j
Againybourhoicecofcoordinatesystemthecomptonenoftummomenisalsoalongthezaxisso
theonlynonanishingvcomptonenoftummomenisinthatdirection.

1(22)
P=ϵE+µHδ
j=32003j
j=3
=1(ϵE2+µH2)
200
Theexpressiononthetrighistheexpressionfortheenergyydensitofelectromagneticevawso
rtheexpressioncanbewrittenas
P3=u
whereuistheenergy.ydensitSincetheforceisthehangecintummomenperunittime,andsince
theinitialtummomeniszero,ewget
¯
F=(P−0)/t=P
33
¯
where·isthetimeeragedvatum.momenhWhicisequaltotimeeragedvaenergy,ydensitusthew
get
F=u¯
Thiswsshothattheenergyydensitisenergyydensitoftheifeld.□
(b)Inthebneighorhoodoftheearththelfuxofelectromagneticenergyfromthesuisximatelyppro
21g
1.4W/km.Ifanterplanetaryin“sailplane”hadasailofmassm2ofareaandnegligibleother
t,eighwwhatouldwbeitsummaximaccleartioninmeterspersecondsquaredtothethesolare
readiationpressure?wHodoesthiscomparewiththeaccleartionduetosolare“wind”(corpuscular
radiation)?
Solution:
Thelfuxrelationtotheenergyydensitisu=lfuxsoewget
c
3
P=1.4×10=5×10−6N
82
3.0×10m
CHAPTER11.CLASSICALYNAMICSODELECTRII247
Sotheaccleartion(a)canbecalculatedas
−6
a=PA=P=5×10=5×10−3m
m−32
mA1×10s
Thetheaccleartionisa=5×10−3m□
2
s
11.2.2.kson(Jac7.1)orFheacsetofesStokparameterssets0=3,s1=−1,s2=2,s3=−2,deduce
theamplitudeoftheelectricifeld,uptoanerallvophase,inbothlinearpolarizationandcircular
polarizationbasesandemakanaccuratewingdrasimilartoFig.7.4wingshothelengthsoftheaxesof
oneoftheellipsesanditstationorien
Solution:
Theesstokparametersaredeifnedforlinearpolarizationwiththewingfollorelations
22
s=|E|+|E|
012
22
s=|E|−|E|
112
s=2Re(E∗E)=2|E||E|cos(θ−θ)
2121221
s=2Im(E∗E)=2|E||E|sin(θ−θ)
3121221
ertingvIntheserelationsewget
√s+s√s−s√
|E|=01=2|E|=01=2
1222
θ−θ=acos(√s2)=π
21224
s−s
01
Withthesepaametersewgetthecomptsonenofelectricifeldas
(iθ1iθ2)iθ1(iθ2−iθ1)
E=|E|e,|E|e=e|E|,|E|e
1212
Sicethephasefactorinfronisarbitraryewcanignoreitbecauseewcanysaalwehievaczerophase
factorybrotationofhoicecofaxes.Similarlyforthecircularpolarizationcaseewevhatheesstok
parametersdeifnedas
22
s0=|E+|+|E−|
s1=2|E+||E−|cos(θ−−θ+)
s=2|E||E|sin(θ−θ)
2+−−+
Similarlyertingvintheseifeldamplitudesintermsofparametersegiv
√s+s1√s−s√5
|E|=03=√|E|=03=
+22−22
(s1)(−3)
θ−−θ+=acos√22=acos√
s−s5
03
wNotheifeldcmptsonenare
(iθ1iθ2)iθ1(iθ2−iθ1)
E=|E1|e,|E2|e=e|E1|,|E2|e
WiththeparameterforE1andE2andthephasedifferencethediagramcanbeplotted.□
CHAPTER11.CLASSICALYNAMICSODELECTRII248
11.2.3.kson(Jac7.3)owTplanesemi-inifniteslabsofthesameuniform,isotropic,nonpermeable,lossless
dielectricwithindexofrefractionnareparallelandseparatedybanairgap(n=1)withwidthd.A
planeelectromagneticevawoffrequencyωistincidenonthegapfromoneoftehslabswiththeangle
foincidencei.orFlinearpolarizationothbparallelandperpendiculartotheplaneofincidence
(a)Calculatetheratiooferpwotransmittedtointhesecondslabtotheincidenerpwoandtheratio
ofrelfectedtotincidener.pwo
Solution:
Letiisthetincidenangleandristheangleofrefractionybsnellswlaewevha
nsini=sinr
wherenistheerefractivindex.eWcanrearrangethistoget
√2√22
cosr=1−sinr=1−nsini
Thealuevofcosrispurelyimaginarywheniisgreaterthancriticalanglefortotalternalin
relfection.oTifndthetransmittednandreplectedcomptsonenintermsofthteincidencomptonen
ewcanusetheterfaceinhing.matcIntheifrstterfacein
E=E+E=E+E
pir+−
H=n(E−E)cosi=(E−E)cosr
pir+−
HereewevhaEandHaretheparallelcomptsonenofelectricandmagneticifeld.
pp
Inthesecondterfaceinewevha
Eeik·d+Ee−ik·d=E
+−t
(Eeik·d−Eeik·d)cosr=nEcosi
+−t
SolvingforEandEintermsofEandEewget
+−ri
1(ncosi)1(ncosi)
E=E1++E1−
+2icosr2rcosr
1(ncosi)1(ncosi)
E=E1−+E1+(11.4)
−3icosr2rcosr
Similarlytheconditionwiththesecondterfaceincanbeedsolvtoget
1ik·d(ncosi)
E+=2eEt1+cosr
1ik·d(ncosi)
E=eE1−(11.5)
−2tcosr
Letuswriteϵ=ncosiEuation.(11.4)and(11.5)canbeedsolvtoget
cosr
Et=4ϵ(11.6)
2−ik·d2ik·d
Ei(1+ϵ)e−(1−ϵ)e
2id·k
E(1−ϵ)(e)
r=(11.7)
2ik·d2ik·d
E(1+ϵ)e−(1−ϵ)e
i
Thisesgivtheratiooftransmittedtorelfectedamplitudes.Theerpwoisproportionaltothesquare
amplitudessotheratiooftransmittederpwotothetincidenerpwois
[]
PE2E2
t=t=t
PiE2Ei
i
CHAPTER11.CLASSICALYNAMICSODELECTRII249
andsimilarlytherelfectederpworatiois
[]
PE2E2
r=r=r
PiE2Ei
i
Thesearetherequiredratioswheretheratiosofamplitudesarecalculated.□
(b)forigreaterthanthecriticalanglefortotalternalinrelfection,hetcsktheratiooftransmitted
erpwototincidenerpwoasafunctionofdinunitsofelengthvawinthegap.
Solution:
Intheequations(11.6)and(11.7)ewcanwritetherationϵandthephasek·daspurelyimaginary
bumersnandsimplifythosequationstogetthefunctionofthratios.Soassumingthephaseand
theratiotobecomplexewget
ϵ=iαk·d=iβ
Usingthesein(11.6)and(11.7)ewget
[]
T2iα24α2
t==
T2iαcoshβ+(1−α2)sinhβ4α2+(1+α2)sinh2β
i
andsmimilarlytheratioofrelfectedtotransmittederpwois
222
Tr=(1+α)sinhβ
Ti4α2+(1+α2)sinh2β
Substutingβ=kdandalson=1ewget
T(1+α2)2sinh2kd
r=
Ti4α2+(1+α2)sinh2kd
d
Graphingthisfunctionasafunctionofλewget.
□
11.3orkHomewThree
11.3.1.kson(Jac7.12)Thetimedependenceofelectricaldistrubaancesingoodconductorsisernedvgoyb
thefrequency-deptenden.yconductivitConsiderlongitudinalelectricifeldsinaconductor,usingOhm’s
w,latheyuittunconequation,andthetdifferenformofb’sCoulomw.la
(a)wShothattheouriertime-Ftransformedhargecydensitsatisifestheequation
[σ(ω)−iωϵ0]ρ(x,ω)=0
Solution:
Letusassumethetimearyingvtitiesquanbehargecydensitρ(t),tcurrenydensitJ(t)andelectric
ifeldE(t).akingTthefouriertransformtoetaktofrequencyspace
1∫iωt
ρ(ω)=√2πρ(t)edt
1∫iωt
J(ω)=√2πJ(t)edt
1∫iωt
E(ω)=√2πE(t)edt
CHAPTER11.CLASSICALYNAMICSODELECTRII250
wNoTheytuinitconequation
∇·J=−∂ρ
∂t
Inthefrequencyspace,thisbecomes
∇·J(ω)=iωρ(ω)
TheOhm’swlarelateshagectcurrenydensitandelectricifeldas,
J(ω)=σ(ω)E(ω)
Thebscoloumwlacanbeusedexpresstherelationeenbwettheelectricifeldandhargecydensit
as
∇·E(ω)=ρ(ω)
ϵ
0
biningComalltheseewobtain
(σ(ω)−iωϵ0)ρ(ω)=0
Thisistherequiredexpression.□
(b)Usingthetationrepresenσ(ω)=σ/(1−iωτ)whereσ=ϵω2τandτisadampinngtime,wsho
000p
thattheximationapproωτ≫1yaninitialdistrubancewilloscillatewithplasmaycfrqueynand
p
ydecaamplitudewithaydecaconstatnλ=1/2τ.
Solution:
Usingthetationrepresenσ(ω)=σ0(1−iωτ)ewget
(σ)
0−iωϵρ(ω)=0
1−iωτ0
substutingσ=ϵω2τ
00p
[ω2τ]
p−iω=0
1−iωτ
thisisaquadraticequationinωhwhiccanberearrangedtogetτω2+iω−ω2τ=0.Thesolutions
p
are
√22
ω=−i±4τωp−1
2τ
Usingtheengivximationapproωpτ≫1ewobtain
ω=±ω−i
p2τ
1
Thiswsshothatinfrequencyspacethesignalisedydelayb2τ.ertingRevkbactotimespacewith
ersevinfouerirtransformewget
f(t)=F−1F(ω−i1)
p2τ
f(t)=f0(t)e−t/2τ
Thiswsshothatthesignalisysdecaattherate1□
2τ
CHAPTER11.CLASSICALYNAMICSODELECTRII251
11.3.2.kson(Jac7.19)Anximatelyappromonohormaticcplaneevawetkpacinonedimensionhasthein-
kx
taneousstanformu(x,0)=f(x)e0,withf(x)themodulationelopve.enorFheacoftehformsf(x)
222
bw,elocalculatetheevawbumernspectrum|A(k)|oftheet,kpachetcsk|u(x,0)|and|A(k)|,aluateev
explicitlythermsdeviationsfromthemeans∆xand∆k
(a)f(x)=Neα|x|/2
Solution:
kx=α|x|/2
Theinitialeformvawforthisproblemisu(x,0)=Ne0Theevawbumernspectrumcan
beobtainedas
∞
1∫ikx
A(k)=√u(x,0)eds
2π
−∞
∞
1∫ikx+ik0x−α|x|
=√Ne2dx
2π
−∞
Thistegralinisafunctionofαandsinceitisenevfunctionofxewcanwriteeabvotegralinas
∞
1∫−αx/2
A(k)=√2Ncos(k−k0)xe
2π
0
Thistegralincanbecomputedandtheifnalexpressionforthetegralinesgiv
1[Nα]
A(k)=√22
2πα/4+(k−k)
0
Themeansquarealuevforafunctionf(x)isengivybtheexpression
∞
∫22
x[f(x)]dx
MeanSquare=−∞
∞
∫2
[f(x)]dx
−∞
orFthemeansqureddeviationofxewcanwrite
∞
∫2−α|x|
xedx
σ2=−∞
x∞
∫α|x|
edx
−∞
Thesetegralsincanbecalulatedwithgammafunctions,andtheifnalresultaftertegrationinis
√
∆x=2
α
SimilarlywithsameentokforthespreadofA(x)ewobtain
v
u∞[]
u∫212
ukdk
uα2/4+k2
u
u
−∞
∆k=u∞[]
u∫2
u1
t−∞α2/4+k2dk
CHAPTER11.CLASSICALYNAMICSODELECTRII252
Thistegralinaswobtainedusingcomputeralgebrasystemandtheifnalexpressionis
∆k=α
2
kingChecfortheproductof∆x∆kewget
√11
∆x∆k=2/2=√2≥2
□
(b)f(x)=Neα2x2/4
Solution:
akingTthefouerietransformtogetthefrequencycomptonenfunctions
∞
1∫ikx
A(k)=√u(x,0)edx
2π
−∞
Thiscanbetegratedinfortheengevinitialshapeas
∞
1∫22
√ik0x−ikx−αx/4
A(k)=Ne
π
2
−∞
Thsicanbecalculatetoobtain
√22
A(k)=N2/α2e−(k−K0)/α
Thespreadcanbewbesimilarlycalculagedaseabvo
v
u∞
u∫22
u2−αx/2
uxedx
u
u
−∞
∆x=u∞
u∫22
u−αx/2
t−∞edx
Thetegralsincanbecalculatedusinggammfunctiontitiesidenandtheifnalexpression(with
computeralgebrasystemused)is
∆x=1
α
Similarlythespreadinthefrequencycomptonencanbecalculated
v
u∞
u∫2−2k2/α2
ukedk
u
u
u
−∞
∆k=u∞
u∫22
u−2k/α
t−∞edk
Thisaswalsoedsolvusingcomputeralgebrasystemtoobtain
∆k=α
2
CHAPTER11.CLASSICALYNAMICSODELECTRII253
orFthissignalalsotheyinequalit∆x∆k≥1holdsas
2
∆x∆k=1·α=1≥1
α222
Soboththeevawtrainsatisfytheyuncertainitprinciple.□
11.3.3.kson(Jac8.2)Atransmissionlinesonsistingofowttricconcencircularcylindersofmetalwithcon-
yductivitσandskindepthδ,aswn,shoisiflledwithauniformlosslessdielectric(µ,ϵ).ATEMmode
ispropagatedalongthisline,
(a)wShothattheeragedvtime-aerpwowlfoalongthelineis
õ22(b)
P=πa|H|ln
ϵ0a
whereHisthepeakaluevoftheuthalazimmagneticifeldandthesurfaceoftheinnserconductor.
0
Solution:
BydeifnitionaTEMmodeisasignle-frequencyevawcmptonenwithboththeelectricifeldand
magneticifeldersetransvtothedirectioofpropagationalongtheeaxis.vawTheinnderconductor
hastoevhasomehargecperunitlength,ysaλ.Withacylindricalgaussiansurfacearoundthe
innerconductorewifndtheelectricifeldis
λˆ
E=2πϵρρ
Sincetheeguidevawaxisisalongthezaxis,themagneticifeldcanbeobtainedfromelectricifeld
as
√ˆ√λˆ
B=µϵz×E=µϵ2πϵρϕ
SinceengivintheproblemthatHisthepeakaluevofmagneticifeldintheinnerconductor,ew
0
obtainHas
0
H=H(ρ=a)=B(ρ=a)1=1λ
0√
µϵµϵ2πa
Thisexpressionesgivthetotalhargecperunitlengthequalto
√
λ=2πaH0µϵ
Substutingthisintheexpressionforelectricifeldewget
√µqˆaˆ
E=HρB=µHϕ
ϵ0ρ0ρ
Theseowtifeldsarecorrectforthestaticproblem.troInducingthetimedependenceinthe
eguidevawewobtain
√µqikz−iωtˆaikz−iωtˆ
E=HeρB=µHeϕ
ϵ0ρ0ρ
wNoewcancalculatetheenergylfuxusingthetingpynoectorvas
S=E×H
[][]
1√aikz−iωtaikz−iωt
S=µµ/ϵH0ρe×µH0ρe
CHAPTER11.CLASSICALYNAMICSODELECTRII254
SinceingeneralewtheytitquanHisacomplexbumernewcanwritethisas
0
iθ
H=|H|e
00
Substutingthisinaboeexpressionandcarryingoutthecorssproductewget
√2
µ2a2ˆ
S=|H|cos(kz−ωt+θ)z
02
ϵρ
Thetimeeragedvaerpwolfuxisusththeeragevaofeabvoexpression.Buttheeragevaofcos2is
⟨2⟩1
cosα=2
Soewget
√2
1µ2aˆ
⟨S⟩=|H|z
2ϵ0z2
Thetotalerpwocanwnobeobtainedybtegratingintheerpwolfuxervothewholearea
Iˆ
P=Sz·⟨S⟩dA
2πb√
∫∫2
=1µ|H|2aρdρdϕ
02
2ϵρ
0a
Thetegralininϕisjustthealuev2πandtherhotegralinisjustlogarithm.Soewget
õ22(b)
P=πa|H|ln(11.8)
ϵ0a
Thisistherequriederpwow.lfo□
(b)wShothattehtransmittederpwoisuatedattenalongthelineas
P(z)=Pe−2γz
0
where
√1+1
1ϵbb
γ=2σδµln(b)
a
Solution:
Therateoferpwolossperunitareawithskindepthδisengivyb
dP1
2

da=4µcωδH∥
Theareatelemenincanbewrittenas
da=ρdϕdz
Usingthisexpressionintheerpwowlfoequationewget
2π
dP1∫
2

daρdϕ=4µcωδH∥ρdϕ
0
CHAPTER11.CLASSICALYNAMICSODELECTRII255
Thereareowtboundariesthesurfacesoewget
dPπ[]
22

dz=2µcωδH∥(a)a+bH∥(b)
Themagneticifeldpartoftheexpressioncanbesubstutitedtoget
dP=π|H|2a[1+a](11.9)
dzσδ0b
Asengivintheproblem,assumingtheerpwolossalongthelineas
P(z)=Pe−2γz
0
tiatingDifferenwithrespecttozewget
dP(z)=−2γP=⇒γ=−1dP
dz2Pdz
SubstutingPfrom(11.8)anditseativderivfrom(11.9)ewget
√1(π2[a])
γ=()|H|a1+
µbσδ0b
22
πa|H|ln
ϵ0a
Simpliifcationyields
1√ϵ1(1+1)
ab
γ=2µσδln(b)
a
Thisistherequiredexpression.□
11.4orkHomewourF
11.4.1.kson(Jac8.4)erseransvTelectricandmagneticesvawarepropagatedalongaw,hollotrighcircular
cylinderwithinnerradiusRandyconductivitσ
(a)FindthecutofffrequenciesoftheariousvTEandTMmodes.Determineumericallyntheestwlo
cutofffrequency(thetdominanmode)intermofthetuberadiusandtheratioofcutofffrequencies
ofthenextfourhighermodestothatofthetdominanmode.orFthispartassumethatthe
yconductivitofthecylinderisinifnite.
Solution:
ThealueveneivequationforboththeTEandTMmodeis
(22)
∇+γψ(r,ϕ)=0
t
whereψ(R,ϕ)=0.orFTEmodethereisnoaxialelectricifeld,soewcanesolvforBz.Thereare
nohargescandtscurrenintheeguidevawsotheyobyetthehomogenousevawequation
1∂2Bz
2
∇Bz−c2∂t2=0
Theevawisfreealongtheaxisofeguide.vawsoewcanssumethatthesolutionforthemagnetic
ifeldhasharmonicdependenceintimeinthedirectionofpropogationusthewcanwrite
ikz−iωt
B=Be
z
CHAPTER11.CLASSICALYNAMICSODELECTRII256
Substutingthisintheexpressionforthehomognneousequationewget
2(ω22)
∇B=−kB
tzc2z
Thelaplacianoperatorinthisexpressionisonlyinthetransersedirectionn.Becauseofthe
cylindricalsymmetriewcanwritethelapalacianinthecylindricalcoordinatesystemas
1∂(∂B)1∂2B(ω2)
rz+z=k2−B
r∂r∂rr2∂ϕ2c2z
Makingasubstution
k′=(ω2−k2)
c2
′imϕ
IfewassumethesolutionofthemagneticifeldB=R(kr)eewget
z
2∂2R(k′r)∂R(k′r)(222)
r∂r2+r∂r+kr−mR(kr)=0
Thistialdifferenequationcanbeertedvcontoabesseltialdifferenequationwithkr=x.The
equationthenbecomes
∂2R(x)∂R(x)()
222
x∂x2+x∂x+x−R(x)=0
Thisisbtialesseldifferenequatio.thesolutionofthisequationis
R(x)=AJ(x)+BN(x)
mm
Substutingthisinthemagneticifeldexpressionewget
Bz=(Am(x)+BNm(x))eikz−ωt+mϕ
SincethefunctionsN(x)wbloupatx=0,andthattheifeldisifniteattheaxisewevhato
m
evhaB=0.Thesolutionthenbecomes
′i(kz−ωt+mϕ)
B=AJ(kr)e
zm
tAthesurfaceoftheperfectconductiorconstutingtheallsofweheguide,vawewevhatheboundary
condidtion

∂Bz
=1
∂r
r=R
Applyingthisconditionewget
(∂′)
∂r(Jm(kr))=0
r=R
Thezerosoftheequationaresimplythezerosofesativderivofbesselfunctions.Assumingthe
zerosareαewget
mn
α
k′r=α=⇒k′=mn
mnR
Substutingthisfortheexpressionrelatingkandk′ewget
√ω2α2
k=−mn
22
cR
CHAPTER11.CLASSICALYNAMICSODELECTRII257
usThthemagneticifeldbecomes
B=AJ(αr)ei(kz−ωt+mϕ)
zmmnR
orFTEmodetheaxialelectricifeldobeyssameequationandewgetsimilartialdifferenequation
whosesolutionis
′i(kz−ωt+mϕ)
E=AJ(kr)e
zm
TheboundaryconditionisthatheelectricifeldiszeroattheallswE(r=R)=0soewgetwNo
z
insteadofthezerosofesativderivofbesselfunctionthezerosareatthezerosofbesselfunction
βsoewget
mn
k′=βmn
R
Thesolutionthenbecomes
ri(kz−ωt+mϕ)
E=AJ(β)e
zmmnR
Thecutofffrequenciesarethefrequencieswherethebumerenvawequalszero.Soewget
β
ω=cmnforTEmode
mnR
α
ω=cmnforTMmode
mnR
ThesearetherequiredcutofffrequenciesforTEmodeandTMmode.□
(b)CalculateforTMmodetheuationattentsconstanoftheeguidevawasafunctionoffrequencyfor
theestwloowtdistinctmodesandplotthemasafunctionof.frequency
Solution:
orFTMmode,theerpwolossisengivyb
()2I
dP1ω1∂ψ
−=dl
dz2σδωmncµ2ω2∂a
mn
□
11.4.2.kson(Jac8.6)Atresonanyvitcaofcopperconsistsofaw,hollotrighcircularcylinderofinnerradius
RandlengthL,withthelfatendfaces.Determinethetresonanfrequenciesoftheyvitcaforallyptes
√1
ofes.vawWithµϵRasaunitof.frequencyPlottheestwlotresonanfrequenciesofheacypteass
functionofRfor0<R<2.DoesthesamemodeevhatheestwlofrequencyforR?
LLL
Solution:
orFthe,yvitcathenormalmodesinTMmodesareengivyb
x
ψ(,ϕ)=EJ(γr)e±imϕwhereγ=mn
0mmnmnR
HerexarethezerosofbesslefunctionJ.AsengivinksonJaceq.8.81ewgetthetresonanfrequency
mnm
√()
1pπR2
2
w=√x+
mnpµϵRmnL
Thezerosofbesselare
x01=2.405,x12=3.832,x21=5.136and
□
CHAPTER11.CLASSICALYNAMICSODELECTRII258
11.5orkHomeweFiv
11.5.1.kson(Jac9.3)owTeshalvofasphericalmetallicshellofradiusRandinifniteyconductivitaresepa-
ratedybaeryvsmallinsulationgap.Analternatingptialotenisappliedeenbwettheowteshalvofthe
spheresothattheptialsotenare±Vcosωt.Inthelongelengthvawlimit,ifndtheradiationifelds,the
angulardistributionofradiatederpwoandthetotalradiatederpwofromthesphere.
Solution:
owToppositehargedceshalvofspherecreatesadipolesothedipoletermintheptialotenexpansionis
thetdominanterm.ThetdominantermontheptialotenexpansionintermsofLegendrepolynomial
expansionis
3R2
Φ=V2r2cosθ
Theptialotenduetoelectricdipoleptingoinizthepeositivzdirectionisengivyb
Φ1pcosθ
dip4πϵr2
0
Thetdonimantermustmbeequaltothedipoleptial.otenEquatingthese
3R21p
V2r2cosθ=4πϵr2cosθ
0
2ˆ
=⇒p=6πϵ0VRz
Theptialoteninthesphereisoscillationwiththefrequencyωascosωt,Themagneticifeldofhsuc
oscillatingifeldcanbewrittenas
2()i(kr−ωt)
µ0ckpˆˆe
B=4πk×pr
Subsistingthealuevofthedipoletmomenewget
3Vk2R2ei(kr−ωt)ˆ
B=−2crsinθϕ
Theelectricifeldissimilarlyengivyb
k2pˆ(ˆˆ)ei(kr−ωt)
E=−4πϵ0k×k×pr
Simplifyingtheectorvcrossproductsewsimplifythiswndoto
322ei(kr−ωt)ˆ
E=−2VkRrsinθθ
wNotheerallvoradiatederpwopersolidangleisengivyb
dP1(2ˆ∗)
dΩ=2Rerr·E×H
Subsistingthealuesvofelectricifeldandmagneticifeldinthisexpressionewget
({i(kr−ωt)}{22i(kr−ωt)})
dP12ˆ322eˆ13VkReˆ
dΩ=2Rerr·−2VkRrsinθθ×−µ22rsinθϕ
0
ˆˆˆ
Sinceθ×ϕ=r,theeabvoexpressionsimpliifesto
dP9V2k4R4
=sin2θ
dΩ8µ0c
CHAPTER11.CLASSICALYNAMICSODELECTRII259
Thetotalradiatederpwoisusththetegralinoftheeabvoexpressionervothetotalsolidangleinthe
tireensphericalshell
IdPI9V2k4R42
P=dΩdΩ=8µcsinθdΩ
0
tegralInofoftheytitquansin2θervothetotalsolidangleisjust8usthgivingustheifnalexpression
3
3πV2K4R4
P=µc
0
Thisesgivthetotalradiateder.pwo□
Chapter12
GeneralyRelativit
12.1orkHomewOne
n
12.1.1.(GeometrizedUnits)Expressheacofthepwingollotitiesquaninowtys:awi)inm,asmetersraised
tosomeappropriateer,pwoandii)inkgnaskilogramsraisedtotheappropriateer.pwo
(a)Thetummomenofanelectronvingmoat0.8c.
Solution:
Thegammafactorγis
γ=√1=√1=1.67
222
1−v/c1−.8
−31
Themassofelectronism=1.21×10kg.Sothtummomenis
e
3130
−−
p=mvγ=9.1×10·0.8·1.67=1.21×10kg
27
Sincetheersionvconfactoris1m=1.35×10kgewget
−30(27)−1−58
p=1.21×101.35×10=8.96×10m
Thesearetherequiredaluesvoftummomeninheacunit.□
(b)Theageoferseuniv(13.8)Gy
Solution:
Theage(A)insecondsis
917
A=13.8×10·365·24·60·60=4.35×10s
Theersionvconfactoris1s=3×108msoewget
17826
A=4.35×10·3×10=1.3×10m
27
Sincetheersionvconfactoris1m=1.35×10kgewget
262753
A=1.3×10·1.35×10=1.74×10kg
Thesearetherequiredalues.v□
(c)Theorbitalspeedoftheearth.
Solution:
2427
ThemassofEarthisM=6×10kghwhicwiththeersionvconfactor1m=1.35×10kgbecomes
260
CHAPTER12.GENERALTIVITYRELA261
−36
M=4.45×10mandthereadiusofearth(R)isR=6.4×10mandforourunitsG=1The
orbitalspeed(v)isengivyb
v2=GM=4.45×10−3m=6.97×10−10m0
6
R6.4×10m
−50
v=2.64×10m
Sincetheorbitalspeedisdimensionless,ithastoevhasamealuevinkgunitalsoso
−50
v=2.64×10kg
Thesearetherequiredaluesvfororbitalspeedinheacunits.□
12.1.2.utzh(Sc1.3)wDratandxaxesofthespacetimecoordinatesofanerobservOandthenw:dra
(a)TheorldwlineO’sclokcatx=1m.dx
(b)Theorldwlineofaparticlevingmowitheloyvcitdt=0.1,andhwhicisatx=0.5mandwhen
t=0.
¯¯
(c)Thetandx¯axesofanerobservOwhioesvmowitheloyvcitv=0.5inthepeositivxdirection
¯
erelativtoOandwhoseoriginx¯=t=0coincideswiththatofO.
22
(d)Thelocusoftsenevwhosealtervin∆sfromoriginis−1m.
22
(e)Thelocusoftsenevwhosealtervin∆sfromoriginis+1m.
¯
(f)Thecalibrationksticatonemeteralstervinalongthex¯andtaxes.
12.1.3.utzh(Sc2.1)enGivthebumersn{A0=5,A1=0,A2=−1,A3=−6},{B=0,B=−2,B=
012
4,B=0},{C=1,C=0,C=3,C=−1,C=6,C=−2,C=−2,C=0,C=
3000103301011121321
5,C22=2,C23=−2,C20=4,C32=−1,C32=−3,C33=0},ifnd:
(a)AαBα
Solution:
α
AB=5∗0+0∗−2+−1∗4+6∗0=−4
α
□
(b)AαCαβforallβ
Solution:
forβ=0
α0123
ACα0=AC00+AC10+AC20+AC30
=5∗1+0∗5+−1∗4−6∗−1=7
Similarly
AαC=0+0+−5+6=1
α1
AαC=10+0+−2+18=26
α2
AαC=15+0+3+0=18
α3
□
(c)AγCγσforallσ
Solution:
Thisissameasthepreviousonebecausetheydummindexistheonlyonet.differen□
CHAPTER12.GENERALTIVITYRELA262
(d)AνCforallµ
µν
Solution:
□
(e)AαBβforallα,β
(f)AiB
i
(g)AjBkforallj,k
¯
12.1.4.utzh(Sc2.14)ThewingfollomatrixesgivatsLorentransformationfromOtoO:
1.25000.75
0100

0010
0.75001.25
¯
(a)WhatistheeloyvcitofOerelativtoO?
(b)Whatistheersevinmatrixtotheengivone?
(c)FindthecomptsoneninOofaectorvA→(1,2,0,0).
12.1.5.utzh(Sc2.22)
(a)Findthe,energyrestmassandyeloitthree-vvofaparticlewhosefourtusmomenhasthecompo-
tsnen(0,1,1,0)kg.
(b)Thecollisionofowtparticlesoftumfour-momen
p−→(3,−1,0,0)kg,p−→(2,1,1,0)kg
1O2O
resultsinthedestructionfotheowtparticleandtheproductionfothreenewones,owtofhwhic
evhatafour-memen
p−→(1,1,0,0)kg,p−→(1,−1/2,0,0)kg
3O4O
Findthetum,four-meomen,energyrestmassandthreeeloyvcitofthethirdparticleproduced.
FindtheCMframe’s.eloythree-vcit
12.1.6.utzh(Sc2.30)Theeloyfour-vcitofaetrokcshipisU−→(2,1,1,1).Ittersencounaeloyhigh-vcit
O
−27
cosmicyrawhosetummemenisP−→(300,299,0,0)×10kg.Computetheenergyofthecosmicyra
O
asmeasuredybtheetrokcship’spassengers,usingheacoftheowtwingfollomethods.
(a)FindthetzLorentransformationfromOtotheMCRFoftheetrokcship,anduseittotranform
thecomponetnsofP.
(b)Useeq2.35
(c)hWhicmethodiser?kquicy?Wh
12.2orkHomewowT
12.2.1.AparticleinwskioMinkspaceelsvtraalongatrajectory:
x(τ)=ατ2
y(τ)=τ
z(τ)=0
(a)Whataretheespacelikcomptsonenofthe,eloy4-vcitUi?
Solution:
Theespacelikcomptsonenoffoureloyvcitis
∂xi
Ui==(2ατ,1,0)
∂τ
CHAPTER12.GENERALTIVITYRELA263
□
(b)UsingtherelationU·U=−1,computeU0.
Solution:
TheinnerproductofthefoureloyvcitectorvUµ=(U0U1U2U3)is
U·U=−(U0)2+(U1)2+(U2)2+(U3)2=−1
=⇒−(U0)2+4α2τ2+1+0=−1
0√2
=⇒U=±2+(2ατ)
Thisistheetimelikcomptonenofeloyvcitfourector.v□
(c)Whatistheeloy3-vcitoftheparticleasafunctionofτ?
Solution:
Theespacelikcomptsonenareengivyb
Vi=Ui=(√2ατ,√1,0)
U02+(2ατ)22+(2ατ)2
□
()
12.2.2.utzh(Sc3.24)eGivthecomptsonenof2tensorMαβasthematrix
0

0100

1−102


2001
10−20
ifnd:
(a)thecomptsonenofsymmetrictensorM(αβ)andtisymmetricantensorM[αβ]
Solution:
Thesymmetrictensorcanbewrittenas
(αβ)1(αββα)
M=2M+M
Whentheindicesarehedswitcthetselemenofthetensorare
0121
0−100


000−2
0210
Usingthisewgetthesymmetricform
0111/2
(αβ)1−101
M=

100−1/2
1/21−1/20
Similarlythetiansymmetrictensoris

00−1−1/2
[αβ]0001
M=
1003/2
1/2−1−3/20
Thesearetherequiredmatrices.□
CHAPTER12.GENERALTIVITYRELA264
(b)thecomptsonenofMαβ
Solution:
Thiscanbewrittenwiththemetrictensoras
0100

αασ−1−102
Mβ=gσβM=

−2001
−10−20
□
(c)thecomptsonenofMβ
α
Solution:
Thiscanbewrittenwiththmetricas

0−100

βσβ1−102
M=gM=
αασ
2001
10−20
□
(d)thecomptsonenofM
αβ
Solution:
Theprevioustensorcanbeusedtocalculatethis

0−10

σ−1−102
M=gM=
αβσβα
−2001
−10−20
□
12.2.3.utzh(Sc3.30)InsomeO,theectorvUandDevhathecomptsonen
U→(1+t2,t2,√2t,0)
D→(x,5tx,√2t,0)
andthescalarρhasthealuev
ρ=x2+t2−y2
(a)FindU·U,U·D,D·D.IsUsuitableaseloyfour-vcitifeld?IsD?
Solution:√
ThecomptsonenofUareU=(−(1+t2),t2,2t,0)andthecomptsonenofDareDu=
√µµµm
(−x,5tx,2t,0)sothedotproductsare
U·U=UµUu=(−(1+t2)2+t4+2t2+0)=−1−2t2−t4+t4+2t2=−1
m
µ2222222
D·D=DD=(−x+25tx+2t+0)=x(25t−1)+2t
µ
U·D=UµD=−x(1+t2)+5t3x+2t2=x(5t3−t2−1)+2t2
µ
SincetheinnerproductofUwithitselfis−1itsissuitableforafoureloyvcitwhileDisnot
(exceptpossiblyforifxedaluesvofxandt).□
CHAPTER12.GENERALTIVITYRELA265
(b)Findthespatialeloyvcitvofaparticlewhoseeloyfour-vcitisU„forarbitraryt.Whathappens
toitinthelimitst→0andt→∞?
Solution:
i(2√)
vi=U=t,2t,0
U01+t21+t2
Inthelimitt→∞ewgetv=(1,0,0)andinthelimitt→0ewgetv=(0,0,0)□
(c)FindUforallα
α
Solution:√
WiththewskioMinkmetricthealuesvofUisU=(−(1+t)2,t2,2t,0)□
αα
(d)FindUα,βforallα,β
Solution:
Thealesvare
α2t000
α∂U2t000

U,β==√
∂xβ2000
0000
□
(e)wShothatUα=0forallβ.wShothatUαU=0forallβ.
α,βα,β
Solution:
orFariousvaluesvofβUαUαis
,β
β=0::UUα=∂(−(1+t2)2+t4+2t)=−2(1+t2)·2t+4t3+4t=0
α,0∂t
α∂(224)
β=1::UU=−(1+t)+t+2t=0
α,1∂x
β=2::UUα=∂(−(1+t2)2+t4+2t)=0
α,2∂y
α∂(224)
β=3::UU=−(1+t)+t+2t=0
α,3∂z
eWevhaUαUistheinnerproductofU·UandsoU·U=UαU=UUαsotheexpression
ααα
UαU=(UUα)=0,∀β
α,βα,β
□
(f)FindDβ,β
Solution:
ItissimplytheergencedivofectorvDsoewget
β∂x∂5tx∂√2t∂0
D,β=∂t+∂x+∂y+∂z=5t
□
(g)Find(UαDβ),βforallα.
Solution:
CHAPTER12.GENERALTIVITYRELA266
ThecomptsonenoftensorUαDβare
22√2
(1+t)x5tx(1+t)2t(1+t)0
23√3
αβtx5tx2t0
UD=√√22

2tx52tx2t0
0000
wNotheesativderiv(UαDβ),βhasthecomptsonen
α=0:2tx+5t(1+t2)+0+0=2tx+5t(1+t2)
33
α−1:2tx+5t+0+0=2tx+5t
√√2√√2
α=2:2x+52t+0+0=2x+52t
α=3:0
Sothecomptsonenare(UαDβ),β=(2tx+5t(1+t2),2tx+5t3,√2x+5√2t2).□
(h)FindU(UαDβ)andcompareresult.
α,β
Solution:√
eWevhathecomptsonenofUα=(−(1+t2),t2,2t,0)andewevhaobtained
ααβ23√√2
M=(UD)=(2tx+5t(1+t),2tx+5t,2x+52t)
,β
U(UαDβ)=UMα
α,βα√√√
22232
=(−(1+t)(2tx+5t(1+t))+t(2tx+5t)+2t(2xt+52t))
=−5t
eWseethatthisisequalto−DβandusingthefactthatUUα=−1ewcanrewrite
,βα
U(UαDβ)=−Dβ=(UUα)Dβ
α,β,βα,β
Thiswsshothattheassoeciativpropyertintensorshold.□
(i)Findρforallα.Findρ,αforallα
,α
Solution:
Thecomptsonenare
ρ=(∂ρ,∂ρ,∂ρ,∂ρ)=(2t,2x,−2y,0)
,α∂t∂x∂y∂z
Theraisedersionvis
,β
ρ=(−2t,2x,−2y,0)
□
µµβ
12.2.4.utsh(Sc4.17)eWevhadeifneda=U,βU.Gotothenon-relativisticlimitandwshothat
iii
a=v˙+(v·∇)v
Solution:
ritingWoutthecomptsonenoftheeabvoexpressionewget
∂Uµ∂Ui∂Ui∂Ui
µ0123
a=∂x0U+∂x1U+∂x2U+∂x3U
CHAPTER12.GENERALTIVITYRELA267
Thespatialcomptsonenare
∂Ui∂Ui∂Ui∂Ui
ai=U0+U1+U2+U3
∂x0∂x1∂x2∂x3
InthenonrelativisticlimitU0=1andUi=viwhereviisthecomptonenofeloyvcitsoewobtain
∂vi∂vi∂vi∂vi
ixyz
a=∂t+∂xv+∂yv+∂zv
Thisexpressioncanberearrangedtoin
iixyzˆ(∂∂ˆ∂)i
ˆˆˆˆ
a=v˙+(vi+vj+vk)·i∂x+j∂y+k∂zv
Sincethenablaoperatoristhemiddletermineabvoexpressionewget
iii
a=v˙+(v·∇)v
Thisistherequiredexpression.□
12.2.5.Considera,stationaryideallfuidoftheform:

ρ000
µν0P00
T=
00P0
000P
orFthet,momenouyshouldassumethatthestress-energytensoristconstanintimeandthroughout
space
(a)ComputethestressenergytensorTµ¯ν¯inaframevingmoataspeed,vwithrespecttothframe
alongthex-axis.
Solution:
Thetransformationmatrixis

γγβ00

µ¯γβγ00
Λ
µ0010
0000
Thecomptsonenofthetransformedtensorare
µ¯ν¯µ¯[ν¯µν]
T=ΛΛT
µ[ν]
µ¯ν¯µ0ν¯µ1ν¯µ2ν¯µ3
=ΛΛT+ΛT+ΛT+ΛT
µ0123
SincetheoffdiagonaltselemenofTµνareallzerosewgetzerosforallj
µ¯ν¯µ¯[ν¯00]µ¯[ν¯11]µ¯[ν¯22]µ¯[ν¯33]
T=ΛΛT+ΛΛT+ΛΛT+ΛΛT
00112233
Soewgetthetransformedtensoras
22222222
γρ+γvPγvρ+γvP00γ(ρ+vP)γv(ρ+P)00
22222222
γvρ+γvPγvP+γρ00γv(ρ+P)γ(vρ+P)00
=
00P000P0
000P000P
Thisistherequiredtransformedtensor.□
CHAPTER12.GENERALTIVITYRELA268
(b)Supposethepressureisaifxedratiotothe.ydensitComputethestressenergytensorinthe
vingmoframefori)P=0(dust),ii)P=1/3ρ(radiation)iii)P=−ρ(cosmologicalt).constan
Solution:
forP=0ewget
22
γργvρ00
222
γvργvρ00

0000
0000
forP=1/3ρewget
22124
γ(ρ+v3ρ)γv(3ρ)00
γ2v(4ρ)γ2(v2ρ+1ρ)00
331
003ρ0
0001ρ
3
forP=−ρewget
ρ000
0−ρ00

00−ρ0
000−ρ
Thesearethetransformedtensor.□
12.3orkHomewThree
12.3.1.Inalfatspace,themetricinsphericalcoordinates,r,θ,ϕis
100
g=0r20
µν
22
00rsinθ
(a)Computeallnon-zeroChristoffelbsymolsforthissystem.
(b)ComputetheergencedivVα;α
12.3.2.Consideraectrovin2-dspace:
ˆ
v=i
startingatr=1,θ=0,andvingmoaroudntheunitcirclewithconstatnr=1,butaryingvθ.The
assumptionisthattheectorvitselfshouldnot.aryv
rite,Wandesolvatialdifferenequationdescribingthehangescinthecomptsonenofvasouyparallel-
transportitaroundtheunitcircle.
12.3.3.utzh(Sc5.14)orFthetensorwhosepolarcomptsonenareArr=r2,Arθ=rsinθ,Aθr=rcosθ,Aθθ=
tanθ,compute
µνµνανµµαν
∇βA=A,β+AΓαβ+AΓαβ
inpolarsforallpossibleindices.
12.3.4.utzh(Sc7.3)CalculatealltheChristoffelbsymolsforthemetric,
22(222)
ds=−(1+2ϕ)dt+(1−2ϕ)dx+dy+dz
,toifrstorderinϕ.Assumeϕisageneralfunctionoft,x,yandz.
CHAPTER12.GENERALTIVITYRELA269
12.3.5.Acosmicstringisatheoreticalconstructwhcihisinifnitelylong,andhasamassydensitperunitlength
λ.Thecoordinatesdescribibgthespacetimesurroundingacosmicstringare
t
µR
x=
ϕ
z
andhwhichasametric:

−1000

0100
2
00R(1−4λ)0
0001
(a)Computetheolumevt,elemendV,nearthecosmicstring.
(b)Computeallnon-zeroChristoelffbsymols.
µ
(c)Computethedistanceeenbwetowtptsoinseparatedybdx=dR,andallothercoordinatesequal
tozero.romFthat,cmptutethedistancefromthestringitselfouttotdistanceR=1
(d)Computethedistanceeenbwetowtpts,oinheacR=1fromthestringseparatedybanangledϕ
µ
(withallotherdx=0)Usingthat,whatisthetotaldistancetraersedybaparticleeringvcoa
circularorbitR=1aroundthecosmicstring?
(e)Compare(12.3.5c)and(12.3.5d)inthetextconofthenormalrelationshipeenbwetradiusand
circumference.Thatis,doesC=2πr?ifnot,whatshoulditbereplacedwith?
12.4orkHomewourF
12.4.1.utzh(Sc6.29)Inpolarcoordinates,calculatetheRiemannaturecurvtensorofthesphereofunitradius
whosemetricisg=r2,g=r2sin2θ,g=0.
θθϕϕθϕ
Solution:
Themetricforpolarcoordinateonthesurfaceofunitsphereis
[]
10
0sin2θ
Thehristoffelcbsymolsareengivyb
µ1µσ()
Γ=gg+g−g
νρ2νσ,ρρσ,ννρ,σ
Theonlynonzeroeativderivofmetriciswithrespecttoθsoewget
θ1θθ1
Γ=g(−g)=−sin2θ
ϕϕ2ϕϕ,θ2
SimilarlytheothernonzeroChristoffelbsymolsare
ϕϕcosθ
Γθϕ=Γϕθ=sinθ
AndtheRiemanntensorisengivyb
αασασαα
Rβµν=ΓΓ−ΓΓ−Γ+Γ
σµβνσνβµβµ,νβν,µ
(λσλσλλ)
R=gΓΓ−ΓΓ−Γ+Γ
αβµναλσµβνσνβµβµ,νβν,µ
CHAPTER12.GENERALTIVITYRELA270
Calculating
(ϕσϕσϕϕ)
Rϕθϕθ=gϕϕΓΓ−ΓΓ−Γ+Γ
σϕθθσθθϕθϕ,θθθ,ϕ
2(ϕσϕσϕϕ)
=sinθΓΓ−ΓΓ−Γ+Γ
σϕθθσθθϕθϕ,θθθ,ϕ
2(ϕϕϕ)
=sinθ−ΓΓ−Γ
ϕθθϕθϕ,θ
=sin2θ(−cos2θ+1)
22
sinθsinθ
=sin2θ
wNoewcanputeermthecoordinatewiththesymmetrypropyerttoobtain
222
R=−sinθR=−sinθR=sinθ
ϕθθϕθϕϕθθϕθϕ
ThesearethenonzerocomptsonenofRiemanntensor.□
12.4.2.utzh(Sc6.30)CalculatetheRiemannaturecurvtensorofthecylinder.
Solution:
Thelinetelemeninsthecylindricalcoordinatesystemis
22222
ds=dr+rdϕ+dz
Sothemetricinis
100
gµν=0r20
001
TheChristoffelbsymolsareengivyb
µ1µσ()
Γ=gg+g−g
νρ2νσ,ρρσ,ννρ,σ
Theonlynonzeroeativderivofmetriciswithrespecttoθsoewget
r1rr
Γ=g(−g)=−r
ϕϕ2ϕϕ,r
SimilarlytheothernonzeroChristoffelbsymolsare
ϕϕ1
Γrϕ=Γϕr=r
AndtheRiemanntensorisengivyb
αασασαα
Rβµν=ΓΓ−ΓΓ−Γ+Γ
σµβνσνβµβµ,νβν,µ
(λσλσλλ)
R=gΓΓ−ΓΓ−Γ+Γ
αβµναλσµβνσνβµβµ,νβν,µ
Calculating
(ϕσϕσϕϕ)
R=gΓΓ−ΓΓ−Γ+Γ
ϕrϕrϕϕσϕrrσrrϕrϕ,rrr,ϕ
2(ϕσϕσϕϕ)
=rΓΓ−ΓΓ−Γ+Γ
σϕrrσrrϕrϕ,rrr,ϕ
2(ϕϕϕ)
=r−ΓΓ−Γ
ϕrrϕrϕ,r
2(11)
=r−r2+r2
=0
CHAPTER12.GENERALTIVITYRELA271
wNoewcanputeermthecoordinatesandwithsymmetryalltherestarezerotoo.
R=0R=0R=0
ϕrrϕrϕϕrrϕrϕ
SoallthecomptsonenofRiemanntensorarezero,wingshothatthesurfaceofcylinderisalfatsurface.
□
12.4.3.Oneyawofdescribingthemetricofalfat,homogeneous,expandingerseunivis:

−10200
0a(t)00
2
00a(t)02
000a(t)
wherea(t)isafunctionoftime,onlyandthecoordinatesare
t

µx
x=
y
z
(a)ComputeallnonanishingvtermsoftheRiemannensor.T
Solution:
TheChristoffelbsymolsareengivyb
µ1µσ()
Γ=gg+g−g
νρ2νσ,ρρσ,ννρ,σ
Theonlynonzeroeativderivofmetriciswithrespecttotsoewget
t1tt
Γxx=2g(−gxx,t)=aa˙
Thesearetrueforyandzcoordinates.
t1ttt1tt
Γ=g(−g)=aa˙Γ=g(−g)=aa˙
yy2yy,tzz2zz,t
SimilarlytheothernonzeroChristoffelbsymolsare
xxa˙
Γtx=Γxt=a
Thesearealsotrueforyandz.
yya˙zza˙
Γty=Γyt=aΓtz=Γzt=a
AndtheRiemanntensorisengivyb
αασασαα
Rβµν=ΓΓ−ΓΓ−Γ+Γ
σµβνσνβµβµ,νβν,µ
(λσλσλλ)
R=gΓΓ−ΓΓ−Γ+Γ
αβµναλσµβνσνβµβµ,νβν,µ
Calculating
(xσxσxx)
R=gΓΓ−ΓΓ−Γ+Γ
xtxtxxσxttσttxtx,ttt,x
2(xσxσxx)
=aΓΓ−ΓΓ−Γ+Γ
σxttσttxtx,ttt,x
2(xxx)
=a−ΓΓ−Γ
xttxtx,t
(22)
2a˙a˙a¨
=a−2+2+
aaa
=aa¨
CHAPTER12.GENERALTIVITYRELA272
wNoewcanputeermthecoordinatewiththesymmetrypropyerttoobtain
Rxttx=−aa¨Rtxxt=−aa¨Rtxtx=aa¨
Similarlytherestofthealuesvcanbecalculatedas
22
Ryxxy=−aa˙
Therestofthemcanbeobtainedybputingermtheindexusingtheti-)symmetry(an.propyert
2
R=R=R=R=R=−aa˙
zxxzzyyzyzzyxzzxxyyx
□
(b)ComputeallanishingNon-vtermsoftheRicciensor.T
Solution:
TheraisedersionvofRiemanntensoris
Rα=gασR
βγµσβγµ
Theifrstindexnonanishingvtermis
xtt✘✿0xxyy✟✯0zz✘✿0
R=gR✘+gR+gR✟+gR✘
✘✘
ttxtttxxttxyttxzttx
✟
−2a¨
=a(−aa¨)=−a
UsingthesymmetrypropyertandthetselemenofmetricewgettherestofcomptsonenofRiemann
tensoras
Rx=−a¨Rx=a¨
ttxatxta
Ry=−a¨Ry=a¨
ttyatyta
Rz=−a¨Rz=a¨
ttzatzta
wNothecomptsonenofRiccitensorintermsoftselemenofRiemannRensorare
R=gµνRν
αβαµβ
SpeciifcallyforRttewget
0
✚❃y
tttxxxyyzzz
✚
R=gR+gR+gR+gR
ttttttxttyttzt
✚
−2−2−2
=−aaa¨−aaa¨−aaa¨
=−3a/a¨
Similarlyrestofthecomptsonencanbecalculated.Theyare
2
R=R=R=aa¨+2a˙
xxyyzz
ThesearethecomptsonenofRiccitensor□
(c)ComputeEinsteinensor.T
Solution:
ThecomptsonenofEinsteintensorareengivyb
G=R−1gR(12.1)
µνµν2µν
CHAPTER12.GENERALTIVITYRELA273
TheRicciscalarcanbecalculatedybtractingcontheRiccitensoras
2
R=Rt+Rx+Ry+Rz=6aa¨+a˙(12.2)
txyza2
wNotheEinsteintensorsimplyisthesubstitution(12.3)tointhe(12.4).Theifrstcomptonenof
thistensoris
(2)2
G=R−1gR=−3a¨+16aa¨+a˙=3a˙
tttttt22
2a2aa
Similarlytherestofthecomptsonencanbecalculated.
2
12126(aa¨+a˙)2
G=R−gR=aa¨+2a˙−a·=−2aa¨−a˙
xxxxxx2
22a
2
G=G=G=−2aa¨−a˙
xxyyzz
1
TheraisedersionvofEinsteintensorsimilarlyare.
22
tt3a˙xxyyzza˙+2aa¨
G=2G=G=G=−4
2aa
ThesearetherequiredcomptsonenofEinsteintensor.□
2
12.4.4.utzh(Sc6.35)Compute20indeptendencomptsonenofRforamanifoldwithlinetelemends=
()αβµν
2Φ22Λ22222
−edt+edr+rdθ+sinθdϕ,whereΦandΛarearbitraryfunctionsfothecoordinater
alone.
Solution:
ritingWwndothemetricfromtheengivexpressionforlinetelemen
2Φ2Λ222
g=−e;g=e;g=r;g=rsinθ
ttrrθθϕϕ
Theersevinmetricis
tt−2Φrr−2Λθθ1ϕϕ1
g=−e;g=e;g=r2;g=22
rsinθ
TheChristoffelbsymolscanbecalculatedybtheexpression
µ1µσ()
Γ=gg+g−g
νρ2νσ,ρρσ,ννρ,σ
aluatingEvthetheseewget
tt
Γ=Γ=Φ,r
rttr
r−2Λ2r−2Λ+2Φrr−2Λ
Γ=−resinθΓ=−reΦ,rΓ=Λ,rΓ=−re
ϕϕttrrθθ
θ1θθ1
Γ=sin2θΓ=Γ=
ϕϕ2θrrθr
ϕϕ1ϕϕcosθ
Γ=Γ=Γ=Γ=
ϕrrϕrθϕϕθsinθ
TheRiemanntensorisengivyb
αααασασ
R=Γ+Γ−ΓΓ−ΓΓ
βµνβµ,νβν,µσµβνσνβµ
λλλσλσ
Rαβµν=gλα(Γ+Γ−ΓΓ−ΓΓ)
βµ,νβν,µσµβνσνβµ
1
ThisaswedsolvmostlyusingCadabra.https://www.physics.drexel.edu/~pgautam/courses/PHYS631/
einstein-tensor-expanding-universe.html
CHAPTER12.GENERALTIVITYRELA274
ExplicitlyforRewget
trtr
R=gRt
trtrttrtr
2Φ[rr✟✯0rσrσ]
=−eΓ+Γ✟+ΓΓ−ΓΓ
rt,rrr,tσrrrσrrt
✟
2Φrrrr
=−e[Φ+ΓΓ−ΓΓ]
,rrrrrrrrrt
2Φ[2]
=−eΦ,rr+(Φ,r)−Φ,rΛ,r
2
Therestofthecomptsonencanbesimilarlycalculated
R=((Λ−Φ)Φ−Φ)e2Φ
trrt,r,r,r,rr
R=(−(Λ−Φ)Φ+Φ)e2Φ
trtr,r,r,r,rr
R=re−2Λ+2Φsin2θΦ
tϕtϕ,r
R=re−2Λ+2ΦΦ
tθtθ,r
−2Λ+2Φ2
R=−resinθΦ
tϕϕt,r
R=−re−2Λ+2ΦΦ
tθθt,r
2
R=rsinθΛ
rϕrϕ,r
R=rΛ
rθrθ,r
R=(−(Λ−Φ)Φ+Φ)e2Φ
rtrt,r,r,r,rr
2
R=−rsinθΛ
rϕϕr,r
R=−rΛ
rθθr,r
2Φ
Rrttr=((Λ,r−Φ,r)Φ,r−Φ,rr)e
R=−rΛ
θrrθ,r
R=rΛ
θrθr,r()
122Λ−12Λ−2Λ
R=resin(2θ)(tanθ)−2ecos(2θ)+cos(2θ)−1e
θϕθϕ2
Rθtθt=re−2Λ+2ΦΦ,r
2(2Λ)−2Λ2
R=r1−eesinθ
θϕϕθ
R=−re−2Λ+2ΦΦ
θttθ,r
2
R=−rsinθΛ
ϕrrϕ,r
2(2Λ)−2Λ2
R=r1−eesinθ
ϕθθϕ
2
R=rsinθΛ
ϕrϕr(,r)
22Λ−2Λ2
R=re−1esinθ
ϕθϕθ
R=re−2Λ+2Φsin2θΦ
ϕtϕt,r
−2Λ+2Φ2
R=−resinθΦ
ϕttϕ,r
ThesearethenonzerocomptsonenofRiemanntensor.□
12.4.5.utzh(Sc7.7)Considerthewingfollofourtdifferenmetrics,asengivybtheirlinets:elemen
22222
i.ds=−dt+dx+dy+dz;
22−22222
ii.ds=−(1−2M/r)dt+(1−2M/r)1dr+r(dθ+sinθdϕ)whereMisat.constan
2
IdidthisusingCadabra.Thedetailofthisexerciseisathttps://www.physics.drexel.edu/~pgautam/courses/PHYS631/
HW4Schutz6.35.html
CHAPTER12.GENERALTIVITYRELA275
222[2−22222]
iii.ds=−dt+R(t)(1−kr)1dr+r(dθ+sinθdϕ),wherekisatconstanandR(t)isan
arbitraryfunctionoftalone.
(a)orFheacmetricifndasymanedconservcomptsonenpofafreelyfallingparticle’sfourtummomen
α
aspossible.
Solution:
Therateofhangecoftummomenisengivyb
dp1
βνα
m=gpp
dτ2µα,β
Thetummomenpisedconservwheng=0.romFtheengivmetrictheedconservtitiesquan
βµα,β
are
fori.:pt,px,py,pz
forii.:pt,pϕ
foriii.:pϕ
□
(b)riteWi.intheform
2222(222)
ds=−dt+dr+rdθ+sinθdϕ
romFthisarguethatii.iii.aresphericallysymmetric.Doesthisincreasethebumernofedconserv
comptsonenofp?
α
Solution:
ThecoordinatetransformationfromCartesiantopolaris
x=rsinθcosϕ=⇒dx=sinθ+cosϕdr++rcosθcosϕdθ−rsinθsinϕdϕ
y=rsinθsinϕ=⇒dy=sinθ+sinϕdr++rcosθsinϕdθ+rsinθcosϕdϕ
z=rcosθ=⇒dzcosθdr=−sinθdθ
Substitutingtheseinthelinetelemenewget
22222222
dl=−dt+dr(sinθsinϕsinθcosϕ+cos)+
2(22222222)
+dθrcosθcosϕ+rcosθcosϕ+rsinθ
2(222222)
+dϕrsinθsinϕ+rsinθcosϕ
222(222)
=−dt+dr+rdθ+sinθdϕ
Thisistherequiredtransformationinsphericalform.□
πθ
(c)Itcanbewnshothatforii.andiii.ageodesicthatbeginswithθ=2andp=0-i.e.,one
πθ
hwhicbeginsttangentotheequatorialplane-ysaalwhasθ=2andp=0.orFthesecasesuse
2r
theequationp·p=−mtoesolvforpintermsofm,otheredconservtities,quanandwnkno
functionsofposition.
Solution:
Expandingtherelationp·p=−m2ewget
()()2()2
2t2r2θϕ
−m=gp+g(p)+gp+gp
ttrrθθϕϕ
θ
enGivθ=π/2andp=0ewget
√2t2ϕ2
r−m−gtt(p)+gϕϕ(p)
p=g
rr
CHAPTER12.GENERALTIVITYRELA276
tϕ
Sincepandpareedconservsubstitutingthecorrespondingmetricaluesvgαβesgivtheytitquan
r
p
√21−2Mt22ϕ2
forii.;pr=−m+r(p)+r(p)
√1−2M/r
r1−kr2(2t22ϕ2)
foriii.;p=R2(t)−m+(p)+(R(t)r)(p)
r
Thesearetherequiredexpressionforpintermsofedconservtities.quan□
θϕ
(d)orFiii.,sphericalsymmetryimpliesthatifageodesicbeginswithp=p=0,theseremainzero.
Usethistowshothatwhenk=0,pisaedconserv.ytitquan
r
Solution:
Therateofhangecoftummomenisengivyb
dp1
βνα
m=gpp
dτ2µα,β
dp1()
rt2r2θ2ϕ2
m=g(p)+g(p)+g(p)+g(p)
dτ2tt,rrr,rθθ,rϕϕ
θϕ
Butfork=0,g=0andg=0andengivp=p=0ewget
rr,rtt,r
mdpr=0
dτ
Thisesvprothatpisaedconserv.ytitquan□
r
12.4.6.Whatfractionalenergydoesaphotonloseifitgoesfromthesurfaceoftheearthtodeepspace?
Solution:
Whenthephotongoesfromthesurfaceofearthtoouterspace,itustmlosethevitationalgraptialoten
energythatishasnearthesurfaceofearth.Sothephotonustmlosethis.energyorFphoton
02
(U)g00=−1
Onsurfaceofearthwitheakwifeldlimit
g=−(1−2ϕ)
00
Sonearthesurfaceofearth
U0≃1+ϕ
InfarspacemetricwskioMinkg00=−1soinfarspace
U0=1
Soratioofenergy
1=1−ϕ
1+ϕ
Sohangecinenergyis∼ϕOnthesurfaceofearththevitationalgraptialotenis
−1124
ϕ=−GM=−6.672×10×6.0×10≈7×10−10
c2r6.4×106×9×1016
Sothephotonustmlosethisenergy.fractionally□
CHAPTER12.GENERALTIVITYRELA277
12.5orkHomeweFiv
12.5.1.Considera1+1dimensionalspace(t,x)withthemetric:
(kx)
gµν=−e0
01
wherekisadimensionalt.constan
(a)Thismetrichasastress-energysourcehwhicis(ptially)otennon-zero.wingKnonothingelse,what
isthescalingoftheydensitρintermsofk?
Solution:
Sincetheexptoneninthemetrichastobedimensionlessthedimensionofkis
[k]=[L]=[M]
Thedimensionofydensitis
[ρ]=[M]=[L]2
3
[L]
romFtheseowtexpressions
ρ∼k2
So,intermsofdimensiononlytheydensithastoscaleasthesquareofk.□
(b)Computeallnon-zeroChristoffelbsymols.
Solution:
Thenonzeroeativderivofthemetricisintermsofxonlyandtheonlynonzeroeativderivis
g=−kekx
tt,x
TheChristoffelbsymolsareengivyb
µ1µσ()
Γ=gg+g−g
νρ2νσ,ρρσ,ννρ,σ
ThenonzeroChristoffelbsymolsare
x1xx1kx1kx
Γ=g(−g)=·(−1)·−ke=ke
tt2tt,x22
Theotherare
tt1
Γ=Γ=k
txxt2
ThesearetherequirednonzeroChristoffelbsymols.□
(c)Aemassivparticleistaneouslyinstanatrestatex=0.Whatisthetaneousinstanacceleration
oftheparticle?
Solution:
Thegeodesicequationcanbeusedtocalculatetheaccelerationoftheparticle.romFthegeodesic
equationewevha
∂Uµ
µαβ
=−ΓUU
∂ταβ
orFparticleatrestvi=0,=⇒Ui=0.UsingU·U=−1ewget
020kx/2
(U)g00=−1U=e
CHAPTER12.GENERALTIVITYRELA278
tx
SincetheonlynonzeroChristoffelbsymolsareΓandΓewget
txtt
∂Ux1
x00kxkx
=−ΓUU=−kee
∂τtt2
tAtheoriginusthx=0ewget
∂Ux1
∂τ=−2k
Thisesgivtheaccelerationoftheparticle.□
(d)Computethenon-zerocomptsonenoftheRiemanntensor.
Solution:
AndtheRiemanntensorisengivyb
αασασαα
Rβµν=ΓΓ−ΓΓ−Γ+Γ
σµβνσνβµβµ,νβν,µ
calculating
xxσxσxx
Rtxt=ΓΓ−ΓΓ−Γ+Γ
σxttσttxtx,ttt,x
xtx
=−ΓΓ+Γ
tttxtt,x
=−1k·1kekx+1k2ekx
222
=1k2ekx
4
SimilarlytheothercomptonenofRiemanntensorare
Rt=−1k2
xtx4
Theothercomptsonenaresimplythecyclicputationermoftheindices.□
(e)Whatarethenon-zerotermsintheRicciensorTandRicciScalar?
Solution:
ThecomptsonenofRiccitensorintermsoftselemenofRiemannensorTare
R=gµνRν
αβαµβ
SpeciifcallyforRttewget
✚❃0
tttxxx
✚
Rtt=gRttt+gRtxt
✚
=1k2ekx
4
SimilarlytheothercomptonenRxxis
✘✿0
tttxxx✘
✘
✘
R=gR+gR
xxxtx✘xtx
−kx12kx
=−e·4ke
=−1k2
4
TheRicciscalarcanbecalculatedybtractingcontheRiccitensoras
R=Rt+Rx=gttR+gxxR=−1k2−1k2=−1k2(12.3)
txttxx442
SotheRiccisscalaris−1/2k2.□
CHAPTER12.GENERALTIVITYRELA279
(f)WhatistheEinsteintensor?
Solution:
ThecomptsonenofEinsteintensorareengivyb
G=R−1gR(12.4)
µνµν2µν
Theifrstcomptonenofthistensoris
112kx1−kx12
G=R−gR=ke+e·−k=0
tttt2tt422
Theothercomptonenis
G=R−1gR=−1k2−1·−1k2=0
xxxx2xx422
SotheEinsteintensoristicallyidenzero.□
12.5.2.Inthegeneralizedlinearmetricewfoundinclass:

−1−2ψ000
01−2ϕ00

001−2ϕ0
0001−2ϕ
where,foranon-relativisticallyvingmosource:
22
∇=4π(ρ+3P);∇ϕ=4πρ
supposeouyerewintheteriorinofasphericallysymmetricdistributionwithtconstanydensitandifxed
equationofsatew=−1
3
(a)Whatistheaccelerationonatestparticleplacesadistancerfromthetercenofthecloud.ouldW
itfallardwinorard?wout
Solution:
Sinceψandϕarefunctionsofronlyewevhanonzeroeativderivofthecomptsonenofmetric
onlywithrespecttor.TheChristoffelbsymolsareengivyb
µ1µσ()
Γ=gg+g−g
νρ2νσ,ρρσ,ννρ,σ
ThenonzeroChristoffelbsymolsare
tt1tt(0)
✟✯✘✿0
Γ=Γ=gg+g✟−g✘
trrttt,rrt,ttr,r
2✟✘
=1−1(−2ψ)
21+2ψ,r
ψ
=,r
1+2ψ
SimilarlytheothernonzeroChristoffelbsymolsare
ψϕ
r,rr,r
Γtt=1−2ϕΓrr=−1−2ϕ
orFastationaryparticlevi=0=⇒Ui=0.UsingU·U=−1ewget
020√
(U)g=−1=⇒U=1+2ψ
00
CHAPTER12.GENERALTIVITYRELA280
Thegeodesicequationcanbeusedtocalculatetheaccelerationoftheparticle.Thegeodesic
equationis
∂Uµ
µαβ
=−ΓUU
∂ταβ
Thespatialaccelerationoftheparticleis
∂Urψ
=ΓrU0U0=,r(1+2ψ)
∂τtt1−2ϕ
TheytitquanψcanbecalculatedybusingthefactthattheLaplacianofψisen.givIna
,r
2∂2
sphericallysymmetricsystem∇≡∂r2,soewget
∂2ψ
∂r2=4π(ρ+3P)
tegratingInoncewithrespecttorewget
dψ(P)
=ψ=4πρ1+3r
dr,rρ
Subsistingthisintheexpressionforaccelerationewget
∂Urψ1+2ψ
=,r(1+2ψ)=4πρ(1+3w)r·
∂τ1−2ϕ1−2ϕ
enGivthatw=−1ewget
3
∂Ur1+2ψ
∂τ=0·1−2ϕ=0
ii
Sotheradialaccelerationoftheparticleiszero.Sincefori̸=r,U=0andΓtt=0allother
spatialcomptsonenofaccelerationiszero.Sothespatialaccelerationoftheparticleisticallyiden
zero.□
(b)Whatistheaccelerationonaphotonelingvtraperpendiculartothecloudalsoadistancerfrom
theter.cenouldWitbelensedardwinorard?wout
Solution:
Sincethephotonistrailingperpendiculartothecloud(inatstraighline),ewcanassume(without
lossofy)generalittheradialanduthalazimcomptsonenoftheeloyvcitarezero,ybhocosingthe
directionofelvtrasameastheradialcoordinate.So,Uθ=0,Uϕ=0Thespatialaccelerationof
thephotonis
∂Ur
r00rrr
∂τ=ΓttUU+ΓrrUU
SubsistingtheChristoffelbsymolsewget
∂Urψϕ
=,r(U0)2−,(Ur)2
∂τ1−2ϕ1−2ϕ
Againybtsargumenofpreviousproblemψ=0,soewget
,r
∂Urϕ,r
=−(Ur)2
∂τ1−2ϕ
CHAPTER12.GENERALTIVITYRELA281
2∂2
Again,inasphericallysymmetricsystem∇≡2,ewcansimilarlyobtainϕandϕybte-in
∂r,r
gratingtheLaplacianofϕwithrespecttoronceandwicet.respelyectiv
ϕ=4πρrϕ=2πρr2
,r
Sinceϕ≪1,(Ur)2>0andϕ,r=4πρr>0theifnalexpressionfortheaccelerationwillturnoutto
bee.negativusThtheaccelerationouldwbeardwinandhencethephotonwillbelensedard.win□
12.5.3.utzh(Sc8.17)
11
(a)Asmallplanetorbitsastaticneutronstarinacircularorbitwhosepropercircumferenceis6×10
m.Theorbitalperiodestakys200daoftheplanet’spropertime.EstimatethemassMofthe
star.
Solution:
orFthepurposeofestimationewcanassumethatNewton’swslaholdandthatthetimedilation
effectisnegligible.Inthatlimitthepropertimeisjustthetimemeasuredyber.observromF
Kepler’sthirdwlaewevha
t2=(4π2)r3
GM
Ifcisthecircumference,itisengivintermsofradius,ybc=2πrsubsistingcewget
c31c3
t2==⇒M=
2πGM2πGT2
Sofortheengivplanet
711
t≈τ=200days=1.728×10sc=6×10m
Sothemassisengivyb
113
1(6×10)30
M=−1172=1.726×10kg
2π·6.672×10(1.728×10)
30
Sothemassoftheneutronstaris1.726×10kg.□
(b)eFivsatellitesareplacedtoinacircularorbitaroundastatickblachole.Thepropercircumferences
andproperperiodsoftheirorbitsareengivinatablebw.eloUsethemethodof12.5.3atoestimate
thehole’smass.Explaintheresultsouygetforthesatellites
66789
circumference2.5×10m6.3×10m6.3×103.1×10m6.3×10m
−33
properperiod8.4×10s0.055s2.1s23s2.1×10s
Solution:
Usingthemethodof12.5.3aewget
(3)
c(m)t(s)1ckg
2
2πGt
2.5×1068.4e-35.28×1032
632
6.3×100.0551.97×10
732
6.3×102.11.35×10
832
3.1×10231.34×10
9332
6.3×102.1×101.35×10
32
Theobtainedaluevforthemassseemtobeergingvconardswto1.35×10kgwiththeesuccessiv
increaseintheorbitalcircumference.Sofurtheryawathesatellite,theNewtonianximationappro
aremorecorrect.□
CHAPTER12.GENERALTIVITYRELA282
12.5.4.utzh(Sc8.18)Considertheifeldequationwithcosmologicalt.constanWithΛarbitraryandk=8π.
(a)FindtheNewtonianlimitandwshothatewervrecothemotionoftheplanetsonlyif12|Λ|iseryv
small.enGivtheradiusofPluto’sorbitis5.9×10m,setanupperboundon|Λ|fromsolar-
systemtsmeasuremen
Solution:
Theifeldequationis
G+Λg=8πT
µνµνµν
Newtonianequationofmotionisengivyb
∇2ϕ=4πρ
speciifcallytheifrstcomptonenofifeldequation
G=8πT−Λg
000000
IntheNewtonianlimit,sinceT=ρandg=−1,Iouldwexpectinifeldequationterm
0000
ρ→ρ+Λ
8π
IamassumingthelimittoΛcomesfromtheummaximestimationofthemassydensitρinthe
solarsystem.enEvifthespacewreyemptandonlycosmologicaltconstanerewtpresenofthat
Λ×π
alue,vewouldwgettheorbitalradiusofPluto.Soummaximaluev<ρ8.Themeasured
ydensitofthesolarsystemisintheorder
−22g−19kg
∼1.3×10=1.3×103
ccm
andsoummaximΛshouldbethesameorder.□
(b)BybringingΛervoththeRHSofutzhSceq8.7ewcanregard−Λgµν/8πasthestress-energy
tensorofy‘empt.space’enGivthatheedobservmassoftheregionoftheerseunivnearourGalaxy
ouldwevhaaydensitofabout1×10−27kgm3ifiterewuniformlydistributed,doouythinkthat
aaluevof|Λ|nearthelimitouyestablishedin12.5.4acouldevhaableobservconsequencesfor
cosmology?erselyvConifΛiscomparabletothemassydensitoftheerse,univdoewneedto
includeitintheequationswhenewdiscussthesolarsystem?
Solution:
IfΛisisintheorderaspredictedin12.5.4a,andtheydensitofgalaxyisintheorderof1×
−273
10kg/mthen
Λ≫ρgalaxy
Inthatcaseρ→ρ+ΛouldwbedominatedybΛ,soewouldwevhatoableobserveffectofthe
8π
cosmologicalt.constan
IfthealuevofΛiscomparabletotheydensitoftheerse,univthenIouldwstillassumethatew
ouldwneedtoincludeinthecalculationofsolarsystem.□
12.5.5.utsh(Sc10.9)
(a)DeifneanewradialcoordinateintermsofthehildarzscwhScryb
r=r¯(1+M)2.
2r¯
CHAPTER12.GENERALTIVITYRELA283
Noticethatasr→∞,r¯→r,whilethetenevhorizonr=2M,whereewevhar¯=1M.wSho
2
thatthemetricforsphericalsymmetryestaktheform
[][]
1−2M/r¯2M4[]
22222
ds=−1+M/r¯dt+1+2r¯dr¯+r¯dΩ
Solution:
ThehildarzscwhScmetricis
−(1−2M/r)000
01/(1−2M/r)00
g=
µν00r20
22
000rsinθ
Thetransformationofthetselemenofmetriccanbeobtainedyb
µν
gµ¯ν¯=Λµ¯Λgµν
ν¯
µ
wherethetselemenofthetransformationmatrixΛareengivyb
µ¯
∂xµ
Λµ=
µ¯∂xµ¯
Sincetheengivmetricisdiagonal,theonlynonzeroterminthemetricgµνarewithµ=ν.
Expandingthemetrictransformationexplicitlyasasum
g=ΛµΛµg
µ¯ν¯µ¯ν¯µµ
enGivthetransformationr→r¯(1+M/2r¯)2andallothercoordinatesarehangeduncewget
(())
∂r∂M2
Λr==r¯1+
r¯∂r¯∂r¯2r¯
()()()
M2MM
=1++2r¯1+−2
2r¯2r¯2r¯
=(1+M)(1+M−M)
(2r¯)(2r¯)r¯
=1+M1−M
2r¯2r¯
¯¯¯
forallothercoordinatest=t,ϕ=ϕ,θ=θsoewget
Λθ=Λϕ=Λt=1
¯¯¯
θϕt
µ
Λ=0ifµ̸=ν
ν¯
usThexpandingthetransformationofthemetricexplicitlyewget
tt(2M)
g¯¯=ΛΛg=g=−1−
tt¯¯tttt
ttr
Undertheengivtransformationewevha
2M2Mr¯(1−M/2r¯)2−2M(1−M)2
1−=1−==(2r¯)(12.5)
r22M2
r¯(1+M/2r¯)r¯(1+M/2r¯)1+2r¯
CHAPTER12.GENERALTIVITYRELA284
Sounderthetransformedcoordinatesystemewget
(1−M/2r¯)2
g¯¯=−
tt(1+M/2r¯)2
Thenexecomptonenofthemetricis
[()()]()
rrMM22M−1
g=ΛΛg=1+1−1−
r¯r¯r¯r¯rr2r¯2r¯r
Using(12.5)ewgetinthisexpressionewget
gr¯r¯=[(1+M)(1−M)]2(1+M/2r¯)2
2r¯2r¯(1−M/2r¯)2
()
M4
=1+2r¯
Thenextcomptonenofthetransformedmetricis
θθ22(M)4
g¯¯=ΛΛg=g=r=r¯1+
θθ¯¯θθθθ
θθ2r¯
Theifnalnonzerocomptonenis
ϕϕ222(M)42
g¯¯=ΛΛg=g=rsinθ=r¯1+sinθ
ϕϕ¯¯ϕϕϕϕ
ϕϕ2r¯
usThtheifnaltransformedmetricis
−(1−M/2r¯)2000
(1+M/2r¯)2
0(1+M/2r¯)400
gµ¯ν¯=24
00r¯(1+M/2r¯)2042
000r¯(1+M/2r¯)sinθ
Thelinetelemeninthismetricisengivyb
[][]
1−2M/r¯2M4[]
22222222
ds=−1+M/r¯dt+1+2r¯dr¯+r¯dθ+r¯sinθdϕ(12.6)
hWhicistherequiredexpression.□
(b)Deifneaquasi-Cartesiancoordinatesybtheusualequationsx=r¯cosϕsinθ,y=r¯sinϕsinθ,
222222
andz=r¯cosθsothat,dr¯+r¯dΩ=dx+dy+dzusThthemetrichasbeenertedvcontoin
coordinates(x,y,z),hwhicarecalledisotropiccoordinates.wNoetakthelimitasr¯→∞and
wsho
2[2M(1)]2[2M(1)](222)
ds=−1−+O2dt+1++O2dx+dy+dz
r¯r¯r¯r¯
Solution:
Underthetransformationengiv
222222222
dr¯+r¯dθ+r¯sinθdϕ=dx+dy+dz
CHAPTER12.GENERALTIVITYRELA285
Underthelimitr¯→∞,themetrictelemeng¯¯canbesimpliifed
tt
()()
(1−M/2r¯)2M2M−2
g¯¯==−1−1+
tt(1+M/2r¯)22r¯2r¯
(M(1))(M(1))
=1−+O21−+O2
(r¯r¯(r¯))r¯
MMM21
=1−−+2+O2
r¯r¯r¯r¯
(2M(1))
=1−+O2
r¯r¯
Similarlyundertheximationapprog
rr
()()
M42M1
gr¯r¯=1+=1++O2
2r¯r¯r¯
Subsistingthisinthelinetelemenexpression(12.6)ewget
2[2M(1)]2[2M(1)](222)
ds=−1−+O2dt+1++O2dx+dy+dz
r¯r¯r¯r¯
Thisistherequiredexpression.□
(c)Computethepropercircumferenceofacircleatradiusr¯
Solution:
Thecircumferenceisengivybthetotaldistanceeledvtraybaparticlegoingatatconstandistance
r¯fromtheter,cenhwhicisthelengthofthelineunderϕ:0→2πThelinetelemenis
22
ds=g¯¯dϕ
ϕϕ
Sothetotalcircumferenceis
2π√()()
∫2M4M2
C=r¯1+2r¯dϕ=2πr¯1+2r¯
0
hWhicisthepropercircumference.□
(d)Computetheproperdistanceinelingvtrafromr¯tor¯+dr¯.
Solution:
Thelinetelemenis
22
ds=gdr¯
r¯r¯
Thelengthgoingfromr¯→r¯+dr¯is
√
()()
√M4M2
ds=gdr¯=1+dr¯=1+dr¯
rr
¯¯2r¯2r¯
Thisesgivthedistancegoingfromr¯→r¯+dr¯.□
CHAPTER12.GENERALTIVITYRELA286
12.6orkHomewSix
12.6.1.utzh(Sc11.7)Aclokcisinacircularorbitatr=10MinahildarzscwhScmetric.
(a)wHohucmtimeelapsesontheclokcduringoneorbit?
Solution:
22
Thepropertimeandthealtervinarerelatedybtheexpressiondτ=ds.orFcircularorbit
dr=dϕ=0soewget
22ϕϕ22ϕ
dτ=ds=g(U)dϕ=⇒dτ=Udϕ
ϕ
˜
Butforcircularorbittheytitquanp=mLusthewobtain
ϕ
1p1
ϕϕϕϕϕ˜
U=mp=gm=r2L
˜2Mr
TheytitquanL=1−3M/rsubstitutingtheseewget
Uϕ=1√Mr
r21−3M
r
Thetimeelapsedisengivyb
τ2π2π√
τ=∫dτ=∫1dϕ=∫r4(1−3M/r)dϕ
UϕMr
000
Notingthat,thetegrandinisindeptendenofϕ,forcircularorbitatr=10Mewobtain
√1000M3(3M)√
τ=2πM1−10M=2π107M
Thisisthetimeelapsedintheclok.c□
(b)Itsendsoutasignaltotatdistanerobservonceheacorbit.Whattimealtervindoesthetdistan
erobservmeasureeenbwetreceivingyanowtsignals?
Solution:
ThetimeelapsedforatdistanerobservisthecoordinatetimeforthehildarzscwhScmetric.If
itsendssignaleryevorbit,thetimeelapsedfortdistanerobservisthecoordinatetimeforone
fullorbit.oTifndthecoordinatetimeewevhatogetexpressionfordt=f(x)dϕ,wheretisthe
coordinatetime.romFthedeifnitionoftheϕcomptonenoffoureloyvcit
dϕpϕp1
ϕϕϕϕϕϕ˜˜
dτ=U=m=gm=gL=r2L
Similarlyfromthe0thcomptonenoffoureloyvcitewget
0˜
dtppE
000000˜
dτ=U=m=gm=g(−E)=1−2M/r(12.7)
biningComtheseowtewget
()
dtdt/dτr31/2
dϕ=dϕ/dτ=M
wNothatewevhaobtainedthefunctionalformconnectingthecoordinatetimeanduthalazim
angle.eWcantegrateintoifnd
(3)1
r2
t=2πM
CHAPTER12.GENERALTIVITYRELA287
orFr=10Mewobtain
√r3√1000M3√
t=2πM=2πM=2π1010M.(12.8)
Thisisthecoordinatetimethatpassesforoneorbithwhicisthetimemeasuredybthetdistan
erobservandisalsothetimeitelapsesfortdistanerobservforacompleteolution.rev□
(c)Asecondclokcislocatedatrestatr=10nexttotheorbitoftheifrstclok.cwHohucmtime
elapsesoniteenbwetesuccessivpassesoftheorbitingclok?c
Solution:
Thetimeisdilatedintheorbitingclokcybthetimedilationfactorhwhicissimply
dt√√
=−1/g=1−2M/r
dτ00
wNothepropertimeisengivyb
√2M
τ=1−rt
Substitutingthecoordinatetimeexpressionform(12.8)ewget
√2M√r3
τ=1−r2πM(12.9)
Substitutingr=10Mewobtain
√
τ=2π8M.
Thisesgivthetimeelapsedinthestationaryclokcastheclokcesmakoneorbit.□
(d)Calculate(12.6.1b)againinsecondsforanorbitatr=6MwhereM=14M⊙.Thisisthe
umminimlfuctuationtimeewexpectintheyX-raspectrumofCygX-1:y?wh
Solution:
orFr=6Msubstitutingr=6Min(12.8)ewget
√√
t=2π216M=12π6·14M
⊙
303
ThemassofsunM=1.9×10kg=1.476×10m.Substationthese
⊙
√3
12π6·1.476×10−3
t=3×108=0.00636s=6.36×10s
thisisthetimeelapsed.□
(e)Iftheorbiting‘clok’cisthewintArtemis,intheobitin(12.6.1d),whohucmdoessheageduring
thetimeherwintDianaeslivears40yfarfromthekblacholeandatrestwithrespecttoit?
Solution:
eWalreadyevhaforacircularorbitfrom(12.7)ewevha
˜
dt=E
dτ1−2M
r
orFastableorbitinthehildarzscwhScmetricewevha
˜1−2M/r
E=√
1−3M/r
CHAPTER12.GENERALTIVITYRELA288
Substitutingewget
dt=√1
dτ1−3M/r
Solvingthistialdifferenequationewget
∫dτ=∫√1−3Mdt
r
Settingr=6Mesgiv
τ=t√1
2
orFt=40yrewget
40
τ=√=28.28yr
2
ThisistheageofArtemiswhenherwintDianaesliv40yr.□
12.6.2.utzh(Sc11.21)Aparticleofm̸=0fallsradiallyardwtothehorizonofahildarzscwhSckblacholeof
˜
massM.ThegeodesicitwsfollohasE=0.95
(a)Findthepropertimerequiredtohreacr=2Mfromr=3M.
Solution:
eWevhaforaemassivobjecttheradialmotionnearthehildarzscwhScmetricsatisifes:
()()
dr22M
˜2
dτ=E−1−r
Thepropertimeisthenengivyb
τ=∫√˜2dr2M(12.10)
E−1+r
˜2
Makingsubstitutingα=E−1ewgetthewingfollotegralin∫
τ=√dr
α+2M
r
Thetegralinis
(√√√)
√2αr2M
2Masinh√3
2Mr2Mr2
τ=√−3+√
α2M+αrα22M+αr
3M(√)
√
√3√3√3√3√2Masinh6α
√33M2√22M2√23M2√22M22Masinh(α)2
=−+−+3−3
3Mα+2M2Mα+2Mα3Mα+2Mα2Mα+2Mα2α2
2
Substitutingα=0.95−1ewobtain
τ=1.1917M
Thisistherequiredtimeforthejourneyfrom3Mto2Mforainifllingparticle.□
CHAPTER12.GENERALTIVITYRELA289
(b)Findthepropertimerequiredtohreacr=0fromr=2M.
Solution:
Similartopreviouspartthepropertimerequiredis
(√√√)
√2αr0
2Masinh√3
2Mr2Mr2
τ=√−3+√
α2M+αrα22M+αr
√√√2M
33
√22M2√22M22Masinh(α)
=+−3
2Mα+2Mα2Mα+2Mα2
2
Substitutingα=0.95−1ewobtain
τ=1.3745M
Thisistherequiredtimeforthejourneyfrom2Mtotercenforainfallingparticle.□
(c)Find,onthehildarzscwhSccoordinatebasis,itseloyfour-vcitcomptsonenater=2.001M.
Solution:
orFradiallyvingmoobjectUϕ=Uθ=0.Theetimelikcomptonenisengivyb
˜
000˜E0.95
U=−gE=1−2M=1−2=1900.95
r2.001
Theradialcomptonencanbeobtainedybreusing(12.7)as
r2˜2(2M)r√22
(U)=E−1−r=⇒U=0.95−1+2.001=0.949
usThthefoureloyvcitis

1900.95
µ0.949
U=
0
0
Thisisthecomptonenoffoureloyvcitatr=2.001M□
(d)Asitpasses2.001M,itsendsaphotonoutradiallytoatdistanstationaryer.observCompute
theredshiftofthephotonwhenithesreactheer.observ
Solution:
Theenergyedobservybthetdistanerobservisengivyb
(0r)
Eobs=−U·p=−−U0p+Urp
eWcancalculatethecomptonenprybusingthefactthatphotonismassless.Sinceforphoton
ewevha
22
p=−m=0
Expandingthedotproductofthetummomenewget
t2r2
g(p)+g(p)=0
ttrr
Butewevhapt=Esoewcanrewritethisas
tt2r2r√gtt√1−2M
g(p)+g(p)=0=⇒p=−p=rE=E
trrgt1−2M
rrr
CHAPTER12.GENERALTIVITYRELA290
Substitutingthisintheedobservenergyexpressionewget
0r(0r)0
E=(Up−Up)=Up−Up=E(U−U)
obs0r0rr
ButU=gUr=UrsubstitutingU0=1900.95andUr=0.949ewget
rrr1−2M/r
(0.949)
E=E1900.95+=3801.95E
obs1−2
2.001
Thisesgivtheedobservenergyofthephoton.Sotheredshiftfactorissimply
z=Eobs−E=3801.95E−E=3800.95
EE
hWhicistherequiredredshiftfactor.□
12.6.3.Usingtherelationsthatewedderivinclass:
a=2M∆yanda=M∆x
hingy-stretcr3x-compressingr3
Throughoutthisproblem,assumethatouydroppedfromrestat.yinifnit
(a)Findthesmallestkblacholeinhwhicouycouldesurvivlongenoughtopassthetenevhorizon.
Solution:
Inthetenevhorizonr=2M.Theummaximaccelerationthatumanhcanesurvivisa∼9g.
max
Soewget
a=M∆x=⇒M=√∆x
max(2M)34a
max
Substituting∆x∼1mg∼10mewget
s2
√11
M=360=√s
610
27
Since1s=299792458mand1m=1.34×10kgewget
127344
M=√·299792458·1.34×10=2.12×10kg=1.07×10M
610⊙
Thisisthemostemassivkblacholeonecanesurvivnearthetenevhorizon.□
(b)orFa1Mkblachole,wholongdoesitetakeenbwetthetimeouyfeelmildlyuncomfortable(tidal
forceeenbwetheadandfeetis2⊙g)andouydie?Thisshouldbeinpropertime,ofcourse.
Solution:
Thetidalforcewillhstretcsoewevhafromtheengivhingstretcexpression
()1
2M2M∆y3
a=∆y=⇒r=
hingy-stretcr3a
orFjustbeing‘uncomfortable‘a=2gesgiv
()1
2M∆y3
r=⊙
20
30
SubstitutingM=1.98×10kgand∆y∼0.5m
⊙
921/3
r=4.56×10(skg)
CHAPTER12.GENERALTIVITYRELA291
Substituting1s=299792458mand1kg=7.42×10−28mewget
92−281/36
r=4.56×10(299792458·7.42×10)=1.85×10m
orFdyinga=9gewgetthroughsimilarprocess
921/392−281/36
r=2.76×10(skg)=2.76×10(299792458·7.42×10)=1.12×10m
Thepropertimetoelvtraeenbwettheseowtdistancecanbeobtainedybtheexpressionasin
Equation.(12.10)eabvo
τ=∫√˜2dr2M
E−1+r
˜˜
HereEisproportionaltoinitialenergyforysimplicitassumingE=1ewget
∫1√123√2√r3
τ=√rdr=√r2=
2M2M33M
Propertimeeenbwettheseowtdistancesis
[√√3]r2
τ=2r
3M
r1
orFM=1Mewget
⊙
[√√]6
1.12×10()
2r3m31/2
τ=30=4.45×10−7
31.98×106kg
1.85×10
−281
Substituting1kg=7.42×10mand1m=299792458sewget
−7(272)1−75−2
τ=4.45×101.34×10m2=4.45×10·1.22×10s=5.44×10s
Thisesgivthetimeformilduncomfortablyanddeath.□
(c)wHoabouta10M⊙
Solution:
RepeatingthesameprocessforM=10Mewget
⊙
r=9.83×109(s2kg)1/3=3.98×106m
1
r=5.95×109(s2kg)1/3=2.41×106m
2
()
m31/3
τ=4.44×10−7kg=5.426×10−2s
Sofora10M⊙thetimealtervinforthefallingpersonfrommildyuncomfortabilittodeathis
5.42×10−2s.□
12.6.4.utsh(Sc12.9)
(a)wShothataphotonhwhicpropagatesinaradialullngeodesicofthemetric,,hasenergy−p
0
erselyvinproportionaltoR(t).
CHAPTER12.GENERALTIVITYRELA292
Solution:
Theengivmetricis

−1000
R2(t)
g=01−kr200
µν00R2(t)r20
000R2(t)r2sin2θ
orFradialgeodesicUϕ=Uθ=0.Sincephotonismasslessewget
002rr2
p·p=0=⇒g(p)+g(p)=0
0r
Simplifyingesgiv
grrR2(t)
222
(p)=−(p)=(p)(12.11)
0g00r1−kr2r
eWwnoevhatoifndtherelationshipeenbwetpandthetelemenofmetric.Thenextrelationship
r
comesfromthegeodesicequationas
ρν
µµpp
p˙=Γρν0
p
Speciifcallyforµ=0ewget
ρν
p˙0=Γ0pp
ρν0
p
eWneedtheChristoffelbsymolsforthis.TheChristoffelbsymolsareengivyb
µ1µσ()
Γ=gg+g−g
νρ2νσ,ρρσ,ννρ,σ
0
TheonlyrequiredChristoffelbsymolsareΓ
αβ
0100()
Γ=gg+g−g
νρ2ν0,ρρ0,ννρ,0
Explicitly
()((2))˙
01001R(t)R(t)R(t)
Γ=gg+g−g=(−1)−∂=
rr2r0,rr0,rrr,02t1−kr21−kr2
Substitutingthisinthegeodesicequationewget
˙rr
0R(t)R(t)pp
p˙=−20
1−krp
r2
Butfrom(12.11)ewevha(p)andsubstitutingewget
˙202
0R(t)R(t)(1−kr)(p)
p˙=−220
1−krR(t)p
˙
0R(t)0
p˙=−R(t)p
Thisisatialdifferenequation,solvingewget
dp0dR(t)()1
00
0=−lnp=−ln(R(t))=⇒p∝
pR(t)R(t)
CHAPTER12.GENERALTIVITYRELA293
0
eringwLotheindexofpintheLHSewget
001
p=gp=−p=⇒p∝−
0000R(t)
hWhicistherequiredexpression.□
(b)wShofromthisthataphotonemittedattimeteandedreceivatimetrybersobservatrestinthe
cosmologicalreferencefromisredshiftedyb
1+z=R(tr)
R(te)
Solution:
orFanerobservatrestvi=0=⇒Ui=0.UsingU·U=−1esgiv
g00(U0)2=−1=⇒U0=√−1=1
g
00
usththeedobservenergyis
E=−p·U=−pU0=−p
obsobs00
calculatingtheredshiftewget
E(t)−E(t)−1+1
obseobsrR(t)R(t)
z==er=
Eobs(tr)−1
R(t)
r
Simplifying
−1+1R(t)
1+z=1+R(te)R(tr)=r
−1R(te)
R(tr)
Thisistherequiredexpression.□
12.6.5.utsh(Sc12.20)AssumethattheerseunivismatterdominatedandifndthealuevofρΛthatpermits
theerseunivtobestatic.
[]
R3
(a)Becausetheerseunivismatter-dominatedatthetpresentime,ewcanetakρm(t)=ρ00
R(t)
wherethesubscript0referstothestaticsolutionewarelookingfor.tiateDifferenthe‘energy’
equation
1˙2142
R=−k+πR(ρ+ρ)(12.12)
223mΛ
withrespecttotimetoifndthedynamicalequationerningvgoamatterdominatederse:univ
¨843−2
R=3πρΛR−3πρ0R0R
Setthistozerotoifndthesolution
ρ=1ρ
Λ20
orFEinstein’sstaticsolution,thecosmologicaltconstanenergyydensithastobehalfofthmatter
energy.ydensit
CHAPTER12.GENERALTIVITYRELA294
Solution:
Asinstructed,tiatingdifferenwithrespecttotieewget
1¨˙8˙42
·2·RR=πRR(ρ+ρ)+πR(ρ˙+ρ˙)
23mΛ3mΛ
¨842
R=3πR(ρm+ρΛ)+3πR(ρ˙m+ρ˙Λ)
Butthefunctionalformofρm(t)isengivtiatingdifferenewget
3˙
Rρ0R
ρ˙=−30
mR4
Andformatterdominatederseunivρ˙Λ=0substitutingthese
3˙
8ρRR
¨˙˙00
RR=3πRR(ρm+ρΛ)−4πR2
8ρ0R3
¨0
R=πR(ρ+ρ)−4π
3mΛR2
hWhicistherequireddynamicalequation.tAtcurrentimeewevhaR=Rsoewget
0
¨884
R=R(ρ+ρ)−4ρR=Rρ−Rρ
300Λ0030Λ300
Settingthisequaltozeroewget
8Rρ=4Rρ
30Λ300
eWobtain
ρ=1ρ
Λ20
Thisistherequiredexpression.□
(b)Putourexpressionforρmtointhet-hand-siderighof(12.12)togenaneenergy-likexpression
hwhichasaeativderivthathastoanishvforastaticsolution.erifyVthattheeabvoconditionof
ρdoesindeedemaktheifrsteativderivanish.v
Λ
Solution:
Substitutingρewobtain
m
114(ρR31)
˙2200
2R=−2k+3πRR3+2ρ0
orFstaticsolutionthesecondtermonthetrighhastoevhaanishingveativderivbecausetheifrst
beingtconstanhaszeroeativderiv.alreadykingChec
∂[4(ρ0R31)]4∂[(R3R2)]4[(R3)]
πR20+ρ=πρ0+=πρ−0+R
∂R3R32030∂RR230R2
orFinitialtimeewevhaR=R0thisexpressionaluatesevtozero.□
(c)Computethesecondeativderivofthet-hand-siderighof(12.12)withrespecttoRandwshothat,
thestaticsolution,itispe.ositivThismeansthatthe‘ptial‘otenisaumminimand’sEinstein
staticsolutionisstable.
CHAPTER12.GENERALTIVITYRELA295
Solution:
Thesecondeativderivis
4∂[(R3)]4[(R3)]
πρ−0+R=πρ20+1
30∂RR230R3
orFtoydaR=R0andewget
4[(R3)]4[(R3)]
πρ20+1=πρ20+1=4πρ
30R330R30
00
orFρ>0thesecondeativderivispe.ositivThismeansthesolutionisstable.□
